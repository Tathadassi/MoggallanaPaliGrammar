% 《墨甘兰语法》主文档
% 南传上座部佛教巴利语法文献译疏主文档
% 请使用LuaLaTeX编译(支持缅甸文和巴利文排版)
%%%%%%%%%%%%%%%%%%%%%%%%%%%%%%%%%%%%%%%%%%%%%%%%%%%%%%%%%%%%%%%

% 版本信息
% ======================================================================
% Version: 1.0.0
% Release Date: 2025-10-09
% Author: 德升尊者 (Bhikkhu Gunodaya)
% Maintainer: 德升尊者
% License: 法布施 (Dhamma Dana)
% 
% 版本历史:
% 1.0.0 (2025-10-09) - 初始版本发布
%   - 完成墨甘兰语法底本
%   - 实现中文-巴利文-缅甸文混合排版
%   - 建立完整的语法注释系统
%   - 设计专业的学术排版格式
%
% 编译要求:
% - 引擎: LuaLaTeX (必需)
% - 字体: Padauk (缅甸文), TeX Gyre Schola (拉丁文)
% - 宏包: 详见 moggallana.cls 文档
%
% 项目结构:
% - text/cover.tex     : 封面设计
% - text/000.tex       : 体例说明  
% - text/01-07.tex     : 各章正文
% - references_Manual.bib: 参考文献数据库
%
% 注意事项:
% 1. 必须使用LuaLaTeX编译以支持缅甸文渲染
% 2. 确保系统中安装有Padauk字体
% 3. 参考文献使用biblatex处理,非BibTeX
% =====================================================================%

% 文档类设置
\documentclass[]{moggallana}

% 参考文献数据库设置
\addbibresource{references_Manual.bib} % 为biblatex引入参考文献数据库
\bibliography{references_Manual} % 利用biblatex工具生成参考文献

%%%%%%%%%%%%%%%%%%%%%%%%%%%%%%%%%%%%%%%%%%%%%%%%%%%%%%%%%%%%%%%

% 文档元数据设置
\title{\textbf{မောဂ္ဂလ္လာနဗျာကရဏံ}} % 巴利文书名
\author{%
    အာစရိ​ယေန\\ % 作者信息(巴利文)
    ဘဒန္တမောဂ္ဂလ္လာ​နေန\\ % 尊师名号
    ဝိရစိတံ % 撰述标识
}
\date{} % 出版日期留空

%%%%%%%%%%%%%%%%%%%%%%%%%%%%%%%%%%%%%%%%%%%%%%%%%%%%%%%%%%%%%%%

\begin{document}

% 封面页
% 《阿毗达摩点津》封面

%%%%%%%%%%%%%%%%%%%%%%%%%%%%%%%%%%%%%%%%%
% 封面页
%%%%%%%%%%%%%%%%%%%%%%%%%%%%%%%%%%%%%%%%%
% Formal Book Title Page
% LaTeX Template
% Version 2.0 (23/7/17)
%
% This template was downloaded from:
% http://www.LaTeXTemplates.com
%
% Original author:
% Peter Wilson (herries.press@earthlink.net) with modifications by:
% Vel (vel@latextemplates.com)
%
% License:
% CC BY-NC-SA 3.0 (http://creativecommons.org/licenses/by-nc-sa/3.0/)
% 
% This template can be used in one of two ways:
%
% 1) Content can be added at the end of this file just before the \end{document}
% to use this title page as the starting point for your document.
%
% 2) Alternatively, if you already have a document which you wish to add this
% title page to, copy everything between the \begin{document} and
% \end{document} and paste it where you would like the title page in your
% document. You will then need to insert the packages and document 
% configurations into your document carefully making sure you are not loading
% the same package twice and that there are no clashes.
%
%%%%%%%%%%%%%%%%%%%%%%%%%%%%%%%%%%%%%%%%%

%----------------------------------------------------------------------------------------
%	PACKAGES AND OTHER DOCUMENT CONFIGURATIONS
%----------------------------------------------------------------------------------------

%\documentclass[a4paper, 11pt, oneside]{book} % A4 paper size, default 11pt font size and oneside for equal margins

% 虚拟出版社
\newcommand{\plogo}{\faBook \space DhammaDāna\space\faLightbulb[regular]} % Generic dummy publisher logo

% 西文字体宏包,启用则巴利字母显示异常,禁用则正常
%\usepackage[utf8]{inputenc} % Required for inputting international characters
%\usepackage[T1]{fontenc} % Output font encoding for international characters
%\usepackage{fouriernc} % Use the New Century Schoolbook font

%----------------------------------------------------------------------------------------
%	TITLE PAGE
%----------------------------------------------------------------------------------------

%\begin{document}
\frontmatter
% 设置新的页面布局
\newgeometry{top=3cm, bottom=3cm, outer=3cm, inner=3cm}
\begin{titlepage} % Suppresses headers and footers on the title page

	\centering % Centre everything on the title page
	
	\scshape % Use small caps for all text on the title page
	
	\vspace*{\baselineskip} % White space at the top of the page
	
	%------------------------------------------------
	%	Title
	%------------------------------------------------
	
	\rule{\textwidth}{1.6pt}\vspace*{-\baselineskip}\vspace*{2pt} % Thick horizontal rule
	\rule{\textwidth}{0.4pt} % Thin horizontal rule
	
	\vspace{0.75\baselineskip} % Whitespace above the title
	
	{\LARGE \textbf{Moggallānabyākaraṇaṃ}\\} % Title
	
	\vspace{0.75\baselineskip} % Whitespace below the title
	
	\rule{\textwidth}{0.4pt}\vspace*{-\baselineskip}\vspace{3.2pt} % Thin horizontal rule
	\rule{\textwidth}{1.6pt} % Thick horizontal rule
	
	\vspace{2\baselineskip} % Whitespace after the title block
	
	%------------------------------------------------
	%	Subtitle
	%------------------------------------------------
	
	\paliroman{A distinctive and classical treatise on Pāḷi grammar.} % Subtitle or further description
	
	\vspace*{3\baselineskip} % Whitespace under the subtitle
	
	%------------------------------------------------
	%	Editor(s)
	%------------------------------------------------
	
%	Edited By
	
	\vspace{0.5\baselineskip} % Whitespace before the editors
	
	{\scshape\Large \begin{table}[h]
	\centering
	\begin{tabular}{rl}
%		\kaishu{墨甘兰尊师} & \heiti{撰述} \\
        \textbf{MoggallānaMahā​therena}\\
%        Viracitaṃ
%	    \kaishu{善吉祥尊师} & \heiti{新解}   \\
%	    \kaishu{历代大德} & \heiti{训}   \\
%		\kaishu{德升比库}  &  \heiti{译疏}
	\end{tabular}
%	\caption{}
%	\label{tab:my-table}
\end{table}
} % Editor list
	
	\vspace{0.5\baselineskip} % Whitespace below the editor list
	
%	\textit{The University of California \\ Berkeley} % Editor affiliation
	
	\vfill % Whitespace between editor names and publisher logo
	
	%------------------------------------------------
	%	Publisher
	%------------------------------------------------
	
	\plogo % Publisher logo
	
	\vspace{0.3\baselineskip} % Whitespace under the publisher logo
	
	2025 % Publication year
	
%	{\large publisher} % Publisher

\end{titlepage}

%\end{document}
%---------------------------------------------------------------------------------------- % 封面设计文件
%% 《阿毗达摩点津》封面

%%%%%%%%%%%%%%%%%%%%%%%%%%%%%%%%%%%%%%%%%
% 封面页
%%%%%%%%%%%%%%%%%%%%%%%%%%%%%%%%%%%%%%%%%
% Formal Book Title Page
% LaTeX Template
% Version 2.0 (23/7/17)
%
% This template was downloaded from:
% http://www.LaTeXTemplates.com
%
% Original author:
% Peter Wilson (herries.press@earthlink.net) with modifications by:
% Vel (vel@latextemplates.com)
%
% License:
% CC BY-NC-SA 3.0 (http://creativecommons.org/licenses/by-nc-sa/3.0/)
% 
% This template can be used in one of two ways:
%
% 1) Content can be added at the end of this file just before the \end{document}
% to use this title page as the starting point for your document.
%
% 2) Alternatively, if you already have a document which you wish to add this
% title page to, copy everything between the \begin{document} and
% \end{document} and paste it where you would like the title page in your
% document. You will then need to insert the packages and document 
% configurations into your document carefully making sure you are not loading
% the same package twice and that there are no clashes.
%
%%%%%%%%%%%%%%%%%%%%%%%%%%%%%%%%%%%%%%%%%

%----------------------------------------------------------------------------------------
%	PACKAGES AND OTHER DOCUMENT CONFIGURATIONS
%----------------------------------------------------------------------------------------

%\documentclass[a4paper, 11pt, oneside]{book} % A4 paper size, default 11pt font size and oneside for equal margins

% 虚拟出版社
\newcommand{\plogo}{\faBook 法布施\faLightbulb[regular]} % Generic dummy publisher logo

% 西文字体宏包,启用则巴利字母显示异常,禁用则正常
%\usepackage[utf8]{inputenc} % Required for inputting international characters
%\usepackage[T1]{fontenc} % Output font encoding for international characters
%\usepackage{fouriernc} % Use the New Century Schoolbook font

%----------------------------------------------------------------------------------------
%	TITLE PAGE
%----------------------------------------------------------------------------------------

%\begin{document}
\frontmatter
% 设置新的页面布局
\newgeometry{top=3cm, bottom=3cm, outer=3cm, inner=3cm}
\begin{titlepage} % Suppresses headers and footers on the title page

	\centering % Centre everything on the title page
	
	\scshape % Use small caps for all text on the title page
	
	\vspace*{\baselineskip} % White space at the top of the page
	
	%------------------------------------------------
	%	Title
	%------------------------------------------------
	
	\rule{\textwidth}{1.6pt}\vspace*{-\baselineskip}\vspace*{2pt} % Thick horizontal rule
	\rule{\textwidth}{0.4pt} % Thin horizontal rule
	
	\vspace{0.75\baselineskip} % Whitespace above the title
	
	{\LARGE \songti{\textbf{墨甘兰语法} \Large{\kaishu{译疏}}}\\} % Title
	
	\vspace{0.75\baselineskip} % Whitespace below the title
	
	\rule{\textwidth}{0.4pt}\vspace*{-\baselineskip}\vspace{3.2pt} % Thin horizontal rule
	\rule{\textwidth}{1.6pt} % Thick horizontal rule
	
	\vspace{2\baselineskip} % Whitespace after the title block
	
	%------------------------------------------------
	%	Subtitle
	%------------------------------------------------
	
	独树一帜的巴利语法经典 % Subtitle or further description
	
	\vspace*{3\baselineskip} % Whitespace under the subtitle
	
	%------------------------------------------------
	%	Editor(s)
	%------------------------------------------------
	
%	Edited By
	
	\vspace{0.5\baselineskip} % Whitespace before the editors
	
	{\scshape\Large \begin{table}[h]
	\centering
	\begin{tabular}{rl}
		\kaishu{墨甘兰尊师} & \heiti{撰述} \\
%	    \kaishu{善吉祥尊师} & \heiti{新解}   \\
%	    \kaishu{历代大德} & \heiti{训}   \\
		\kaishu{德升比库}  &  \heiti{译疏}
	\end{tabular}
%	\caption{}
%	\label{tab:my-table}
\end{table}
} % Editor list
	
	\vspace{0.5\baselineskip} % Whitespace below the editor list
	
%	\textit{The University of California \\ Berkeley} % Editor affiliation
	
	\vfill % Whitespace between editor names and publisher logo
	
	%------------------------------------------------
	%	Publisher
	%------------------------------------------------
	
	\plogo % Publisher logo
	
	\vspace{0.3\baselineskip} % Whitespace under the publisher logo
	
	2025 % Publication year
	
%	{\large publisher} % Publisher

\end{titlepage}

%\end{document}
%---------------------------------------------------------------------------------------- % 封面设计文件

% 前文部分(罗马数字页码)
\frontmatter 

% 书名页
\maketitle % 打印书名页信息
\cleardoublepage % 空白页

% 索引收集设置(注释状态,按需启用)
%\makeindex % 开启索引的收集功能

%%%%%%%%%%%%%%%%%%%%%%%%%%%%%%%%%%%%%%%%%%%%%%%%%%%%%%%%%%%%%%%

% 目录系统设置
% 使用etocsetnexttocdepth替代etocsettocdepth

% 简目设置(显示章名级别)
\setcounter{tocdepth}{0} % 显示章名级别
\renewcommand{\contentsname}{မာတိကာ} % 简目标题
%\renewcommand{\contentsname}{简\quad 目} % 简目标题
\tableofcontents % 打印简目目录

% 细目设置(显示详细目录结构)
\renewcommand{\contentsname}{မာတိကာ} % 细目标题
%\renewcommand{\contentsname}{细\quad 目} % 细目标题

%%%%%%%%%%%%%%%%%%%%%%%%%%%%%%%%%%%%%%%%%%%%%%%%%%%%%%%%%%%%%%%

% 体例说明章节
\include{text/000} % 体例说明文件

% 正文部分(阿拉伯数字页码)
\mainmatter

% 经文条目
\chapter*{မောဂ္ဂလ္လာနသုတ္တပါဌော}
\addcontentsline{toc}{chapter}{မောဂ္ဂလ္လာနသုတ္တပါဌော}
\kaishi


\section{သညာဒိကဏ္ဍော ပဌမော}
\markboth{မောဂ္ဂလ္လာနသုတ္တပါဌေ}{သညာဒိကဏ္ဍော ပဌမော}

\begin{suttalist}
\suttaitem{1}{1}{အအာဒယော တိတာလီသ ဝဏ္ဏာ။}
\suttaitem{2}{2}{ဒသာ-ဒေါ သရာ။}
\suttaitem{3}{3}{ဒွေဒွေ သဝဏ္ဏာ။}
\suttaitem{4}{4}{ပုဗ္ဗော ရဿော။}
\suttaitem{5}{5}{ပရော ဒီဃော။}
\suttaitem{6}{6}{ကာဒယော ဗျဉ္ဇနာ။}
\suttaitem{7}{7}{ပဉ္စ ပဉ္စကာ ဝဂ္ဂါ။}
\suttaitem{8}{8}{ဗိန္ဒု နိဂ္ဂဟီတံ။}
\suttaitem{9}{9}{ဣယုဝဏ္ဏာ ဈလာ နာမဿန္တေ။}
\suttaitem{10}{10}{ပိတ္ထိယံ။}
\suttaitem{11}{11}{ဃာ။}
\suttaitem{12}{12}{ဂေါ သျာလပနေ။}
\begin{jieshu}
    ဣတိ သညာ။
\end{jieshu}

\suttaitem{13}{13}{ဝိဓိဗ္ဗိသေသနန္တဿ။}
\suttaitem{14}{14}{သတ္တမိယံ ပုဗ္ဗဿ။}
\suttaitem{15}{15}{ပဉ္စမိယံ ပရဿ။}
\suttaitem{16}{16}{အာဒိဿ။}
\suttaitem{17}{17}{ဆဋ္ဌိယန္တဿ။}
\suttaitem{18}{18}{ငါနုဗန္ဓော။}
\suttaitem{19}{19}{ဋာနုဗန္ဓာ-နေကဝဏ္ဏာ သဗ္ဗဿ။}
\suttaitem{20}{20}{ဉ ကာနုဗန္ဓာဒျန္တာ။}
\suttaitem{21}{21}{မာနုဗန္ဓော သရာနမန္တာ ပရော။}
\suttaitem{22}{22}{ဝိပ္ပဋိသေဓေ။}
\suttaitem{23}{23}{သင်္ကေတော-နဝယဝေါ-နုဗန္ဓော။}
\suttaitem{24}{24}{ဝဏ္ဏပရေန သဝဏ္ဏော-ပိ။}
\suttaitem{25}{25}{န္တု ဝန္တုမန္တာဝန္တုတဝန္တုသမ္ဗန္ဓီ။}
\begin{jieshu}
    ဣတိ ပရိဘာသာ။
\end{jieshu}

\suttaitem{26}{26}{သရော လောပေါ သရေ။}
\suttaitem{27}{27}{ပရော ကွစိ။}
\suttaitem{28}{28}{န ဒွေ ဝါ။}
\suttaitem{29}{29}{ယုဝဏ္ဏာနမေဩ လုတ္တာ။}
\suttaitem{30}{30}{ယဝါ သရေ။}
\suttaitem{31}{31}{ဧဩနံ။}
\suttaitem{32}{32}{ဂေါဿာဝင်။}
\suttaitem{33}{33}{ဗျဉ္ဇနေ ဒီဃရဿာ။}
\suttaitem{34}{34}{သရမှာ ဒွေ။}
\suttaitem{35}{35}{စတုတ္ထဒုတိယေသွေသံ တတိယပဌမာ။}
\suttaitem{36}{36}{ဝီ-တိဿေ-ဝေ ဝါ။}
\suttaitem{37}{37}{ဧဩနမ ဝဏ္ဏေ။}
\suttaitem{38}{38}{နိဂ္ဂဟီတံ။}
\suttaitem{39}{39}{လောပေါ။}
\suttaitem{40}{40}{ပရသရဿ။}
\suttaitem{41}{41}{ဝဂ္ဂေ ဝဂ္ဂန္တော။}
\suttaitem{42}{42}{ယေဝဟိသု ဉော။}
\suttaitem{43}{43}{ယေ သံဿ။}
\suttaitem{44}{44}{မယဒါ သရေ။}
\suttaitem{45}{45}{ဝ-န-တ-ရ-ဂါ စာ-ဂမာ။}
\suttaitem{46}{46}{ဆာ ဠော။}
\suttaitem{47}{47}{တဒမိနာဒီနိ။}
\suttaitem{48}{48}{တဝဂ္ဂ-ဝ-ရ-ဏာနံ ယေ စဝဂ္ဂ-ဗ-ယ-ဉာ။}
\suttaitem{49}{49}{ဝဂ္ဂလသေဟိ တေ။}
\suttaitem{50}{50}{ဟဿ ဝိပလ္လာသော။}
\suttaitem{51}{51}{ဝေ ဝါ။}
\suttaitem{52}{52}{တထနရာနံ ဋဌဏလာ။}
\suttaitem{53}{53}{သံယောဂါဒိ လောပေါ။}
\suttaitem{54}{54}{ဝိစ္ဆာဘိက္ခညေသု ဒွေ။}
\suttaitem{55}{55}{သျာဒိလောပေါ ပုဗ္ဗဿေ-ကဿ။}
\suttaitem{56}{56}{သဗ္ဗာဒီနံ ဝီတိဟာရေ။}
\suttaitem{57}{57}{ယာဝဗောဓံ သမ္ဘမေ။}
\suttaitem{58}{58}{ဗဟုလံ။}
\end{suttalist}

\begin{jieshu}
ဣတိ မောဂ္ဂလ္လာနေ ဗျာကရဏေ သညာဒိကဏ္ဍော ပဌမော။
\end{jieshu}


\section{သျာဒိကဏ္ဍော ဒုတိယော}
\markboth{မောဂ္ဂလ္လာနသုတ္တပါဌေ}{ သျာဒိကဏ္ဍော ဒုတိယော}

\begin{suttalist}
\suttaitemmulti{59}{1}{ဒွေ ဒွေကာ-နေကေသု နာမသ္မာ သိ ယော၊ အံ ယော၊ နာ ဟိ၊ သ နံ၊ သ္မာ ဟိ၊ သ နံ၊ သ္မိံ သု။}
\suttaitem{60}{2}{ကမ္မေ ဒုတိယာ။}
\suttaitem{61}{3}{ကာလဒ္ဓါနမစ္စန္တသံယောဂေ။}
\suttaitem{62}{4}{ဂတိဗောဓာဟာရသဒ္ဒတ္ထာကမ္မကဘဇ္ဇာဒီနံ ပယောဇ္ဇေ။}
\suttaitem{63}{5}{ဟရာဒီနံ ဝါ။}
\suttaitem{64}{6}{န ခါဒါဒီနံ။}
\suttaganaitem{65}{1}{ဝဟိဿာ-နိယန္တုကေ။}
\suttaganaitem{66}{2}{ဘက္ခိဿာဟိံသာယံ။}
\suttaitem{67}{7}{ဓျာဒီဟိ ယုတ္တာ။}
\suttaitem{68}{8}{လက္ခဏိတ္ထမ္ဘူတဝိစ္ဆာသွဘိနာ။}
\suttaitem{69}{9}{ပတိပရီဟိ ဘာဂေ စ။}
\suttaitem{70}{10}{အနုနာ။}
\suttaitem{71}{11}{သဟတ္ထေ။}
\suttaitem{72}{12}{ဟီနေ။}
\suttaitem{73}{13}{ဥပေန။}
\suttaitem{74}{14}{သတ္တမျာဓိကျေ။}
\suttaitem{75}{15}{သာမိတ္တေ-ဓိနာ။}
\suttaitem{76}{16}{ကတ္တုကရဏေသု တတိယာ။}
\suttaitem{77}{17}{သဟတ္ထေန။}
\suttaitem{78}{18}{လက္ခဏေ။}
\suttaitem{79}{19}{ဟေတုမှိ။}
\suttaitem{80}{20}{ပဉ္စမီဏေ ဝါ။}
\suttaitem{81}{21}{ဂုဏေ။}
\suttaitem{82}{22}{ဆဋ္ဌီ ဟေတွတ္ထေဟိ။}
\suttaitem{83}{23}{သဗ္ဗာဒိတော သဗ္ဗာ။}
\suttaitem{84}{24}{စတုတ္ထီ သမ္ပဒါနေ။}
\suttaitem{85}{25}{တာဒတ္ထျေ။}
\suttaitem{86}{26}{ပဉ္စမျဝဓိသ္မာ။}
\suttaitem{87}{27}{အပပရီဟိ ဝဇ္ဇနေ။}
\suttaitem{88}{28}{ပဋိနိဓိပဋိဒါနေသု ပတိနာ။}
\suttaitem{89}{29}{ရိတေ ဒုတိယာ စ။}
\suttaitem{90}{30}{ဝိနာ-ညတြ တတိယာ စ။}
\suttaitem{91}{31}{ပုထနာနာဟိ။}
\suttaitem{92}{32}{သတ္တမျာဓာရေ။}
\suttaitem{93}{33}{နိမိတ္တေ။}
\suttaitem{94}{34}{ယမ္ဘာဝေါ ဘာဝလက္ခဏံ။}
\suttaitem{95}{35}{ဆဋ္ဌီ စာ-နာဒရေ။}
\suttaitem{96}{36}{ယတော နိဒ္ဓါရဏံ။}
\suttaitem{97}{37}{ပဌမာ-တ္ထမတ္တေ။}
\suttaitem{98}{38}{အာမန္တဏေ။}
\suttaitem{99}{39}{ဆဋ္ဌီ သမ္ဗန္ဓေ။}
\suttaitem{100}{40}{တုလျတ္ထေန ဝါ တတိယာ။}
\suttaitem{101}{41}{အတော ယောနံ ဋာဋေ။}
\suttaitem{102}{42}{နိနံ ဝါ။}
\suttaitem{103}{43}{သ္မာသ္မိန္နံ။}
\suttaitem{104}{44}{သဿာ-ယ စတုတ္ထိယာ။}
\suttaitem{105}{45}{ဃပတေ-ကသ္မိံ နာဒီနံ ယ-ယာ။}
\suttaitem{106}{46}{ဿာ ဝါ တေ-တိ-မာမူဟိ။}
\suttaitem{107}{47}{နမှိ နုက် ဒွါဒီနံ သတ္တရသန္နံ။}
\suttaitem{108}{48}{ဗဟုကတိန္နံ။}
\suttaitem{109}{49}{ဏ္ဏံဏ္ဏန္နံ တိတော ဈာ။}
\suttaitem{110}{50}{ဥဘိန္နံ။}
\suttaitem{111}{51}{သုဉ် သဿ။}
\suttaitem{112}{52}{ဿံ-ဿာ-ဿာယေသွိ-တရေ-က-ညေ-တိမာန-မိ။}
\suttaitem{113}{53}{တာယ ဝါ။}
\suttaitem{114}{54}{တေတိမာတော သဿ ဿာယ။}
\suttaitem{115}{55}{ရတျာဒီဟိ ဋော သ္မိနော။}
\suttaitem{116}{56}{သုဟိသု-ဘဿော။}
\suttaitem{117}{57}{လ္တုပိတာဒီနမာ သိမှိ။}
\suttaitem{118}{58}{ဂေ အ စ။}
\suttaitem{119}{59}{အယူနံ ဝါ ဒီဃော။}
\suttaitem{120}{60}{ဃဗြဟ္မာဒိတေ။}
\suttaitem{121}{61}{နာ-မ္မာဒီဟိ။}
\suttaitem{122}{62}{ရဿော ဝါ။}
\suttaitem{123}{63}{ဃော ဿံဿာဿာယံတိံသု။}
\suttaitem{124}{64}{ဧကဝစနယောသွဃောနံ။}
\suttaitem{125}{65}{ဂေ ဝါ။}
\suttaitem{126}{66}{သိသ္မိံ နာ-နပုံသကဿ။}
\suttaitem{127}{67}{ဂေါဿာ-ဂ-သိ-ဟိ-နံသု ဂါဝ-ဂဝါ။}
\suttaitem{128}{68}{သုမှိ ဝါ။}
\suttaitem{129}{69}{ဂဝံ သေန။}
\suttaitem{130}{70}{ဂုန္နံ စ နံနာ။}
\suttaitem{131}{71}{နာဿာ။}
\suttaitem{132}{72}{ဂါဝုမှိ။}
\suttaitem{133}{73}{ယံ ပီတော။}
\suttaitem{134}{74}{နံ ဈီတော။}
\suttaitem{135}{75}{ယောနံ နောနေ ပုမေ။}
\suttaitem{136}{76}{နော။}
\suttaitem{137}{77}{သ္မိံနော နိ။}
\suttaitem{138}{78}{အမ္ဗာဒီဟိ။}
\suttaitem{139}{79}{ကမ္မာဒိတော။}
\suttaitem{140}{80}{နာဿေ-နော။}
\suttaitem{141}{81}{ဈလာ သဿ နော။}
\suttaganaitem{142}{3}{ဣတော ကွစိ သဿ ဋာနုဗန္ဓော။}
\suttaitem{143}{82}{နာ သ္မာဿ။}
\suttaitem{144}{83}{လာ ယောနံ ဝေါ ပုမေ။}
\suttaitem{145}{84}{ဇန္တာဒိတော နော စ။}
\suttaitem{146}{85}{ကူတော။}
\suttaitem{147}{86}{လောပေါ-မုသ္မာ။}
\suttaitem{148}{87}{န နော သဿ။}
\suttaitem{149}{88}{ယောလောပ-နိသု ဒီဃော။}
\suttaitem{150}{89}{သုနံဟိသု။}
\suttaitem{151}{90}{ပဉ္စာဒီနံ စုဒ္ဒသန္နမ။}
\suttaitem{152}{91}{ယွာဒေါ န္တုဿ။}
\suttaitem{153}{92}{န္တဿ စ ဋ ဝံသေ။}
\suttaitem{154}{93}{ယောသု ဈိဿ ပုမေ။}
\suttaitem{155}{94}{ဝေဝေါသု လုဿ။}
\suttaitem{156}{95}{ယောမှိ ဝါ ကွစိ။}
\suttaitem{157}{96}{ပုမာ-လပနေ ဝေဝေါ။}
\suttaitem{158}{97}{သ္မာဟိသ္မိန္နံ မှာဘိမှိ။}
\suttaitem{159}{98}{သုဟိသွဿေ။}
\suttaitem{160}{99}{သဗ္ဗာဒီနံ နံမှိ စ။}
\suttaganaitem{161}{4}{ပုဗ္ဗပရာဝရဒက္ခိဏုတ္တရာဓရာနိ ဝဝတ္ထာယမသညာယံ။}
\suttaitem{162}{100}{သံသာနံ။}
\suttaitem{163}{101}{ဃပါ သဿ ဿာ ဝါ။}
\suttaitem{164}{102}{သ္မိနော ဿံ။}
\suttaitem{165}{103}{ယံ။}
\suttaitem{166}{104}{တိံ သဘာ-ပရိသာယ။}
\suttaitem{167}{105}{ပဒါဒီဟိ သိ။}
\suttaitem{168}{106}{နာဿ သာ။}
\suttaitem{169}{107}{ကောဓာဒီဟိ။}
\suttaitem{170}{108}{အတေန။}
\suttaitem{171}{109}{သိဿော။}
\suttaitem{172}{110}{ကွစေ ဝါ။}
\suttaitem{173}{111}{အံ နပုံသကေ။}
\suttaitem{174}{112}{ယောနံ နိ။}
\suttaitem{175}{113}{ဈလာ ဝါ။}
\suttaitem{176}{114}{လောပေါ။}
\suttaitem{177}{115}{ဇန္တုဟေတွီဃပေဟိ ဝါ။}
\suttaitem{178}{116}{ယေ ပဿိဝဏ္ဏဿ။}
\suttaitem{179}{117}{ဂသီနံ။}
\suttaitem{180}{118}{အသင်္ချေဟိ သဗ္ဗာသံ။}
\suttaitem{181}{119}{ဧကတ္ထတာယံ။}
\suttaitem{182}{120}{ပုဗ္ဗသ္မာ-မာဒိတော။}
\suttaitem{183}{121}{နာ-တော-မပဉ္စမိယာ။}
\suttaitem{184}{122}{ဝါ တတိယာသတ္တမီနံ။}
\suttaitem{185}{123}{ရာဇဿိ နာမှိ။}
\suttaitem{186}{124}{သုနံဟိသူ။}
\suttaitem{187}{125}{ဣမဿာနိတ္ထိယံ ဋေ။}
\suttaitem{188}{126}{နာမှ-နိ-မိ။}
\suttaitem{189}{127}{သိမှ-နပုံသကဿာ-ယံ။}
\suttaitem{190}{128}{တျတေတာနံ တဿ သော။}
\suttaitem{191}{129}{မဿာ-မုဿ။}
\suttaitem{192}{130}{ကေ ဝါ။}
\suttaitem{193}{131}{တ တဿ နော သဗ္ဗာသု။}
\suttaitem{194}{132}{ဋ သ-သ္မာ-သ္မိံ-ဿာယ-ဿံ-ဿာ-သံ-မှာ-မှိ-သွိ-မဿ စ။}
\suttaitem{195}{133}{ဋေ သိဿိသိသ္မာ။}
\suttaitem{196}{134}{ဒုတိယဿ ယောဿ။}
\suttaitem{197}{135}{ဧကစ္စာဒီဟ-တော။}
\suttaitem{198}{136}{န နိဿ ဋာ။}
\suttaitem{199}{137}{သဗ္ဗာဒီဟိ။}
\suttaitem{200}{138}{ယောနမေဋ်။}
\suttaitem{201}{139}{နာညံ စ နာမပ္ပဓာနာ။}
\suttaitem{202}{140}{တတိယတ္ထယောဂေ။}
\suttaitem{203}{141}{စတ္ထသမာသေ။}
\suttaitem{204}{142}{ဝေဋ်။}
\suttaitem{205}{143}{ပုဗ္ဗာဒီဟိ ဆဟိ။}
\suttaitem{206}{144}{မနာဒီဟိ သ္မိံသံနာသ္မာနံ သိသောဩသာသာ။}
\suttaganaitem{207}{5}{သုမေဓာဒီန-မဝုဒ္ဓိစ။}
\suttaganaitemmulti{208}{6}{သရ-ဝယာ-ယ-ဝါသ-စေတာ ဇလာ-သယ-က္ခယ-\\လောဟ-ပဋ-မနေသု။}
\suttaitem{209}{145}{သတော သဗ် ဘေ။}
\suttaitem{210}{146}{ဘဝတော ဝါ ဘောန္တော ဂ-ယော-နာသေ။}
\suttaitem{211}{147}{သိဿဂ္ဂိတော နိ။}
\suttaitem{212}{148}{န္တဿံ။}
\suttaitem{213}{149}{ဘူတော။}
\suttaitem{214}{150}{မဟန္တာရဟန္တာနံ ဋာ ဝါ။}
\suttaitem{215}{151}{န္တုဿ။}
\suttaitem{216}{152}{အံငံ နပုံသကေ။}
\suttaitem{217}{153}{ဟိမဝတော ဝါ ဩ။}
\suttaitem{218}{154}{ရာဇာဒိယုဝါဒိတွာ။}
\suttaganaitem{219}{7}{ဓမ္မော ဝါ-ညတ္ထေ။}
\suttaganaitem{220}{8}{ဣမော ဘာဝေ။}
\suttaitem{221}{155}{ဝါ-မှာ-နင်။}
\suttaitem{222}{156}{ယောနမာနော။}
\suttaitem{223}{157}{အာယောနော စ သခါ။}
\suttaitem{224}{158}{ဋေ သ္မိနော။}
\suttaitem{225}{159}{နောနာသေသွိ။}
\suttaitem{226}{160}{သ္မာနံသု ဝါ။}
\suttaitem{227}{161}{ယောသွံဟိသု စာရင်။}
\suttaitem{228}{162}{လ္တုပိတာဒီနမသေ။}
\suttaitem{229}{163}{နံမှိ ဝါ။}
\suttaitem{230}{164}{အာ။}
\suttaitem{231}{165}{သလောပေါ။}
\suttaitem{232}{166}{သုဟိသွာရင်။}
\suttaitem{233}{167}{နဇ္ဇာ ယောသွာမ်။}
\suttaitem{234}{168}{ဋိ ကတိမှာ။}
\suttaitem{235}{169}{ဋ ပဉ္စာဒီဟိ စုဒ္ဒသဟိ။}
\suttaitem{236}{170}{ဥဘ-ဂေါဟိ ဋော။}
\suttaitem{237}{171}{အာရင်သ္မာ။}
\suttaitem{238}{172}{ဋောဋေ ဝါ။}
\suttaitem{239}{173}{ဋာ နာသ္မာနံ။}
\suttaitem{240}{174}{ဋိ သ္မိနော။}
\suttaitem{241}{175}{ဒိဝါဒိတော။}
\suttaitem{242}{176}{ရဿာရင်။}
\suttaitem{243}{177}{ပိတာဒီနမနတွာဒီနံ။}
\suttaitem{244}{178}{ယုဝါဒီနံ သုဟိသွာနင်။}
\suttaitem{245}{179}{နောနာနေသွာ။}
\suttaitem{246}{180}{သ္မာသ္မိံနံ နာနေ။}
\suttaitem{247}{181}{ယောနံ နောနေ ဝါ။}
\suttaitem{248}{182}{ဣတော-ညတ္ထေ ပုမေ။}
\suttaitem{249}{183}{နေ သ္မိနော ကွစိ။}
\suttaitem{250}{184}{ပုမာ။}
\suttaitem{251}{185}{နာမှိ။}
\suttaitem{252}{186}{သုမှာ စ။}
\suttaitem{253}{187}{ဂဿံ။}
\suttaitem{254}{188}{သာဿံ-သေ စာနင်။}
\suttaitem{255}{189}{ဝတ္တဟာ သနန္နံ နောနာနံ။}
\suttaitem{256}{190}{ဗြဟ္မဿု ဝါ။}
\suttaitem{257}{191}{နာမှိ။}
\suttaitem{258}{192}{ပုမကမ္မထာမဒ္ဓါနံ ဝါ သသ္မာသု စ။}
\suttaitem{259}{193}{ယုဝါ သဿိနော။}
\suttaitem{260}{194}{နော-တ္တာ-တုမာ။}
\suttaitem{261}{195}{သုဟိသု နက်။}
\suttaitem{262}{196}{သ္မာဿ နာ ဗြဟ္မာ စ။}
\suttaitem{263}{197}{ဣမေ-တာန-မေနာ-နွာဒေသေ ဒုတိယာယံ။}
\suttaitem{264}{198}{ကိဿ ကော သဗ္ဗာသု။}
\suttaitem{265}{199}{ကိ သသ္မိံသု ဝါ-နိတ္ထိယံ။}
\suttaitem{266}{200}{ကိမံသိသု သဟ နပုံသကေ။}
\suttaitem{267}{201}{ဣမဿိဒံ ဝါ။}
\suttaitem{268}{202}{အမုဿာဒုံ။}
\suttaitem{269}{203}{သုမှာ-မုဿာ-သ္မာ။}
\suttaitem{270}{204}{နံမှိ တိစတုန္နမိတ္ထိယံ တိဿ-စတဿာ။}
\suttaitem{271}{205}{တိဿော စတဿော ယောမှိ သဝိဘတ္တီနံ။}
\suttaitem{272}{206}{တီဏိစတ္တာရိ နပုံသကေ။}
\suttaitem{273}{207}{ပုမေ တယောစတ္တာရော။}
\suttaitem{274}{208}{စတုရော ဝါ စတုဿ။}
\suttaitem{275}{209}{မယ-မသ္မာ-မှာဿ။}
\suttaitem{276}{210}{နံ-သေသွ-သ္မာကံ-မမံ။}
\suttaitem{277}{211}{သိမှ-ဟံ။}
\suttaitem{278}{212}{တုမှဿ တုဝံ တွမမှိ စ။}
\suttaitem{279}{213}{တယာတယီနံ တွ ဝါ တဿ။}
\suttaitem{280}{214}{သ္မာမှိ တွမှာ။}
\suttaitem{281}{215}{န္တန္တူနံ န္တော ယောမှိ ပဌမေ။}
\suttaitem{282}{216}{တံ နံမှိ။}
\suttaitem{283}{217}{တောတာတိတာ သသ္မာသ္မိံနာသု။}
\suttaitem{284}{218}{ဋဋာအံ ဂေ။}
\suttaitem{285}{219}{ယောမှိ ဒွိန္နံ ဒုဝေ ဒွေ။}
\suttaitem{286}{220}{ဒုဝိန္နံ နံမှိ ဝါ။}
\suttaitem{287}{221}{ရာဇဿ ရညံ။}
\suttaitem{288}{222}{နာသ္မာသု ရညာ။}
\suttaitem{289}{223}{ရညောရညဿရာဇိနော သေ။}
\suttaitem{290}{224}{သ္မိံမှိ ရညေရာဇိနိ။}
\suttaitem{291}{225}{သမာသေ ဝါ။}
\suttaitem{292}{226}{သ္မိံမှိ တုမှ-မှာနံ တယိ-မယိ။}
\suttaitem{293}{227}{အံမှိ တံ-မံ-တဝံ-မမံ။}
\suttaitem{294}{228}{နာသ္မာသု တယာ-မယာ။}
\suttaitem{295}{229}{တဝ-မမ-တုယှံ-မယှံ သေ။}
\suttaitem{296}{230}{ငံ-ငါကံ နံမှိ။}
\suttaitem{297}{231}{ဒုတိယေ ယောမှိ ဝါ။}
\suttaitem{298}{232}{အပါဒါ-ဒေါ ပဒတေ-ကဝါကျေ။}
\suttaitem{299}{233}{ယောနံဟိသွ-ပဉ္စမျာ ဝေါ-နော။}
\suttaitem{300}{234}{တေမေ နာသေ။}
\suttaitem{301}{235}{အနွာဒေသေ။}
\suttaitem{302}{236}{သပုဗ္ဗာ ပဌမန္တာ ဝါ။}
\suttaitem{303}{237}{န စ-ဝါ-ဟ-ဟေ-ဝယောဂေ။}
\suttaitem{304}{238}{ဒဿနတ္ထေ-နာ-လောစနေ။}
\suttaitem{305}{239}{အာမန္တဏံ ပုဗ္ဗ-မသန္တံ-ဝ။}
\suttaitem{306}{240}{န သာမညဝစနမေကတ္ထေ။}
\suttaitem{307}{241}{ဗဟူသု ဝါ။}
\end{suttalist}

\begin{jieshu}
ဣတိ မောဂ္ဂလ္လာနေ ဗျာကရဏေ သျာဒိကဏ္ဍော ဒုတိယော။
\end{jieshu}


\section{သမာသကဏ္ဍော တတိယော }
\markboth{မောဂ္ဂလ္လာနသုတ္တပါဌေ}{သမာသကဏ္ဍော တတိယော}

\begin{suttalist}
\suttaitem{308}{1}{သျာဒိ သျာဒိနေ-ကတ္ထံ။}
\suttaitemmulti{309}{2}{အသင်္ချံ ဝိဘတ္တိ-သမ္ပတ္တိ-သမီပ-သာကလျာ-ဘာဝ-ယထာ-ပစ္ဆာ-ယုဂပဒတ္ထေ။}
\suttaitem{310}{3}{ယထာ န တုလျေ။}
\suttaitem{311}{4}{ယာဝါ-ဝဓာရဏေ။}
\suttaitem{312}{5}{ပယျပါ-ဗဟိ-တိရော-ပုရေ-ပစ္ဆာ ဝါ ပဉ္စမျာ။}
\suttaitem{313}{6}{သမီပါ-ယာမေသွ-နု။}
\suttaitem{314}{7}{တိဋ္ဌဂွါဒီနိ။}
\suttaitem{315}{8}{ဩရေ-ပရိ-ပတိ-ပါရေ-မဇ္ဈေ-ဟေဋ္ဌု-ဒ္ဓါ-ဓောန္တော ဝါ ဆဋ္ဌိယာ။}
\suttaitem{316}{9}{တံ နပုံသကံ။}
\suttaitem{317}{10}{အမာဒိ။}
\suttaitem{318}{11}{ဝိသေသနမေကတ္ထေန။}
\suttaitem{319}{12}{နဉ်။}
\suttaitem{320}{13}{ကု-ပါ-ဒယော နိစ္စ-မသျာ-ဒိဝိဓိမှိ။}
\suttaganaitem{321}{9}{ပါဒယော ဂတာဒျတ္ထေ ပဌမာယ။}
\suttaganaitem{322}{10}{အစ္စာဒယော ကန္တာဒျတ္ထေ ဒုတိယာယ။}
\suttaganaitem{323}{11}{အဝါဒယော ကုဋ္ဌာဒျတ္ထေ တတိယာယ။}
\suttaganaitem{324}{12}{ပရိယာဒယော ဂိလာနာဒျတ္ထေ စတုတ္ထိယာ။}
\suttaganaitem{325}{13}{နျာဒယော ကန္တာဒျတ္ထေ ပဉ္စမိယာ။}
\suttaitem{326}{14}{စီ ကြိယတ္ထေဟိ။}
\suttaitem{327}{15}{ဘူသနာ-ဒရာ-နာဒရေသွ-လံ-သာ-သာ။}
\suttaitem{328}{16}{အညေ စ။}
\suttaitem{329}{17}{ဝါ-နေက-ညတ္ထေ။}
\suttaitem{330}{18}{တတ္ထ ဂဟေတွာ တေန ပဟရိတွာ ယုဒ္ဓေ သရူပံ။}
\suttaitem{331}{19}{စတ္ထေ။}
\suttaitem{332}{20}{သမာဟာရေ နပုံသကံ။}
\suttaitem{333}{21}{သင်္ချာဒိ။}
\suttaitem{334}{22}{ကွစေ-ကတ္တဉ္စ ဆဋ္ဌိယာ။}
\suttaitem{335}{23}{သျာဒီသု ရဿော။}
\suttaitem{336}{24}{ဃ-ပဿ-န္တဿာ-ပ္ပဓာနဿ။}
\suttaitem{337}{25}{ဂေါဿု။}
\suttaitem{338}{26}{ဣတ္ထိယမတွာ။}
\suttaitem{339}{27}{နဒါဒိတော ငီ။}
\suttaganaitem{340}{14}{ဂေါတော ဝါ။}
\suttaitem{341}{28}{ယက္ခာဒိတွိနီ စ။}
\suttaitem{342}{29}{အာရာမိကာဒီဟိ။}
\suttaganaitem{343}{15}{သညာယံ မာနုသော။}
\suttaitem{344}{30}{ယုဝဏ္ဏေဟိ နီ။}
\suttaitem{345}{31}{က္တိမှာ-ညတ္ထေ။}
\suttaitem{346}{32}{ဃရဏျာဒယော။}
\suttaganaitem{347}{16}{အာစရိယာ ဝါ ယလောပေါ စ။}
\suttaitem{348}{33}{မာတုလာဒိတွာနီ ဘရိယာယံ။}
\suttaganaitem{349}{17}{အဘရိယာယံ ခတ္တိယာ ဝါ။}
\suttaganaitem{350}{18}{ပုန္နာမသ္မာ ယောဂါ အပါလကန္တာ။}
\suttaitemmulti{351}{34}{ဥပမာ-သံဟိတ-သဟိတ-သညတ-သဟ-သဖ-ဝါမ-\\လက္ခဏာဒိတူ-ရုတူ။}
\suttaitem{352}{35}{ယုဝါ တိ။}
\suttaitem{353}{36}{န္တန္တူနံ ငီမှိ တော ဝါ။}
\suttaitem{354}{37}{ဘဝတော ဘောတော။}
\suttaitem{355}{38}{ဂေါဿာ-ဝင်။}
\suttaitem{356}{39}{ပုထုဿ ပထဝပုထဝါ။}
\suttaitem{357}{40}{သမာသန္တွ။}
\suttaitem{358}{41}{ပါပါဒီဟိ ဘူမိယာ။}
\suttaitem{359}{42}{သင်္ချာဟိ။}
\suttaitem{360}{43}{နဒီဂေါဒါဝရီနံ။}
\suttaitem{361}{44}{အသင်္ချေဟိ စာ-င်္ဂုလျာ-နညာ-သင်္ချတ္ထေသု။}
\suttaitem{362}{45}{ဒီဃာ-ဟော-ဝဿေ-ကဒေသေဟိ စ ရတ္တျာ။}
\suttaitem{363}{46}{ဂေါတွစတ္ထေ စာလောပေ။}
\suttaitem{364}{47}{ရတ္တိန္ဒိဝ-ဒါရဂဝ-စတုရဿာ။}
\suttaitem{365}{48}{အာယာမေ-နုဂဝံ။}
\suttaitem{366}{49}{အက္ခိသ္မာ-ညတ္ထေ။}
\suttaitem{367}{50}{ဒါရုမျင်္ဂုလျာ။}
\suttaitem{368}{51}{စိ ဝီတိဟာရေ။}
\suttaitem{369}{52}{လ္တိ-တ္ထိ-ယူဟိ ကော။}
\suttaitem{370}{53}{ဝါ-ညတော။}
\suttaitem{371}{54}{ဥတ္တရပဒေ။}
\suttaitem{372}{55}{ဣမဿိဒံ။}
\suttaitem{373}{56}{ပုံ ပုမဿ ဝါ။}
\suttaitem{374}{57}{ဋ န္တန္တူနံ။}
\suttaitem{375}{58}{အ။}
\suttaitem{376}{59}{မနာ-ဒျာ-ပါဒီန-မော မယေ စ။}
\suttaitem{377}{60}{ပရဿ သင်္ချာသု။}
\suttaitem{378}{61}{ဇနေ ပုထဿု။}
\suttaitem{379}{62}{သော ဆဿာ-ဟာ-ယတနေ ဝါ။}
\suttaitem{380}{63}{လ္တုပိတာဒီန-မာရင်-ရင်။}
\suttaitem{381}{64}{ဝိဇ္ဇာယောနိသမ္ဗန္ဓာနမာ တတြ စတ္ထေ။}
\suttaitem{382}{65}{ပုတ္တေ။}
\suttaitem{383}{66}{စိသ္မိံ။}
\suttaitem{384}{67}{ဣတ္ထိယံ ဘာသိတပုမိတ္ထီ ပုမေဝေ-ကတ္ထေ။}
\suttaitem{385}{68}{ကွစိ ပစ္စယေ။}
\suttaitem{386}{69}{သဗ္ဗာဒယော ဝုတ္တိမတ္ထေ။}
\suttaitem{387}{70}{ဇာယာယ ဇယံ ပတိမှိ။}
\suttaitem{388}{71}{သညာယ-မုဒေါ-ဒကဿ။}
\suttaitem{389}{72}{ကုမ္ဘာဒီသု ဝါ။}
\suttaitem{390}{73}{သောတာဒီသူလောပေါ။}
\suttaitem{391}{74}{ဋ နဉ်ဿ။}
\suttaitem{392}{75}{အန် သရေ။}
\suttaitem{393}{76}{နခါဒယော။}
\suttaitem{394}{77}{နဂေါ ဝါ-ပ္ပါဏိနိ။}
\suttaitem{395}{78}{သဟဿ သောညတ္ထေ။}
\suttaitem{396}{79}{သညာယံ။}
\suttaitem{397}{80}{အပ္ပစ္စက္ခေ။}
\suttaitem{398}{81}{အကာလေ သကတ္ထေ။}
\suttaitem{399}{82}{ဂန္ထန္တာ-ဓိကျေ။}
\suttaitem{400}{83}{သမာနဿ ပက္ခာဒီသု ဝါ။}
\suttaitem{401}{84}{ဥဒရေ ဣယေ။}
\suttaitem{402}{85}{ရီရိက္ခကေသု။}
\suttaitem{403}{86}{သဗ္ဗာဒီနမာ။}
\suttaitem{404}{87}{န္တကိမိမာနံ ဋာကီဋီ။}
\suttaitem{405}{88}{တုမှာ-မှာနံ တာ-မေ-ကသ္မိံ။}
\suttaitem{406}{89}{တံ-မ-မညတြ။}
\suttaitem{407}{90}{ဝေ-တဿေ-ဋ်။}
\suttaitem{408}{91}{ဝိဓာဒီသု ဒွိဿ ဒု။}
\suttaitem{409}{92}{ဒိ ဂုဏာဒီသု။}
\suttaitem{410}{93}{တီသွ။}
\suttaitem{411}{94}{အာ သင်္ချာယာ-သတာဒေါ-နညတ္ထေ။}
\suttaitem{412}{95}{တိဿေ။}
\suttaitem{413}{96}{စတ္တာလီသာ-ဒေါ ဝါ။}
\suttaitem{414}{97}{ဒွိဿာ စ။}
\suttaitem{415}{98}{ဗာစတ္တာလီသာ-ဒေါ။}
\suttaitem{416}{99}{ဝီသတိဒသေသု ပဉ္စဿ ပဏ္ဏပန္နာ။}
\suttaitem{417}{100}{စတုဿ စုစော ဒသေ။}
\suttaitem{418}{101}{ဆဿ သော။}
\suttaitem{419}{102}{ဧကဋ္ဌာနမာ။}
\suttaitem{420}{103}{ရ သင်္ချာတော ဝါ။}
\suttaitem{421}{104}{ဆတီဟိ ဠော စ။}
\suttaitem{422}{105}{စတုတ္ထ-တတိယာန-မဍ္ဎု-ဍ္ဎတိယာ။}
\suttaitem{423}{106}{ဒုတိယဿ သဟ ဒိယဍ္ဎဒိဝဍ္ဎာ။}
\suttaitem{424}{107}{သရေ ကဒ် ကုဿု-တ္တရတ္ထေ။}
\suttaitem{425}{108}{ကာ-ပ္ပတ္ထေ။}
\suttaitem{426}{109}{ပုရိသေ ဝါ။}
\suttaitem{427}{110}{ပုဗ္ဗာ-ပရ-ဇ္ဇ-သာယ-မဇ္ဈေဟာ-ဟဿ ဏှော။}
\end{suttalist}

\begin{jieshu}
ဣတိ မောဂ္ဂလ္လာနေ ဗျာကရဏေ သမာသကဏ္ဍော တတိယော။
\end{jieshu}

%校对至此
\section{ဏာဒိကဏ္ဍော စတုတ္ထော}
\markboth{မောဂ္ဂလ္လာနသုတ္တပါဌေ}{ဏာဒိကဏ္ဍော စတုတ္ထော}

\begin{suttalist}
\suttaitem{428}{1}{ဏော ဝါ ပစ္စေ။}
\suttaitem{429}{2}{ဝစ္ဆာဒိတော ဏာနဏာယနာ။}
\suttaganaitem{430}{19}{ကတာ ဏိယောဝ။}
\suttaganaitem{431}{20}{ကဏှော ဗြာဟ္မဏေ။}
\suttaitem{432}{3}{ကတ္တိကာဝိဓဝါဒီဟိ ဏေယျဏေရာ။}
\suttaitem{433}{4}{ဏျ ဒိစ္စာဒီဟိ။}
\suttaitem{434}{5}{အာ ဏိ။}
\suttaitem{435}{6}{ရာဇတော ညော ဇာတိယံ။}
\suttaitem{436}{7}{ခတ္တာ ယိယာ။}
\suttaitem{437}{8}{မနုတော ဿသဏ။}
\suttaitem{438}{9}{ဇနပဒနာမသ္မာ ခတ္တိယာ ရညေ စ ဏော။}
\suttaitem{439}{10}{ဏျ ကုရုသိဝီဟိ။}
\suttaitem{440}{11}{ဏ ရာဂါ တေန ရတ္တံ။}
\suttaitem{441}{12}{နက္ခတ္တေ-နိန္ဒုယုတ္တေန ကာလေ။}
\suttaitem{442}{13}{သာ-ဿ ဒေဝတာ ပုဏ္ဏမာသီ။}
\suttaitem{443}{14}{တမဓီတေ တံ ဇာနာတိ ကဏိကာ စ။}
\suttaitem{444}{15}{တဿ ဝိသယေ ဒေသေ။}
\suttaitem{445}{16}{နိဝါသေ တန္နာမေ။}
\suttaitem{446}{17}{အဒူရဘဝေ။}
\suttaitem{447}{18}{တေန နိဗ္ဗတ္တေ။}
\suttaitem{448}{19}{တမီဓတ္ထိ။}
\suttaitem{449}{20}{တတြ ဘဝေ။}
\suttaitem{450}{21}{အဇ္ဇာဒီဟိ တနော။}
\suttaitem{451}{22}{ပုရာတော ဏော စ။}
\suttaitem{452}{23}{အမာတွစ္စော။}
\suttaitem{453}{24}{မဇ္ဈာဒိတွိမော။}
\suttaitem{454}{25}{ကဏ ဏေယျ ဏေယျက ယိယာ။}
\suttaitem{455}{26}{ဏိကော။}
\suttaitem{456}{27}{တမဿ သိပ္ပံ သီလံ ပဏျံ ပဟရဏံ ပယောဇနံ။}
\suttaitem{457}{28}{တံ ဟန္တ ရဟတိ ဂစ္ဆတုဉ္ဆတိ စရတိ။}
\suttaitemmulti{458}{29}{တေန ကတံ ကီတံ ဗဒ္ဓမဘိသင်္ခတံ သံသဋ္ဌံ ဟတံ ဟန္တိ ဇိတံ ဇယတိ ဒိဗ္ဗတိ ခဏတိ တရတိ စရတိ ဝဟတိ ဇီဝတိ။}
\suttaitem{459}{30}{တဿ သံဝတ္တတိ။}
\suttaitem{460}{31}{တတော သမ္ဘူတမာဂတံ။}
\suttaitem{461}{32}{တတ္ထ ဝသတိ ဝိဒိတော ဘတ္တော နိယုတ္တော။}
\suttaitem{462}{33}{တဿိဒံ။}
\suttaitem{463}{34}{ဏော။}
\suttaitem{464}{35}{ဂဝါဒီဟိ ယော။}
\suttaitem{465}{36}{ပိတိတော ဘာတရိ ရေယျဏ။}
\suttaitem{466}{37}{မာတိတော စ ဘဂိနိယံ ဆော။}
\suttaitem{467}{38}{မာတာပိတူသွာ-မဟော။}
\suttaitem{468}{39}{ဟိတေ ရေယျဏ။}
\suttaitem{469}{40}{နိန္ဒာ-ညာတ-ပ္ပပဋိဘာဂရဿ ဒယာသညာသု ကော။}
\suttaganaitem{470}{21}{ဝတ္ထိတော ဣဝတ္ထေ ဧယျော။}
\suttaganaitem{471}{22}{သိလာယ ဏေယျော စ။}
\suttaganaitem{472}{23}{သာခါဒီဟိ ဣယော။}
\suttaganaitem{473}{24}{မုခါဒီဟိ ယော။}
\suttaganaitem{474}{25}{အာကသ္မိကေ ဘိဓေယေ ဤယော။}
\suttaganaitem{475}{26}{သက္ကရာဒီဟိ ဏော။}
\suttaganaitem{476}{27}{အင်္ဂုလျာဒီဟိ ဏိကော။}
\suttaitem{477}{41}{တမဿ ပရိမာဏံ ဏိကော စ။}
\suttaitem{478}{42}{ယတေ-တေဟိ တ္တကော။}
\suttaitem{479}{43}{သဗ္ဗာ စာ-ဝန္ထု။}
\suttaitem{480}{44}{ကိမှာ ရတိ ရီဝ ရီဝတက ရိတ္တကာ။}
\suttaitem{481}{45}{သဉ္ဇာတံ တာရကာဒိတွိတော။}
\suttaitem{482}{46}{မာနေ မတ္တော။}
\suttaitem{483}{47}{တဂ္ဃော စုဒ္ဓံ။}
\suttaitem{484}{48}{ဏော စ ပုရိသာ။}
\suttaitem{485}{49}{အယုဘဒွိတီဟံသေ။}
\suttaitemmulti{486}{50}{သင်္ချာယ သစ္စုတီသာ-သ၊ ဒသန္တာ-ဓိကာ-သ္မိံ သတသဟဿေ ဍော။}
\suttaitem{487}{51}{တဿ ပူရဏေ-ကာဒသာဒိတော ဝါ။}
\suttaitem{488}{52}{မ ပဉ္စာဒိကတီဟိ။}
\suttaitem{489}{53}{သတာဒီနမိ စ။}
\suttaitem{490}{54}{ဆာ ဋ္ဌဋ္ဌမာ။}
\suttaitem{491}{55}{ဧကာ ကာကျ-သဟာယေ။}
\suttaitem{492}{56}{ဝစ္ဆာဒီဟိ တနုတ္တေ တရော။}
\suttaitem{493}{57}{ကိမှာ နိဒ္ဓါရဏေ ရတရ ရတမာ။}
\suttaitem{494}{58}{တေန ဒတ္တေ လိယာ။}
\suttaitem{495}{59}{တဿ ဘာဝကမ္မေသု တ္တ တာ တ္တန ဏျ ဏေယျဏိယ ဏိယာ။}
\suttaitem{496}{60}{ဗျ ဝဒ္ဓဒါသာ ဝါ။}
\suttaitem{497}{61}{နဏ ယုဝါ ဗော စ ဝဿ။}
\suttaitem{498}{62}{အဏွာဒိတွိမော။}
\suttaitem{499}{63}{ဘာဝါ တေန နိဗ္ဗတ္တေ။}
\suttaitem{500}{64}{တရ တမိ-ဿိကိယိဋ္ဌာတိသယေ။}
\suttaitem{501}{65}{တန္နိဿိတေ လ္လော။}
\suttaitem{502}{66}{တဿ ဝိကာရာဝယဝေသု ဏ ဏိက ဏေယျ မယာ။}
\suttaitem{503}{67}{ဇတုတော ဿဏ ဝါ။}
\suttaitem{504}{68}{သမူဟေ ကဏ ဏ ဏိကာ။}
\suttaitem{505}{69}{ဇနာဒီဟိ တာ။}
\suttaitem{506}{70}{ဣယော ဟိတေ။}
\suttaitem{507}{71}{စက္ခွာဒိတော ဿော။}
\suttaitem{508}{72}{ဏျော တတ္ထ သာဓု။}
\suttaitem{509}{73}{ကမ္မာ နိယညာ။}
\suttaitem{510}{74}{ကထာဒိတွိကော။}
\suttaitem{511}{75}{ပထာဒီဟိ ဏေယျော။}
\suttaitem{512}{76}{ဒက္ခိဏာယာ-ရဟေ။}
\suttaitem{513}{77}{ရာယော တုမန္တာ။}
\suttaitem{514}{78}{တမေတ္ထ-ဿ-တ္ထီတိ မန္တု။}
\suttaitem{515}{79}{ဝန္တွဝဏ္ဏာ။}
\suttaitem{516}{80}{ဒဏ္ဍာဒိတွိက ဤ ဝါ။}
\suttaganaitem{517}{28}{ဥတ္တမီဏေ ဝ ဓနာ ဣကော။}
\suttaganaitem{518}{29}{အသန္နိဟိတေ အတ္ထာ။}
\suttaganaitem{519}{30}{တဒန္တာ စ။}
\suttaganaitem{520}{31}{ဝဏ္ဏန္တာ ဤယေဝ။}
\suttaganaitem{521}{32}{ဟတ္ထ ဒန္တေဟိ ဇာတိယံ။}
\suttaganaitem{522}{33}{ဝဏ္ဏတော ဗြဟ္မစာရိမှိ။}
\suttaganaitem{523}{34}{ပေါက္ခရာဒိတော ဒေသေ။}
\suttaganaitem{524}{35}{နာဝါယိ-ကော။}
\suttaganaitem{525}{36}{သုခဒုက္ခာ ဤ။}
\suttaganaitem{526}{37}{သိခါဒီဟိ ဝါ။}
\suttaganaitem{527}{38}{ဗလာ ဗာဟူရုပုဗ္ဗာ စ။}
\suttaitem{528}{81}{တပါဒီဟိ ဿီ။}
\suttaitem{529}{82}{မုခါဒိတော ရော။}
\suttaganaitem{530}{39}{ဒန္တဿု စ ဥန္နတဒန္တေ။}
\suttaitem{531}{83}{တုန္ဒျာဒီဟိ ဘော။}
\suttaitem{532}{84}{သဒ္ဓါဒိတွ။}
\suttaitem{533}{85}{ဏော တပါ။}
\suttaitem{534}{86}{အာလွဘိဇ္ဈာဒီဟိ။}
\suttaitem{535}{87}{ပိစ္ဆာဒိတွိလော။}
\suttaitem{536}{88}{သီလာဒိတော ဝေါ။}
\suttaganaitem{537}{40}{အဏ္ဏာ နိစ္စံ။}
\suttaganaitem{538}{41}{ဂါဏ္ဍိရာဇီဟိ သညာယံ။}
\suttaitem{539}{89}{မာယာမေဓာဟိ ဝီ။}
\suttaitem{540}{90}{သိဿရေ အာမျုဝါမီ။}
\suttaitem{541}{91}{လက္ချာ ဏော အ စ။}
\suttaitem{542}{92}{အင်္ဂါ နော ကလျာဏေ။}
\suttaitem{543}{93}{သော လောမာ။}
\suttaitem{544}{94}{ဣမိယာ။}
\suttaitem{545}{95}{တော ပဉ္စမျာ။}
\suttaitem{546}{96}{ဣတော တေတ္တော ကုတော။}
\suttaitem{547}{97}{အဘျာဒီဟိ။}
\suttaitem{548}{98}{အာဒျာဒီဟိ။}
\suttaitem{549}{99}{သဗ္ဗာဒိတော သတ္တမျာ တြတ္ထာ။}
\suttaitem{550}{100}{ကတ္ထေ-တ္ထကုတြာ-တြကွေ-ဟိဓ။}
\suttaitem{551}{101}{ဓိ သဗ္ဗာ ဝါ။}
\suttaitem{552}{102}{ယာ ဟိံ။}
\suttaitem{553}{103}{တာ ဟံ စ။}
\suttaitem{554}{104}{ကုဟိံ ကဟံ။}
\suttaitem{555}{105}{သဗ္ဗေ-ကည ယ တေဟိ ကာလေ ဒါ။}
\suttaitem{556}{106}{ကဒါ ကုဒါ သဒါ-ဓုနေ-ဒါနိ။}
\suttaitem{557}{107}{အဇ္ဇသဇ္ဇွပရဇ္ဇွေ-တရဟိကရဟာ။}
\suttaitem{558}{108}{သဗ္ဗာဒီဟိ ပကာရေ ထာ။}
\suttaitem{559}{109}{ကထမိတ္ထံ။}
\suttaitem{560}{110}{ဓာ သင်္ချာဟိ။}
\suttaitem{561}{111}{ဝေကာ ဇ္ဈံ။}
\suttaitem{562}{112}{ဒွိတီဟေဓာ။}
\suttaitem{563}{113}{တဗ္ဗတိ ဇာတိယော။}
\suttaitem{564}{114}{ဝါရသင်္ချာယ က္ခတ္တုံ။}
\suttaitem{565}{115}{ကတိမှာ။}
\suttaitem{566}{116}{ဗဟုမှာ ဓာ စ ပစ္စာသတ္တိယံ။}
\suttaitem{567}{117}{သကိံ ဝါ။}
\suttaitem{568}{118}{သော ဝီစ္ဆာ ပကာရေသု။}
\suttaitem{569}{119}{အဘူတတဗ္ဘာဝေ ကရာသဘူယောဂေ ဝိကာရာ စီ။}
\suttaitem{570}{120}{ဒိဿန္တညေပိ ပစ္စယာ။}
\suttaitem{571}{121}{အညသ္မိံ။}
\suttaitem{572}{122}{သကတ္ထေ။}
\suttaitem{573}{123}{လောပေါ။}
\suttaitem{574}{124}{သရာနမာဒိဿာ-ယုဝဏ္ဏဿာ ဧ ဩ ဏာနုဗန္ဓေ။}
\suttaitem{575}{125}{သံယောဂေ ကွစိ။}
\suttaitem{576}{126}{မဇ္ဈေ။}
\suttaitemmulti{577}{127}{ကောသဇ္ဇာဇ္ဇဝ ပါရိသဇ္ဇ သောဟဇ္ဇ မဒ္ဒဝါရိဿာသဘာဇည ထေယျ ဗာဟုသစ္စာ။}
\suttaitem{578}{128}{မနာဒီနံ သက။}
\suttaitem{579}{129}{ဥဝဏ္ဏဿာ-ဝင သရေ။}
\suttaitem{580}{130}{ယမှိ ဂေါဿ စ။}
\suttaitem{581}{131}{လောပေါ-ဝဏ္ဏိဝဏ္ဏာနံ။}
\suttaitem{582}{132}{ရာနုဗန္ဓေ-န္တသရာဒိဿ။}
\suttaitem{583}{133}{ကိသမဟတမိမေ ကသမဟာ။}
\suttaitem{584}{134}{အာယုဿာ-ယသ မန္တုမှိ။}
\suttaitem{585}{135}{ဇော ဝုဒ္ဓဿိယိဋ္ဌေသု။}
\suttaitem{586}{136}{ဗာဠှန္တိကပသတ္ထာနံ သာဓ နေဒ သာ။}
\suttaitem{587}{137}{ကဏကနာ-ပ္ပယုဝါနံ။}
\suttaitem{588}{138}{လောပေါ ဝီမန္တုဝန္တူနံ။}
\suttaitem{589}{139}{ဍေ သတိဿ တိဿ။}
\suttaitem{590}{140}{ဧတဿေဋ တ္တကေ။}
\suttaitem{591}{141}{ဏိကဿိ ယော ဝါ။}
\suttaitem{592}{142}{အဓာတုဿ ကေ-သျာဒိတော ဃေ-ဿိ။}
\end{suttalist}

\begin{jieshu}
ဣတိ မောဂ္ဂလ္လာနေ ဗျာကရဏေ ဏာဒိကဏ္ဍော စတုတ္ထော။
\end{jieshu}


\section{ခါဒိကဏ္ဍော ပဉ္စမော}
\markboth{မောဂ္ဂလ္လာနသုတ္တပါဌေ}{ခါဒိကဏ္ဍော ပဉ္စမော}

\begin{suttalist}
\suttaitem{593}{1}{တိဇမာနေဟိ ခသာ ခမာဝီမံသာသု။}
\suttaitem{594}{2}{ကိတာ တိကိစ္ဆာသံသယေသု ဆော။}
\suttaitem{595}{3}{နိန္ဒာယံ ဂုပဗဓာ ဗဿ ဘောစ။}
\suttaitem{596}{4}{တုံသ္မာ လောပေါ စိစ္ဆာယံ တေ။}
\suttaitem{597}{5}{ဤယော ကမ္မာ။}
\suttaitem{598}{6}{ဥပမာ-နာစာရေ။}
\suttaitem{599}{7}{အာဓာရာ။}
\suttaitem{600}{8}{ကတ္တုတာ-ယော။}
\suttaitem{601}{9}{စျတ္ထေ။}
\suttaitem{602}{10}{သဒ္ဒါဒီနိ ကရောတိ။}
\suttaitem{603}{11}{နမောတွဿော။}
\suttaitem{604}{12}{ဓာတွတ္ထေ နာမသ္မိ။}
\suttaitem{605}{13}{သစ္စာဒီဟာပိ။}
\suttaitem{606}{14}{ကြိယတ္ထာ။}
\suttaitem{607}{15}{စုရာဒိတော ဏိ။}
\suttaitem{608}{16}{ပယောဇကဗျာပါရေ ကာပိ စ။}
\suttaitem{609}{17}{ကျော ဘာဝကမ္မေသွပရောက္ခေသု မာနန္တတျာဒီသု။}
\suttaitem{610}{18}{ကတ္တရိ လော။}
\suttaitem{611}{19}{မံ စ ရုဓာဒီနံ။}
\suttaitem{612}{20}{ဏိဏာပျာပီဟိ ဝါ။}
\suttaitem{613}{21}{ဒိဝါဒီဟိ ယက။}
\suttaitem{614}{22}{တုဒါဒီဟိ ကော။}
\suttaitem{615}{23}{ဇျာဒီဟိက္နာ။}
\suttaitem{616}{24}{ကျာဒီဟိ က္ဏာ။}
\suttaitem{617}{25}{သွာဒီဟိ က္ဏော။}
\suttaitem{618}{26}{တနာဒိတွော။}
\suttaitem{619}{27}{ဘာဝကမ္မေသု တဗ္ဗာနီယာ။}
\suttaitem{620}{28}{ဃျဏ။}
\suttaitem{621}{29}{အာဿေ စ။}
\suttaitem{622}{30}{ဝဒါဒီဟိ ယော။}
\suttaganaitem{623}{42}{ဘုဇာန္နေ။}
\suttaitem{624}{31}{ကိစ္စ ဃစ္စ ဘစ္စ ဘဗ္ဗ လေယျာ။}
\suttaganaitem{625}{43}{သညာယံ ဘရာ။}
\suttaitem{626}{32}{ဂုဟာဒီဟိ ယက။}
\suttaitem{627}{33}{ကတ္တရိ လ္တုဏကာ။}
\suttaitem{628}{34}{အာဝီ။}
\suttaitem{629}{35}{အာသိံသာယ-မကော။}
\suttaitem{630}{36}{ကရာ ဏနော။}
\suttaitem{631}{37}{ဟာတော ဝီဟိကာလေသု။}
\suttaitem{632}{38}{ဝိဒါ ကူ။}
\suttaitem{633}{39}{ဝိတော ဉာတော။}
\suttaitem{634}{40}{ကမ္မာ။}
\suttaitem{635}{41}{ကွ စဏ။}
\suttaitem{636}{42}{ဂမာ ရူ။}
\suttaitemmulti{637}{43}{သမာနညဘဝန္တယာဒိတူပမာနာ ဒိသာ ကမ္မေရီရိက္ခာကာ။}
\suttaitem{638}{44}{ဘာဝကာရကေ သွဃဏဃကာ။}
\suttaitem{639}{45}{ဒါဓာတွိ။}
\suttaitem{640}{46}{ဝမာဒီဟျထု။}
\suttaitem{641}{47}{ကွိ။}
\suttaitem{642}{48}{အနော။}
\suttaitem{643}{49}{ဣတ္ထိယမဏ တ္တိ က ယကယာ စ။}
\suttaitem{644}{50}{ဇာဟာဟိ နိ။}
\suttaitem{645}{51}{ကရာ ရိရိယော။}
\suttaitem{646}{52}{ဣ ကိ တီ သရူပေ။}
\suttaitem{647}{53}{သီလာ-ဘိက္ခညာ-ဝဿကေသု ဏီ။}
\suttaitemmulti{648}{54}{ထာဝရိ-တ္တရ၊ ဘင်္ဂုရ၊ ဘိဒုရ၊ ဘာသုရ၊ ဘဿရာ။}
\suttaitem{649}{55}{ကတ္တရိ ဘူတေ က္တွန္တုတ္တာဝီ။}
\suttaitem{650}{56}{က္တော ဘာဝကမ္မေသု။}
\suttaitem{651}{57}{ကတ္တရိ စာရမ္ဘေ။}
\suttaitemmulti{652}{58}{ဌာ-သ၊ ဝသ၊ သိလိသ၊ သီ၊ ရုဟ၊ ဇရ၊ ဇနီဟိ။}
\suttaitem{653}{59}{ဂမနတ္ထာ ကမ္မကာဓာရေ စ။}
\suttaitem{654}{60}{အာဟာရတ္ထာ။}
\suttaitemmulti{655}{61}{တုံ တာယေ တဝေ ဘာဝေ ဘဝိဿတိ ကြိယာယံ တဒတ္ထာယံ။}
\suttaitem{656}{62}{ပဋိသေဓေ-လံခလူနံ၊ တုနက္တွာန၊ က္တွာ ဝါ။}
\suttaitem{657}{63}{ပုဗ္ဗေ-ကကတ္တုကာနံ။}
\suttaitem{658}{64}{န္တော ကတ္တရိ ဝတ္တမာနေ။}
\suttaitem{659}{65}{မာနော။}
\suttaitem{660}{66}{ဘာဝကမ္မေသု။}
\suttaitem{661}{67}{တေ ဿပုဗ္ဗာ-နာဂတေ။}
\suttaitem{662}{68}{ဏွာဒယော။}
\suttaitem{663}{69}{ခဆသာနမေကဿရောဒိ ဒွေ။}
\suttaitem{664}{70}{ပရောက္ခာယဉ္စ။}
\suttaitem{665}{71}{အာဒိသ္မာ သရာ။}
\suttaitem{666}{72}{န ပုန။}
\suttaitem{667}{73}{ယထိဋ္ဌံ သျာဒိနော။}
\suttaitem{668}{74}{ရဿော ပုဗ္ဗဿ။}
\suttaitem{669}{75}{လောပေါ-နာဒိဗျဉ္ဇနဿ။}
\suttaitem{670}{76}{ခဆသေသွဿိ။}
\suttaitem{671}{77}{ဂုပိဿုဿ။}
\suttaitem{672}{78}{စတုတ္ထ ဒုတိယာနံ တတိယပဌမာ။}
\suttaitem{673}{79}{ကဝဂ္ဂဟာနံ စဝဂ္ဂဇာ။}
\suttaitem{674}{80}{မာနဿ ဝီ ပရဿ စ မံ။}
\suttaitem{675}{81}{ကိတဿာ-သံသယေ တိ ဝါ။}
\suttaitem{676}{82}{ယုဝဏ္ဏာနမေ ဩ ပစ္စယေ။}
\suttaitem{677}{83}{လဟုဿုပါန္တဿ။}
\suttaitem{678}{84}{အဿာ ဏာနုဗန္ဓေ။}
\suttaitem{679}{85}{န တေ ကာနုဗန္ဓနာဂမေသု။}
\suttaitem{680}{86}{ဝါ ကွစိ။}
\suttaitem{681}{87}{အညတြာ ပိ။}
\suttaitem{682}{88}{ပျေ သိဿာ။}
\suttaitem{683}{89}{ဧဩနမယဝါ သရေ။}
\suttaitem{684}{90}{အာယာဝါ ဏာနုဗန္ဓေ။}
\suttaitem{685}{91}{အာဿာ ဏာပိမှိ ယုက။}
\suttaitem{686}{92}{ပဒါဒီနံ ကွစိ။}
\suttaitem{687}{93}{မံ ဝါ ရုဓာဒီနံ။}
\suttaitem{688}{94}{ကွိမှိ လောပေါ-န္တ ဗျဉ္ဇနဿ။}
\suttaitem{689}{95}{ပရရူပမယကာရေ ဗျဉ္ဇနေ။}
\suttaitem{690}{96}{မနာနံ နိဂ္ဂဟီတံ။}
\suttaitem{691}{97}{န ဗြူဿော။}
\suttaitem{692}{98}{ကဂါ စဇာနံ ဃာနုဗန္ဓေ။}
\suttaitem{693}{99}{ဟနဿ ဃာတော ဏာနုဗန္ဓေ။}
\suttaitem{694}{100}{ကွိမှိ ဃော ပရိပစ္စာသမောဟိ။}
\suttaitem{695}{101}{ပရဿ ဃံသေ။}
\suttaitem{696}{102}{ဇိဟရာနံ ဂီ။}
\suttaitem{697}{103}{ဓာဿ ဟော။}
\suttaitem{698}{104}{ဏိမှိ ဒီဃော ဒုသဿ။}
\suttaitem{699}{105}{ဂုဟိဿ သရေ။}
\suttaitem{700}{106}{မုဟဗဟာနဉ္စ တေ ကာနုဗန္ဓေ တွေ။}
\suttaitem{701}{107}{ဝဟဿုဿ။}
\suttaitem{702}{108}{ဓာဿ ဟိ။}
\suttaitem{703}{109}{ဂမာဒိရာနံ လောပေါ-န္တဿ။}
\suttaitem{704}{110}{ဝစာဒီနံ ဝဿုဋ ဝါ။}
\suttaitem{705}{111}{အဿု။}
\suttaitem{706}{112}{ဝဒ္ဓဿ ဝါ။}
\suttaitem{707}{113}{ယဇဿ ယဿ ဋိယီ။}
\suttaitem{708}{114}{ဌာဿိ။}
\suttaitem{709}{115}{ဂါပါနမီ။}
\suttaitem{710}{116}{ဇနိဿာ။}
\suttaitem{711}{117}{သာသဿ သိသ ဝါ။}
\suttaitem{712}{118}{ကရဿာ တဝေ။}
\suttaitem{713}{119}{တုံတုနတဗ္ဗေသု ဝါ။}
\suttaitem{714}{120}{ဉာဿ နေ ဇာ။}
\suttaitem{715}{121}{သကာပါနံ ကုဏကူ ဏေ။}
\suttaitem{716}{122}{နိတော စိဿ ဆော။}
\suttaitem{717}{123}{ဇရသဒါနမီမ ဝါ။}
\suttaitemmulti{718}{124}{ဒိသဿ ပဿ ဒဿ ဒသ ဒ ဒက္ခာ။}
\suttaitem{719}{125}{သမာနာ ရော ရီရိက္ခကေသု။}
\suttaitem{720}{126}{ဒဟဿ ဒဿ ဍော။}
\suttaitem{721}{127}{အနဃဏသွာပရီဟိ ဠော။}
\suttaitem{722}{128}{အတျာဒိန္တေသွတ္ထိဿ ဘူ။}
\suttaitem{723}{129}{အအာဿအာဒီသု။}
\suttaitem{724}{130}{န္တမာနာန္တိယိယုံ သွာဒိလောပေါ။}
\suttaitem{725}{131}{ပါဒိတော ဌာဿ ဝါ ဌဟော ကွစိ။}
\suttaitem{726}{132}{ဒါဿိ ယင။}
\suttaitem{727}{133}{ကရောတိဿ ခေါ။}
\suttaitem{728}{134}{ပုရာ သ္မာ။}
\suttaitem{729}{135}{နိတော ကမဿ။}
\suttaitem{730}{136}{ယုဝဏ္ဏာနမိယငုဝင သရေ။}
\suttaitem{731}{137}{အညာဒိဿာဿီ ကျေ။}
\suttaitem{732}{138}{တနဿာ ဝါ။}
\suttaitem{733}{139}{ဒီဃော သရဿ။}
\suttaitem{734}{140}{သာ-နန္တရဿ တဿ ဌော။}
\suttaitem{735}{141}{ကသဿိမ စ ဝါ။}
\suttaitem{736}{142}{ဓသ္တောတြသ္တာ။}
\suttaitem{737}{143}{ပုစ္ဆာဒိတော။}
\suttaitemmulti{738}{144}{သာသ၊ ဝသ၊ သံသ၊ သသာ ထော။}
\suttaitem{739}{145}{ဓော ဓဟဘေဟိ။}
\suttaitem{740}{146}{ဒဟာ ဎော။}
\suttaitem{741}{147}{ဗဟဿုမ စ။}
\suttaitem{742}{148}{ရုဟာဒီဟိ ဟော ဠ စ။}
\suttaitem{743}{149}{မုဟာ ဝါ။}
\suttaitem{744}{150}{ဘိဒါဒိတော နော က္တက္တဝန္တူနံ။}
\suttaitem{745}{151}{ဒါတွိန္နော။}
\suttaitem{746}{152}{ကိရာဒီဟိ ဏော။}
\suttaitem{747}{153}{တရာဒီဟိ ရိဏ္ဏော။}
\suttaitem{748}{154}{ဂေါ ဘန္ဇာဒီဟိ။}
\suttaitem{749}{155}{သုသာ ခေါ။}
\suttaitem{750}{156}{ပစာ ကော။}
\suttaitem{751}{157}{မုစာ ဝါ။}
\suttaitem{752}{158}{လောပေါ ဝဍ္ဎာ က္တိဿ။}
\suttaitem{753}{159}{ကွိဿ။}
\suttaitem{754}{160}{ဏိဏာပီနံ တေသု။}
\suttaitem{755}{161}{ကွစိ ဝိကရဏာနံ။}
\suttaitem{756}{162}{မာနဿ မဿ။}
\suttaitem{757}{163}{ဉိ လဿေ။}
\suttaitem{758}{164}{ပျော ဝါ တွာဿ သမာသေ။}
\suttaitem{759}{165}{တုံယာနာ။}
\suttaitem{760}{166}{ဟနာ ရစ္စော။}
\suttaitem{761}{167}{သာသာဓိကရာ စ စ ရိစ္စာ။}
\suttaitem{762}{168}{ဣတော စ္စော။}
\suttaitem{763}{169}{ဒိသာ ဝါနဝါသ စ။}
\suttaitem{764}{170}{ဉိ ဗျဉ္ဇနဿ။}
\suttaitem{765}{171}{ရာ နဿ ဏော။}
\suttaitem{766}{172}{န န္တမာနတျာဒီနံ။}
\suttaitem{767}{173}{ဂမယမိသာသဒိသာနံ ဝါ စ္ဆင။}
\suttaitem{768}{174}{ဇရမရာနမီယင။}
\suttaitem{769}{175}{ဌာပါနံ တိဋ္ဌ ပိဝါ။}
\suttaitemmulti{770}{176}{ဂမဝဒဒါနံ ဃမ္မ ဝဇ္ဇ ဒဇ္ဇာ။}
\suttaitemmulti{771}{177}{ကရဿ သောဿ ကုဗ္ဗ ကုရု ကယိရာ။}
\suttaitem{772}{178}{ဂဟဿ ဃေပ္ပော။}
\suttaitem{773}{179}{ဏော နိဂ္ဂဟီတဿ။}
\end{suttalist}

\begin{jieshu}
ဣတိ မောဂ္ဂလ္လာနေ ဗျာကရဏေ ခါဒိကဏ္ဍော ပဉ္စမော။
\end{jieshu}


\section{တျာဒိကဏ္ဍော ဆဋ္ဌော}
\markboth{မောဂ္ဂလ္လာနသုတ္တပါဌေ}{တျာဒိကဏ္ဍော ဆဋ္ဌော}

\begin{suttalist}
\suttaitemmulti{774}{1}{ဝတ္တမာနေ တိ အန္တိ၊ သိ ထ၊ မိ မ၊ တေ အန္တေ၊ သေ ဝှေ၊ ဧ မှေ။}
\suttaitemmulti{775}{2}{ဘဝိဿတိ ဿတိ ဿန္တိ၊ ဿသိ ဿထ၊ ဿာမိ ဿာမ၊ ဿတေ ဿန္တေ၊ ဿသေ ဿဝှေ၊ ဿံ ဿာမှေ။}
\suttaitem{776}{3}{နာမေ ဂရဟာဝိမှယေသု။}
\suttaitemmulti{777}{4}{ဘူတေ ဤ ဥံ၊ ဩ တ္ထ၊ ဣံ မှာ၊ အာ ဦ၊ သေ ဝှံ၊ အ မှေ။}
\suttaitemmulti{778}{5}{အနဇ္ဇတနေ အာ ဦ၊ ဩ တ္ထ၊ အ မှာ၊ တ္ထ တ္ထုံ၊ သေ ဝှံ၊ ဣံ မုသေ။}
\suttaitemmulti{779}{6}{ပရောက္ခေ အ ဥ၊ ဧ တ္ထ၊ အ မှ၊ တ္ထ ရေ၊ တ္ထော ဝှော၊ ဣ မှေ။}
\suttaitemmulti{780}{7}{ဧယျာဒေါ ဝါ တိပတ္တိယံ ဿာ ဿံသု၊ ဿေ ဿထ၊ ဿံ ဿာမှာ၊ ဿထ ဿိံသု၊ ဿသေ ဿဝှေ၊ ဿိံ ဿာမှသေ။}
\suttaitemmulti{781}{8}{ဟေတုဖလေသွေယျ ဧယျုံ၊ ဧယျာသိ ဧယျာထ၊ ဧယျာမိ ဧယျာမ၊ ဧထ ဧရံ၊ ဧထော ဧယျာဝှော၊ ဧယျံ ဧယျာမှေ။}
\suttaitem{782}{9}{ပဉ္စပတ္ထနာဝိဓီသု။}
\suttaitemmulti{783}{10}{တု အန္တု၊ ဟိ ထ၊ မိမ၊ တံ အန္တံ၊ ဿု ဝှော၊ ဧ အာမဓသ။}
\suttaitem{784}{11}{သတျရဟေသွေယျာဒီ။}
\suttaitem{785}{12}{သမ္ဘာဝနေ ဝါ။}
\suttaitem{786}{13}{မာယောဂေ ဤအာအာဒီ။}
\suttaitemmulti{787}{14}{ပုဗ္ဗာပရစ္ဆက္ကာန မေကာနေကေသု တုမှာမှသေသေသု ဒွေဒွေ မဇ္ဈိမုတ္တမပဌမာ။}
\suttaitem{788}{15}{အာဤဿာဒီသွဥ ဝါ။}
\suttaitem{789}{16}{အအာဒီသွာဟော ဗြူဿ။}
\suttaitem{790}{17}{ဘူဿ ဝုက။}
\suttaitem{791}{18}{ပုဗ္ဗဿ အ။}
\suttaitem{792}{19}{ဥဿံ သွာဟာ ဝါ။}
\suttaitem{793}{20}{တျန္တီနံ ဋဋူ။}
\suttaitem{794}{21}{ဤအာဒေါ ဝစဿောမ။}
\suttaitem{795}{22}{ဒါဿ ဒံ ဝါ မိမေသွဒွိတ္တေ။}
\suttaitem{796}{23}{ကရဿ သောဿ ကုံ။}
\suttaitem{797}{24}{ကာ ဤအာဒီသု။}
\suttaitem{798}{25}{ဟာဿ စာဟင ဿေန။}
\suttaitem{799}{26}{လဘဝသစ္ဆိဒဘိဒရုဒါနံ စ္ဆင။}
\suttaitem{800}{27}{ဘုဇ မူစ ဝစ ဝိသာနံ က္ခင။}
\suttaitem{801}{28}{အာဤအာဒီသု ဟရဿာ။}
\suttaitem{802}{29}{ဂမိဿ။}
\suttaitem{803}{30}{ဍံသဿ စ ဆင။}
\suttaitem{804}{31}{ဟူဿ ဟေ ဟေဟိ ဟောဟီ ဿတျာဒေါ။}
\suttaitem{805}{32}{ဏာနာသု ရဿော။}
\suttaitem{806}{33}{အာဤဦမှာဿာဿမှာနံ ဝါ။}
\suttaitem{807}{34}{ကုသရုဟေဟီ-ဿ ဆိ။}
\suttaitem{808}{35}{အဤဿအာဒီနံ ဗျဉ္ဇနဿိဥ။}
\suttaitem{809}{36}{ဗြူတော တိဿီဥ။}
\suttaitem{810}{37}{ကျဿ။}
\suttaitemmulti{811}{38}{ဧယျာထဿေအအာဤထာနံ ဩအအံတ္ထတ္ထောဝှောက။}
\suttaitem{812}{39}{ဥံဿိံ သွံသု။}
\suttaitem{813}{40}{ဧဩတ္တာ သုံ။}
\suttaitem{814}{41}{ဟူတော ရေသုံ။}
\suttaitem{815}{42}{ဩဿ အဣတ္ထတ္ထော။}
\suttaitem{816}{43}{သိ။}
\suttaitem{817}{44}{ဒီဃာ ဤဿ။}
\suttaitem{818}{45}{မှာတ္ထာန မှ။}
\suttaitem{819}{46}{ဣံဿ စ သိဥ။}
\suttaitem{820}{47}{ဧယျုံဿုံ။}
\suttaitem{821}{48}{ဟိဿ-တော လောပေါ။}
\suttaitem{822}{49}{ကျဿ ဿေ။}
\suttaitem{823}{50}{အတ္ထိတေယျာဒိစ္ဆန္နံ သ သု သ သထ သံ သာမ။}
\suttaitem{824}{51}{အာဒိဒွိန္နမိယာဣယုံ။}
\suttaitem{825}{52}{တဿ ထော။}
\suttaitem{826}{53}{သိဟိသွဋ။}
\suttaitem{827}{54}{မိမာနံ ဝါ မိမှာ စ။}
\suttaitem{828}{55}{ဧသုင။}
\suttaitem{829}{56}{ဤအာဒေါ ဒီဃော။}
\suttaitem{830}{57}{ဟိမိမေသွဿ။}
\suttaitem{831}{58}{သကာ ဏာဿ ခ ဤအာဒေါ။}
\suttaitem{832}{59}{ဿေ ဝါ။}
\suttaitem{833}{60}{တေသု သုတော က္ဏောက္ဏာနံ ရောဋ။}
\suttaitem{834}{61}{ဉာဿ သနာဿ နာယော တိမှိ။}
\suttaitem{835}{62}{ဉာမှိ ဇံ။}
\suttaitem{836}{63}{ဧယျာဿိယာဉာ ဝါ။}
\suttaitem{837}{64}{ဤဿတျာဒီသု က္နာလောပေါ။}
\suttaitem{838}{65}{ဿဿ ဟိ ကမ္မေ။}
\suttaitem{839}{66}{ဧတိသ္မာ။}
\suttaitem{840}{67}{ဟနာ ဆ ခါ။}
\suttaitem{841}{68}{ဟတော ဟ။}
\suttaitem{842}{69}{ဒက္ခခဟေဟိ ဟောဟီဟိ လောပေါ။}
\suttaitem{843}{70}{ကယိရေယျဿေယျုမာဒီနံ။}
\suttaitem{844}{71}{ဋာ။}
\suttaitem{845}{72}{ဧထဿာ။}
\suttaitem{846}{73}{လဘာ ဣံဤနံ ထံထာ ဝါ။}
\suttaitem{847}{74}{ဂုရုပုဗ္ဗာ ရဿာ ရေ-န္တေ-န္တိနံ။}
\suttaitem{848}{75}{ဧယျေယျာသေယျန္နံ ဋေ။}
\suttaitem{849}{76}{ဩဝိကရဏဿု ပရစ္ဆက္ကေ။}
\suttaitem{850}{77}{ပုဗ္ဗစ္ဆက္ကေ ဝါ ကွစိ။}
\suttaitem{851}{78}{ဧယျာမဿေ မုစ။}
\end{suttalist}

\begin{jieshu}
ဣတိ မောဂ္ဂလ္လာနေ ဗျာကရဏေ တျာဒိကဏ္ဍော ဆဋ္ဌော။
\end{jieshu}


\section{ဏွာဒိကဏ္ဍော သတ္တမော}
\markboth{မောဂ္ဂလ္လာနသုတ္တပါဌေ}{ဏွာဒိကဏ္ဍော သတ္တမော}

\begin{suttalist}
\suttaitemmulti{852}{1}{စရ ဒရ ကရ ရဟ ဇန သန တလ သာဒ သာဓ ကသာသ စဋာ ယ ဝါဟိ ဏု။}
\suttaitemmulti{853}{2}{ဘရ မရ စရ တရ အရ ဂရ ဟန တန မန ဘမ ကိတ ဓန ဗံဟ ကမ္ဗမ္ဗ စက္ခ ဘိက္ခ သံကိန္ဒန္ဒ ယဇ ပဋာဏာသ ဝသ ပသ ပံသ ဗန္ဓာ ဥ။}
\suttaitem{854}{3}{ဗန္ဓာ ဦ ဝဓော စ။}
\suttaitem{855}{4}{ဇမ္ဗာဒယော။}
\suttaitem{856}{5}{တပုသ ဝိဓ ကုရ ပုထ မုဒါ ကု။}
\suttaitem{857}{6}{သိန္ဓာဒယော။}
\suttaitem{858}{7}{ဣ။}
\suttaitem{859}{8}{ဒဓျာဒယော။}
\suttaitem{860}{9}{ယုဝဏ္ဏုပန္တာ ကိ။}
\suttaitemmulti{861}{10}{ဝပ ဝရ ဝသ ရသ နဘ ဟရ ဟန ပဏာ ဤဏ။}
\suttaitem{862}{11}{ဘူ ဂမာ ဤဏ။}
\suttaitem{863}{12}{တန္ဒ လက္ခာ ဤ။}
\suttaitem{864}{13}{ဂမာ ရော။}
\suttaitem{865}{14}{ဣ ဘီ ကာ ကရာရ ဝက သက ဝါဟိ ကော။}
\suttaitem{866}{15}{ဦကာဒကော။}
\suttaitem{867}{16}{ဘီတွာ နကော။}
\suttaitem{868}{17}{သိင်္ဃာ အာဏိ ကာဋကာ။}
\suttaitem{869}{18}{ကရာဒိတွကော။}
\suttaitem{870}{19}{ဗလ ပတေ ဟျာကော။}
\suttaitem{871}{20}{သာမာကာဒယော။}
\suttaitemmulti{872}{21}{ဝိစ္ဆာ လ ဂမ မုသာ ကိကော။}
\suttaitem{873}{22}{ကိံ ကဏိကာဒယော။}
\suttaitem{874}{23}{ဣ သာ ကီကော။}
\suttaitem{875}{24}{ကမ ပဒါ ဏုကော။}
\suttaitem{876}{25}{မဏ္ဍ သလာ ဏူကော။}
\suttaitem{877}{26}{လူကာဒယော။}
\suttaitem{878}{27}{ကသာ သကော။}
\suttaitem{879}{28}{ကရာ တိကော။}
\suttaitem{880}{29}{ဣသာ ဌကန။}
\suttaitem{881}{30}{သမာ ခေါ။}
\suttaitem{882}{31}{မုခါဒယော။}
\suttaitemmulti{883}{32}{အဇ ဝဇ မုဒ ဂဒ ဂမာ ဂက။}
\suttaitem{884}{33}{သိင်္ဂါဒယော။}
\suttaitem{885}{34}{အဂါ ဂိ။}
\suttaitem{886}{35}{ယာဝလာ ဂု။}
\suttaitem{887}{36}{ဖေဂွာဒယော။}
\suttaitem{888}{37}{ဇနာ ဃော။}
\suttaitem{889}{38}{မေဃာဒယော။}
\suttaitem{890}{39}{စုသရ ဝရာ စော။}
\suttaitem{891}{40}{မရာ စုဤစီစ။}
\suttaitem{892}{41}{ကုသ ပသာ ဆိက။}
\suttaitem{893}{42}{ကသဥသာ ဆုက။}
\suttaitemmulti{894}{43}{အသ မသ ဝသ ကုစ ကစာ ဆော။}
\suttaitem{895}{44}{ဂုစ္ဆာဒယော။}
\suttaitem{896}{45}{အရာ ဇု ဥဋ စ။}
\suttaitem{897}{46}{ရဇ္ဇာဒယော။}
\suttaitem{898}{47}{ဂိဓာ ဈက။}
\suttaitem{899}{48}{ဝဉ္စျာဒယော။}
\suttaitem{900}{49}{ကမ ယဇာ ဉော။}
\suttaitem{901}{50}{ပုညံ။}
\suttaitemmulti{902}{51}{အရဟာညော ဟာဿ ဟိရ စ။}
\suttaitem{903}{52}{ကိရ တရာ ကီဋော။}
\suttaitem{904}{53}{သကာဒီဟျဋော။}
\suttaitemmulti{905}{54}{မကုဋာဝါဋ ကဝါဋ ကုက္ကုဋာ။}
\suttaitem{906}{55}{ကမုသ ကုသ ကသာ ဌော။}
\suttaitem{907}{56}{ကုဋ္ဌာဒယော။}
\suttaitem{908}{57}{ဝရ ကရာ အဏ္ဍော။}
\suttaitem{909}{58}{မနန္တာ ဍော။}
\suttaitem{910}{59}{ကုဏ္ဍာဒယော။}
\suttaitemmulti{911}{60}{တိဇ ကသ တသ ဒက္ခာ ကိဏော ဇဿ ခေါ စ။}
\suttaitem{912}{61}{ဝီအာဒိတော ဏိ။}
\suttaitem{913}{62}{ဂဟာဒီဟျ ဏိ။}
\suttaitem{914}{63}{ရီဝီဘာဟိ ဏု။}
\suttaitem{915}{64}{ခါဏွာဒယော။}
\suttaitem{916}{65}{ကွာဒိတောဏော။}
\suttaitem{917}{66}{သုဝီဟိ ဏက။}
\suttaitem{918}{67}{တိဏာဒယော။}
\suttaitem{919}{68}{ရဝဏ ဝရဏ ပူရဏာဒယော။}
\suttaitem{920}{69}{ပါဝသာ အတိ။}
\suttaitemmulti{921}{70}{ဓာဟိသိ တန ဇန ဇရ ဂမ သစာ တု။}
\suttaitem{922}{71}{အရိဿုဋ စ။}
\suttaitem{923}{72}{ပိတွာဒယော။}
\suttaitem{924}{73}{ဇန ကရာ ရတု။}
\suttaitem{925}{74}{သကာ ဥန္တော။}
\suttaitem{926}{75}{ကပါ ဩတော။}
\suttaitem{927}{76}{ဝသာဒီဟျန္တော။}
\suttaitem{928}{77}{ဟိသီနံ မုက စ။}
\suttaitem{929}{78}{ဟရ ရုဟ ကုလာ ဣတော။}
\suttaitem{930}{79}{ဘရာဒီဟျတော။}
\suttaitem{931}{80}{ကိရာဒီဟျာ-တက။}
\suttaitem{932}{81}{အမာဒီဟျ-တ္တော။}
\suttaitem{933}{82}{ဝါဒီဟိ တော။}
\suttaitem{934}{83}{ဃရာဒီဟိ တက။}
\suttaitem{935}{84}{နေတ္တာဒယော။}
\suttaitem{936}{85}{သမာဒီဟျထော။}
\suttaitem{937}{86}{ဥပဝသာ ဝဿောဋ စ။}
\suttaitem{938}{87}{ရမာ ထက။}
\suttaitem{939}{88}{တိတ္ထာဒယော။}
\suttaitem{940}{89}{ဝသ မသ ကုသာ ထု။}
\suttaitem{941}{90}{သက ဝသာ ထိ။}
\suttaitem{942}{91}{ဝီတော ထိက။}
\suttaitem{943}{92}{သာရိသ္မာ ရထိ။}
\suttaitem{944}{93}{တာ တာ ဣထိ။}
\suttaitem{945}{94}{ဣသာ ထီ။}
\suttaitemmulti{946}{95}{ရုဒ ခုဒ မုဒ မဒ ဆိဒ သူဒ သပ ကမာ ဒက။}
\suttaitem{947}{96}{ကုန္ဒာဒယော။}
\suttaitem{948}{97}{ဒဒါ ဒု။}
\suttaitem{949}{98}{ခနာန ဒမ ရမာ ဓော။}
\suttaitem{950}{99}{မုဒ္ဓါဒယော။}
\suttaitem{951}{100}{သီတော ဓုက။}
\suttaitemmulti{952}{101}{ဝရာရ ကရ တရ ဒရ ယမအဇ္ဇ မိထသကာ ကုနော။}
\suttaitem{953}{102}{အဇာ ဣနော။}
\suttaitem{954}{103}{ဝိပိနာဒယော။}
\suttaitem{955}{104}{ကိရာ ကနော။}
\suttaitem{956}{105}{ဒီ ဇိ ဣ မီဟိ နက။}
\suttaitem{957}{106}{သိ ဓာ ဝီ ဝါဟိ နော။}
\suttaitem{958}{107}{ဦနာဒယော။}
\suttaitem{959}{108}{ဝီပတာ တနော။}
\suttaitem{960}{109}{ရမာ တနက။}
\suttaitem{961}{110}{သူ ဘာဟိ နုက။}
\suttaitem{962}{111}{ဓာဿေ စ။}
\suttaitem{963}{112}{ဝတ္တာ ဋာဝ ဓမာသေဟျနိ။}
\suttaitem{964}{113}{ယုတော နိ။}
\suttaitem{965}{114}{စမာပ ပါ ဝပါ ပေါ။}
\suttaitem{966}{115}{ယု ထု ကုနံ ဒီဃော စ။}
\suttaitem{967}{116}{ခိပ သုပ နီ သူ ပူဟိ ပက။}
\suttaitem{968}{117}{သိပ္ပာဒယော။}
\suttaitem{969}{118}{သာသာ အပေါ။}
\suttaitem{970}{119}{ဝိဋပါဒယော။}
\suttaitem{971}{120}{ဂုပါ ဖော။}
\suttaitem{972}{121}{ဂရ သရာဒီဟိ ဗော။}
\suttaitem{973}{122}{နိမ္ဗာဒယော။}
\suttaitem{974}{123}{ဒရာ ဗိ။}
\suttaitemmulti{975}{124}{ကရ သရ သလ ကလ ဝလ္လ ဝသာ အဘော။}
\suttaitem{976}{125}{ဂဒါ ရဘော။}
\suttaitem{977}{126}{ဥသရာ သာ ကတော။}
\suttaitem{978}{127}{ဣတော ဘက။}
\suttaitem{979}{128}{ဂရာဝါ ဘော။}
\suttaitem{980}{129}{သောဗ္ဘာဒယော။}
\suttaitemmulti{981}{130}{ဥသ ကုသ ပဒ သုခါ ကုမော။}
\suttaitem{982}{131}{ဝဋုမာဒယော။}
\suttaitem{983}{132}{ဂုဓာ ဥမော။}
\suttaitem{984}{133}{ပဌ စရာ အမိမာ။}
\suttaitem{985}{134}{ဟိ ဓူဟိ မက။}
\suttaitem{986}{135}{တီတော ရီသနော စ။}
\suttaitemmulti{987}{136}{ခီ သု ဝီ ယာ ဂါ ဟိ သာ လူ ခု ဟု မရ ဓရ ကရ ဃရ ဇမာ မ သာမာ မော။}
\suttaitem{988}{137}{အသ္မာဒယော။}
\suttaitem{989}{138}{နီတော မိ။}
\suttaitem{990}{139}{ဦမိ ဘူမိ နိမိ ရသ္မိ။}
\suttaitem{991}{140}{မာ ဆာဟိ ယော။}
\suttaitem{992}{141}{ဇနိဿ ဇာ စ။}
\suttaitem{993}{142}{ဟဒယာဒယော။}
\suttaitemmulti{994}{143}{ခီ သိ သိနီ သီ သု ဝီ ကု သူ ဟိ ရက။}
\suttaitem{995}{144}{ဟိစိ ဒု မိနံ ဒီဃော စ။}
\suttaitem{996}{145}{ဓာတာ နမီ စ။}
\suttaitem{997}{146}{ဘဒြာဒယော။}
\suttaitemmulti{998}{147}{မန္ဒင်္က သသာ သ မ ဓ စတာ ဥရော။}
\suttaitem{999}{148}{ဝိဓုရာဒယော။}
\suttaitemmulti{1000}{149}{တိမရုဟရုဓဗဓမဒမန္ဒဝဇာ ဇရုစကသာ ကိရော။}
\suttaitem{1001}{150}{ထိရာဒယော။}
\suttaitem{1002}{151}{ဒဒဂရေဟိ ဒုရ ဘရာ။}
\suttaitem{1003}{152}{စရ ဒရ ဇရ ဂရ မရေဟိတေ။}
\suttaitem{1004}{153}{ပီတော ကွရော။}
\suttaitem{1005}{154}{စီဝရာဒယော။}
\suttaitem{1006}{155}{ကုတော ကြရော။}
\suttaitem{1007}{156}{ဝသာသာ ဆရော။}
\suttaitem{1008}{157}{မသာ ဆေရော စ။}
\suttaitem{1009}{158}{ဓူဝါတော သရော။}
\suttaitem{1010}{159}{ဘမာဒီဟျရော။}
\suttaitem{1011}{160}{ဝဒိဿ ဗဒ စ။}
\suttaitem{1012}{161}{ဝဒဇနာနံ ဌင စ။}
\suttaitem{1013}{162}{ပစိဿိဌင စ။}
\suttaitem{1014}{163}{ဝကာ အရဏ။}
\suttaitemmulti{1015}{164}{သိဂျင်္ဂါဂ မဇ္ဇကလာ လာ အာရော။}
\suttaitem{1016}{165}{ကမိဿ ဿု စ။}
\suttaitem{1017}{166}{ဘိင်္ဂါ (င်္ကာ) ရာဒယော။}
\suttaitem{1018}{167}{ကရာ မာရော။}
\suttaitem{1019}{168}{ပုသ သရေဟိ ခရော။}
\suttaitem{1020}{169}{သရ ဝသ ကလာ ကီရော ဝဿုဋ စ။}
\suttaitem{1021}{170}{ဂဗ္ဘီရာဒယော။}
\suttaitem{1022}{171}{ခဇ္ဇ ဝလ္လ မသာ ဦရော။}
\suttaitem{1023}{172}{ကပ္ပူရာဒယော။}
\suttaitem{1024}{173}{ကဌ စကာ ဩရော။}
\suttaitem{1025}{174}{မောရာဒယော။}
\suttaitem{1026}{175}{ကုတော ဧရက။}
\suttaitem{1027}{176}{ဘူသူဟိ ရိက။}
\suttaitem{1028}{177}{မီကသီနီဟိ ရု။}
\suttaitem{1029}{178}{သိနာ ဧရု။}
\suttaitem{1030}{179}{ဘီရုဟိ ရုက။}
\suttaitem{1031}{180}{တမာ ဗူလော။}
\suttaitem{1032}{181}{သိတော လကဝါလာ။}
\suttaitemmulti{1033}{182}{မင်္ဂ ကမ သမ္ဗ သဗ သက ဝသ ဝိသ ကေဝ ကလ ပလ္လ ကဌ ပဋ ကုဏ္ဍ မဏ္ဍာ အလော။}
\suttaitem{1034}{183}{မုသာ ကလော။}
\suttaitem{1035}{184}{ထလာဒယော။}
\suttaitem{1036}{185}{ကုလာ ကာလော စ။}
\suttaitem{1037}{186}{မုဠာလာဒယော။}
\suttaitem{1038}{187}{စဏ္ဍ ပတာ ဏာလော။}
\suttaitem{1039}{188}{မာဒိတော လော။}
\suttaitemmulti{1040}{189}{အန သန ကလ ကုက သဌ မဟာ ဣလော။}
\suttaitem{1041}{190}{ကုဋာ ကိလော။}
\suttaitem{1042}{191}{သိထိလာဒယော။}
\suttaitem{1043}{192}{စဋ ကဏ္ဍ ဝဋ္ဋ ပုထာ ကုလော။}
\suttaitem{1044}{193}{တုမုလာဒယော။}
\suttaitemmulti{1045}{194}{ကလ္လ ကပ တက္က ပဋာ ဩလော။}
\suttaitem{1046}{195}{အင်္ဂါ ဥလော လိ။}
\suttaitem{1047}{196}{အဉ္ဇာ လိ။}
\suttaitem{1048}{197}{ဆဒါ လိ။}
\suttaitem{1049}{198}{အလျာဒယော။}
\suttaitem{1050}{199}{ပိလာဒီ ဟျ ဝေါ။}
\suttaitem{1051}{200}{သာဠဝါဒယော။}
\suttaitem{1052}{201}{သရာ အာဝေါ။}
\suttaitem{1053}{202}{အလ မလ ဗိလာ ဏုဝေါ။}
\suttaitem{1054}{203}{ဂါတွီဝေါ။}
\suttaitem{1055}{204}{သုတော ကွ ကွာ။}
\suttaitem{1056}{205}{ဝိဒွါ။}
\suttaitem{1057}{206}{ထုတော ရေ ဝေါ။}
\suttaitem{1058}{207}{သမာ ရိဝေါ။}
\suttaitem{1059}{208}{ဆဒါ ရဝိ။}
\suttaitem{1060}{209}{ပူရ တိမာ ကိသော ရဿော စ။}
\suttaitem{1061}{210}{ကရာ ဤသော။}
\suttaitem{1062}{211}{သိရီသာဒယော။}
\suttaitem{1063}{212}{ကရာ ရိဗ္ဗိ သော။}
\suttaitemmulti{1064}{213}{သသာသ ဝသ ဝိသ ဟန ဝန မနာန ကမာ သော။}
\suttaitem{1065}{214}{အာမိ ထု ကုသီတော သက။}
\suttaitem{1066}{215}{ဖဿာဒယော။}
\suttaitem{1067}{216}{သုတော ဏိသက။}
\suttaitemmulti{1068}{217}{ဝေ တာ တ ယု ပနာ လ ကလ စမာ အသော။}
\suttaitemmulti{1069}{218}{ဝယ ဒိဝ ကရ ကရေ ဟျ သဏသကပါသ ကသာ။}
\suttaitem{1070}{219}{သသ မသ ဒံသာ သာ သု။}
\suttaitem{1071}{220}{ဝိဒါ ဒသုက။}
\suttaitem{1072}{221}{သသာ ရီဟော။}
\suttaitem{1073}{222}{ဇီဝါမာ ဟော ဝမာ စ။}
\suttaitem{1074}{223}{တဏှာဒယော။}
\suttaitem{1075}{224}{ပဏုဿဟာ ဟီဟီ ဏောလင စ။}
\suttaitemmulti{1076}{225}{ခီ မိ ပီ စု မာ ဝါ ကာဟိ ဠော ဥဿ ဝါ ဒီဃော စ။}
\suttaitem{1077}{226}{ဂုတော ဠက စ။}
\suttaitem{1078}{227}{ပင်္ဂုဠာဒယော။}
\suttaitem{1079}{228}{ပါတော ဠိ။}
\suttaitem{1080}{229}{ဝီတော ဠု။}
\end{suttalist}

\begin{jieshu}
ဣတိ မောဂ္ဂလ္လာနေ ဗျာကရဏေ ဏွာဒိကဏ္ဍော သတ္တမော။
\end{jieshu}

\begin{jieshu}
မောဂ္ဂလ္လာနသုတ္တပါ​ဌော နိဋ္ဌိ​တော။
\end{jieshu}



% 正文
\chapter{သညာဒိကဏ္ဍော ပဌမော}
\markboth{မောဂ္ဂလ္လာနဗျာကရဏေ}{သညာဒိကဏ္ဍော ပဌမော}

\begin{song}
    သိဒ္ဓမိဒ္ဓဂုဏံ သာဓု၊ နမဿိတွာ တထာဂတံ။\\
    သဓမ္မသင်္ဃံ ဘာသိဿံ၊ မာဂဓံ သဒ္ဒလက္ခဏံ။
\end{song}


\sutta{1}{1}{အအာဒယော တိတာလီသ ဝဏ္ဏာ။}
\vutti{အကာရာဒယော နိဂ္ဂဟီတန္တာ တေစတ္တာလီသ-က္ခရာ ဝဏ္ဏာ နာမ ဟောန္တိ။ အ အာ ဣ ဤ ဥ ဦ ဧ ဧ ဩ ဩ၊ က ခ ဂ ဃ င စ ဆ ဇ ဈ ည၊ ဋ ဌ ဍ ဎ ဏ၊ တ ထ ဒ ဓ န၊ ပ ဖ ဗ ဘ မ၊ ယ ရ လ ဝ သ ဟ ဠ အံ။ တေန ကွတ္ထော? \suttalink{37}{ဧဩနမ ဝဏ္ဏေ။} တိတာလီသာဘိ ဝစနံ ကတ္ထစိ ဝဏ္ဏလောပံ ဉာပေတိ။ တေန “ပဋိသင်္ခါ ယောနိသော ”တိအာဒိ သိဒ္ဓံ။}

\sutta{2}{2}{ဒသာဒေါ သရာ။}
\vutti{တတ္ထာဒိမှိ ဒသ ဝဏ္ဏာ သရာ နာမ ဟောန္တိ။ တေန ကွတ္ထော? \suttalink{26}{သရော လောပေါ သရေ။} ဣစ္စာဒိ။}

\sutta{3}{3}{ဒွေဒွေ သဝဏ္ဏာ။}
\vutti{တေသု ဒွေဒွေ သရာ သဝဏ္ဏာ နာမ ဟောန္တိ။ တေန ကွတ္ထော? “ဝဏ္ဏပရေန သဝဏ္ဏောပိ ” ၁၊ ၂၄။}

\sutta{4}{4}{ပုဗ္ဗော ရဿော။}
\vutti{တေသု ဒွီသု ယော ယော ပုဗ္ဗော၊ သော သော ရဿသညော ဟောတိ။ တေသု ဧ.ဩ.သံယောဂတော ပုဗ္ဗာဝ ဒိဿန္တိ။ တေန ကွတ္ထော? \suttalink{122}{ရဿော ဝါ။} ဣစ္စာဒိ။}

\sutta{5}{5}{ပရော ဒီဃော။}
\vutti{တေ သွေဝ ဒွီသုယော ယော ပရော၊ သော သော ဒီဃသညော ဟောတိ။ တေန ကွတ္ထော? \suttalink{149}{ယောလောပနိသု ဒီဃော။} ဣစ္စာဒိ။}

\sutta{6}{6}{ကာဒယော ဗျဉ္ဇနာ။}
\vutti{ကကာရာဒယော ဝဏ္ဏာ နိဂ္ဂဟီတပရိယန္တာ ဗျဉ္ဇနသညာဟောန္တိ။ တေန ကွတ္ထော? “ဗျဉ္ဇနေ ဒီဃရဿာ ” ၁၊၃၃ ဣစ္စာဒိ။}

\sutta{7}{7}{ပဉ္စ ပဉ္စကာ ဝဂ္ဂါ။}
\vutti{ကာဒယော ပဉ္စ ပဉ္စကာ ဝဂ္ဂ၊ နာမ ဟောန္တိ။ တေန ကွတ္ထော? \suttalink{41}{ဝဂ္ဂေ ဝဂ္ဂန္တော။} ဣစ္စာဒိ။}

\sutta{8}{8}{ဗိန္ဒု နိဂ္ဂဟီတံ။}
\vutti{ယွာယံ ဝဏ္ဏော ဗိန္ဒုမတ္တော၊ သော နိဂ္ဂဟီတသညော ဟောတိ။ တေန ကွတ္ထော? “နိဂ္ဂဟီတံ ” ၁၊၃၈ ဣစာဒိ။ ဂရုသညာကရဏံ အနွတ္ထသညတ္ထံ။}

\sutta{9}{9}{ဣယုဝဏ္ဏာ ဈလာ နာမဿန္တေ။}
\vutti{နာမံ ပါဋိပဒိကံ၊ တဿအန္တေ ဝတ္တမာနာ ဣဝဏ္ဏုဝဏ္ဏာ ဈလသညာ ဟောန္တိ ယထာက္ကမံ။ တေန ကွတ္ထော? “ဈလာ ဝါ ” ၂၊၁၁၁ ဣစ္စာဒိ။}

\sutta{10}{10}{ပိတ္ထိယံ။}
\vutti{ဣတ္ထိယံ ဝတ္တမာနဿ နာမဿ-န္တေ ဝတ္တမာနာ ဣဝဏ္ဏုဝဏ္ဏာ ပသညာ ဟောန္တိ။ တေန ကွတ္ထော? “ယေ ပဿိဝဏ္ဏဿ ” ၂၊၁၁၆ ဣစ္စာဒိ။}

\sutta{11}{11}{ဃာ။}
\vutti{ဣတ္ထိယံ ဝတ္တမာနဿ နာမဿ-န္တေ ဝတ္ထမာနော အာကာရော ဃသညော ဟောတိ။ တေန ကွတ္ထော? “ဃဗြဟ္မာဒိတေ ” ၂၊၆၀ ဣစ္စာဒိ။}

\sutta{12}{12}{ဂေါ သျာလပနေ။}
\vutti{အာလပနေ သိ ဂသညော ဟောတိ။ တေန ကွတ္ထော? “ဂေဝါ ” ၂၊၆၅ ဣစ္စာဒိ။ (သညာ)}

\sutta{13}{13}{ဝိဓိဗ္ဗိသေသနန္တဿ။}
\vutti{ယံ ဝိသေသနံ၊ တဒန္တဿ ဝိဓိ ဉာတဗ္ဗော “အတော ယောနံ ဋာဋေ ” ၂၊၄၁ နရာနရေ။}

\sutta{14}{14}{သတ္တမိယံ ပုဗ္ဗဿ။}
\vutti{သတ္တမီနိဒ္ဒေသေ ပုဗ္ဗဿေဝ ကာရိယံ ဉာတဗ္ဗံ “သရော လောပေါ သရေ ” ၁၊၂၆ ဝေဠဂ္ဂံ။ ‘တမဟ ’န္တီဓ ကသ္မာ န ဟောတိ?၊ သရေတောပသိလေသိကာဓာရော တတ္ထေတာဝ ဝုစ္စတေ ‘ပုဗ္ဗဿေဝ ဟောတိ န ပရဿာ ’တိ။}

\sutta{15}{15}{ပဉ္စမိယံ ပရဿ။}
\vutti{ပဉ္စမီနိဒ္ဒေသေ ပရဿ ကာရိယံ ဉာတဗ္ဗံ “အတောယောနံ ဋာဋေ ” ၂၊၄၁ နရာနရေ။ ဣဓ န ဟောတံ ‘ဇန္တုဟော အနတ္တာ ’။ ဣဓ ကသ္မာ န ဟောတိ? ဩသချော၊ အနန္တရေ ကတတ္ထတာယ န ဗျဝဟိတဿ ကာရိယံ။}

\sutta{16}{16}{အာဒိဿ။}
\vutti{ပရဿ ဿိဿမာနံ ကာရိယမာဒိဝဏ္ဏဿ ဉာတဗ္ဗံ “ရ သင်္ချာတော ဝါ ” ၃၊၁၀၃ တေရသ။}

\sutta{17}{17}{ဆဋ္ဌိယန္တဿ။}
\vutti{ဆဋ္ဌီနိဒ္ဒိဋ္ဌဿ ယံ ကာရိယံ၊ တဒန္တဿ ဝဏ္ဏဿ ဝိညေယျံ “ရာဇဿိ နာမှိ ” ၂၊၂၃ ရာဇိနာ။}

\sutta{18}{18}{ငါနုဗန္ဓော။}
\vutti{ငကာရော အနုဗန္ဓော ယဿ၊ သော အန္တဿ ဟောတိ “ဂေါဿာ ဝင ” ၁၊၃၂ ဂဝါဿံ။}

\sutta{19}{19}{ဋာနုဗန္ဓာနေကဝဏ္ဏာ သဗ္ဗဿ။}
\vutti{ဋကာရော-နုဗန္ဓော ယဿ၊ သော-နေကက္ခရော စာဒေသော သဗ္ဗဿ ဟောတိ “ဣမဿာနိတ္ထိယံ ဋေ ” ၂၊၁၂၈ ဧသု၊ “နာမှ-နိမိ ” ၂၊၁၂၆ အနေန။}

\sutta{20}{20}{ဉ ကာနုဗန္ဓာဒျန္တာ။}
\vutti{ဆဋ္ဌီနိဒ္ဒိဋ္ဌဿ ဉာနုဗန္ဓကာနုဗန္ဓာ အာဒျန္တာ ဟောန္တိ “ဗြူတော တိဿီဉ ” ၆၊၃၆ ဗြဝီတိ၊ “ဘူဿ ဝုက ” ၆၊၁၇ ဗဘူဝ။}

\sutta{21}{21}{မာနုဗန္ဓော သရာနမန္တာ ပရော။}
\vutti{မကာရော-နုဗန္ဓော ယဿ၊ သော သရာ နမန္တာ သရာ ပရော ဟောတိ “မဉ္စ ရုဓာဒီနံ ” ၅-၁၉ ရုန္ဓတိ။}

\sutta{22}{22}{ဝိပ္ပဋိသေဓေ။}
\vutti{ဒွိန္နံ သာဝကာသာနမေကတ္ထပ္ပသင်္ဂေ ပရော ဟောတိ။ ယထာ ဒွိန္နံ တိဏ္ဏံ ဝါပုရိသာနံ သဟပ္ပတ္တိယံ ပရော၊ သော စ (ဂစ္ဆတိ) တွံ စ (ဂစ္ဆသိ၊ တုမှေ) ဂစ္ဆထ။ သော စ (ဂစ္ဆတိ၊) တွံ စ (ဂစ္ဆသိ၊) အဟံ စ (ဂစ္ဆာမိ၊ မယံ) ဂစ္ဆာမ။}

\sutta{23}{23}{သင်္ကေတော-နဝယဝေါ-နုဗန္ဓော။}
\vutti{ယော နဝယဝဘူတောသင်္ကေတော၊ သော-နုဗန္ဓောတိ ဉာတဗ္ဗော၊ “လုပိတာဒီနမာသိမှိ ” ၂-၅၇ ကတ္တာ။ သင်္ကေတဂ္ဂဟဏံ ကိံ? ပကတိယာဒိသမုဒါယဿာနုဗန္ဓတာ မာ ဟောတူတိ၊ အနဝယဝေါဟိ သမုဒါယော။ သမုဒါယရူပတ္တာယေဝ။ အနဝယဝဂ္ဂဟဏံ ကိံ? “အတေန ” ၂-၁၀၈ ဇနေန။ ဣမိနာဝ လောပဿာဝဂတတ္တာ နာနုဗန္ဓလောပါယ ဝစနမာရဒ္ဓံ။}

\sutta{24}{24}{ဝဏ္ဏပရေန သဝဏ္ဏောပိ။}
\vutti{ဝဏ္ဏသဒ္ဒေါ ပရော ယသ္မာ တေန သဝဏ္ဏောပိ ဂယှတိ သံစ ရူပံ “ယုဝဏ္ဏာနမေဩ လုတ္တာ ” ၁-၂၉ ဝါကေရိတံ၊ သမောနာ။}

\sutta{25}{25}{န္တု ဝန္တုမန္တွာဝန္တုတဝန္တုသမ္ဗန္ဓီ။}
\vutti{ဝန္တာဒိသမ္ဗန္ဓီယေဝ န္တု ဂယှတိ၊ \suttalink{281}{န္တန္တူနံ န္တော ယောမှိ ပဌမေ။} ဂုဏဝန္တော။ ဝန္တာဒိသမ္ဗန္ဓီတိ ကိံ? ဇန္တူ တန္တူ။}

\begin{jieshu}
     ပရိဘာသာယော။
\end{jieshu}

\sutta{26}{26}{သရော လောပေါ သရေ။}
\vutti{သရေ သရော လောပနီယော ဟောတိ။ တတြိ-မေ၊ သဒ္ဓိ-န္ဒြိယံ၊ နောဟေ-တံ၊ ဘိက္ခုနော-ဝါဒေါ၊ သမေတာ-ယသ္မာ၊ အဘိဘာ-ယတနံ၊ ပုတ္တာမ-တ္ထိ၊ အသန္တေ-တ္ထ။}

\sutta{27}{27}{ပရော ကွစိ။}
\vutti{သရမှာ ပရော သရော ကွစိ လောပနီယော ဟောတိ။ သော-ပိ၊ သာဝ၊ ယတော-ဒကံ၊ တတော-ဝ။ ကွစီတိကိံ? သဒ္ဓိ-န္ဒြိယံ၊ အယမဓိကာရော အာပရိစ္ဆေဒါဝသာနာ၊ တေန နာတိပ္ပသင်္ဂေါ။}

\sutta{28}{28}{န ဒွေဝါ။}
\vutti{ပုဗ္ဗပရသရာ ဒွေပိ ဝါ ကွစိ န လုပျန္တေ၊ လတာ ဣဝ၊ လတေ-ဝ၊ လတာ-ဝ။}

\sutta{29}{29}{ယုဝဏ္ဏာနမေဩ လုတ္တာ။}
\vutti{လုတ္တာ သရာ ပရေသံ ဣဝဏ္ဏုဝဏ္ဏာနံ ဧဩ ဟောန္တိ ဝါ ယထာက္ကမံ။ တဿေ-ဒံ၊ ဝါတေ-ရိတံ၊ နော-ပေတိ၊ ဝါမော-ရူ၊ အတေ-ဝညေဟိ၊ ဝေါ-ဒကံ။ ကထံ “ပစ္စောရသ္မိ ”န္တိ? ယောဂဝိဘာဂါ။ ဝါတွေဝ? တဿိဒံ။ လုတ္တေတိ ကိံ? လတာ ဣဝ။}

\sutta{30}{30}{ယဝါ သရေ။}
\vutti{သရေ ပရေ ဣဝဏ္ဏုဝဏ္ဏာနံ ယကာရဝကာရာ ဟောန္တိ ဝါ ယထာက္ကမံ။ ဗျာကတော၊ ဣစ္စဿ၊ အဇ္ဈိဏမုတ္တော၊ သွာဂတံ၊ သွာပနလာနိလံ၊ ဝါတွေဝ? ဣတိ-ဿ။ ကွစိတွေဝ? ယာနီ-ဓ၊ သူ-ပဋ္ဌိတံ။}

\sutta{31}{31}{ဧဩနံ။}
\vutti{ဧဩနံယဝါယောန္တိ ဝါ သရေ ယထာက္ကမံ။ တျဇ္ဇ တေ-ဇ္ဇ၊ သွာဟံ သော-ဟံ။ ကွစိတွေဝ? ပုတ္တာမ-တ္ထိ၊ အသန္တေ-တ္ထ။}

\sutta{32}{32}{ဂေါဿာ-ဝင်။}
\vutti{သရေ ဂေါဿ အဝင ဟောတိ။ ဂဝါ-ဿံ။ ‘ယထရီဝ၊ တထရိဝေ ’တိ နိပါတာဝ၊ ‘ဘုသာမိဝေ ’တိ ဣဝသဒ္ဒေါ ဧဝတ္ထော။}

\sutta{33}{33}{ဗျဉ္ဇနေ ဒီဃရဿာ။}
\vutti{ရဿဒီဃာနံ ကွစိ ဒီဃရဿာ ဟောန္တိ ဗျဉ္ဇနေ။ တတြာ-ယံ၊ မုနီစရေ၊ သမ္မဒေဝ၊ မာလဘာရီ။}

\sutta{34}{34}{သရမှာ ဒွေ။}
\vutti{သရမှာ ပရဿ ဗျဉ္ဇနဿ ကွစိ ဒွေ ရူပါနိ ဟောန္တိ။ ပဂ္ဂဟော။ သရမှာတိ ကိံ? တံ ခဏံ။}

\sutta{35}{35}{စတုတ္ထဒုတိယေသွေသံ တတိယပဌမာ။}
\vutti{စတုတ္ထဒုတိယေသု ပရေသွေသံ စတုတ္ထဒုတိယာနံ တဗ္ဗဂ္ဂေ တတိယ ပဌမာ ဟောန္တိ။ ပစ္စာသတ္ထျာ၊ နိဂ္ဃောသော၊ အက္ခန္တိ၊ ဗောဇ္ဈင်္ဂါ၊ သေတစ္ဆတ္တံ၊ ဒဍ္ဎော၊ နိဋ္ဌာနံ၊ မဟဒ္ဓနော၊ ယသတ္ထေရော၊ အပ္ဖုဋံ၊ အဗ္ဘုဂ္ဂတော။ ဧသွီတိ ကိံ? ထေရော။ ဧသန္တိ ကိံ? ပတ္ထော။}

\sutta{36}{36}{ဝီတိဿေဝေ ဝါ။}
\vutti{ဧဝသဒ္ဒေ ပရေ ဣတိဿ ဝေါ ဟောတိ ဝါ။ ဣတွေဝ၊ ဣစ္စေဝ။ ဧဝေတိ ကိံ? ဣစ္စာဟ။}

\sutta{37}{37}{ဧဩနမ ဝဏ္ဏေ။}
\vutti{ဧဩနံ ဝဏ္ဏေ ကွစိ အ ဟောတိ ဝါ။ ဒိသွာ ယာစက မာဂတေ၊ အကရမှသ တေ၊ ဧသ အတ္ထော၊ ဧသ ဓမ္မော၊ အဂ္ဂမက္ခာယတိ၊ သွာတနံ ၊ ဟိယျတ္တနံ၊ ကရဿု။ ဝါတွေဝ? ယာစကေ အာဂတေ ဧသော ဓမ္မော။ ဝဏ္ဏေတိ ကိံ? သော။}

\sutta{38}{38}{နိဂ္ဂဟီတံ။}
\vutti{နိဂ္ဂဟီတမာဂမော ဟောတိ ဝါ ကွစိ။ စက္ခုံ ဥဒပါဒိ စက္ခုဥဒပါဒိ၊ ပုရိမံ ဇာတိံ ပုရိမဇာတိံ၊ ကတ္တဗ္ဗံ ကုသလံ ဗဟုံ။ အဝံသိရောတိအာဒီသု နိစ္စံ။ ဝဝတ္ထိတဝိဘာသတ္တာ ဝါဓိကာရဿ၊ သာမတ္တိယေနာဂမောဝ၊ သ စ ရဿ သရဿေဝ ဟောတိ။ တဿ ရဿာနုဂတတ္တာ။}

\sutta{39}{39}{လောပေါ။}
\vutti{နိဂ္ဂဟီတဿ လောပေါ ဟောတိ ဝါ ကွစိ။ ကျာဟံ ကိမဟံသာရတ္တော သံရတ္တော။ သလ္လေ ခေါ-ဂန္တုကာမော ဂန္တုမနောတိ၊ အာဒီသု နိစ္စံ။}

\sutta{40}{40}{ပရသရဿ။}
\vutti{နိဂ္ဂဟီတမှာ ပရဿ သရဿ လောပေါ ဟောတိ ဝါ ကွစိ။ တွံ-သိ တွမသိ။}

\sutta{41}{41}{ဝဂ္ဂေ ဝဂ္ဂန္တော။}
\vutti{နိဂ္ဂဟီတဿ ခေါ ဝဂ္ဂေ ဝဂ္ဂန္တော ဝါ ဟောတိ ပစ္စာသတ္ထျာ။ တင်္ကရောတိ တံ ကရောတိ၊ တဉ္စရတိ တံ စရတိ၊ တဏ္ဌာနံ တံ ဌာနံ၊ တန္ဓနံ တံ ဓနံ၊ တမ္ပာတိ တံ ပါတိ။ နိစ္စံ ပဒမဇ္ဈေ ဂန္တွာ၊ ကွစညာတြပိ သန္တိဋ္ဌတိ။}

\sutta{42}{42}{ယေဝဟိသု ဉော။}
\vutti{ယ ဧဝ ဟိ သဒ္ဒေသု နိဂ္ဂဟီတဿ ဝါ ဉော ဟောတိ။ ယညဒေဝ၊ တညေဝ၊ တဉှိ၊ ဝါတွေဝ? ယံ ယဒေဝ။}

\sutta{43}{43}{ယေ သံဿ။}
\vutti{သံသဒ္ဒဿ ယံ နိဂ္ဂဟီတံ တဿ ဝါ ဉော ဟောတိ ယကာရေ။ သညမော သံယမော။}

\sutta{44}{44}{မယဒါ သရေ။}
\vutti{နိဂ္ဂဟီတဿ မယဒါ ဟောန္တိ ဝါ သရေ ကွစိ။ တမဟံ၊ တယိဒံ၊ တဒလံ။ ဝါ တွေဝ? တံ အဟံ။}

\sutta{45}{45}{ဝ-န-တ-ရ-ဂါ စာဂမာ။}
\vutti{ဧတေ မယဒါ စ အာဂမာ ဟောန္တိ သရေ ဝါ ကွစိ။ တိဝင်္ဂိကံ၊ ဣတော နာယတိ၊ စိနိတွာ၊ တသ္မာတိဟ၊ နိရောဇံ၊ ပုထဂေဝ၊ ဣဓမာဟု၊ ယထယိဒံ၊ အတ္တဒတ္ထံ။ ဝါ တွေဝ? အတ္တတ္ထံ။ ‘အတိပ္ပဂေါ ခေါ တာဝါ ’တိ-ပဌမန္တော ပဂသဒ္ဒေါဝ။}

\sutta{46}{46}{ဆာ ဠော။}
\vutti{ဆသဒ္ဒါ ပရဿ သရဿ ဠကာရော အာဂမော ဟောတိ ဝါ။ ဆဠင်္ဂံ၊ ဆဠာယတနံ။ ဝါတွေဝ? ဆအဘိညာ။}

\sutta{47}{47}{တဒမိနာဒီနိ။}
\vutti{တဒမိနာဒီနိ သာဓူနိ ဘဝန္တိ။ တံ ဣမိနာ တဒမိနာ၊ သကိံ အာဂါမီ သကဒါဂါမီ၊ ဧကံ ဣဓ အဟံ ဧကမိဒါဟံ၊ သံဝိဓာယ အဝဟာရော သံဝိဒါဝဟာရော၊ ဝါရိနော ဝါဟကော ဝလာဟကော၊ ဇီဝနဿ မူတော ဇီမူတော၊ ဆဝဿ သယနံ သုသာနံ၊ ဥဒ္ဓံ ခမဿ ဥဒုက္ခလံ၊ ပိသိတာသော ပိသာစော၊ မဟိယံ ရဝတီတိ မယူရော၊ ဧဝမညေပိ ပယောဂတော-နုဂန္တဗ္ဗာ၊ ပရေသံ ပိသောဒရာဒိမိဝေဒံ ဒဋ္ဌဗ္ဗံ။}

\sutta{48}{48}{တဝဂ္ဂ-ဝ-ရ-ဏာနံ ယေ စဝဂ္ဂ-ဗ-ယ-ဉာ။}
\vutti{တဝဂ္ဂဝရဏာနံ ကွစိ စဝဂ္ဂဗယဉာ ဟောန္တိ ယထာက္ကမံ ယကာရေ။ အပုစ္စဏ္ဍတာယ၊ တစ္ဆံ၊ ယဇ္ဇေဝံ၊ အဇ္ဈတ္တံ၊ ထညံ၊ ဒိဗ္ဗံ၊ ပယျောသနာ၊ ပေါက္ခရညော။ ကွစိတွေဝ? ရတ္တျာ}

\sutta{49}{49}{ဝဂ္ဂလသေဟိ တေ။}
\vutti{ဝဂ္ဂလသေဟိ ပရဿ ယကာရဿ ကွစိ တေ ဝဂ္ဂလသာ ဟောန္တိ။ သက္ကတေ၊ ပစ္စတေ၊ အဋ္ဋတေ၊ ကုပ္ပတေ၊ ဖလ္လတေ၊ အဿတေ။ ကွစိတွေဝ? ကျာဟံ။}

\sutta{50}{50}{ဟဿ ဝိပလ္လာသော။}
\vutti{ဟဿ ဝိပလ္လာသော ဟောတိ ယကာရေ။ ဂုယှံ။}

\sutta{51}{51}{ဝေ ဝါ။}
\vutti{ဟဿ ဝိပလ္လာသော ဟောတိ ဝါ ဝကာရေ။ ဗဝှာဗာဓော ဗဝှာဗာဓော။}

\sutta{52}{52}{တထာနရာနံ ဋဌဏလာ}
\vutti{တထနရာနံ ဋဌဏလာ ဟောန္တိ ဝါ။ ဒုက္ကဋံ၊ အဋ္ဌကထာ၊ ဂဟဏံ၊ ပလိဃော၊ ပလာယတိ။ ဝါတွေဝ? ဒုက္ကတံ။ ကွစိတွေဝ? သုဂတော။}

\sutta{53}{53}{သံယောဂါဒိ လောပေါ။}
\vutti{သံယောဂဿ ယော အာဒီဘူတော-ဝယဝေါ တဿ ဝါ ကွစိ လောပေါ ဟောတိ။ ပုပ္ဖံ-သာ ဇာယတေ-ဂိနိ။}

\sutta{54}{54}{ဝိစ္ဆာဘိက္ခညေသု ဒွေ။}
\vutti{ဝိစ္ဆာယမာဘိက္ခညေ စ ယံ ဝတ္တတေ၊ တဿ ဒွေ ရူပါနိ ဟောန္တိ။ ကြိယာယ ဂုဏေန ဒဗ္ဗေန ဝါ ဘိန္နေ အတ္ထေ ဗျာပိတုမိစ္ဆာ ဝိစ္ဆာ။ ရုက္ခံ ရုက္ခံ သိဉ္စတိ၊ ဂါမော ဂါမော ရမဏီယော၊ ဂါမေ ဂါမေ ပါနီယံ၊ ဂေဟေ ဂေဟေ ဣဿရော၊ ရသံ ရသံ ဘက္ခယတိ၊ ကိရိယံ ကိရိယမာရဘတေ။ အတ္ထိယေဝါ-နုပုဗ္ဗိယေပိ ဝိစ္ဆာ မူလေ မူလေ ထူလာ၊ အဂ္ဂေ အဂ္ဂေ သုခုမာ၊ ယဒိ ဟိ ဧတ္ထ မူလဂ္ဂဘေဒေါ န သိယာ၊ အာနုပုဗ္ဗိယမ္ပိ န ဘဝေယျ။ မာသကံ မာသကံ ဣမမှာ ကဟာပဏာ ဘဝန္ထာနံ ဒွိန္နံ ဒေဟီတိ မာသကံ မာသကမိစ္စေတသ္မာ ဝိစ္ဆာဂမျတေ၊ သဒ္ဒန္တရတော ပန ဣမမှာ ကဟာပဏာတိ အဝဓာရဏံ။ ပုဗ္ဗံ ပုဗ္ဗံ ပုပ္ဖန္တိ၊ ပဌမံ ပဌမံ ပစ္စန္တီတျတြာပိ ဝိစ္ဆာဝ။ ဣမေ ဥဘော အဍ္ဎာ ကတရာ ကတရာ ဧသံ ဒွိန္နမဍ္ဎတာ၊ သဗ္ဗေ ဣမေ အဍ္ဎာ ကတမာ ကတမာ ဣမေသံ အဍ္ဎုတာ ဣဟာပိ ဝိစ္ဆာဝ။ အာဘိက္ခညံ ပေါနောပုညံ ပစတိ ပစတိ၊ ပပစတိ ပပစတိ၊ လုနာဟိ လုနာဟိတွေဝါယံ လုနာတိ၊ ဘုတွာ ဘုတွာ ဂစ္ဆတိ၊ ပဋပဋာ ကရောတိ၊ ပဋပဋာယတိ။}

\sutta{55}{55}{သျာဒိလောပေါ ပုဗ္ဗဿေကဿ။}
\vutti{ဝိစ္ဆာယမေကဿ ဒွိတ္တေ ပုဗ္ဗဿ သျာဒိလောပေါ ဟောတိ။ ဧကေကဿ။ ကထံ မတ္ထကမတ္ထကေနာတိ? ‘သျာဒိလောပေါ ပုဗ္ဗဿာ ’တိ ယောဂဝိဘာဂါ၊ နာစာတိပ္ပသင်္ဂေါ ယောဂဝိဘာဂါ ဣဋ္ဌပ္ပသိဒ္ဓီတိ။}

\sutta{56}{56}{သဗ္ဗာဒီနံ ဝီတိဟာရေ။}
\vutti{သဗ္ဗာဒီနံ ဝီတိဟာရေ ဒွေ ဘဝန္တိ ပုဗ္ဗဿ သျာဒိလောပေါ စ။ အညမညဿ ဘောဇကာ၊ ဣတရီတရဿ ဘောဇကာ။}

\sutta{57}{57}{ယာဝဗောဓံ သမ္ဘမေ။}
\vutti{တုရိတေနာပါယဟေတုပဒဿနံ သမ္ဘမော၊ တသ္မိံသတိ ဝတ္ထု ယာဝန္တေဟိ သဒ္ဒေဟိ သော-တ္ထော ဝိညာယတေ၊ တာဝန္တော သဒ္ဒါ ပယုဇ္ဇန္တေ။ သပ္ပော သပ္ပော သပ္ပော၊ ဗုဇ္ဈဿု ဗုဇ္ဈဿု ဗုဇ္ဈဿု၊ ဘိန္နော ဘိက္ခုသင်္ဃော ဘိန္နော ဘိက္ခုသင်္ဃော။}

\sutta{58}{58}{ဗဟုလံ။}
\vutti{အယမဓိကာရော အာသတ္ထပရိသမတ္တိယာ။ တေန နာတိပ္ပသင်္ဂေါ ဣဋ္ဌပ္ပသိဒ္ဓိ စ။}

\begin{jieshu}
ဣတိ မောဂ္ဂလ္လာနေ ဗျာကရဏေ ဝုတ္တိယံ 

ပဌမော ကဏ္ဍော။
\end{jieshu}
\chapter{သျာဒိကဏ္ဍော ဒုတိယော}
\markboth{မောဂ္ဂလ္လာနဗျာကရဏေ}{ သျာဒိကဏ္ဍော ဒုတိယော}

\sutta{59}{1}{ဒွေ ဒွေ-ကာနေကေသု နာမသ္မာ သိ ယော၊ အံ ယော၊ နာ ဟိ၊ သ နံ၊ သ္မာ ဟိ၊ သ နံ၊ သ္မိံ သု။}
\vutti{ဧတေသံ ဒွေ ဒွေ ဟောန္တိ ဧကာနေကတ္ထေသု ဝတ္တမာနတော နာမသ္မာ။ မုနိ မုနယော၊ မုနိံ မုနယော၊ မုနိနာ မုနီဟိ၊ မုနိဿ မုနီနံ၊ မုနိသ္မာ မုနီဟိ၊ မုနိဿ မုနီနံ၊ မုနိသ္မိံ မုနီသု၊ ဧဝံ ကုမာရီ ကုမာရိယော၊ ကညာ ကညာယောတိ။}

\sutta{60}{2}{ကမ္မေ ဒုတိယာ။}
\vutti{ကရီယတိ ကတ္တု ကိရိယာယာ-ဘိသမ္ဗန္ဓီယတီတိ ကမ္မံ၊ တသ္မိံ ဒုတိယာဝိဘတ္တိ ဟောတံ။ ကဋံ ကရောတိ၊ ဩဒနံ ပစတိ၊ အာဒိစ္စံ ပဿတိ။}

\sutta{61}{3}{ကာလဒ္ဓါနမစ္စန္တသံယောဂေ။}
\vutti{ကိရိယာ၊ ဂုဏ၊ ဒဗ္ဗေဟိ သာကလ္လေန ကာလဒ္ဓါနံ သမ္ဗန္ဓော အစ္စန္တသံယောဂေါ။ တသ္မိံ ဝိညာယမာနေ ကာလသဒ္ဒေဟိ အဒ္ဓသဒ္ဒေဟိ စ ဒုတိယာ ဟောတိ။ မာသမဓီတေ၊ မာသံ ကလျာဏိ၊ မာသံ ဂုဠဓာနာ၊ ကောသမဓီတေ၊ ကောသံ ကုဋိလာ နဒီ၊ ကောသံ ပဗ္ဗတော။}

\sutta{62}{4}{ဂတိ ဗောဓာဟာရ သဒ္ဒတ္ထာကမ္မက ဘဇ္ဇာဒီနံ ပယောဇ္ဇေ။}
\vutti{ဂမနတ္ထာနံ ဗောဓတ္ထာနံ အာဟာရတ္ထာနံ သဒ္ဒတ္ထာနမကမ္မကာနံ ဘဇ္ဇာဒီနဉ္စ ပယောဇ္ဇေ ကတ္တရိ ဒုတိယာ ဟောတိ။ ဂမယတိ မာဏဝကံ ဂါမံ၊ ယာပယတိ မာဏဝကံ ဂါမံ၊ ဗောဓယတိ မာဏဝကံ ဓမ္မံ၊ ဝေဒယတိ မာဏဝကံ ဓမ္မံ၊ ဘောဇယတိ မာဏဝကံ မောဒကံ။}

\sutta{63}{5}{ဟရာဒီနံ ဝါ။}
\vutti{ဟရာဒီနံ ပယောဇ္ဇေ ကတ္တရိ ဒုတိယာ ဟောတိ ဝါ။ ဟာရေတိ ဘာရံ ဒေဝဒတ္တံ ဒေဝဒတ္တေနေတိ ဝါ၊ အဇ္ဈောဟာရေတိ သတ္တုံ ဒေဝဒတ္တံ ဒေဝဒတ္တေနေတိ ဝါ။}

\sutta{64}{6}{န ခါဒါဒီနံ။}
\vutti{ခါဒါဒီနံ ပယောဇ္ဇေ ကတ္တရိ ဒုတိယာ န ဟောတိ။ ခါဒယတိ ဒေဝဒတ္တေန၊ အာဒယတိ ဒေဝဒတ္တေန၊ အဝှ၊ ပယတိ ဒေဝဒတ္တေန။}

\suttagana{65}{1}{ဝဟိဿာနိယန္တုကေ။}
\vutti{ဝါဟယတိ ဘာရံ ဒေဝဒတ္တေန အနိယန္တုကေတိ ကိံ? ဝါဟ-ယတိ ဘာရံ ဗလီဗဒ္ဒေ။}

\suttagana{66}{2}{ဘက္ခိဿာဟိံသာယံ။}
\vutti{ဘက္ခယတိ မောဒကေ ဒေဝဒတ္တေန။ အဟိံသာယန္တိ ကိံ? ဘက္ခယတိ ဗလီဗဒ္ဒေ သဿံ။}

\sutta{67}{7}{ဓျာဒီဟိ ယုတ္တာ။}
\vutti{ဓီအာဒီဟိ ယုတ္တတော ဒုတိယာ ဟောတိ။ ဓိရတ္ထု မံ ပူတိကာယံ၊ အန္တရာ စ ရာဇဂဟံ အန္တရာ စ နာလန္ဒံ၊ သမာဓာနမန္တရေန၊ မုစလိန္ဒမဘိတော သရမိစ္စာဒိ။}

\sutta{68}{8}{လက္ခဏိတ္ထမ္ဘူတဝိစ္ဆာသွဘိနာ။}
\vutti{လက္ခဏာဒီသွတ္ထေသွဘိနာ ယုတ္တမှာ ဒုတိယာ မဟာတိ။ ရုက္ခမဘိ ဝိဇ္ဇောတတေ ဝိဇ္ဇု၊ သာဓု ဒေဝဒတ္တော မာတရမဘိ၊ ရုက္ခံ ရုက္ခမဘိတိဋ္ဌတိ။}

\sutta{69}{9}{ပတိပရီဟိ ဘာဂေ စ။}
\vutti{ပတိပရီဟိ ယုတ္တမှာ လက္ခဏာဒီသု ဘာဂေ စတ္ထေ ဒုတိယာ ဟောတိ။ ရုက္ခံ ပတိ ဝိဇ္ဇောတတေ ဝိဇ္ဇု၊ သာဓု ဒေဝဒတ္တော မာတရံ ပတိ၊ ရုက္ခံ ရုက္ခံ ပတိ တိဋ္ဌတိ။}

\sutta{70}{10}{အနုနာ။}
\vutti{လက္ခဏာဒီသွတ္ထေသွနုနာ ယုတ္တမှာ ဒုတိယာ ဟောတိ။ ရုက္ခမနု ဝိဇ္ဇောတတေ ဝိဇ္ဇု၊ သစ္စကိရိယမနု ဝုဋ္ဌိ ပါဝဿိ၊ ဟေတု စ လက္ခဏံ ဘဝတိ။}

\sutta{71}{11}{သဟတ္ထေ။}
\vutti{သဟတ္ထေ-နုနာ ယုတ္တမှာ ဒုတိယာ ဟောတိ။ ပဗ္ဗတမနု သေနာ တိဋ္ဌတိ။}

\sutta{72}{12}{ဟီနေ။}
\vutti{ဟီနတ္ထေ-နုနာ ယုတ္တမှာ ဒုတိယာ ဟောတိ။ အနု သာရိပုတ္တံ ပညဝန္တော။}

\sutta{73}{13}{ဥပေန။}
\vutti{ဟီနတ္ထေ ဥပေန ယုတ္တမှာ ဒုတိယာ ဟောတိ။ ဥပ သာရိပုတ္တံ ပညဝန္ထော။}

\sutta{74}{14}{သတ္တမျာဓိကျေ။}
\vutti{အာဓိကျတ္ထေ ဥပေန ယုတ္တမှာ သတ္တမီ ဟောတိ။ ဥပ ခါရိယံ ဒေါဏော။}

\sutta{75}{15}{သာမိတ္တေ-ဓိနာ။}
\vutti{သာမိဘာဝတ္ထေ-ဓိနာ ယုတ္တမှာ သတ္တမီ ဟောတိ။ အဓိ ဗြဟ္မဒတ္တေ ပဉ္စာလာ၊ အဓိ ပဉ္စာလေသု ဗြဟ္မဒတ္တော။}

\sutta{76}{16}{ကတ္တုကရေဏေသု တတိယာ။}
\vutti{ကတ္တရိ ကရဏေ စ ကာရကေ တတိယာ ဟောတိ။ ပုရိသေန ကတံ၊ အသိနာ ဆိန္ဒတိ။ ပကတိယာ-ဘိရူပေါ၊ ဂေါတ္ထေန ဂေါတမော၊ သုမေဓော နာမ နာမေန၊ ဇာတိယာ သတ္တဝဿိကောတိ။}

\sutta{77}{17}{သဟတ္ထေန။}
\vutti{သဟတ္ထေန ယောဂေ တတိယာ သိယာ။ ပုတ္တေန သဟ ဂတော၊ ပုတ္တေန သဒ္ဓိံ အာဂတော။}

\sutta{78}{18}{လက္ခဏေ။}
\vutti{လက္ခဏေ ဝတ္တမာနတော တတိယာ သိယာ။ တိဒဏ္ဍကေန ပရိဗ္ဗာဇကမဒ္ဒက္ခီ၊ အက္ခိနာ ကာဏော၊ တေန ဟိ အင်္ဂေန အင်္ဂိနော ဝိကာရော လက္ခီယတေ။}

\sutta{79}{19}{ဟေတုမှိ။}
\vutti{တက္ကိရိယာ ယောဂ္ဂေ တတိယာ သိယာ။ အန္တေန ဝသတိ၊ ဝိဇ္ဇာယ ယသော။}

\sutta{80}{20}{ပဉ္စမီဏေ ဝါ။}
\vutti{ဣဏေ ဟေတုမှိ ပဉ္စမီ ဟောတိ ဝါ။ သတသ္မာ ဗဒ္ဓေါ၊ သတေန ဝါ။}

\sutta{81}{21}{ဂုဏေ။}
\vutti{ပရင်္ဂဘူတေ ဟေတုမှိ ပဉ္စမီ ဟောတိ ဝါ။ ဇဠတ္တာ ဗဒ္ဓေါ ဇဠတ္တေန ဝါ၊ ပညာယ မုတ္တော၊ ဟုတွာ အဘာဝတောအနိစ္စာ။}

\sutta{82}{22}{ဆဋ္ဌီ ဟေတွတ္ထေဟိ။}
\vutti{ဟေတွတ္ထဝါစီဟိ ယောဂေ ဟေတုမှိ ဆဋ္ဌီ သိယာ။ ဥဒရဿ ဟေတု၊ ဥဒရဿ ကာရဏာ။}

\sutta{83}{23}{သဗ္ဗာဒိတော သဗ္ဗာ။}
\vutti{ဟေတွတ္ထေဟိ ယောဂေ သဗ္ဗာဒီဟိ သဗ္ဗာ ဝိဘတ္တိယော ဟောန္တိ။ ကော ဟေတု၊ ကံ ဟေတုံ၊ ကေန ဟေတုနာ၊ ကဿ ဟေတုဿ၊ ကသ္မာ ဟေတုသ္မာ။}

\sutta{84}{24}{စတုတ္ထီ သမ္ပဒါနေ။}
\vutti{ယဿ သမ္မာ ပဒီယတေ တသ္မိံ စတုတ္ထီ သိယာ။ သင်္ဃဿ ဒဒါတိ။ အာဓာရဝိဝက္ခာယံ သတ္တမီပိ သိယာ သင်္ဃေ ဒေဟိ။}

\sutta{85}{25}{တာဒတ္ထျေ။}
\vutti{တဿေ-ဒံ တဒတ္ထံ၊ တဒတ္ထဘာဝေ ဇောတနီယေ နာမသ္မာ စတုတ္ထီ သိယာ။ သီတဿ ပဋိဃာတာယ၊ အတ္ထာယ ဟိတာယ (သုခါယ) ဒေဝမနုဿာနံ၊ နာလံ ဒါရဘရဏာယ။}

\sutta{86}{26}{ပဉ္စမျဝဓိသ္မာ။}
\vutti{ပဒတ္ထာဝဓိသ္မာ ပဉ္စမီဝိဘတ္တိ ဟောတိ။ ဂါမသ္မာ အာဂစ္ဆတု၊ ဧဝံ စောရသ္မာ ဘာယတိ၊ စောရသ္မာ ဥတ္တသတိ၊ ဩရသ္မာ တာယတိ။}

\sutta{87}{27}{အပပရီဟိ ဝဇ္ဇနေ။}
\vutti{ဝဇ္ဇနေ ဝတ္တမာနေဟိ အပပရီဟိ ယောဂေ ပဉ္စမီ ဟောတိ။ အပ သာလာယ အာယန္တိ ဝါဏိဇာ၊ ပရိ သာလာယ အာယန္တိ ဝါဏိဇာ၊ သာလံ ဝဇ္ဇေတွာတိ အတ္ထော။}

\sutta{88}{28}{ပဋိနိဓိပဋိဒါနေသု ပတိနာ။}
\vutti{ပဋိနိဓိမှိ ပဋိဒါနေ စ ဝတ္တမာနေန ပတိနာ ယောဂေ နာမသ္မာ ပဉ္စမီ ဝိဘတ္တိ ဟောတိ။ ဗုဒ္ဓသ္မာ ပတိ သာရိပုတ္တော၊ ဃတမဿ တေသသ္မာ ပတိ ဒဒါတိ။}

\sutta{89}{29}{ရိတေ ဒုတိယာ စ။}
\vutti{ရိတေသဒ္ဒေန ယောဂေ နာမသ္မာ ဒုတိယာ ဟောတိ ပဉ္စမီ စ။ ရိတေ သဒ္ဓမ္မံ၊ ရိတေ သဒ္ဓမ္မာ။}

\sutta{90}{30}{ဝိနာ-ညတြ တတိယာ စ။}
\vutti{ဝိနာ-ညတြသဒ္ဒေဟိ ယောဂေ နာမသ္မာ တတိယာ စ ဟောတိ ဒုတိယာ ပဉ္စမိယော စ။ ဝိနာ ဝါတေန၊ ဝိနာ ဝါတံ၊ ဝိနာ ဝါတသ္မာ။}

\sutta{91}{31}{ပုထနာနာဟိ။}
\vutti{ဧတေဟိ ယောဂေ တတိယာ ဟောတိ ပဉ္စမီ စ။ ပုထဂေဝ ဇနေန၊ ပုထဂေဝ ဇနသ္မာ၊ ဇနေန နာနာ၊ ဇနသ္မာ နာနာ။}

\sutta{92}{32}{သတ္တမျာဓာရေ။}
\vutti{ကိရိယာဓာရ ဘူတ ကတ္တု ကမ္မာနံ ဓာရဏေန ယော ကိရိယာယာဓာရော တသ္မိံ ကာရကေ နာမသ္မာ သတ္တမီ ဟောတိ။ ကဋေ နိသီဒတိ (ဒေဝဒတ္တော)၊ ထာလိယံ ဩဒနံ ပစတိ၊ အာကာသေ သကုနာ။}

\sutta{93}{33}{နိမိတ္တေ။}
\vutti{နိမိတ္တတ္ထေ သတ္တမီ ဟောတိ။ အဇိနမှိ ဟညတေ ဒီပိ၊ မုသာဝါဒေ ပါစိတ္ထိယံ။}

\sutta{94}{34}{ယမ္ဘာဝေါ ဘာဝလက္ခဏံ။}
\vutti{ယဿ ဘာဝေါ ဘာဝန္တရဿ လက္ခဏံ ဘဝတိ၊ တတော သတ္တမီ ဟောတိ။ ဂါဝီသု ဒုယှမာနာသု ဂတော၊ ဒုဒ္ဓါသု အာဂတော။}

\sutta{95}{35}{ဆဋ္ဌီ စာနာဒရေ။}
\vutti{ယဿ ဘာဝေါ ဘာဝန္တရဿ လက္ခဏံ ဘဝတိ၊ တတော ဆဋ္ဌီ ဘဝတိ သတ္တမီ စ အနာဒရေ ဂမျမာနေ။ “အာကောဋယန္တော သော နေတိ၊ သိဝိရာဇဿ ပေက္ခတော ”။}

\sutta{96}{36}{ယတော နိဒ္ဓါရဏံ။}
\vutti{ဇာတိဂုဏကိရိယာဟိ သမုဒါယတေကဒေသဿ ပုထက္ကရဏံ နိဒ္ဓါရဏံ။ ယတော တံ ကရီယတိ၊ တတော ဆဋ္ဌီသတ္တမိယော ဟောန္တိ။ သာလယော သူကဓညာနံ ပထျတမာ၊ သာလယော သူကဓညေသု ပထျတမာ။}

\sutta{97}{37}{ပဌမာတ္ထမတ္တေ။}
\vutti{နာမဿာဘိဓေယျမတ္တေ ပဌမာဝိဘတ္တိ ဟောတိ။ ရုက္ခော။ ဣတ္ထိ ပုမာ နပုံသကန္တိ လိင်္ဂမ္ပိ သဒ္ဒတ္ထောဝ၊ တထာ ဒေါဏော ခါရီ အာဠှကန္တိ ပရိမာဏမ္ပိ သဒ္ဒတ္ထောဝ။}

\sutta{98}{38}{အာမန္တဏေ။}
\vutti{သတော သဒ္ဒေနာဘိမုခီကရဏမာမန္တဏံ။ တသ္မိံ ဝိသယေ ပဌမာ ဝိဘတ္တိ ဟောတိ။ ဘောပုရိသ၊ ဘောက္ကတ္ထိ၊ ဘော နပုံသက။}

\sutta{99}{39}{ဆဋ္ဌီ သမ္ဗန္ဓေ။}
\vutti{ကိရိယာကာရကသဉ္ဇာတော အဿေဒမ္ဘာဝဟေတုကော သမ္ဗန္ဓော နာမ။ တသ္မိံ ဆဋ္ဌီ ဝိဘတ္တိ ဟောတိ။ ရညော ပုရိသော၊ သရတိ ရဇ္ဇဿာတိ သမ္ဗန္ဓေ ဆဋ္ဌီ။}

\sutta{100}{40}{တုလျတ္ထေန ဝါ တတိယာ။}
\vutti{တုလျတ္ထေန ယောဂေ ဆဋ္ဌီ ဟောတိ တတိယာ ဝါ၊ တုလျော ပိတု၊ တုလျော ပိတရာ၊ သဒိသော ပိတု၊ သဒိသော ပိတရာ။}

\sutta{101}{41}{အတော ယောနံ ဋာဋေ။}
\vutti{အကာရန္ထတော နာမသ္မာ ယောနံ ဋာဋေ ဟောန္တိ ယထာက္ကမံ၊ ဋကာရာ သဗ္ဗဒေသတ္ထာ၊ ဗုဒ္ဓါ ဗုဒ္ဓေ၊ အတောတိ ကိံ? ကညာယော၊ ဣတ္ထိယော၊ ဝဓုယော။}

\sutta{102}{42}{နိနံ ဝါ။}
\vutti{အကာရန္တတော နာမသ္မာ နိနံ ဋာဋေ ဝါ ဟောန္တိ ယထာက္ကမံ။ ရူပါ၊ ရူပေ၊ ရူပါနိ၊ အတောတွေဝ အဋ္ဌီနိ။}

\sutta{103}{43}{သ္မာသ္မိံနံ။}
\vutti{အကာရန္တတော နာမသ္မာ သ္မာသ္မိန္နံ ဋာဋေ ဝါ ဟောန္တိ ယထာက္ကမံ။ ဗုဒ္ဓါ ဗုဒ္ဓသ္မာ၊ ဗုဒ္ဓေ ဗုဒ္ဓသ္မိံ၊ အတောတွေဝ အဂ္ဂိသ္မာ အဂ္ဂိသ္မိံ။}

\sutta{104}{44}{သဿာယ စတုတ္ထိယာ။}
\vutti{အကာရန္တတော ပရဿ သဿ စတုတ္ထိယာ အာယော ဟောတိ ဝါ။ ဗုဒ္ဓါယ ဗုဒ္ဓဿ၊ ဘိယျော တာဒတ္ထျေယေဝါယမာယော ဒိဿတေ။}

\sutta{105}{45}{ဃပတေကသ္မိံ နာဒီနံ ယ-ယာ။}
\vutti{ဃပတော နာဒီနမေကသ္မိံ ယယာ ဟောနိ ယထာက္ကမံ။ ကညာယ၊ ရတ္တိယာ၊ ဣတ္ထိယာ၊ ဓေနုယာ၊ ဝဓုယာ၊ ဧကသ္မိန္တိ ကိံ? ကညာဟိ၊ ရတ္တီဟိ။}

\sutta{106}{46}{ဿာ ဝါ တေတိမာမူဟိ။}
\vutti{ဃပသညေဟိ တေတိမာမူဟိ နာဒီနမေကသ္မိံဿာ ဝါ ဟောတိ၊ တဿာ ကတံ၊ တဿာ ဒီယတေ၊ တဿာ နိဿဋံ၊ တဿာ ပရိဂ္ဂဟော၊ တဿာ ပတိဋ္ဌိတံ၊ တာယ ဝါ၊ ဧဝံ ဧတိဿာ ဧတာယ၊ ဣမိဿာ ဣမာယ၊ အမုဿာ အမုယာ။}

\sutta{107}{47}{နံမှိ နုက် ဒွါဒီနံ သတ္တရသန္နံ။}
\vutti{ဒွါဒီနံ သတ္တရသန္နံ သင်္ချာနံ နုက ဟောတိ နံမှိ ဝိဘတ္တမှိ၊ ဒွိန္နံ စတုန္နံ၊ ပဉ္စန္နံ၊ ဧဝံ ယာဝ အဋ္ဌာရသန္နံ၊ ဥကာရော ဥစ္စာရဏတ္ထော၊ ကကာရော အန္တာဝယဝတ္ထော။}

\sutta{108}{48}{ဗသုကတိန္နံ။}
\vutti{နံမှိ ဗဟုနော ကတိဿ စ နုက ဟောတိ၊ ဗဟုန္နံ၊ ကတိန္နံ။}

\sutta{109}{49}{ဏ္ဏံဏ္ဏန္နံ တိတော ဈာ။}
\vutti{ဈသညာ တိတော နံဝစနဿ ဏ္ဏံဏ္ဏန္နံ ဟောတိ၊ တိဏ္ဏံ၊ တိဏ္ဏန္နံ၊ ဈာတိ ကိံ တိဿန္နံ။}

\sutta{110}{50}{ဥဘိန္နံ။}
\vutti{ဥဘာ နံဝစနဿ ဣန္နံ ဟောတိ၊ ဥဘိန္နံ။}

\sutta{111}{51}{သုဉ် သဿ။}
\vutti{နာမသ္မာ သဿ သုဉ် ဟောတိ၊ ဗုဒ္ဓဿ၊ ဒွိသကာရပါဌေန သိဒ္ဓေ လာဃဝတ္ထမိဒံ။}

\sutta{112}{52}{ဿံ-ဿာ-ဿာယေသွိ-တရေ-က-ညေ-တိမာန-မိ။}
\vutti{ဿမာဒီသွိတရာဒီနမိ ဟောတိ၊ ဣတရိဿံ၊ ဣတရိဿာ၊ ဧကိဿံ၊ ဧကိဿာ၊ အညိဿံ၊ အညိဿာ၊ ဧတိဿံ၊ ဧတိဿာ၊ ဧတိဿာယ၊ ဣမိဿံ၊ ဣမိဿာ၊ ဣမိဿာယ။}

\sutta{113}{53}{တာယ ဝါ။}
\vutti{ဿမာဒီသု တဿာ ဝါ ဣ ဟောတိ၊ တိဿံ တဿံ၊ တိဿာ တဿာ၊ တိဿာယ တဿာယ၊ ဿံဿာဿာယေသွိတွေဝ? တာယ။}

\sutta{114}{54}{တေတိမာတော သဿ ဿာယ။}
\vutti{တာဧတာဣမာတော သဿ ဿာယော ဟောတိ ဝါ။ တဿာယ တာယ၊ ဧတိဿာယ ဧတာယ၊ ဣမိဿာယ ဣမာယ။}

\sutta{115}{55}{ရတျာဒီဟိ ဋော သ္မိနော။}
\vutti{ရတျာဒီဟိ သ္မိနော ဋော ဟောတိ ဝါ၊ ရတ္တော ရတ္တိယံ၊ အာဒေါ အာဒိသ္မိံ။}

\sutta{116}{56}{သုဟိသုဘဿော။}
\vutti{ဥဘဿ သုဟိသွော ဟောတိ။ ဥဘောသု၊ ဥဘောဟိ။}

\sutta{117}{57}{လ္တုပိတာဒိနမာ သိမှိ။}
\vutti{လ္တုပ္ပစ္စယန္တာနံ ပိတာဒီနံ စ အာ ဟောတိ သိမှိ။ ကတ္တာ၊ ပိတာ။ ပိတု၊ မာတု၊ ဘာဝု၊ မီတု၊ ဒုဟိတု၊ ဇာမာတု၊ နတ္တု၊ ဟောတု၊ ပေါတု။}

\sutta{118}{58}{ဂေ အ စ။}
\vutti{လ္တုပိတာဒီနံ အ ဟောတိ ဂေ အာ စ၊ ဘော ကတ္တ၊ ဘော ကတ္တာ၊ ဘော ပိတ၊ ဘော ပိတာ။}

\sutta{119}{59}{အယုနံ ဝါ ဒီဃော။}
\vutti{အ ဣ ဥ ဣစ္စေသံ ဝါ ဒီဃော ဟောတိ ဂေ ပရေ တိလိင်္ဂေ။ ဘော ပုရိသာ၊ ဘော ပုရိသ၊ ဘော အဂ္ဂီ၊ ဘော အဂ္ဂိ၊ ဘော ဘိက္ခူ၊ ဘော ဘိက္ခူ။}

\sutta{120}{60}{ဃဗြဟ္မာဒိတေ။}
\vutti{ဃတော ဗြဟ္မာဒိတော စ ဂဿေ ဝါ ဟောတိ။ ဘောတိ ကညေ၊ ဘောတိ ကညာ၊ ဘော ဗြဟ္မေ၊ ဘော ဗြဟ္မ၊ ဘော ခတ္တေ၊ ဘော ခတ္တ၊ ဘော ဣသေ၊ ဘော ဣသိ၊ ဘော သခေ၊ ဘော သခ။}

\sutta{121}{61}{နမ္မာဒီဟိ။}
\vutti{အမ္မာဒီဟိ ဂဿေ န ဟောတိ။ ဘောတိ အမ္မာ၊ ဘောတိ အန္တာ၊ ဘောတိ အမ္ဗာ။}

\sutta{122}{62}{ရဿော ဝါ။}
\vutti{အမ္မာဒီနံ ဂေ ရဿော ဟောတိ ဝါ။ ဘောတိ အမ္မ၊ ဘောတိ အမ္မာ။}

\sutta{123}{63}{ဃော ဿံ၊ ဿာ၊ ဿာယံ တိံသု။}
\vutti{ဿမာဒီသု ဃော ရဿော ဟောတိ။ တဿံ၊ တဿာ၊ တဿာယ၊ တံ၊ သဘတိံ၊ ဧသွိတိ ကိံ? တာယ၊ သဘာယ။}

\sutta{124}{64}{ဧကဝစန ယောသွ-ဃောနံ။}
\vutti{ဧကဝစနေ ယောသု စ ဃဩကာရန္တဝဇ္ဇိတာ နံ နာမာနံ ရဿော ဟောတိ တိလိင်္ဂေ။ ဣတ္ထိံ၊ ဣတ္ထိယာ၊ ဣတ္ထိယော၊ ဝဓုံ ဝဓုယာ၊ ဝဓုယော၊ ဒဏ္ဍိံ၊ ဒဏ္ဍိနာ၊ ဒဏ္ဍိနော၊ သယမ္ဘုံ၊ သယမ္ဘုနာ၊ သယမ္ဘုဝေါ။}

\sutta{125}{65}{ဂေ ဝါ။}
\vutti{အဃောနံ ဂေ ဝါ ရဿော ဟောတိ တိလိင်္ဂေ၊ ဣတ္ထိ၊ ဣတ္ထီ၊ ဝဓု၊ ဝဓူ၊ ဒဏ္ဍိ၊ ဒဏ္ဍီ၊ သယမ္ဘု၊ သယမ္ဘူ။}

\sutta{126}{66}{သိသ္မိံ နာနပုံသကဿ။}
\vutti{နပုံသကဝဇ္ဇိတဿ နာမဿ သိသ္မိံ ရဿော န ဟောတိ။ ဣတ္ထီ၊ ဒဏ္ဍီ၊ ဝဓူ၊ သယမ္ဘူ။}

\sutta{127}{67}{ဂေါဿာ-ဂ-သိ-ဟိ-နံသု ဂါဝ-ဂဝါ။}
\vutti{ဂသိဟိနံဝဇ္ဇိတာသု ဝိဘတ္တီသု ဂေါသဒ္ဒဿ ဂါဝဂဝါ ဟောန္တိ။ (ဂါဝံ၊ ဂဝံ)၊ ဂါဝေါ၊ ဂဝေါ၊ ဂါဝေန၊ ဂဝေန၊ ဂါဝဿ၊ ဂဝဿ၊ ဂါဝသ္မာ၊ ဂဝသ္မာ၊ ဂါဝေ၊ ဂဝေ။}

\sutta{128}{68}{သုမှိ ဝါ။}
\vutti{ဂေါဿ သုမှိ ဂါဝဂဝါ ဟောန္တိ ဝါ။ ဂါဝေသု၊ ဂဝေသု၊ ဂေါသု။}

\sutta{129}{69}{ဂဝံ သေန။}
\vutti{ဂေါဿ သေ ဝါ ဂဝံ ဟောတိ သဟ သေန။ ဂဝံ၊ ဂါဝဿ၊ ဂဝဿ။}

\sutta{130}{70}{ဂုန္နံ စ နံနာ။}
\vutti{နံဝစနေန သဟ ဂေါဿ ဂုန္နံ ဟောတိ ဂဝံစ ဝါ။ ဂုန္နံ၊ ဂဝံ၊ ဂေါနံ။}

\sutta{131}{71}{နာဿာ။}
\vutti{ဂေါတော နာဿ အာ ဟောတိ ဝါ။ ဂါဝါ၊ ဂဝါ၊ ဂါဝေန၊ ဂဝေန။}

\sutta{132}{72}{ဂါဝုမှိ။}
\vutti{အံဝစနေ ဂေါဿ ဂါဝှ ဝါ ဟောတိ။ ဝါဝှံ၊ ဂါဝံ၊ ဂဝံ။}

\sutta{133}{73}{ယံ ပီတော။}
\vutti{ပသညီတော အံဝစနဿ ယံ ဝါ ဟောတိ။ ဣတ္ထိယံ၊ ဣတ္ထိံ။ ပီတောတိ ကိံ? ဒဏ္ဍိံ၊ ရတ္တိံ။}

\sutta{134}{74}{နံ ဈီတော။}
\vutti{ဈသညီတော အံဝစနဿ နံ ဝါ ဟောတိ။ ဒဏ္ဍိနံ၊ ဒဏ္ဍိံ။}

\sutta{135}{75}{ယောနံ နောနေ ပုမေ။}
\vutti{ဈီတော ယောနံ နောနေ ဝါ ဟောန္တိ ယထာက္ကမံ ပုလ္လိင်္ဂေ။ ဒဏ္ဍီနော၊ ဒဏ္ဍိနေ၊ ဒဏ္ဍီ၊ ဈီတော တွေဝ? ဣတ္ထိယော၊ ပုမေတိ ကိံ? ဒဏ္ဍီနိ ကုလာနိ။}

\sutta{136}{76}{နော။}
\vutti{ဈီတော ယောနံ နော ဝါ ဟောတိ ပုလ္လိင်္ဂေ။ ဒဏ္ဍိနော တိဋ္ဌန္တိ၊ ဒဏ္ဍိနော ပဿ၊ ဒဏ္ဍီ ဝါ။}

\sutta{137}{77}{သ္မိနော နိ။}
\vutti{ဈီတော သ္မိံဝစနဿ နိ ဟောတိ ဝါ၊ ဒဏ္ဍိနိ၊ ဒဏ္ဍိသ္မိံ၊ ဈီတော တွေဝ? အဂ္ဂိသ္မိံ။}

\sutta{138}{78}{အမ္ဗာဒီဟိ။}
\vutti{အမ္ဗုအာဒီဟိ သ္မိနောနိ ဟောတိ ဝါ၊ ဖလံ ပတတိ အမ္ဗုနိ၊ ပုပ္ဖံ ယထာ ပံသုနိ အာတပေ ကတံ။}

\sutta{139}{79}{ကမ္မာဒိတော။}
\vutti{ကမ္မာဒိတော သ္မိနော နိ ဟောတိ ဝါ။ ကမ္မနိ ကမ္မေ။ ကမ္မ၊ စမ္မ၊ ဝေသ္မ၊ ဘသ္မ (အသ္မ)၊ ဗြဟ္မ၊ အတ္တ၊ အာတုမ၊ ဃမ္မ၊ မုဒ္ဓ။}

\sutta{140}{80}{နာဿေနော။}
\vutti{ကမ္မာဒိတော နာဝစနဿ ဧနော ဝါ ဟောတိ။ ကမ္မေန၊ ကမ္မနာ၊ စမ္မေန၊ စမ္မနာ။}

\sutta{141}{81}{ဈလာ သဿ နော။}
\vutti{ဈလတော သဿ နော ဝါ ဟောတိ။ အဂ္ဂိနော အဂ္ဂိဿ၊ ဒဏ္ဍိနော ဒဏ္ဍိဿ၊ ဘိက္ခုနော ဘိက္ခုဿ၊ သယမ္ဘုနော သယမ္ဘုဿ။}

\suttagana{142}{3}{ဣတော ကွစိ သဿ ဋာနုဗန္ဓော။}

\sutta{143}{82}{နာ သ္မာဿ။}
\vutti{ဈလတော သ္မာဿ နာ ဟောတိ ဝါ။ အဂ္ဂိနာ အဂ္ဂိသ္မာ၊ ဒဏ္ဍိနာ ဒဏ္ဍိသ္မာ၊ ဘိက္ခုနာ ဘိက္ခုသ္မာ၊ သယမ္ဘုနာ သယမ္ဘုသ္မာ။}

\sutta{144}{83}{လာ ယောနံ ဝေါ ပုမေ။}
\vutti{လတော ယောနံ ဝေါ ဟောတိ ဝါ ပုလ္လိင်္ဂေ။ ဘိက္ခဝေါ ဘိက္ခူ၊ သယမ္ဘုဝေါ သယမ္ဘူ။}

\sutta{145}{84}{ဇန္တာဒိတော နော စ။}
\vutti{ဇန္တာဒိတော ယောနံ နော ဟောတိ ဝေါ စ ဝါ ပုလ္လိင်္ဂေ။ ဇန္တုနော၊ ဇန္တဝေါ ဇန္တုယော၊ ဂေါတြဘုနော၊ ဂေါတြဘုဝေါ ဂေါတြဘူ။}

\sutta{146}{85}{ကူတော။}
\vutti{ကူပစ္စယန္တတော ယောနံ နော ဝါ ဟောတိ ပုလ္လိင်္ဂေ၊ ဝိဒုနော ဝိဒူ၊ ဝိညုနော ဝိညူ၊ သဗ္ဗညုနော သဗ္ဗညူ။}

\sutta{147}{86}{လောပေါ-မုသ္မာ။}
\vutti{အမုသဒ္ဒတော ယောနံ လောပေါဝ ဟောတိ ပုလ္လိင်္ဂေ၊ အမူ၊ ပုမေတွေဝ? အမုယော အမူနိ။}

\sutta{148}{87}{န နော သဿ။}
\vutti{အမုသ္မာ သဿ နော န ဟောတိ၊ အမုဿ၊ နောတိ ကိံ? အမုယာ။}

\sutta{149}{88}{ယောလောပ-နိသု ဒီဃော။}
\vutti{ယောနံ လောပေ နိသု စ ဒီဃော ဟောတိ၊ အဋ္ဌီ အဋ္ဌီနိ၊ ယောလောပနိသူတိ ကိံ? ရတ္တိယော။}

\sutta{150}{89}{သုနံဟိသု။}
\vutti{ဧသု နာမဿ ဒီဃော ဟောတိ။ အဂ္ဂီသု၊ အဂ္ဂီနံ၊ အဂ္ဂီဟိ။}

\sutta{151}{90}{ပဉ္စာဒီနံ စုဒ္ဒသန္နမ။}
\vutti{ပဉ္စာဒီနံ စုဒ္ဒသန္နံ သုနံဟိသွ ဟောတိ။ ပဉ္စသု၊ ပဉ္စန္နံ၊ ပဉ္စဟိ၊ ဆသု၊ ဆန္နံ၊ ဆဟိ၊ ဧဝံ ယာဝ အဋ္ဌရသာ။}

\sutta{152}{91}{ယွာဒေါ န္တုဿ။}
\vutti{ယွာဒီသု န္တုဿ အ ဟောတိ။ ဂုဏဝန္တာ၊ ဂုဏဝန္တံ၊ ဂုဏဝန္တေ၊ ဂုဏဝန္တေန ဣစ္စာဒိ။}

\sutta{153}{92}{န္တဿ စ ဋ ဝံသေ။}
\vutti{အံသေသု န္တပ္ပစ္စယဿ ဋ ဟောတိ ဝါ န္တုဿ စ။ ယံ ယံ ဟိ ရာဇ ဘဇတိ သန္တံဝါယဒိ ဝါ အသံ၊ ကိစ္စာ-နကုဗ္ဗဿ ကရေယျ ကိစ္စံ။}

\sutta{154}{93}{ယောသု ဈိဿ ပုမေ။}
\vutti{ဈသညဿ ဣဿ ယောသု ဝါ ဋ ဟောတိ ပုလ္လိင်္ဂေ။ အဂ္ဂယော အဂ္ဂီ၊ ဈဂ္ဂဟဏံ ကိံ? ဣကာရန္တသမုဒါယဿ ဋ မာ သိယာဘိ။}

\sutta{155}{94}{ဝေဝေါသု လုဿ။}
\vutti{လသညဿ ဥဿ ဝေဝေါသု ဋ ဟောတိ။ ဘိက္ခဝေ၊ ဘိက္ခဝေါ၊ ဝေဝေါသူတိ ကိံ? ဇန္တုယော။}

\sutta{156}{95}{ယောမှိ ဝါ ကွစိ။}
\vutti{ယောမှိ ကွစိ လသညဿ ဥဿ ဝါ ဋ ဟောတိ။ ဟေတယော၊ နန္ဒန္တိ တံ ကုရယော ဒဿနေန။}

\sutta{157}{96}{ပုမာ-လပနေ ဝေဝေါ။}
\vutti{လသညတော ဥတော ယောဿာလပနေ ဝေဝေါ ဟောန္တိ ဝါ ပုလ္လိင်္ဂေ။ ဘိက္ခဝေ၊ ဘိက္ခဝေါ ဘိက္ခူ။}

\sutta{158}{97}{သ္မာဟိသ္မိန္နံ မှာဘိမှိ။}
\vutti{နာမသ္မာ ပရေသံ သ္မာဟိသ္မိန္နံ မှာ ဘိမှိ ဝါ ဟောန္တိ ယထာက္ကမံ။ ဗုဒ္ဓမှာ ဗုဒ္ဓသ္မာ၊ ဗုဒ္ဓေဘိ ဗုဒ္ဓေဟိ၊ ဗုဒ္ဓမှိ ဗုဒ္ဓသ္မိံ။}

\sutta{159}{98}{သုဟိသွဿေ။}
\vutti{အကာရန္တဿ သုဟိသွေ ဟောတိ။ ဗုဒ္ဓေသု ဗုဒ္ဓေဟိ။}

\sutta{160}{99}{သဗ္ဗာဒီနံ နံမှိ စ။}
\vutti{အကာရန္တာနံ သဗ္ဗာဒီနံ ဧ ဟောတိ နံမှိ သုဟိသု စ။ သဗ္ဗေသံ၊ သဗ္ဗေသု၊ သဗ္ဗေဟိ။}

\suttagana{161}{4}{ပုဗ္ဗပရာဝရဒက္ခိဏုတ္တရာဓရာနိ ဝဝတ္ထာယမသညာယံ။}

\sutta{162}{100}{သံသာနံ။}
\vutti{သဗ္ဗာဒိတော နံဝစနဿ သံသာနံ ဟောန္တိ။ သဗ္ဗေသံ၊ သဗ္ဗေသာနံ။}

\sutta{163}{101}{ဃပါ သဿ ဿာ ဝါ။}
\vutti{သဗ္ဗာဒီနံ ဃပတော သဿ ဿာ ဝါ ဟောတိ၊ သဗ္ဗဿာ သဗ္ဗာယ။}

\sutta{164}{102}{သ္မိနော ဿံ။}
\vutti{သဗ္ဗာဒီနံ ဃပတော သ္မိနော ဿံ ဝါ ဟောတိ၊ သဗ္ဗဿံ သဗ္ဗာယ။}

\sutta{165}{103}{ယံ။}
\vutti{ဃပတော သ္မိနော ယံ ဝါ ဟောတိ၊ ကညာယံ ကညာယ၊ ရတ္တိယံ ရတ္တိယာ။}

\sutta{166}{104}{တိံ သဘာပရိသာယ။}
\vutti{သဘာပရိသာဟိ သ္မိနော တိံ ဝါ ဟောတိ၊ သဘတိံ သဘာယ၊ ပရိသတိံ ပရိသာယ။}

\sutta{167}{105}{ပဒါဒီဟိ သိ။}
\vutti{ဧဟိ သ္မိနော သိ ဟောတိ ဝါ၊ ပဒသိ ပဒသ္မိံ၊ ဗိလသိ ဗိလသ္မိံ။}

\sutta{168}{106}{နာဿ သာ။}
\vutti{ပဒါဒီဟိ နာဿ သာ ဟောတိ ဝါ၊ ပဒသာ ပဒေန၊ ဗိလသာ ဗိလေန။}

\sutta{169}{107}{ကောဓာဒီဟိ။}
\vutti{ဧဟိ နာဿ သာ ဟောတိ ဝါ၊ ကောဓသာ ကောဓေန၊ အတ္ထသာ အတ္ထေန။}

\sutta{170}{108}{အတေန။}
\vutti{အကာရန္တတော ပရဿ နာဝစနဿ ဧနာဒေသော ဟောတိ၊ ဗုဒ္ဓေန၊ အတောတိ ကိံ? အဂ္ဂိနာ။}

\sutta{171}{109}{သိဿော။}
\vutti{အကာရန္တတော နာမသ္မာ သိဿ ဩ ဟောတိ၊ ဗုဒ္ဓေါ၊ အတောတွေဝ? အဂ္ဂိ။}

\sutta{172}{110}{ကွစေ ဝါ။}
\vutti{အကာရန္တတော နာမသ္မာ သိဿ ဧ ဟောတိ ဝါ ကွစိ၊ ဝနပ္ပဂုမ္ဗေ ယထာ ဖုဿိတဂ္ဂေ။}

\sutta{173}{111}{အံ နပုံသကေ။}
\vutti{အကာရန္တတော နာမသ္မာ သိဿ အံ ဟောတိ နပုံသကလိင်္ဂေ။ ရူပံ။}

\sutta{174}{112}{ယောနံ နိ။}
\vutti{အကာရန္တတော နာမသ္မာ ယောနံ နိ ဟောတိ နပုံသကေ။ သဗ္ဗာနိ ရူပါနိ။}

\sutta{175}{113}{ဈလာ ဝါ။}
\vutti{ဈလတော ယောနံ နိ ဟောတိ ဝါ နပုံသကေ၊ အဋ္ဌီနိ အဋ္ဌီ၊ အာယူနိ အာယူ။}

\sutta{176}{114}{လောပေါ။}
\vutti{ဈလတော ယောနံ လောပေါ ဟောတိ၊ အဋ္ဌီ၊ အာယူ၊ အဂ္ဂီ၊ ဘိက္ခူ။}

\sutta{177}{115}{ဇန္တုဟေတွီဃပေဟိ ဝါ။}
\vutti{ဇန္တုဟေတူဟိ ဤကာရန္တေဟိ ဃပ သညေဟိ စ ပရေသံ ယောနံ ဝါ လောပေါ ဟောတိ၊ ဇန္တူ ဇန္တုယော၊ ဟေတူ ဟေတုယော၊ ဒဏ္ဍီ ဒဏ္ဍိယော၊ ကညာ ကညာယော။}

\sutta{178}{116}{ယေ ပဿိဝဏ္ဏဿ။}
\vutti{ပသညဿ ဣဝဏ္ဏဿ လောပေါ ဟောတိ ဝါ ယကာရေ၊ ရတျော ရတျာ၊ ရတျံ၊ ပေါက္ခရညော၊ ပေါက္ခရညာ၊ ပေါက္ခရညံ။}

\sutta{179}{117}{ဂသီနံ။}
\vutti{နာမသ္မာ ဂသီနံ လောပေါ ဟောတိ ဝိဇ္ဈန္တရာဘာဝေ၊ ဘော ပုရိသ၊ အယံ၊ ဒဏ္ဍီ။}

\sutta{180}{118}{အသင်္ချေဟိ သဗ္ဗာသံ။}
\vutti{အဝိဇ္ဇမာနသင်္ချေဟိ ပရာသံ သဗ္ဗာသံ ဝိဘတ္တီနံ လောပေါ ဟောတိ၊ စ ဝါ ဧဝ ဧဝံ။}

\sutta{181}{119}{ဧကတ္ထတာယံ။}
\vutti{ဧကတ္ထီဘာဝေ သဗ္ဗာသံ ဝိဘတ္တီနံ လောပေါ ဟောတိ ဗဟုလံ၊ ပုတ္တီယတိ၊ ရာဇပုရိသော။}

\sutta{182}{120}{ပုဗ္ဗသ္မာ-မာဒိတော။}
\vutti{အမာဒေကတ္ထာ ပုဗ္ဗံ ယဒေကတ္ထံ တတော ပရာသံ သဗ္ဗာသံ ဝိဘတ္တီနံ လောပေါ ဟောတိ။ အဓိတ္ထိ။}

\sutta{183}{121}{နာ-တော-မပဉ္စမိယာ။}
\vutti{အမာဒေကတ္ထာ ပုဗ္ဗံ ယဒေကတ္ထမကာရန္တံ၊ တတော ပရာသံ သဗ္ဗာသံ ဝိဘတ္တီနံ လောပေါ န ဟောတိ၊ အန္တု၊ ဘဝတျပဉ္စမျာ။}

\sutta{184}{122}{ဝါ တတိယာသတ္တမီနံ။}
\vutti{အမာဒေကတ္ထာ ပုဗ္ဗံ ယဒေကတ္ထမကာရန္တံ၊ တတော ပရာသံ တတိယာသတ္ထမီနံ ဝါ အံ ဟောတိ၊ ဥပကုမ္ဘေန ကတံ၊ ဥပကုမ္ဘံ ကတံ။}

\sutta{185}{123}{ရာဇဿိ နာမှိ။}
\vutti{နာမှိ ရာဇဿိ ဝါ ဟောတိ၊ သဗ္ဗဒတ္တေန ရာဇိနာ၊ ဝါတွေဝ? ရညာ။}

\sutta{186}{124}{သုနံဟိသူ။}
\vutti{ရာဇဿ ဦ ဟောတိ ဝါ သုနံဟိသု၊ ရာဇူသု ရာဇေသု၊ ရာဇူနံ ရညံ၊ ရာဇူဟိ ရာဇေဟိ။}

\sutta{187}{125}{ဣမဿာနိတ္ထိယံ ဋေ။}
\vutti{ဣမသဒ္ဒဿာနိတ္ထိယံ ဋေ ဟောတိ ဝါ သုနံဟိသု၊ ဧသု ဣမေသု၊ ဧသံ ဣမေသံ၊ ဧသိ ဣမေဟိ။}

\sutta{188}{126}{နာမှ-နိ-မိ။}
\vutti{ဣမသဒ္ဒဿာနိတ္ထိယံ နာမှိ အနဣမိဣစ္စာဒေသာ ဟောန္တိ၊ အနေန ဣမီနာ၊ အနိတ္ထိယံတွေဝ? ဣမာယ။}

\sutta{189}{127}{သိမှ-နပုံသကဿာယံ။}
\vutti{ဣမသဒ္ဒဿာနပုံသကဿ အယံ ဟောတိ သိမှိ၊ အယံ ပုရိသော၊ အယံ ဣတ္ထီ။}

\sutta{190}{128}{တျတေတာနံ တဿ သော။}
\vutti{တျတေတာနမနပုံသကာနံ တဿ သော ဟောတိ သိမှိ၊ သျော ပုရိသော၊ သျာ ဣတ္ထီ၊ ဧဝံ သော၊ သာ၊ ဧသော၊ ဧသာ။}

\sutta{191}{129}{မဿာ-မုဿ။}
\vutti{အနပုံသကဿာမုဿ မကာရဿ သော ဟောတိ သိမှိ၊ အသု ပုရိသော၊ အသု ဣတ္ထီ။}

\sutta{192}{130}{ကေ ဝါ။}
\vutti{အမုဿ မဿ ကေ ဝါ သော ဟောတိ၊ အသုကော အမုကော၊ အသုကာ အမုကာ၊ အသုကံ အမုကံ။}

\sutta{193}{131}{တ တဿ နော သဗ္ဗာသု။}
\vutti{တသဒ္ဒဿ တဿ နော ဝါ ဟောတိ သဗ္ဗာသု ဝိဘတ္တီသု၊ နေ တေ နာယော တာယော၊ နံ တံ၊ နာနိ တာနိ ဣစ္စာဒိ။}

\sutta{194}{132}{ဋ သ-သ္မာ-သ္မိံ-ဿာယ-ဿံ-ဿာ-သံ-မှာ-မှိ-သွိ-မဿ စ။}
\vutti{သာဒီသွိမဿ တတဿ စ ဋော ဝါ ဟောတိ၊ အဿ ဣမဿ၊ အသ္မာ ဣမသ္မာ၊ အသ္မိံ ဣမသ္မိံ၊ အဿာယ ဣမိဿာယ၊ အဿံ ဣမိဿံ၊ အဿာ ဣမိဿာ၊ အာသံ ဣမာသံ၊ အမှာ ဣမမှာ၊ အမှိ ဣမမှိ၊ အဿ တဿ၊ အသ္မာ တသ္မာ၊ အသ္မိံ တသ္မိံ၊ အဿာယ တဿာယ၊ အဿံ တဿံ၊ အဿာ တဿာ၊ အာသံ တာသံ၊ အမှာ တမှာ၊ အမှိ တမှိ၊ ဿာယာဒိဂ္ဂဟဏမာဒေသန္တရေ မာ ဟောတူတိ။}

\sutta{195}{133}{ဋေ သိဿိသိသ္မာ။}
\vutti{ဣသိသ္မာ သိဿ ဋေ ဝါ ဟောတိ၊ ‘ယော နွဇ္ဇ ဝိနယေ ကင်္ခံ၊ အတ္ထဓမ္မဝိဒူ ဣသေ’၊ ဝါတွေဝ? ဣသိ။}

\sutta{196}{134}{ဒုတိယဿ ယောဿ။}
\vutti{ဣသိသ္မာ ပရဿ ဒုတိယာယောဿ ဋေ ဝါ ဟောတိ၊ ‘သမဏေ ဗြာဟ္မဏေ ဝန္ဒေ၊ သမ္ပန္နစရဏေ ဣသေ’၊ ဝါတွေဝ? ဣသယော ပဿ၊ ဒုတိယဿာတိ ကိံ? ဣသယော တိဋ္ဌန္တိ။}

\sutta{197}{135}{ဧကစ္စာဒီဟ-တော။}
\vutti{အကာရန္တေဟိ ဧကစ္စာဒီဟိ ယောနံ ဋေ ဟောတိ၊ ဧကစ္စေ တိဋ္ဌန္တိ၊ ဧကစေ ပဿ၊ အတောတိ ကိံ? ဧကစ္စာယော၊ ဧဝံ ဧသ သ ပဌမ။}

\sutta{198}{136}{န နိဿ ဋာ။}
\vutti{ဧကစ္စာဒီဟိ ပရဿ နိဿ ဋာ န ဟောတိ၊ ဧကစ္စာနိ။}

\sutta{199}{137}{သဗ္ဗာဒီဟိ။}
\vutti{သဗ္ဗာဒီဟိ ပရဿ နိဿ ဋာ န ဟောတိ၊ သဗ္ဗာနိ။}

\sutta{200}{138}{ယောနမေဋ်။}
\vutti{အကာရန္တေဟိ သဗ္ဗာဒီဟိ ယောနမေဋ် ဟောတိ၊ သဗ္ဗေ တိဋ္ဌန္တိ၊ သဗ္ဗေ ပဿ၊ အတောတွေဝ? သဗ္ဗာယော။}

\sutta{201}{139}{နာညံ စ နာမပ္ပဓာနာ။}
\vutti{နာမဘူတေဟီ အပ္ပဓာနေဟိ စ သဗ္ဗာဒီဟိ ယံ ဝုတ္တံ၊ ယဉ္စာညံ သဗ္ဗာဒိကာရိယံ၊ တံ န ဟောတိ၊ တေ သဗ္ဗာ၊ တေ ပိယသဗ္ဗာ၊ တေ အတိသဗ္ဗာ။}

\sutta{202}{140}{တတိယတ္ထယောဂေ။}
\vutti{တတိယတ္ထေန ယောဂေ သဗ္ဗာဒီဟိ ယံ ဝုတ္တံ၊ ယဉ္စာညံ သဗ္ဗာဒိ ကာရိယံ၊ တံ န ဟောတိ၊ မာသေန ပုဗ္ဗာနံ မာသပုဗ္ဗာနံ။}

\sutta{203}{141}{စတ္ထသမာသေ။}
\vutti{စတ္ထသမာသဝိသယေ သဗ္ဗာဒီဟိ ယံ ဝုတ္တံ၊ ယဉ္စာညံ သဗ္ဗာဒိကာရိယံ၊ တံ န ဟောတိ၊ ဣက္ခိဏုတ္တရပုဗ္ဗာနံ၊ သမာသေတိ ကိံ? အမုသဉ္စ တေသဉ္စ ဒေဟိ။}

\sutta{204}{142}{ဝေဋ်။}
\vutti{ဧတ္ထသမာသဝိသဓယ သဗ္ဗာဒီဟိ ယဓဿဋ ဝုတ္တော၊ တဿ ဝါ ဟောတိ၊ ပုဂ္ဂုတ္တရေ၊ ပုဗ္ဗုတ္တရာ။}

\sutta{205}{143}{ပုဗ္ဗာဒီဟိ ဆဟိ။}
\vutti{ဧတေဟိ ပုဗ္ဗာဒီဟိ ဆဟိ သဝိသယေ ဧဋ ဝါ ဟောတိ၊ ပုဗ္ဗေ ပုဗ္ဗာ၊ ပရေ ပရာ၊ အပရေ အပရာ၊ ဒက္ခိဏေ ဒက္ခိဏာ၊ ဥတ္တရေ ဥတ္တရာ၊ အဓရေ အဓရာ၊ ဆဟိတိကိံ? ယေ။}

\sutta{206}{144}{မနာဒီဟိ သ္မိံသံနာသ္မာနံ သိသောဩသာသာ။}
\suttagana{207}{5}{သုမေဓာဒီန-မဝုဒ္ဓိစ။}
\vutti{မနာဒီဟိ သ္မိမာဒီနံ သိသောဩသာသာ ဝါ ဟောန္တိ ယထာက္ကမံ၊ မနသိ မနသ္မိံ၊ မနသော မနဿ၊ မနော မနံ၊ မနသာ မနေန၊ မနသာ မနသ္မာ၊ ကထံ ‘ပုတ္တော ဇာတော အစေတသော၊ ဟိတွာ ယာတိ သုမေဓသော၊ သုဒ္ဓုတ္တရဝါသသာ၊ ဟေမကပ္ပနဝါသသေ’တိ? သကတ္ထေဏတ္ထာ။ မနတမ တပ တေဇ သိရ ဥရ ဝစ ဩဇ ရဇယသ ပယ (၆) “\suttagananormal{208}{6}{သရ-ဝယာ-ယ-ဝါသ-စေတာ ဇလာ-သယ-က္ခယ-လောဟ-ပဋ-မနေသု။}”။}

\sutta{209}{145}{သတော သဗ် ဘေ။}
\vutti{သန္ထသဒ္ဒဿ သဗ် ဘဝတိ ဘကာရေ၊ သဗ္ဘိ။}

\sutta{210}{146}{ဘဝတော ဝါ ဘောန္တော ဂယောနာသေ။}
\vutti{ဘဝန္တသဒ္ဒဿ ဘောန္တာဒေသော ဝါ ဟောတိ ဂယောနာသေ၊ ဘောန္တ ဘဝံ၊ ဘောန္တော ဘဝန္တော၊ ဘောတာ ဘဝတာ၊ ဘောတော ဘဝတော၊ ဘော ဣတိ အာမန္တဏေ နိပါတော ‘ကုတော နု အာဂစ္ဆထ ဘော တယော ဇနာ’၊ ဧဝံ ဘဝန္တတိ၊ ဘဒ္ဒေတိ သဒ္ဒန္ထရေန သိဒ္ဓံ၊ သဒ္ဓန္ထဣတိ ဒဿ ဒွိဘာဝေန။}

\sutta{211}{147}{သိဿဂ္ဂိတော နိ။}
\vutti{အဂ္ဂိသ္မာ သိဿ နိ ယောတိ ဝါ၊ အဂ္ဂိနိ အဂ္ဂိ။}

\sutta{212}{148}{န္တဿံ။}
\vutti{သိမှိ န္တပ္ပစ္စယဿ အံ ဟောတိ ဝါ၊ ဂစ္ဆံ ဂစ္ဆန္တော။}

\sutta{213}{149}{ဘူတော။}
\vutti{ဘူဓာတုတော န္တဿ အံ ဟောတိ သိမှိ နိစ္စံ ပုနဗ္ဗိဓာနာ၊ ဘဝံ။}

\sutta{214}{150}{မဟန္တာရဟန္တာနံ ဋာ ဝါ။}
\vutti{သိမှိ မဟန္တာရဟန္တာနံ န္တဿ ဋာ ဝါ ဟောတိ၊ မဟာ မဟံ၊ အရဟာ အရဟံ။}

\sutta{215}{151}{န္တုဿ။}
\vutti{သိမှိ န္တုဿ ဋာ ဟောတိ၊ ဂုဏဝါ။}

\sutta{216}{152}{အံငံ နပုံသကေ။}
\vutti{န္တုဿ အံငံ ဟောန္တိ သိမှိ နပုံသကေ၊ ဂုဏဝံ ကုလံ၊ ဂုဏဝန္တံ ကုလံ၊ နပုံသကေတိံ ကိံ? သီလဝါ ဘိက္ခု။}

\sutta{217}{153}{ဟိမဝတော ဝါ ဩ။}
\vutti{ဟိမဝတော သိမှီ န္တုဿ ဩ ဝါ ဟောတိ၊ ဟိမဝန္တော ဟိမဝါ။}

\sutta{218}{154}{ရာဇာဒိယုဝါဒိတွာ။}
\vutti{ရာဇာဒီဟိယုဝါဒီဟိ စ သိဿ အာ ဟောတိ၊ ရာဇာ၊ ယုဝါ။ ရာဇ ဗြဟ္မ သခ အတ္တ အာတုမ (၇) “\suttagananormal{219}{7}{ဓမ္မော ဝါ-ညတ္ထေ။}” ဒဠှဓမ္မာ၊ အသ္မ၊ (၈) “\suttagananormal{220}{8}{ဣမော ဘာဝေ။}” အဏိမာ၊ (မဟိမာ၊ ဂရိမာ) လဃိမာ၊ ယုဝ သာ သုဝါ မဃဝ ပုမ ဝတ္တဟ။}

\sutta{221}{155}{ဝါ-မှာ-နင်။}
\vutti{ရာဇာဒီနံ ယုဝါဒီနံ စ အာနင် ဟောတိ ဝါ အံမှိ၊ ရာဇာနံ ရာဇံ၊ ယုဝါနံ ယုဝံ။}

\sutta{222}{156}{ယောနမာနော။}
\vutti{ရာဇာဒီဟိ ယုဝါဒီဟိ စ ယောနံ အာနော ဝါ ဟောတိ၊ ရာဇာနော ယုဝါနော၊ ဝါ တွေဝ? ရာဇာ ရာဇေ၊ ယုဝါ ယုဝေ။}

\sutta{223}{157}{အာယောနော စ သခါ။}
\vutti{သခတော ယောန မာယော နော ဟောန္တိ ဝါ အာနော စ၊ သခါယော၊ သခိနော၊ သခါနော၊ ဝါ တွေဝ? သခါ၊ သခေ။}

\sutta{224}{158}{ဋေ သ္မိနော။}
\vutti{သခတော သ္မိနော ဋေ ဟောတိ၊ သခေ၊ နိစ္စတ္ထော-ယမာရမ္ဘော။}

\sutta{225}{159}{နောနာသေသွိ။}
\vutti{သခဿ ဣ ဟောတိ နောနာသေသု၊ သခိနော၊ သခိနာ၊ သခိဿ။}

\sutta{226}{160}{သ္မာနံသု ဝါ။}
\vutti{သခဿ ဣ ဝါ ဟောတိ သ္မာနံသု၊ သခိသ္မာ သခသ္မာ၊ သခီနံ သခါနံ။}

\sutta{227}{161}{ယောသွံဟိသု စာရင်။}
\vutti{သခဿ အာရင် ဝါ ဟောတိ ယောသွံဟိသု သ္မာနံသု စ၊ သခါရော သခါယော၊ သခါရေသု သခေသု၊ သခါရံ သခံ၊ သခါရေဟိ သခေဟိ၊ သခါရာ သခါ၊ သခသ္မာ၊ သခါရာနံ သခါနံ။}

\sutta{228}{162}{လ္တုပိတာဒီနမသေ။}
\vutti{လ္တုပ္ပစ္စယန္တာနံ ပိတာဒီနံ စ အာရင ဟောတီ သတော-ညတြ၊ ကတ္တာရော၊ ပိတရော၊ ကတ္တာရံ၊ ပိတရံ၊ ကတ္တာရာ၊ ပိတရာ၊ ကတ္တရိ၊ ပိတရိ၊ အသေတိ ကိံ? ကတ္တုနော၊ ပိတုနော။}

\sutta{229}{163}{နံမှိ ဝါ။}
\vutti{နမှိ လ္တုပိတာဒီနုမာနင ဝါ ဟောတိ၊ ကတ္တာရာနံ ကတ္တူနံ၊ ဒိတရာနံ ပိတုန္နံ။}

\sutta{230}{164}{အာ။}
\vutti{နမှိ လ္တုပိတာဒီနမာ ဝါ ဟောတီ၊ ကတ္တာနံ ကတ္တူနံ၊ ပီတာနံ ပိတုန္နံ။}

\sutta{231}{165}{သလောပေါ။}
\vutti{လ္တုပိတာဒိဟိ သဿ လောပေါ ဝါ ဟောတိ၊ ကတ္ထု ကတ္ထုနော၊ သကမန္တာတု သကမန္ဓာတုနော၊ ပီတု ပိတုနော။}

\sutta{232}{166}{သုဟိသွာရင်။}
\vutti{သုဟိသု လ္တုပိတာဒီနမာရင် ဝါ ဟောတိ၊ ကတ္တာရေသု ကတ္တူသု၊ ပိတရေသု ပိတူသု၊ ကတ္တာရေဟိ ကတ္တူဟိ၊ ပိတရေဟိ ပီတူဟီ။}

\sutta{233}{167}{နဇ္ဇာ ယောသွာမ်။}
\vutti{ယောသု နဒိသဒ္ဒဿ အာမ် ဝါ ဟောတိ၊ နဇ္ဇာယော နဒိယော။}

\sutta{234}{168}{ဋိ ကတိမှာ။}
\vutti{ကတိမှာ ဓယာနံ ဋိ ဟောတိ၊ ကတိ တိဋ္ဌန္တိ၊ ကတိ ပဿ။}

\sutta{235}{169}{ဋ ပဉ္စာဒီဟိ စုဒ္ဒသဟိ။}
\vutti{ပဉ္စာဒီဟိ စုဒ္ဒသဟိ သံချာဟိ ယောနံ ဋော ဟောတိ ပဉ္စ၊ ပဉ္စ၊ ဧဝံ ယာဝ အဋ္ဌာရသာ။ ပဉ္စာဒီဟီတိ ကိံ? ဒွေ၊ တယော၊ စတ္တာရော၊ စုဒ္ဓသဟီတိ ကိံ? ဒွေ ဝိသတိယော။}

\sutta{236}{170}{ဥဘဂေါဟိ ဋော။}
\vutti{ဥဘဂေါဟိ ယောနံ ဋော ဟောတိ၊ ဥဘော၊ ဥဘော၊ ဂါဝေါ၊ ဂါဝေါ၊ ကထံ ‘ဣမေကရတ္ထိံ ဥဘယော ဝသာမာ’တိ? ဋောမှိ ယကာရာဂမော။}

\sutta{237}{171}{အာရင်သ္မာ။}
\vutti{အာရင်ဒေသတော ယောနံ ဋော ဟောတိ၊ သခါရော၊ ကတ္တာရော၊ ပိတရော။}

\sutta{238}{172}{ဋောဋေ ဝါ။}
\vutti{အာရဝါဒေသမှာ ယောနံ ဋောဋေ ဝါ ဟောန္တိ ယထာက္ကမံ၊ သခါရော၊ သခါရေ သခါယော၊ ဋောဂ္ဂဟဏံ လာဃဝတ္ထံ။}

\sutta{239}{173}{ဋာ နာသ္မာနံ။}
\vutti{အာရဝါဒေသမှာ နာသ္မာနံ ဋာ ဟောတိ၊ ကတ္တာရာ၊ ကတ္တရာ။ ကူစိ ဝါ ဟောတိ ဗယုလာဓိကာရာ၊ ဧတာဒိသာ သခါရမှာ။}

\sutta{240}{174}{ဋိ သ္မိနော။}
\vutti{အာရဝါဒေသမှာ သ္မိနော ဋိ ဟောတိ၊ ကတ္တရိ၊ ဝိတရိ။}

\sutta{241}{175}{ဒိဝါဒိတော။}
\vutti{ဒိဝါဒီဟိ နာမေဟိ သ္မိနော ဋိ ဟောတိ၊ ဒိဝိ၊ ဘုဝိ။ နိစ္စံ ဝကာရာဂမော။}

\sutta{242}{176}{ရဿာရင်။}
\vutti{သ္မိမှိ အာရော ရဿော ဟောတိ၊ ကတ္တရိ၊ နတ္တရိ။}

\sutta{243}{177}{ပိတာဒီနမနတွာဒီနံ။}
\vutti{နတွာဒိဝဇ္ဇိတာနံ ပိတာဒီနမာရော ရဿော ဟောတိ သဗ္ဗာသု ဝိဘတ္တီသု၊ ပိတရော၊ ပိတရံ၊ အနုတွာဒီနန္တိ ကိံ? နတ္တာရော။}

\sutta{244}{178}{ယုဝါဒီနံ သုဟိသွာနင်။}
\vutti{သုဟိသု ယုဝါဒီနံ အာနင် ဟောတိ၊ ယုဝါနေသု၊ ယုဝါနေဟိ။}

\sutta{245}{179}{နောနာနေသွာ။}
\vutti{ဧသု ယုဝါဒီနမာ ဟောတိ၊ ယုဝါနော၊ ယုဝါနာ၊ ယုဝါနေ။}

\sutta{246}{180}{သ္မာသ္မိံနံ နာနေ။}
\vutti{ယုဝါဒီဟိ သ္မာသ္မိန္နံ နာနေ ဟောန္တိ ယထာက္ကမံ၊ ယုဝါနာ၊ ယုဝါနေ။}

\sutta{247}{181}{ယောနံ နောနေ ဝါ။}
\vutti{ယုဝါဒီဟိ ယောနံ နောနေ ဝါ ဟောန္တိ ယထာက္ကမံ၊ ယုဝါနော၊ ယုဝါနေ၊ ဝါတိ ကိံ? ယုဝေ ပဿ၊ နောဂ္ဂဟဏံ လာဃဝတ္ထံ။}

\sutta{248}{182}{ဣတော-ညတ္ထေ ပုမေ။}
\vutti{အညပဒတ္ထေ ဝတ္တမာနာ ဣကာရန္တတော နာမသ္မာ ယောနံ နောနေ ဝါ ဟောန္တိ ယထာက္ကမံ ပုလ္လိင်္ဂေ၊ တောမရင်္ကုသပါဏိနော၊ တောမရင်္ကုသပါဏိနေ၊ ဝါတွေဝ? တောမရင်္ကုသပါဏယော၊ အညတ္ထေတိ ကိံ? ပါဏယော။}

\sutta{249}{183}{နေ သ္မိနော ကွစိ။}
\vutti{အညပဒတ္ထေ ဝတ္ထမာနာ ဣကာရန္တတော နာမသ္မာ သ္မိနော နေ ဟောတိ ဝါ ကွစိ ပုလ္လိင်္ဂေ၊ ကတညုမှိ စ ပေါသမှိ၊ သီလဝန္တေ အရိယ ဝုတ္တိနေ၊ ဝါတွေဝ? အရိယဝုတ္တိမှိ၊ ပုမေတွေဝ? အရိယဝုတ္တိယာ။}

\sutta{250}{184}{ပုမာ။}
\vutti{ပုမသဒ္ဒတော သ္မိနော ယံ ဝုတ္တံ၊ တံ ဝါ ဟောတိ၊ ပုမာနေ ပုမေ။}

\sutta{251}{185}{နာမှိ။}
\vutti{ပုမဿ နာမှိ ယံ ဝုတ္တံ၊ တံ ဝါ ဟောတိ၊ ပုမာနာ ပုမေန။}

\sutta{252}{186}{သုမှာ စ။}
\vutti{ပုမဿ သုမှိ ယံ ဝုတ္တံ၊ တံ အာ စ ဝါ ဟောတိ၊ ပုမာနေသု၊ ပုမာသု ပုမေသုံ။}

\sutta{253}{187}{ဂဿံ။}
\vutti{ပုမသဒ္ဒတော ဂဿ အံ ဝါ ဟောတိ၊ ဘော ပုမံ ဘော ပုမ၊ ဘော ဣတ္တိပုမံ ဘော ဣတ္ထိပုမ။}

\sutta{254}{188}{သာဿံ-သေ စာနင်။}
\vutti{သာသဒ္ဒဿ အာနင် ဟောတိ အံသေ ဂေ စ၊ သာနံ၊ သာနဿ၊ စဘာ သာန။}

\sutta{255}{189}{ဝတ္တဟာ သနန္နံ နောနာနံ။}
\vutti{ဝတ္တဟာ သနံနံ နောနာနံ ဟောန္တိ ယထာက္ကမံ၊ ဝတ္တဟာနော၊ ဝတ္တဟာနာနံ။}

\sutta{256}{190}{ဗြဟ္မဿု ဝါ။}
\vutti{ဗြဟ္မဿ ဥ ဝါ ဟောတိ သနံသု၊ ဗြဟ္မုနော ဗြဟ္မဿ၊ ဗြဟ္မူနံ ဗြဟ္မာနံ။}

\sutta{257}{191}{နာမှိ။}
\vutti{ဗြဟ္မဿ ဥ ဟောတိ နာမှိ၊ ဗြဟ္မုနာ။}

\sutta{258}{192}{ပုမကမ္မထာမဒ္ဓါနံ ဝါ သသ္မာသု စ။}
\vutti{ပုမာဒိနမု ဟောတိ ဝါ သသ္မာသု နာမှိ စ၊ ပုမုနော ပုမဿ၊ ပုမုနာ ပုမာနာ၊ ပုမုနာ ပုမာနာ၊ ကမ္မုနော ကမ္မဿ၊ ကမ္မုနာ ကမ္မသ္မာ၊ ကမ္မုနာ ကမ္မနာ၊ ထာမုနော ထာမဿ၊ ထာမုနာ ထာမသ္မာ၊ ထာမုနာ ထာမေန၊ အဒ္ဓုနော အဒ္ဓဿ၊ အဒ္ဓုနာ အဒ္ဓသ္မာ၊ အဒ္ဓုနာ အဒ္ဓနာ။}

\sutta{259}{193}{ယုဝါ သဿိနော။}
\vutti{ယုဝါ သဿ ဝါ ဣနော ဟောတိ၊ ယုဝိနော ယုဝဿ။}

\sutta{260}{194}{နော-တ္တာ-တုမာ။}
\vutti{အတ္တာတုမေဟိ သဿ နော ဟောတိ ဝါ၊ အတ္တနော အတ္တဿ၊ အာတုမနော အာတုမဿ။}

\sutta{261}{195}{သုဟိသု နက်။}
\vutti{အတ္တအာတုမာနံ သုဟိသု ဝါ နက ဟောတိ၊ အတ္တနေသု အတ္တေသု အာတုမနေသု အာတုမေသု၊ အတ္တနေဟိ အတ္တေဟိ အာတုမနေဟိ အာတုမေဟိ၊ ကထံ ‘ဝေရိနေသု’တိ? ‘နက’ ဣတိ ယောဂဝိဘာဂါ။}

\sutta{262}{196}{သ္မာဿ နာ ဗြဟ္မာ စ။}
\vutti{ဗြဟ္မာ အတ္တအာတုမေဟိ စ သွာဿ နာ ဟောတိ၊ ဗြဟ္မုနာ၊ အတ္တနာ၊ အာတုမနာ။}

\sutta{263}{197}{ဣမေတာနမေနာ-နွာဒေသေ ဒုတိယာယံ။}
\vutti{ဣမဧတသဒ္ဒါနံ ကထိတာနုကထနဝိသယေ ဒုတိယာယမေနာဒေသော ဟောတိ၊ ဣမံ ဘိက္ခုံ ဝိနယမဇ္ဈာပယ၊ အထော ဧနံ ဓမ္မမဇ္ဈာပယ၊ ဣမေ ဘိက္ခူ ဝိနယမဇ္ဈာပယ၊ အထော ဧနေ ဓမ္မမဇ္ဈာပယ၊ ဧဝမေတဿ စ ယောဇနိယံ။}

\sutta{264}{198}{ကိဿ ကော သဗ္ဗာသု။}
\vutti{သဗ္ဗာသု ဝိဘတ္တိသု ကိဿ ကော ဟောတိ၊ ကော၊ ဧက၊ ကာ၊ ကာယော၊ ကံ၊ ကာနိ၊ ကေနေစ္စာမိ။}

\sutta{265}{199}{ကိ သသ္မိံသု ဝါ-နိတ္ထိယံ။}
\vutti{အနိတ္ထိယံ ကိဿ ကိ ဝါ ဟောတိ သသ္မိံသု၊ ကိဿ ကဿ၊ ကိသ္မိံ ကသ္မိံ၊ အနိတ္ထိယန္တိ ကိံ? ကဿာ၊ ကဿံ။}

\sutta{266}{200}{ကိမံသိသု သဟ နပုံသကေ။}
\vutti{အံသိသု သဟ တေဟိ ကိံသဒ္ဒဿ ကိံ ဟောတိ နပုံသကေ။ ကိံ၊ ကိံ၊ နပုံသကေတိ ကိံ? ကော၊ ကံ။}

\sutta{267}{201}{ဣမဿိဒံ ဝါ။}
\vutti{အံသိသု သဟ တေတိ ဣမဿ ဣဒံ ဟောတိ ဝါ နပုံသကေ၊ ဣဒံ ဣမံ၊ ဣဒံ ဣမံ။}

\sutta{268}{202}{အမုဿာဒုံ။}
\vutti{အံသိသု သဟ တေဟိ အမုဿ အဒုံ ဟောတိ ဝါ နပုံသကေ၊ အဒုံ အမုံ၊ အဒုံ အမုံ။}

\sutta{269}{203}{သုမှာ-မုဿာ-သ္မာ။}
\vutti{အမှဿ အသ္မာ ဟောတိ ဝါ သုမှိ၊ ဘတ္တိရသ္မာသု ယာ တဝ၊ ဝါ တွေဝ? အမှေသု။}

\sutta{270}{204}{နံမှိ တိစတုန္နမိတ္ထိယံ တိဿစတဿာ။}
\vutti{နံမှိ တိစတုန္နံ တိဿစတဿာ ဟောန္တိ ဣတ္ထိယံ ယထာက္ကမံ၊ တိဿန္နံ စတဿန္နံ၊ ဣတ္ထိယန္တိ ကိံ? တိဏ္ဏံ စတုန္နံ။}

\sutta{271}{205}{တိဿော စတဿော ယောမှိ သဝိဘတ္တီနံ။}
\vutti{ဝိဘတ္တိသဟိတာနံ တိစတုန္နံ ယောမှိ တိဿော စတဿော ဟောန္တိ ဣတ္ထိယံ ယထာက္ကမံ၊ တိဿော စတဿော။}

\sutta{272}{206}{တီဏိစတ္တာရိ နပုံသကေ။}
\vutti{ယောမှိ သဝိဘတ္တီနံ တိစတုန္နံ ယထာက္ကမံ တီဏိစတ္တာရိ ဟောန္တိ နပုံသကေ၊ တီဏိ။ စတ္တာရိ။}

\sutta{273}{207}{ပုမေ တယောစတ္တာရော။}
\vutti{ယောမှိ သဝိဘတ္တီနံ တိစတုန္နံ တယောစတ္တာရော ဟောန္တိ ယထာက္ကမံ ပုလ္လိင်္ဂေ၊ တယော၊ စတ္တာရော။}

\sutta{274}{208}{စတုရော ဝါ စတုဿ။}
\vutti{စတုသဒ္ဒဿ သဝိဘတ္တိဿ ယောမှိ စတုရော ဝါ ဟောတိ ပုလ္လိင်္ဂေ၊ စတုရော ဇနာ သံဝိဓာယ၊ ကထံ ‘စတုရော နိမိတ္တေ နာဒ္ဒဿာသိ’န္တိ? လိင်္ဂဝိပလ္လာသာ။}

\sutta{275}{209}{မယ-မသ္မာ-မှာဿ။}
\vutti{ယောသွမှဿ သဝိဘတ္တိဿ မယမသ္မာ ဝါ ဟောန္တိ ယထာက္ကမံ၊ မယံ၊ အသ္မာ၊ အမှေ။}

\sutta{276}{210}{နံ-သေသွ-သ္မာကံ-မမံ။}
\vutti{နံသေသွမုဿ သဝိဘတ္တိဿ အသ္မာကံ မမံ ဟောန္တိ ဝါ ယထာက္ကမံ၊ အသ္မာကံ၊ အမှာကံ၊ မမံ မမ။}

\sutta{277}{211}{သိမှ-ဟံ။}
\vutti{သိမှိ အမှဿ သဝိဘတ္တိဿ အဟံ ဟောတိ၊ အဟံ။}

\sutta{278}{212}{တုမှဿ တုဝံ တွမမှိ စ။}
\vutti{အံမှိ သိမှိ စ တုမှဿ သဝိဘတ္တိဿ တုဝံတွံ ဟောန္တိ ယထာက္ကမံ၊ တုဝံ၊ တွံ။}

\sutta{279}{213}{တယာတယီနံ တွ ဝါ တဿ။}
\vutti{တုမှဿ တယာတယီနံ တကာရဿ တွ ဟောတိ ဝါ၊ တွယာ တယာ၊ တွယိ တယိ။}

\sutta{280}{214}{သ္မာမှိ တွမှာ။}
\vutti{သွာမှိ တုမှဿ သဝိဘတ္တိဿ တွမှာ ဟောတိ ဝါ၊ ပတ္တာ နိဿံ သယံ တွမှာ၊ ဝါ တွေဝ? တွယာ။}

\sutta{281}{215}{န္တန္တူနံ န္တော ယောမှိ ပဌမေ။}
\vutti{ပဌမေ ယောမှိ န္တန္တူနံ သဝိဘတ္တီနံ န္တောဣစ္စာဒေသော ဝါ ဟောတိ၊ ဂစ္ဆန္တော၊ ဂစ္ဆန္တာ၊ ဂုဏဝန္တော ဂုဏဝန္တာ။}

\sutta{282}{216}{တံ နံမှိ။}
\vutti{နံမှိ န္တန္တူနံ သဝိဘတ္ထီနံ တံ ဝါ ဟောတိ၊ ဂစ္ဆတံ ဂစ္ဆန္တာနံ၊ ဂုဏဝတံ ဂုဏဝန္တာနံ။}

\sutta{283}{217}{တောတာတိတာ သသ္မာသ္မိံနာသု။}
\vutti{သာဒီသု န္တန္တူနံ သဝိဘတ္တီနံ တောတာတိတာ ဟောန္တိ ဝါ ယထာက္ကမံ၊ ဂစ္ဆတော ဂစ္ဆန္တဿ၊ ဂုဏဝတော ဂုဏဝန္တဿ၊ ဂစ္ဆတာ ဂစ္ဆန္တမှာ၊ ဂုဏဝတာ ဂုဏဝန္ထမှာ၊ ဂစ္ဆတိ ဂစ္ဆန္တေ၊ ဂုဏဝတိ ဂုဏဝန္တေ၊ ဂစ္ဆတာ ဂစ္ဆန္တေန၊ ဂုဏဝတာ ဂုဏဝန္တေန။}

\sutta{284}{218}{ဋဋာအံ ဂေ။}
\vutti{ဂေ ပရေ န္တန္တူနံ သဝိဘတ္တီနံ ဋဋာအံ ဣစ္စာဒေသာ ဟောန္တိ၊ ဘော ဂစ္ဆ၊ ဘော ဂစ္ဆာ၊ ဘော ဂစ္ဆံ၊ တော ဂုဏဝ၊ ဘော ဂုဏဝါ၊ ဘော ဂုဏဝံ။}

\sutta{285}{219}{ယောမှိ ဒွိန္နံ ဒုဝေ ဒွေ။}
\vutti{ယောမှီ ဒွီဿ သဝိဘတ္တိဿ ဒုဝေဒွေ ဟောန္တိ ပစ္စေကံ၊ ဒုဝေ၊ ဒွေ။}

\sutta{286}{220}{ဒုဝိန္နံ နံမှိ ဝါ။}
\vutti{နံမှိ ဒွိဿ သဝိဘတ္တိဿ ဒုဝိန္နံ ဟောတိ ဝါ၊ ဒုဝိန္နံ၊ ဒွိန္နံ။}

\sutta{287}{221}{ရာဇဿ ရညံ။}
\vutti{နံမှိ ရာဇဿ သဝိဘတ္တိဿ ရညံ ဟောတိ ဝါ၊ ရညံ ရာဇာနံ။}

\sutta{288}{222}{နာသ္မာသု ရညာ။}
\vutti{နာသ္မာသု ရာဇဿ သဝိဘတ္တိဿ ရညာ ဟောတိ၊ ရညာ ကတံ၊ ရညာ နိဿဋံ။}

\sutta{289}{223}{ရညောရညဿရာဇိနော သေ။}
\vutti{သေ ရာဇဿ သဝိဘတ္တိဿ ရညော ရညဿ ရာဇိနော ဟောန္တိ၊ ရညော၊ ရညဿ၊ ရာဇိနော။}

\sutta{290}{224}{သ္မိံမှိ ရညေရာဇိနိ။}
\vutti{သ္မိမှိ ရာဇဿ သဝိဘတ္တိဿ ရညေ ရာဇိနိ ဟောန္တိ၊ ရညေ၊ ရာဇိနိ။}

\sutta{291}{225}{သမာသေ ဝါ။}
\vutti{သမာသဝိသယေ ဧတေ အာဒေသာ ရာဇဿ ဝါ ဟောန္တိ၊ ကာသိရညာ ကာသိရာဇေန၊ ကာသိရညာ ကာသိရာဇသ္မာ၊ ကာသိရညော ကာသိရာဇဿ၊ ကာသိရညေ ကာသိရာဇေ။}

\sutta{292}{226}{သ္မိံမှိ တုမှမှာနံ တယိမယိ။}
\vutti{သ္မိမှိ တုမှအမုသဒ္ဒါနံ သဝိဘတ္တီနံ တယိမယိ ဟောန္တိ ယထာက္ကမံ၊ တယိ၊ မယိ။}

\sutta{293}{227}{အံမှိ တံ မံ တဝံ မမံ။}
\vutti{အံမှိ တုမှအမှသဒ္ဒါနံ သဝိဘတ္တီနံ တံ မံ တဝံ မမံ ဟောန္တိ ယထာက္ကမံ၊ တံ၊ မံ၊ တဝံ၊ မမံ。}

\sutta{294}{228}{နာသ္မာသု တယာမယာ။}
\vutti{နာသ္မာသု တုမှအမှသဒ္ဒါနံ သဝိဘတ္တီနံ တယာမယာ ဟောန္တိ ယထာက္ကမံ၊ တယာ ကတံ၊ မယာ ကတံ၊ တယာ နိဿဋံ၊ မယာ နိဿဋံ။}

\sutta{295}{229}{တဝ မမ တုယှံ မယှံ သေ။}
\vutti{သေ တုမှအမှသဒ္ဒါနံ သဝိဘတ္တီနံ တဝ မမ တုယှံ မယှံ ဟောန္တိ ယထာက္ကမံ၊ တဝ၊ တုယှံ၊ မမ၊ မယှံ။}

\sutta{296}{230}{ငံ-ငါကံ နံမှိ။}
\vutti{နံမှိ တုမှအမှသဒ္ဒါနံ သဝိဘတ္တီနံ ငံငါကံ ဟောန္တိ ပစ္စေကံ၊ တုမှံ၊ တုမှာကံ၊ အမှံ၊ အမှာကံ၊ ယထာသင်္ချမတြ န ဝိဝစ္ဆတေ။}

\sutta{297}{231}{ဒုတိယေ ယောမှိ ဝါ။}
\vutti{တုမှအမှသဒ္ဒါနံ သဝိဘတ္တီနံ ပစ္စေကံ ငံငါကံ ဝါ ဟောန္တိ ယောမှိ ဒုတိယေ၊ တုမှံ၊ တုမှာကံ၊ တုမှေ၊ အမှံ၊ အမှာကံ၊ အမှေ။}

\sutta{298}{232}{အပါဒါဒေါ ပဒတေကဝါကျေ။}
\vutti{ဣဒမဓိကတံ ဝေဒိတဗ္ဗံ။ ပဇ္ဇတေ-နေနတ္ထောတိ ပဒံ-သျာဒျန္တံ တျာဒျန္တံ စ၊ ပဒသမူဟော ဝါကျံ။}

\sutta{299}{233}{ယောနံဟိသွ-ပဉ္စမျာ ဝေါ-နော။}
\vutti{အပဉ္စမိယာ ယောနံဟိသွပါဒါဒေါ ဝတ္တမာနာနံ ပဒသ္မာ ပရေသံ ဧကဝါကျေ ဌိတာနံ တုမှာမှသဒ္ဒါနံ သဝိဘတ္တီနံ ဝေါနော ဟောန္တိ ဝါ ယထာက္ကမံ၊ တိဋ္ဌထ ဝေါ၊ တိဋ္ဌထ တုမှေ၊ တိဋ္ဌာမ နော၊ တိဋ္ဌာမ မယံ၊ ပဿတိ ဝေါ၊ ပဿတိ တုမှေ၊ ပဿတိ နော၊ ပဿတိ အမှေ၊ ဒီယတေ ဝေါ၊ ဒီယတေ တုမှံ၊ ဒီယတေ နော၊ ဒီယတေ အမှံ၊ ဓနံ ဝေါ၊ ဓနံ တုမှံ၊ ဓနံ နော ဓနံ အမှံ၊ ကတံ ဝေါ၊ ကတံ တုမှေဟိ၊ ကတံ နော၊ ကတံ အမှေဟိ၊ အပဉ္စမျာတိ ကိံ? နိဿဋံ တုမှေဟိ၊ နိဿဋံ အမှေဟိ၊ အပါဒါဒေါတွေဝ? ‘ဗလဉ္စ ဘိက္ခူနမနုပ္ပဒိန္နံ၊ တုမှေဟိ ပုညံ ပသုတံ အနပ္ပကံ’၊ ပဒတောတွေဝ? တုမှေ တိဋ္ဌထ၊ ဧကဝါကျေတွေဝ? ဒေဝဒတ္တော တိဋ္ဌတိ ဂါမေ၊ တုမှေ တိဋ္ဌထ နဂရေ၊ သဝိဘတ္တီနံတွေဝ? အရဟတိ ဓမ္မော တုမှာဒိသာနံ၊ အရဟတိ ဓမ္မော အမှာဒိသာနံ။}

\sutta{300}{234}{တေမေ နာသေ။}
\vutti{နာမှိ သေ စ အပါဒါဒေါ ဝတ္တမာနာနံ ပဒသ္မာ ပရေသံ ဧကဝါကျေ ဌိတာနံ တုမှာမှသဒ္ဒါနံ သဝိဘတ္တီနံ တေမေ ဝါ ဟောန္တိ ယထာက္ကမံ၊ ကတံ တေ၊ ကတံ တယာ၊ ကတံ မေ၊ ကတံ မယာ၊ ဒီယတေ တေ၊ ဒီယတေ တဝ ဒီယတေ မေ၊ ဒီယတေ မမ၊ ဓနံ တေ၊ ဓနံ တဝ၊ ဓနံ မေ၊ ဓနံ မမ။}

\sutta{301}{235}{အနွာဒေသေ။}
\vutti{ကထိတာနုကထနဝိသယေ တုမှအမှ-သဒ္ဒါနမာဒေသာ နိစ္စံ ဘဝန္တိ ပုနဗ္ဗိဓာနာ၊ ဂါမော တုမှံ ပရိဂ္ဂဟော၊ အထော ဇနပဒေါ ဝေါ ပရိဂ္ဂဟော။}

\sutta{302}{236}{သပုဗ္ဗာ ပဌမန္တာ ဝါ။}
\vutti{ဝိဇ္ဇမာနပုဗ္ဗသ္မာ ပဌမန္တာ ပရေသံ တုမှအမှသဒ္ဒါနမာဒေသာ ဝါ ဟောန္တိ အနွာဒေသေပိ၊ ဂါမေ ပဋော တုမှာကံ၊ အထော နဂရေ ကမ္ဗလော ဝေါ၊ အထော နဂရေ ကမ္ဗလော တုမှာကံ၊ သပုဗ္ဗာတိ ကိံ? ပဋော တုမှာကံ၊ အထော ကမ္ဗလော ဝေါ၊ ပဌမန္တာတိ ကိ? ပဋော နာဂရေ တုမှာကံ၊ အထော ကမ္ဗလော ဂါမေ ဝေါ။}

\sutta{303}{237}{န စ-ဝါ-ဟ-ဟေ-ဝယောဂေ။}
\vutti{စာဒီဟိ ယောဂေ တုမှအမှသဒ္ဒါနမာဒေသာ န ဟောန္တိ၊ ဂါမော တဝ စ ပရိဂ္ဂဟော၊ မမ စ ပရိဂ္ဂဟော၊ ဂါမော တဝ ဝါ ပရိဂ္ဂဟော၊ မမ ဝါ ပရိဂ္ဂဟော၊ ဂါမော တဝ ဟ ပရိဂ္ဂဟော၊ မမ ဟ ပရိဂ္ဂဟော၊ ဂါမော တဝါဟ ပရိဂ္ဂဟော၊ မမာဟ ပရိဂ္ဂဟော၊ ဂါမော တဝေဝ ပရိဂ္ဂဟော၊ မမေဝ ပရိဂ္ဂဟော၊ ဧဝံ သဗ္ဗတ္ထ ဥဒါဟရိတဗ္ဗံ၊ ယောဂေတိ ကိံ? ဂါမော စ တေ ပရိဂ္ဂဟော၊ နဂရဉ္စ မေ ပရိဂ္ဂဟော။}

\sutta{304}{238}{ဒဿနတ္ထေ-နာ-လောစနေ။}
\vutti{ဒဿနတ္ထေသု အာလောစနဝဇ္ဇိတေသု ပယုဇ္ဇမာနေသု တုမှအမှသဒ္ဒါနမာဒေသာ န ဟောန္တိ၊ ဂါမော တုမှေ ဥဒ္ဒိဿာဂတော၊ ဂါမော အမှေ ဥဒ္ဒိဿာဂတော၊ အနာလောစနေတိ ကိံ? ဂါမော ဝေါ အာလောစေတိ၊ ဂါမော နော အာလောစေတိ။}

\sutta{305}{239}{အာမန္တဏံ ပုဗ္ဗမသန္တံဝ။}
\vutti{အာမန္တဏံ ပုဗ္ဗမဝိဇ္ဇမာနံ ဝိယ ဟောတိ တုမှာမှသဒ္ဒါနမာဒေသဝိသယေ၊ ဒေဝဒတ္တ တဝ ပရိဂ္ဂဟော၊ အာမန္တဏန္တိ ကိံ? ကမ္ဗလော တေ ပရိဂ္ဂဟော၊ ပုဗ္ဗမိတိ ကိံ? ’မယေတံ သဗ္ဗမက္ခာတံ၊ တုမှာကံ ဒွိဇပုင်္ဂဝါ ၊ ပရဿ ဟိ အဝိဇ္ဇမာနတ္တေ ‘အပါဒါဒေါ’တိ ပဋိသေဓော န သိယာ။ ဣဝါတိ ကိံ? သဝနံ ယထာ သိယာ။}

\sutta{306}{240}{န သာမညဝစနမေကတ္ထေ။}
\vutti{သမာနာဓိကရဏေ ပရတော သာမညဝစနမာမန္တဏမသန္တံ ဝိယ န ဟောတိ၊ မာဏဝက ဇဋိလက တေ ပရိဂ္ဂဟော။ ပရဿာဝိဇ္ဇမာနတ္တေပိ ပုဗ္ဗရူပမုပါဒါယာဒေသော ဟောတိ၊ သာမညဝစနန္တိ ကိံ? ဒေဝဒတ္တ မာဏဝက တဝ ပရိဂ္ဂဟော၊ ဧကတ္ထေတိ ကိံ? ဒေဝဒတ္တ ယညဒတ္တ တုမှံ ပရိဂ္ဂဟော။}

\sutta{307}{241}{ဗဟူသု ဝါ။}
\vutti{ဗဟူသု ဝတ္တမာနမာမန္တဏံ သာမညဝစနမေကတ္ထေ အဝိဇ္ဇမာနံ ဝိယ ဝါ န ဟောတိ၊ ဗြာဟ္မဏာ ဂုဏဝန္တော တုမှာကံ ပရိဂ္ဂဟော၊ ဗြာဟ္မဏာ ဂုဏဝန္တော ဝေါ ပရိဂ္ဂဟော။}

\begin{jieshu}
    ဣတိ မောဂ္ဂလ္လာနေ ဗျာကရဏေ ဝုတ္တိယံ 
    
    သျာဒိကဏ္ဍော ဒုတိယော။
\end{jieshu}
\chapter{သမာသကဏ္ဍော တတိယော }
\markboth{မောဂ္ဂလ္လာနဗျာကရဏေ}{သမာသကဏ္ဍော တတိယော}

\sutta{308}{1}{သျာဒိ သျာဒိနေကတ္ထံ။}
\vutti{သျာဒျန္တံ သျာဒျန္တေန သဟေကတ္ထံ ဟောတီတိ ဣဒမဓိကတံ ဝေဒိတဗ္ဗံ၊ သော စ ဘိန္နတ္ထာနမေကတ္ထီဘာဝေါ သမာသောတိ ဝုစ္စတေ။}

\sutta{309}{2}{အသင်္ချံ ဝိဘတ္တိ-သမ္ပတ္တိ-သမီပ-သာကလျာ-ဘာဝ-ယထာ-ပစ္ဆာ-ယုဂပဒတ္ထေ။}
\vutti{အသင်္ချံ ၊ သျာဒျန္တံ ဝိဘတျာဒီနမတ္ထေ ဝတ္တမာနံ သျာဒျန္တေန သဟေကတ္ထံ ဘဝတိ၊ တတ္ထ ဝိဘတျတ္ထေ တာဝ ဣတ္ထီသု ကထာ ပဝတ္တာ အဓိတ္ထိ။ သမ္ပတ္တိ ဒွိဓာ အတ္တသမ္ပတ္တိ သမိဒ္ဓိ စ၊ သမ္ပန္နံ ဗြဟ္မံ သဗြဟ္မံ လိစ္ဆဝီနံ၊ သမိဒ္ဓိ ဘိက္ခာနံ သုဘိက္ခံ။ သမီပေ ကုမ္ဘဿ သမီပမုပကုမ္ဘံ။ သာကလျေသတိဏမဇ္ဈောဟရတိ၊ သာဂျဓီတေ။ အဘာဝေါ သမ္ဗန္ဓိဘေဒါ ဗဟုဝိဓော၊ တတြ ဣဒ္ဓါဘာဝေ-ဝိဂတာ ဣဒ္ဓိ သဒ္ဒိကာနံ ဒုဿဒ္ဒိကံ၊ အတ္ထာဘာဝေ-အဘာဝေါ မက္ခိကာနံ နိမ္မက္ခိကံ၊ အဟိက္ကမာဘာဝေ-အတိဂတာနိ တိဏာနိ နိတ္တိဏံ၊ သမ္ပတျာဘာဝေ-အတိဂတံ လဟုပါဝုရဏံ အတိလဟုပါဝုရဏံ၊ လဟုပါဝုရဏဿ နာယမုပဘောဂကာလောတိ အတ္ထော။ ယထာ ဧတ္ထာ-နေကဝိဓော၊ တတြ ယောဂ္ဂတာယံ-အနုရူပံ သုရူပေါဝါဟတိ၊ ဝိစ္ဆာယံ-အနွဒ္ဓမာသံ၊ အတ္ထာနတိဝတ္တိယံ-ယထာသတ္တိ၊ သဒိသတ္တေ၊ သဒိသော ကိခိယာ သကိခိ၊ အာနုပုဗ္ဗိယေ-အနုဇေဋ္ဌံ၊ ပစ္ဆာဒတ္ထေအနုရထံ၊ ယုဂပဒတ္ထေ-သစက္ကံ နိဓေဟိ။}

\sutta{310}{3}{ယထာ န တုလျေ။}
\vutti{ယထာသဒ္ဒေါ တုလျတ္ထေ ဝတ္တမာနော သျာဒျန္တေန သဟေကတ္ထော န ဘဝတိ၊ ယထာ ဒေဝဒတ္တော တထာ ယညဒတ္တော။}

\sutta{311}{4}{ယာဝါဝဓာရဏေ။}
\vutti{ယာဝသဒ္ဒေါ-ဝဓာရဏေ ဝတ္တမာနော သျာဒျန္တေန သဟေကတ္ထော ဘဝတိ၊ အဝဓာရဏ မေတ္တကတာ ပရိစ္ဆေဒေါ၊ ယာဝါမတ္တံ ဗြာဟ္မဏေ အာမန္တယ၊ ယာဝဇီဝံ၊ အဝဓာရဏေတိ ကိံ? ယာဝ ဒိန္နံ တာဝ ဘုတ္တံ၊ နာဝဓာရယာမိ ကိတ္တကံ မယာ ဘုတ္တန္တိ။}

\sutta{312}{5}{ပယျပါ-ဗဟိ-တိရော-ပုရေ-ပစ္ဆာ ဝါ ပဉ္စမျာ။}
\vutti{ပရိအာဒယော ပဉ္စမျန္တေန သဟေကတ္ထာ ဟောန္တိ ဝါ၊ ပရိပဗ္ဗတံ ဝဿိ ဒေဝေါ ပရိပဗ္ဗတာ၊ အပပဗ္ဗတံ ဝဿိ ဒေဝေါ အပပဗ္ဗတာ၊ အာပါဋလိပုတ္တံ ဝဿိ ဒေဝေါ အာပါဋလိပုတ္တာ၊ ဗဟိဂါမံ ဗဟိ ဂါမာ၊ တိရောပဗ္ဗတံ တိရောပဗ္ဗတာ၊ ပုရေဘတ္တံ ပုရေဘတ္တာ၊ ပစ္ဆာဘတ္တံ ပစ္ဆာဘတ္တာ၊ ဝေတာဓိကာရော။}

\sutta{313}{6}{သမီပါယာမေသွနု။}
\vutti{အနုသဒ္ဒေါ သာမီပျေ အာယာမေ စ ဝတ္တမာနော သျာဒျန္တေန သဟေကတ္ထော ဟောတိ ဝါ၊ အနုဝနမသနိ ဂတာ၊ အနုဂင်္ဂံ ဗာရာဏသီ၊ သမီပါယာမေသွီတိ ကိံ? ရက္ခမနုဝိဇ္ဇောတတေ ဝိဇ္ဇု။}

\sutta{314}{7}{တိဋ္ဌဂွာဒီနိ။}
\vutti{တိဋ္ဌဂုပ္ပဘုတီနိ ဧကတ္ထီဘာဝဝိသယေ နိပါတီယန္တေ၊ တိဋ္ဌန္တီ ဂါဝေါ ယသ္မိံ ကာလေ တိဋ္ဌဂု ကာလော၊ ဝဟဂ္ဂု ကာလော။ အာယတီဂဝံ၊ ခလေယဝံ၊ လူနယဝံ လူယမာနယဝမိစ္စာဒိ၊ စျန္တော ပေတ္ထ ကေသာ ကေသိ၊ ဒဏ္ဍာ ဒဏ္ဍိ၊ တထာ ဝေလာပ္ပဘာဝနတ္ထောပိ၊ ပါတော နဟာနံ ပါတရဟာနံ၊ သာယံ နယာနံ သာယနဟာနံ၊ ပါတကာလံ သာယကာလံ၊ ပါတမေဃံ သာယမေဃံ၊ ပါတမဂ္ဂံ သာယမဂ္ဂံ။}

\sutta{315}{8}{ဩရေ-ပရိ-ပတိ-ပါရေ-မဇ္ဈေ-ဟေဋ္ဌု-ဒ္ဓါ-ဓောန္တော ဝါ ဆဋ္ဌိယာ။}
\vutti{ဩရာဒယော သဒ္ဒါ ဆဋ္ဌိယန္တေန သဟေကတ္ထာ ဝါ ဟောန္တိ၊ ဧကာရန္တတ္တံ နိပါတနတော၊ ဩရေဂင်္ဂံ၊ ပရိသိခရံ၊ ပဋိသောတံ၊ ပါရေယမုနံ၊ မဇ္ဈေဂင်္ဂံ၊ ဟေဋ္ဌာပါသာဒံ၊ ဥဒ္ဓဂင်္ဂံ၊ အဓောဂင်္ဂံ၊ အန္ထောပါသာဒံ၊ ပုန ဝါဝိဓာနာ ‘ဂင်္ဂါဩရ’ မိစ္စာဒီပိ ဟောန္တိ။}

\sutta{316}{9}{တံ နပုံသကံ။}
\vutti{ယဒေတမတိက္ကန္တမေကတ္ထံ ၊ တံ နပုံသကလိင်္ဂံ ဝေဒိတဗ္ဗံ၊ တထာ စေဝေါဒါဟဋံ၊ ဝါ ကွစိ ဗဟုလာဓိကာရာ၊ ယထာပရိသံ ယထာပရိသာယ၊ သကာယ သကာယ ပရိသာယာတိ အတ္ထော။}

\sutta{317}{10}{အမာဒိ။}
\vutti{အမာဒိ သျာဒျန္တံ သျာဒျန္တေန သဟ ဗဟုလမေကတ္ထံ ဟောတိ ဂါမံ ဂတော ဂါမဂတော၊ မုဟုတ္တံ သုခံ မုဟုတ္တသုခံ၊ ဝုတ္တိယေဝေါပပဒသမာသေ ကုမ္ဘကာရော၊ သပါကော၊ တန္တဝါယော၊ ဝရာဟရော။ န္တမာနက္တဝန္တူတိ ဝါကျမေဝ၊ ဓမ္မံ သုဏန္တော၊ ဓမ္မံ သုဏမာနော၊ ဩဒနံ ဘုတ္တဝါ။
ရညာ ဟတော ရာဇဟတော၊ အသိနာ ဆိန္နော အသိစ္ဆိန္နော၊ ပိတုသဒိသော၊ ပိတုသမော၊ သုခသဟဂတံ၊ ဒဓိနာ ဥပသိတ္တံ ဘောဇနံ ဒဓိဘောဇနံ၊ ဂုဠေန မိဿော ဩဒနော ဂုဠောဒနော၊ ဝုတ္တိပဒေနေဝေါပသိတ္တာဒိကိရိယာယာချာပနတော နတ္ထာယုတ္တတ္ထတာ။ ကွစိ ဝုတ္တိယေဝ ဥရဂေါ၊ ပါဒပေါ။ ကွစိ ဝါကျမေဝ ဖရသုနာ ဆိန္နဝါ၊ ဒဿနေန ပဟာတဗ္ဗာ။
ဗုဒ္ဓဿ ဒေယျံ ဗုဒ္ဓဒေယျံ၊ ယူပါယ ဒါရု ယူပါဒါရု၊ ရဇနာယ ဒေါဏိ ရဇနဒေါဏိ။ ဣဓ န ဟောတိ သင်္ဃဿ ဒါတဗ္ဗံ။ ကထံ ‘ဧတဒတ္ထော ဧတဒတ္ထာ ဧတဒတ္ထ’န္တိ? အညပဒတ္ထေ ဘဝိဿတိ။
သဝရေဟိ ဘယံ သဝရဘယံ၊ ဂါမနိဂ္ဂတော၊ မေထုနာပေတော၊ ကွစိ ဝုတ္တိယေဝ ကမ္မဇံ၊ စိတ္တဇံ၊ ဣဓ န ဟောတိ ရုက္ခာ ပတိတော။
ရညော ပုရိသော ရာဇပုရိသော။ ဗဟုလာဓိကာရာ န္တမာနနိဒ္ဓါရိယပူရဏဘာဝတိတ္တတ္ထေဟိ န ဟောတိ-မမာနုကုဗ္ဗံ၊ မမာနုကုရုမာနော၊ ဂုန္နံ ကဏှာ သမ္ပန္နခီရတမာ၊ သိဿာနံ ပဉ္စမော၊ ပဋဿ သုက္ကတာ၊ ကွစိ ဟောတေဝ-ဝတ္တမာနသာမီပျံ၊ ကထံ ‘ဗြာဟ္မဏဿ သုက္ကာ ဒန္တာ’တိ? သာပေက္ခတာယ န ဟောတိ။ ဣဓ ပန ဟောတေဝ ‘စန္ဒနဂေါ၊ နဒိဃောသော၊ ကညာရူပံ၊ ကာယသမ္ဖဿော၊ ဖလရသော’တိ၊ ဖလာနံ တိတ္တော၊ ဖလာနမာသိတော၊ ဖသာနံ သုဟိတော။
ဗြာဟ္မဏဿ ဥစ္စံ ဂေဟန္တိ သာပေက္ခတာယ န ဟောတိ၊ ‘ရညော ပါဋလိပုတ္တကဿ ဓန’န္တိ ဓနသမ္ဗန္ဓေ ဆဋ္ဌီတိ ပါဋလိပုတ္တကေန သမ္ဗန္ဓာဘာဝါ န ဟေဿတိ၊ ‘ရညော ဂေါ စ အဿော စ ပုရိသော စာ’တိ ဘိန္နတ္ထတာယ ဝါကျမေဝ၊ ‘ရညော ဂဝါဿပုရိသာ ရာဇဂဝါဿပုရိသာ’တိ ဝုတ္တိ ဟောတေဝေကတ္တိဘာဝေ။
ဒါနေ သောဏ္ဍော ဒါနသောဏ္ဍော၊ ဓမ္မရတော၊ ဒါနာဘိရတော။ ကွစိ ဝုတ္တိယေဝ ကုစ္ဆိသယော၊ ထလဋ္ဌော၊ ပင်္ကဇံ၊ သရောရုဟံ။ ဣဓ န ဟောတိ ဘောဇနေ မတ္တညုတာ၊ ဣန္ဒြိယေသု ဂုတ္တဒွါရတာ၊ အာသနေ နိသိန္နော၊ အာသနေ နိသီဒိတဗ္ဗံ။}

\sutta{318}{11}{ဝိသေသနမေကတ္ထေန။}
\vutti{ဝိသေသနံ သျာန္တံ ဝိသေဿေန သျာဒျန္တေန သမာနာဓိကရဏေန သဟေကတ္ထံ ဟောတိ၊ နီလဉ္စ တံ ဥပ္ပလဉ္စေတိ နီလုပ္ပလံ၊ ဆိန္နဉ္စ တံ ပရုဠှဉ္စေတိ ဆိန္နပရဠှံ၊ သတ္ထီဝ သတ္ထီ၊ သတ္ထီ စ သာ သာမာ စေတိ သတ္ထိသာမာ၊ သီဟောဝ သီဟော၊ မုနိ စ သော သီဟော စေတိ မုနိသီဟော၊ သီလမေဝ ဓနံ သီလဓနံ။
ကွစိ ဝါကျမေဝ ပုဏ္ဏော မန္တာဏိပုတ္တော၊ စိတ္တော ဂဟပတိ။ ကွစိ ဝုတ္တိယေဝ ကဏှသပ္ပော၊ လောဟိတသာလိ၊ ဝိသေသနန္တိ ကိံ? တစ္ဆကော သပ္ပော၊ ဧကတ္ထေနေတိ ကိံ? ကာဠမှာ အညော။ ကထံ ‘ပတ္တဇီဝိကော၊ အာပန္နဇီဝိကော၊ မာသဇာတော’တိ? အညပဒတ္ထေ ဘဝိဿတိ။}

\sutta{319}{12}{နဉ။}
\vutti{နဉိစ္စေတံ သျာဒျန္တံ သျာဒျန္တေန သဟေကတ္ထံ ဟောတိ၊ န ဗြာဟ္မဏော အဗြာဟ္မဏော၊ ဗဟုလာဓိကာရတော အသမတ္ထတ္ထေဟိ၊ ကေဟိစိ ဟောတိ ‘အပုနဂေယျာ ဂါထာ၊ အနောကာသံ ကာရေတွာ၊ အမူလာ မူလံ ဂန္တွာ။ ဤသံကဠာရော၊ ဤသံပိင်္ဂလောတိ ‘သျာဒိ သျာဒိနေ’တိ သမာသော၊ ဝါကျမေဝ ဝါတိပ္ပသင်္ဂါဘာဝါ။}

\sutta{320}{13}{ကုပါဒယော နိစ္စမသျာဒိဝိဓိမှိ။}
\vutti{ကုသဒ္ဒေါ ပါဒယော စ သျာဒျန္တေန သဟေကတ္ထာ ဟောန္တိ နိစ္စံ သျာဒိဝိဓိဝိသယတော-ဉတ္ထ၊ ကုစ္ဆိတော ဗြာဟ္မဏော ကုဗြာဟ္မဏော၊ ဤသကံ ဥဏှံ ကဒုဏှံ၊ ပနာယကော၊ အဘိသေကော၊ ပကရိတွာ၊ ပကတံ၊ ဒုပ္ပုရိသော၊ ဒုက္ကဋံ၊ သုပုရိသော၊ သုကတံ၊ အဘိတ္ထုတံ၊ အတိတ္ထုတံ၊ အာကဠာရော၊ အာဗဒ္ဓေါ။}

\suttagana{321}{9}{ပါဒယော ဂတာဒျတ္ထေ ပဌမာယ။}
\suttagana{322}{10}{အစ္စာဒယော ကန္တာဒျတ္ထေ ဒုတိယာယ။}
\suttagana{323}{11}{အဝါဒယော ကုဋ္ဌာဒျတ္ထေ တတိယာယ။}
\suttagana{324}{12}{ပရိယာဒယော ဂိလာနာဒျတ္ထေ စတုတ္ထိယာ။}
\suttagana{325}{13}{နျာဒယော ကန္တာဒျတ္ထေ ပဉ္စမိယာ။}
\vutti{ပဂတော အာစရိယော ပါစရိယော၊ ပန္တေဝါသီ။
အတိက္ကန္တော မဉ္စမတိမဉ္စော၊ အတိမာလော။
အဝကုဋ္ဌံ ကောကိလာယ ဝနံ အဝကောကိလံ၊ အဝမယူရံ။
ပရိဂိလာနော အဇ္ဈေနာယ ပရိယဇ္ဈေနော။
နိက္ခန္တော ကောသမ္ဗိယာ နိက္ကောသမ္ပိ၊ အသျာဒိဝိဓိမှီတိ ကိံ? ရုက္ခံ ပတိ ဝိဇ္ဇောတတေ။}

\sutta{326}{14}{စီ ကြိယတ္ထေဟိ။}
\vutti{စီပ္ပစ္စယန္တော ကိရိယတ္ထေဟိ သျာဒျန္တေဟိ သဟေကတ္ထော ဟောတိ၊ မလိနီကရိယ။}

\sutta{327}{15}{ဘူသနာ-ဒရာ-နာဒရေသွ-လံ-သာ-သာ။}
\vutti{ဘူသနာဒိသွတ္ထေ သွလမာဒယော သဒ္ဒါ ကိရိယတ္ထေဟိ သျာဒျန္တေဟိ သဟေကတ္ထာ ဟောန္တိ၊ အလံကရိယ၊ သက္ကစ္စ၊ အသက္ကစ္စ။ ဘူသနာဒီသူတိ ကိံ? အလံဘုတွာ ဂတော၊ သက္ကတွာ ဂတော၊ အသက္ကတွာ ဂတော၊ ပရိယတ္တံ သောဘနမသောဘနန္တိ အတ္ထော။}

\sutta{328}{16}{အညေ စ။}
\vutti{အညေ စ သဒ္ဒါ ကိရိယတ္ထေဟိ သျာဒျန္တေဟိ သဟ ဗဟုလမေကတ္ထာ ဘဝန္တိ၊ ပုရောဘူယ၊ တိရောဘူယ၊ တိရောကရိယ၊ ဥရသိကရိယ၊ မနသိကရိယ၊ မဇ္ဈေကရိယ၊ တုဏှီဘူယ။}

\sutta{329}{17}{ဝါနေကညတ္ထေ။}
\vutti{အနေကံ သျာဒျန္တမညဿ ပဒဿတ္ထေ ဧကတ္ထံ ဝါ ဟောတိ၊ ဗဟူနိ ဓနာနိ ယဿ သော ဗဟုဓနော၊ လမ္ဗာ ကဏ္ဍာ ယဿ သော လမ္ဗကဏ္ဍော၊ ဝဇိရံ ပါဏိမှိ ယဿ သောယံ ဝဇိရပါဏိ၊ မတ္တာ ဗဟဝေါ မာတင်္ဂါ ဧတ္ထ မတ္တဗဟုမာတင်္ဂံ ဝနံ၊ အာရုဠှော ဝါနရော ယံ ရုက္ခံ သော အာရုဠှဝါနရော၊ ဇိတာနိ ဣန္ဒြိယာနိ ယေန သော ဇိတိန္ဒြိယော၊ ဒိန္နံ ဘောဇနံ ယဿ သော ဒိန္နဘောဇနော၊ အပဂတံ ကာဠကံ ယသ္မာ ပဋာ သော-ယမပဂတကာဠကော၊ ဥပဂတာ ဒသ ယေသံ တေ ဥပဒသာ၊ အာသန္နဒသာ၊ အဒူရဒသာ၊ အဓိကဒသာ၊ တယော ဒသ ပရိမာဏမေသံ တိဒသာ၊ ကထံ ဒသသဒ္ဒေါ သင်္ချာနေ ဝတ္တတေ? ပရိမာဏသဒ္ဒသန္နိဓာနာ၊ ယထာ ပဉ္စ ပရိမာဏမေသံ ပဉ္စကာ သကုနာတိ၊ ဒွေ ဝါ တယော ဝါ ပရိမာဏမေသံ ဒွတ္တယော ဝါသဒ္ဒတ္ထေ ဝါ ဒွေ ဝါ တယော ဝါ ဒွတ္တယော။
ဒက္ခိဏဿာ စ ပုဗ္ဗဿာ စ ဒိသာယ ယဒန္တရာဠံ ဒက္ခိဏပုဗ္ဗာ ဒိသာ၊ ဒက္ခိဏာ စ သာ ပုဗ္ဗာ စာတိ ဝါ၊ သဟ ပုတ္တေနာဂတော သပုတ္တော၊ သလောမကော ဝိဇ္ဇမာနလောမကောတိ အတ္ထော၊ ဧဝံ သပက္ခကော၊ အတ္ထီ ခီရာ ဗြာဟ္မဏီတိ အတ္ထိသဒ္ဒေါ ဝိဇ္ဇမာနတ္ထေ နိပါတော၊ ကွစိ ဂတတ္ထတာယ ပဒန္တရာနမပ္ပယောဂေါ၊ ကဏ္ဌဋ္ဌာ ကာဠာ အဿ ကဏ္ဌေကာဠော၊ ဩဋ္ဌဿ မုခမိဝ မုခမဿ ဩဋ္ဌမုခေါ၊ ကေသသင်္ဃာတော စူဠာ အဿ ကေသစူဠော၊ သုဝဏ္ဏဝိကာရော အလင်္ကာရော အဿ သုဝဏ္ဏာလင်္ကာရော၊ ပပတိတံ ပဏ္ဏမဿ ပပတိတပဏ္ဏော၊ ပပဏ္ဏော၊ အဝိဇ္ဇမာနာ ပုတ္တာ အဿ အဝိဇ္ဇမာနပုတ္တော၊ န သန္တိ ပုတ္တာ အဿ အပုတ္တေ၊ ကွစိ န ဟောတိ ပဉ္စ ဘုတ္တဝန္တော အဿ ဘာတုနော ပုတ္တော အဿ အတ္ထီတိ ဗဟုလာဓိကာရတော။}

\sutta{330}{18}{တတ္ထ ဂဟေတွာ တေန ပဟရိတွာ ယုဒ္ဓေ သရူပံ။}
\vutti{သတ္တမျန္တံ တတိယန္တဉ္စ သရူပမနေကံ တတ္ထ ဂဟေတွာ တေန ပဟရိတွာ ယုဒ္ဓေ-ညပဒတ္ထေ ဧကတ္ထံ ဝါ ဟောတိ၊ ကေသေသု စ ကေသေသု စ ဂဟေတွာ ယုဒ္ဓံ ပဝတ္တံ ကေသာကေသိ၊ ဒဏ္ဍေဟိ စ ဒဏ္ဍေဟိ စ ပဟရိတွာ ယုဒ္ဓံ ပဝတ္တံ ဒဏ္ဍာဒဏ္ဍိ၊ မုဋ္ဌာမုဋ္ဌိ၊ “စိ ဝီတိယာရေ” (၃-၅၁) တိ စိ သမာသန္တော၊ “စိသ္မိံ” (၃.၆၆) တိ အကာရော။ တတ္ထ တေနေတိ ကိံ? ကာယဉ္စ ကာယဉ္စ ဂဟေတွာ ယုဒ္ဓံ ပဝတ္တံ။ ဂဟေတွာ ပဟရိတွာတိ ကိံ? ရထေ စ ရထေ စ ဌတွာ ယုဒ္ဓံ ပဝတ္တိ။ ယုဒ္ဓေတိ ကိံ? ဟတ္ထေ စ ဟတ္ထေ စ ဂဟေတွာ သချံ ပဝတ္တံ။ သရူပန္တိ ကိံ? ဒဏ္ဍေဟိ စ မုသလေဟိ စ ပဟရိတွာ ယုဒ္ဓ ပဝတ္တံ။}

\sutta{331}{19}{စတ္ထေ။}
\vutti{အနေကံ သျာဒျန္တံ စတ္ထေ ဧကတ္ထံ ဝါ ဘဝတိ။ သမုစ္စယောနွာစယော ဣတရီတရယောဂေါ သမာဟာရော စ စ သဒ္ဒတ္ထာ၊ တတ္ထ သမုစ္စယာနွာစယေသု နေကတ္ထီဘာဝေါ သမ္ဘဝတိ၊ တေသု ဟိ သမုစ္စယော အညမညနိရပေက္ခာ နမတ္တပ္ပဓာနာနံ ကတ္ထစိ ကိရိယာဝိသေသေ စီယမာနတာ၊ ယထာ ‘ဓဝေ စ ခဒိရေ စ ပလာသေ စ ဆိန္ဒာ’တိ။ အနွာစယော စ ယတ္ထေကော ပဓာနဘာဝေန ဝိဓီယတေ အပရော စ ဂုဏဘာဝေန၊ ယထာ ‘ဘိက္ခဉ္စရ ဂါဝေါ စာနယေ’တိ။ ဣတရဒွယေ တု သမ္ဘဝတိ၊ တေသု ဟိ အညမညသာပေက္ခာနမဝယဝဘဒါနုဂတော ဣတရီတရယောဂေါ၊ ယထာ ‘သာရိပုတ္တမောဂ္ဂလ္လာနာ’တိ၊ အဿာဝယဝပ္ပဓာနတ္တာ ဗဟုဝစနမေဝ။ အညမညသာပေက္ခာနမေဝ တိရောဟိတာဝယဝဘေဒေါ သမုဒါယပ္ပဓာနော သမာဟာရော၊ ယထာ ‘ဆတ္တုပါဟန’န္တိ၊ အဿ ပန သမုဒါယပ္ပဓာနတ္တာ ဧကဝစနမေဝ။
တေ စ သမာဟာရီတရီတရယောဂါ ဗဟုလံ ဝိဓာနာ နိယတဝိသယာယေဝ ဟောန္တိ၊ တတြာယံ ဝိသယဝိဘာဂေါ နိရုတ္ထိပိဋကာဂတော-ပါဏိတူရိယယောဂ္ဂသေနင်္ဂါနံ၊ နိစ္စဝေရီနံ၊ သင်္ချာပရိမာဏ-သညာနံ၊ ခုဒ္ဒဇန္တုကာနံ၊ ပစနစဏ္ဍာလာနံ၊ စရဏသာဓာရဏာနံ၊ ဧကဇ္ဈာယနပါဝစနာနံ၊ လိင်္ဂဝိသေသာနံ၊ ဝိဝိဓဝိရုဒ္ဓါနံ ဒိသာနံ၊ နဒီနဉ္စ နိစ္စံ သမာဟာရေကတ္တံ ဘဝတိ၊ တိဏရုက္ခပသုသကုနဓနဓညဗျဉ္ဇနဇနပ္ပဒါနံ ဝါ၊ အညေသမိတရီတရယောဂေါဝ။
ပါဏျင်္ဂါနံ-စက္ခုသောတံ၊ မုခနာသိကံ၊ ဟနုဂီဝံ၊ ဆဝိမံသလောဟိတံ၊ နာမရူပံ၊ ဇရာမရဏံ။ တုရိယင်္ဂါနံ-အလသတာလမ္ဗရံ၊ မုရဇဂေါမုခံ၊ သင်္ခဒေဏ္ဍိမံ၊ မဒ္ဒဝိကပါဏဝိကံ၊ ဂီတဝါဒိတံ၊ သမ္မတာလံ။ ယောဂ္ဂင်္ဂါနံ ဖာလပါစနံ၊ ယုဂနင်္ဂလံ။ သေနင်္ဂါနံ-အသိသတ္တိတောမရပိဏ္ဍံ၊ အသိစမ္မံ၊ ဓနုကလာပံ၊ ပဟရဏာဝရဏံ။ နိစ္စဝေရီနံ-အဟိနကုလံ၊ ဗီဠာလမူသိကံ၊ ကာကောလူကံ၊ နာဂသုပဏ္ဏံ။ သင်္ချာပရိမာဏ သညာနံ-ဧကကဒုကံ၊ ဒုကတိကံ၊ တိကစတုက္ကံ၊ စတုက္ကပဉ္စကံ၊ ဒသေကာဒသကံ။ ခုဒ္ဒဇန္တုကာနံ ကီဋပဋင်္ဂံ၊ ကုန္ထကိပိလ္လိကံ၊ ဍံသမကသံ၊ မက္ခိကကိပိလ္လိကံ။ ပစနစဏ္ဍာလာနံ-ဩရဗ္ဘိကသူကရိကံ၊ သာကုန္တိ ကမာဂဝိကံ၊ သပါကစဏ္ဍာလံ၊ ဝေနရထကာရံ၊ ပုက္ကုသ ဆဝဍာဟကံ။ စရဏသာဓာရဏာနံ-အတိသဘာရဒွါဇံ၊ ကဌကလာပံ၊ သီလပညာဏံ၊ သမထဝိပဿနံ၊ ဝိဇ္ဇာစရဏံ။ ဧကဇ္ဈာယနပါဝစနာနံ ဒီဃမဇ္ဈိမံ၊ ဧကုတ္တရသံယုတ္တကံ၊ ခန္ဓကဝိဘင်္ဂံ။ လိင်္ဂဝိသေသာနံ-ဣတ္ထိပုခံ၊ ဒါသိဒါသံ၊ စီဝရပိဏ္ဍပါတသေနာသနဂိလာနပ္ပစ္စယဘေသဇ္ဇပရိက္ခာရံ၊ တိဏကဋ္ဌသာခါပလာသံ၊ ‘လာဘီ ဟောတိ စီဝရပိဏ္ဍပါတသေနာ- သနဂိလာနပ္ပစ္စယဘေသဇ္ဇပရိက္ခာရာန’န္တိပိ ဒိဿတိ။ ဝိဝိဓဝိရုဒ္ဓါနံ ကုသလာကုသလံ ၊ သာဝဇ္ဇာနဝဇ္ဇံ၊ ဟီနပ္ပဏီတံ၊ ကဏှသုက္ကံ၊ ဆေကပါပကံ။ ဒိသာနံ ပုဗ္ဗာပရံ၊ ဒက္ခိဏုတ္တရံ၊ ပုဗ္ဗဒက္ခိဏံ၊ ပုဗ္ဗုတ္တရံ၊ အဓရုတ္တရံ၊ အပရဒက္ခိဏံ၊ အပရုတ္တရံ။ နဒီနံ-ဂင်္ဂါယမုနံ၊ မဟိသရဘု။
တိဏဝိသေသာနံ-ကာသကုသံ ကာသကုသာ၊ ဥသီရဗီရဏံ ဥသီရဗီရဏာ၊ မုဉ္ဇပဗ္ဗဇံ မုဉ္ဇပဗ္ဗဇံ မုဉ္ဇပဗ္ဗဇာ။ ရုက္ခဝိသေသာနံ ခဒိရပလာသံ ခဒိရပလာသာ၊ ဝေါဿကဏ္ဍံ ဓဝါဿကဏ္ဍာ၊ ပိလက္ခနိဂြောဓံ ပိလက္ခနိဂြောဓာ၊ အဿတ္ထကပိတ္ထနံ အဿတ္ထကပိတ္ထနာ၊ သာကသာလံ သာကသာလာ။ ပသုဝိသေသာနံ-ဂဇဂဝဇံ ဂဇဂဝဇာ၊ ဂေါမဟိသံ ဂေါမဟိသာ၊ ဧဏေယျဂေါမဟိသံ ဧဏေယျဂေါမဟိသာ၊ ဧဏေယျဝရာဟံ ဧဏေယျဝရာဟာ၊ အဇေဠကံ အဇေဠကာ၊ ကုက္ကုရသူကရံ ကုက္ကုရသူကရာ၊ ဟတ္ထိဂဝါဿဝဠဝံ ဟတ္ထိဂဝါဿဝဠဝါ။ သကုနဝိသေသာနံ-ဟံသဗလာဝံ ဟံသဗလာဝါ၊ ကာရဏ္ဍဝစက္ကဝါကံ ကာရဏ္ဍဝစက္ကဝါကာ၊ ဗကဗလာကံ ဗကဗလာကာ။ ဓနာနံ-ဟိရညသုဝဏ္ဏံ ဟိရညသုဝဏ္ဏာ၊ မဏိသင်္ခမုတ္တာဝေဠုရိယံ မဏိသင်္ခမုတ္တာဝေဠုရိယာ၊ ဇာတရူပရဇတံ ဇာတရူပရဇတာ။ ဓညာနံ-သာလိယဝကံ သာလိယဝကာ၊ တိလမုဂ္ဂမာသံ တိလမုဂ္ဂမာသာ၊ နိပ္ဖာဝကုလတ္ထံ နိပ္ဖာဝကုလတ္ထာ။ ဗျဉ္ဇနာနံ-သာကသုဝံ သာကသုဝါ၊ ဂဗျမာဟိသံ ဂဗျမာဟိသာ၊ ဧဏေယျဝါရာဟံ ဧဏေယျဝါရာဟာ၊ မိဂမာယူရံ မိဂမာယူရာ။ ဇနပဒါနံ-ကာသိကောသလံ ကာသိကောသလာ၊ ဝဇ္ဇိမလ္လံ ဝဇ္ဇိမလ္လာ၊ စေတိဝံသံ စေတိဝံသာ၊ မစ္ဆသူရသေနံ မစ္ဆသူရသေနာ၊ ကုရုပဉ္စာလံ ကုရုပဉ္စာလာ။ ဣတရီတရယောဂေါ ယထာ-စန္ဒိမသူရိယာ၊ သမဏဗြာဟ္မဏာ မာတာပိတရော ဣစ္စာဒိ။
ဧတသ္မိံ ဧကတ္ထီဘာဝကဏ္ဍေ ယံ ဝုတ္တံ ပုဗ္ဗံ၊ တဒေဝ ပုဗ္ဗံ နိပတတိ ကမာတိက္ကမေ ပယောဇနာဘာဝါ။ ကွစိ ဝိပလ္လာသောပိ ဟောတိ ဗဟုလာဓိကာရတော၊ ဒန္တာနံ ရာဇာ ရာဇဒန္တော၊ ကတ္ထစိ ကမံ ပစ္စာနာဒရာ ပုဗ္ဗကာလဿာပိ ပရနိပါတော၊ လိတ္တဝါသိတော၊ နဂ္ဂမုသိတော၊ သိတ္တသမ္မဋ္ဌော၊ ဘဋ္ဌလုဉ္စိတော။ စတ္ထေ ယဒေကတ္ထံ တတ္ထ ကေစိ ပုဗ္ဗပဒံ ဗဟုဓာ နိယမေန္တိ၊ တဒိဟ ဗျဘိစာရဒဿာန န ဝုတ္တန္တိ ဒဋ္ဌဗ္ဗံ။}

\sutta{332}{20}{သမာဟာရေ နပုံသကံ။}
\vutti{စတ္ထေ သမာဟာရေ ယဒေကတ္ထံ၊ တံ နပုံသကလိင်္ဂံ ဘဝတိ၊ တထာစေဝေါဒါဟဋံ၊ ကတ္ထစိ န ဟောတိ ‘သဘာပရိသာယာ’တိ ဉာပကာ၊ အာဓိပစ္စပရိဝါရော၊ ဆန္ဒပါရိသုဒ္ဓိ၊ ပဋိသန္ဓိပဝတ္တိယံ။}

\sutta{333}{21}{သင်္ချာဒိ။}
\vutti{ဧကတ္ထေ သမာဟာရေ သင်္ချာဒိ နပုံသကလိင်္ဂံ ဘဝတိ၊ ပဉ္စဂဝံ၊ စတုပ္ပထံ၊ သမာဟာရဿေကတ္တာ ဧကဝစနမေဝ ဟောတိ၊ သမာဟာရေတွေဝ ပဉ္စကာပါလော ပူဝေါ၊ တိပုတ္တော။}

\sutta{334}{22}{ကွစေကတ္တဉ္စ ဆဋ္ဌိယာ။}
\vutti{ဆဋ္ဌိယေကတ္ထေ ကွစိ နပုံသကတ္တံ ဟောတေကတ္ထဉ္စ၊ သလဘာနံ ဆာယာ သလဘစ္ဆာယံ၊ ဧဝံ သကုန္တာနံ ဆာယာ သကုန္တစ္ဆာယံ၊ ပါသာဒစ္ဆာယံ ပါသာဒစ္ဆာယာ၊ ဃရစ္ဆာယံ ဃရစ္ဆာယာ၊ အမနုဿသဘာယ နပုံသကေကတ္တံ ဘဝတိ ဗြဟ္မသဘံ၊ ဒေဝသဘံ၊ ဣန္ဒသဘံ၊ ယက္ခသဘံ၊ သရဘသဘံ၊ မနုဿသဘာယံ ပန ခတ္တိယသဘာ၊ ရာဇသဘာ ဣစ္စေဝမာဒိ၊ ကွစီတိ ကိံ ရာဇပုရိသော။}

\sutta{335}{23}{သျာဒီသု ရဿော။}
\vutti{နပုံသကေ ဝတ္တမာနဿ ရဿော ဟောတိ သျာဒီသု။ သလဘစ္ဆာယံ၊ သျာဒီသူတိ ကိံ? သလဘစ္ဆာယေ။}

\sutta{336}{24}{ဃပဿန္တဿာပ္ပဓာနဿ။}
\vutti{အန္တဘူတဿ အပ္ပဓာနဿ ဃပဿ သျာဒီသု ရဿော ဟောတိ။ ဗဟုမာလော ပေါသော၊ နိက္ကောသမ္ဗိ၊ အတိဝါမောရု၊ အန္တဿာတိ ကိံ? ရာဇာ ကညာပိယော၊ အပ္ပဓာနဿာတိ ကိံ? ရာဇကုမာရီ ဗြဟ္မဗန္ဓူ။}

\sutta{337}{25}{ဂေါဿု။}
\vutti{အန္တဘူတဿ အပ္ပဓာနဿ ဂေါဿ သျာဒီသု ဥ ဟောတိ။ စိတ္တဂု၊ အပ္ပဓာနဿာတွေဝ? သုဂေါ၊ အန္တဿာတွေဝ? ဂေါကုလံ။}

\sutta{338}{26}{ဣတ္ထိယမတွာ။}
\vutti{ဣတ္ထိယံ ဝတ္တမာနတော အကာရန္တတော နာမသ္မာ အာပစ္စယော ဟောတိ။ ဓမ္မဒိန္နာ။}

\sutta{339}{27}{နဒါဒိတော ဝီ။}
\vutti{နဒါဒီဟိ ဣတ္ထိယံ ဝီပ္ပစ္စယော ဟောတိ။ နဒီ၊ မဟီ၊ ကုမာရီ၊ တရုဏီ၊ ဝါရုဏီ၊ ဂေါတမီ။}

\suttagana{340}{14}{ဂေါတော ဝါ။}

\sutta{341}{28}{ယက္ခာဒိတွိနီ စ။}
\vutti{ယက္ခာဒိတော ဣတ္ထိယံ ဣနီ ဟောတိ ဝီစ။ ယက္ခိနီ ယက္ခီ၊ နာဂိနီ နာဂီ၊ သီဟိနီ သီဟီ။}

\sutta{342}{29}{အာရာမိကာဒီဟိ။}
\vutti{အာရာမိကာဒိတော ဣနီ ဟောတိတ္ထိယံ။ အာရာမိကိနီ၊ အနန္တရာယိကိနီ၊ ရာဇိနီ}

\suttagana{343}{15}{သညာယံ မာနုသော။}
\vutti{မာနုသိနီ၊ အညတြ မာနုသီ။}

\sutta{344}{30}{ယုဝဏ္ဏေဟိ နီ။}
\vutti{ဣတ္ထိယမိဝဏ္ဏုဝဏ္ဏန္တေဟိ နီ ဟောတိ ဗဟုလံ။ သဒါပယတပါဏိနီ၊ ဒဏ္ဍိနီ၊ ဘိက္ခုနီ၊ ခတ္တဗန္ဓုနီ၊ ပရစိတ္တဝိဒုနီ၊ မာတုအာဒိတော ကသ္မာ န ဟောတိ? ဣတ္ထိပ္ပစ္စယံ ဝိနာပိ ဣတ္ထတ္တာဘိဓာနတော။}

\sutta{345}{31}{က္တိမှာညတ္ထေ။}
\vutti{က္တိမှာညတ္ထေယေဝ ဣတ္ထိယံ နီ ဟောတိ ဗဟုလံ။ သာဟံ အဟိံ သာရတိနီ၊ တဿာ မုဋ္ဌဿတိနိယာ၊ သာ ဂါဝီ ဝစ္ဆဂိဒ္ဓိနီ၊ အညတ္ထေတိ ကိံ? ဓမ္မရတိ။}

\sutta{346}{32}{ဃရဏျာဒယော။}
\vutti{ဃရဏိပ္ပဘုတယော နီပ္ပစ္စန္တာသာဓဝေါ ဘဝန္တိ။ ဃရဏီ၊ ပေါက္ခရဏီ၊ ဤဿ-တ္တံ နိပါတနာ၊}

\suttagana{347}{16}{အာစရိယာ ဝါ ယ-လောပေါ စ။}
\vutti{အာစရိနီ၊ အာစရိယာ။}

\sutta{348}{33}{မာတုလာဒိတွာနီ ဘရိယာယံ။}
\vutti{မာတုလာဒိတော ဘရိယာယမာနီ ဟောတိ။ မာတုလာနီ၊ ဝါရုဏာနီ၊ ဂါဟပတာနီ၊ အာစရိယာနီ၊}

\suttagana{349}{17}{အဘရိယာယံ ခတ္တိယာ ဝါ။}
\vutti{ခတ္တိယာနီ ခတ္တိယာ၊ နဒါဒိပါဌာ ဘရိယာယန္တု ခတ္တိယီ။}

\suttagana{350}{18}{ပုန္နာမသ္မာ ယောဂါ အပါလကန္တာ။}

\sutta{351}{34}{ဥပမာ-သံဟိတ-သဟိတ-သညတ-သဟ-သဖ-ဝါမ-\\လက္ခဏာဒိတူ-ရုတူ။}
\vutti{ဦရုသဒ္ဒါ ဥပမာနာဒိပုဗ္ဗာ ဣတ္တိယမူ ဟောတိ။ ကရဘောရူ၊ သံဟိတောရူ၊ သဟိတောရူ၊ သညတောရူ၊ သဟောရူ၊ သဖောရူ၊ ဝါမောရူ၊ လက္ခဏောရူ၊ ဦတိယောဂဝိဘာဂါ ဦ ဗြဟ္မဗန္ဓူ။}

\sutta{352}{35}{ယုဝါ တိ။}
\vutti{ယုဝသဒ္ဒတော တိ ဟောတိတ္ထိယံ။ ယုဝတိ။}

\sutta{353}{36}{န္တန္တူနံ ဝီမှိတော ဝါ။}
\vutti{ဝီမှိ န္တန္တူနံ တော ဝါ ဟောတိ။ ဂစ္ဆတီ ဂစ္ဆန္တီ၊ သီလဝတီ သီလဝန္တီ။}

\sutta{354}{37}{ဘဝတော ဘောတော။}
\vutti{ဝီမှိ ဘဝတော ဘောတာဒေသော ဟောတိ ဝါ။ ဘောတီ ဘဝန္တီ။}

\sutta{355}{38}{ဂေါဿာဝင်။}
\vutti{ဂေါသဒ္ဒဿ ဝီမှာဝင် ဟောတိ။ ဂါဝီ။}

\sutta{356}{39}{ပုထုဿ ပထဝပုထဝါ။}
\vutti{ဝီမှိ ပုထုဿ ပထဝပုထဝါ ဟောန္တိ။ ပထဝီ၊ ပုထဝီ၊ ဌေ ပထဝီ။}

\sutta{357}{40}{သမာသန္တွ။}
\vutti{သမာသန္တွ ဣတိ စာဓိကရီယတိ။}

\sutta{358}{41}{ပါပါဒီဟိ ဘူမိယာ။}
\vutti{ပါပါဒီဟီ ပရာ ယာ ဘူမိ တဿာ သမာသန္တော အ ဟောတိ။ ပါပဘူမံ၊ ဇာတိဘူမံ။}

\sutta{359}{42}{သင်္ချာဟိ။}
\vutti{သင်္ချာဟိ ပရာ ယာ ဘူမိ တဿာ သမာသန္တော အ ဟောတိ။ ဒွိဘူမံ၊ တိဘူမံ။}

\sutta{360}{43}{နဒီဂေါဒါဝရီနံ။}
\vutti{သင်္ချာဟိ ပရာသံ နဒီဂေါဒါဝရီနံ သမာသန္တော အ ဟောတိ၊ ပဉ္စနဒံ၊ သတ္တဂေါဒါဝရံ၊ သင်္ချာဟိတွေဝ? မဟာနဒီ၊ နဒီဂေါဒါဝရီနန္တိ ကိံ? ဒသိတ္ထိ။}

\sutta{361}{44}{အသင်္ချေဟိ စာင်္ဂုလျာနညာသင်္ချတ္ထေသု။}
\vutti{အသင်္ချေဟိ သင်္ချာဟိ စ ပရာယ အင်္ဂုလျာ သမာသန္တော အ ဟောတိ နော စေ အညပဒတ္ထေ အသင်္ချတ္ထေ စ သမာသော ဝတ္တတေ။ နိဂ္ဂတမင်္ဂုလီဟိ နိရင်္ဂုလံ၊ အစ္စင်္ဂုလံ၊ ဒွေ အင်္ဂုလိယော သမာဟဋာ ဒွင်္ဂုလံ၊ အနညာသင်္ချတ္ထေသူတိ ကိံ? ပဉ္စင်္ဂုလိ ဟတ္ထော၊ ဥပင်္ဂုလိ၊ ကထံ ‘ဒွေ အင်္ဂုလီမာနမဿာတိ ဒွင်္ဂုလ’န္တိ? နာတြ သမာသောညပဒတ္ထေ ဝိဟိတော မတ္တာဒီနံ လောပေ ကတေ တတ္ထ ဝတ္တတေ။ အင်္ဂုလသဒ္ဒေါ ဝါ ပမာဏဝါစိ သဒ္ဒန္တရံ၊ ယထာ ‘သေနင်္ဂုလပ္ပမာဏေန အင်္ဂုလာနံ သတံ ပုဏ္ဏံ စတုဒ္ဒသ ဝါ အင်္ဂုလာနီ’တိ။}

\sutta{362}{45}{ဒီဃာဟောဝဿေကဒေသေဟိ စ ရတ္တျာ။}
\vutti{ဒီဃာဒီဟိ အသင်္ချေဟိ သင်္ချာဟိ စ ပရမသ္မာ ရတ္တိယာ သမာသန္တော အ ဟောတိ။ ဒီဃရတ္တံ၊ အဟောရတ္တံ (တ္တော)၊ ဝဿာရတ္တံ (တ္တော)၊ ပုဗ္ဗရတ္တံ၊ အပရရတ္တံ၊ အဍ္ဎုရတ္တံ၊ အတိက္ကန္တော ရတ္တိံ အတိရတ္တော၊ ဒွေရတ္တီ သမာဟဋာ ဒိရတ္တံ (တ္တော)၊ ဝါ ကွစိ ဗဟုလာဓိကာရာ ဧကရတ္တံ (တ္တော)၊ ဧကရတ္တိ၊ အနညာသင်္ချတ္ထေသုတွေဝ? ဒီဃရတ္တိဟေမန္တော၊ ဥပရတ္တိ၊ ကွစိ ဟောတေဝ ဗဟုလံ ဝိဓာနာယထာရတ္တံ။}

\sutta{363}{46}{ဂေါတွစတ္ထေ စာလောပေ။}
\vutti{ဂေါသဒ္ဒါ အလောပဝိသယာ သမာသန္တော အ ဟောတိ န စေ စတ္ထေ သမာသော အညပဒတ္ထေ အသင်္ချတ္ထေ စ၊ ရာဇဂဝေါ၊ ပရမဂဝေါ၊ ပဉ္စဂဝဓနော၊ ဒသဂဝံ၊ အလောပေတိ ကိံ? ပဉ္စဟိ ဂေါဟိ ကီတော ပဉ္စဂု၊ အစတ္ထေတိ ကိ? အဇဿဂါဝေါ၊ အနညာသင်္ချတ္ထေသုတွေဝ? စိတ္တဂု၊ ဥပဂု။}

\sutta{364}{47}{ရတ္တိန္ဒိဝဒါရဂဝစတုရဿာ။}
\vutti{ဧတေ သဒ္ဒါ အအန္တာ နိပစ္စန္တေ။ ရတ္တော စ ဒိဝါ စ ရတ္တိန္ဒိဝံ၊ ရတ္တိ စ ဒိဝါ စ ရတ္တိန္ဒိဝံ၊ ဒါရာ စ ဂါဝေါ စ ဒါရဂဝံ၊ စတဿော အဿိယော အဿ စတုရဿော။}

\sutta{365}{48}{အာယာမေနုဂဝံ။}
\vutti{အနုဂဝန္တိ နိပစ္စတေ အာယာမေ ဂမျမာနေ။ အနုဂဝံ သကဋံ၊ အာယာမေတိ ကိံ? ဂုန္နံ ပဉ္ဆာ အနုဂု။}

\sutta{366}{49}{အက္ခိသ္မာညတ္ထေ။}
\vutti{အက္ခိသ္မာ သမာသန္တော အ ဟောတိ အညတ္ထေ စေ သမာဓသာ။ ဝိသာလက္ခော၊ ဝိသာလက္ခီ။}

\sutta{367}{50}{ဒါရုမျင်္ဂုလျာ။}
\vutti{အင်္ဂုလန္တာ အညပဒတ္ထေ ဒါရုမှိ သမာသန္တော အ ဟောတိ။ ဒွင်္ဂုလံဒါရု၊ ပဉ္စင်္ဂုလံ၊ အင်္ဂုလိသဒိသာဝယဝံ ဓညာဒီနံ ဝိက္ခေပကံ ဒါရုံ ဝုစ္စတေ၊ ပမာဏေ တု ပုဗ္ဗေ ဝိယ သိဒ္ဓံ သခရာဇသဒ္ဒါ အကာရန္တာဝ၊ သိဿောပိ န ဒိဿတိ၊ ဂါဏ္ဍီ ဝဓနွာတိ ပကတန္တရေန သိဒ္ဓံ။}

\sutta{368}{51}{စိ ဝီတိဟာရေ။}
\vutti{ဩဃာဗျတိဟာရေ ဂမျမာနေ အညပဒတ္ထေ ဝတ္တမာနတော စိ ဟောတိ။ ကေသာကေသိ ဒဏ္ဍာဒဏ္ဍိ၊ စကာရော “စိသ္မိ”န္တိ (၃.၆၆) ဝိသေသနတ္ထော၊ သုဂန္ဓိ၊ ဒုဂ္ဂန္တီတိ ပယောဂေါ န ဒိဿတေ။}

\sutta{369}{52}{လ္တိ-တ္ထိ-ယူဟိ ကော။}
\vutti{လ္တုပ္ပစ္စယန္တေဟိ၊ ဣတ္ထိယမီကာရူကာရန္တေဟိ စ ဗဟုလံ ကပ္ပစ္စယော ဟောတိ အညပဒတ္ထေ။ ဗဟုကတ္တုကော၊ ဗဟုကုမာရိကော၊ ဗဟုဗြဟ္မဗန္ဓုကော၊ ဗဟုလံတွေဝ? သုဗ္ဘူ။}

\sutta{370}{53}{ဝါညတော။}
\vutti{အညေဟိ အညပဒတ္ထေ ကော ဝါ ဗဟုလံ ဟောတိ။ ဗဟုမာလကော၊ ဗဟုမာလော။}

\sutta{371}{54}{ဥတ္တရပဒေ။}
\vutti{ဧတမဓိကတံ ဝေဒိတဗ္ဗံ။}

\sutta{372}{55}{ဣမဿိဒံ။}
\vutti{ဥတ္တရပဒေ ပရတော ဣမဿ ဣဒံ ဟောတိ။ ဣဒမဋ္ဌိတာ၊ ဣဒပ္ပစ္စယတာ၊ နိဂ္ဂဟီတလောပေါ ပဿ စ ဒွိဘာဝေါ။}

\sutta{373}{56}{ပုံ ပုမဿ ဝါ။}
\vutti{ပုမဿ ပုံ ဟောတုတ္တရပဒေ ဝိဘာသာ။ ပုလ္လိင်္ဂံ၊ ပုမလိင်္ဂံ။}

\sutta{374}{57}{ဋန္တန္တူနံ။}
\vutti{ဧသံ ဋ ဟောတုတ္တရပဒေ ကွစိ ဝါ။ ဘဝမ္ပတိဋ္ဌာမယံ၊ ဘဂဝံမူလကာ နော ဓမ္မာ၊ ဗဟုလာဓိကာရာ တရာဒီသု စ ပဂေဝ မဟတ္တရီ၊ ရတ္တညုမဟတ္တံ။}

\sutta{375}{58}{အ။}
\vutti{ဧသံ အ ဟောတုတ္တရပဒေ။ ဂုဏဝန္တပတိဋ္ဌော-သ္မိ။}

\sutta{376}{59}{မနာဒျာပါဒီနမော မယေ စ။}
\vutti{မနာဒီနမာပါဒီနံ စ ဩ ဟောတုတ္တရပဒေ မယေ စ။ မနောသေဋ္ဌာ၊ မနောမယာ၊ ရဇောဇလ္လံ၊ ရဇောမယံ၊ အာပေါဂတံ၊ အာပေါမယံ၊ အနုယန္တိ ဒိသောဒိသံ။}

\sutta{377}{60}{ပရဿ သင်္ချာသု။}
\vutti{သင်္ချာသုတ္တရပဒေသု ပရဿ ဩ ဟောတိ။ ပရောသတံ၊ ပရောသဟဿံ၊ သင်္ချာသူတိ ကိံ? ပရဒတ္တူပဇီဝိနော။}

\sutta{378}{61}{ဇနေ ပုထဿု။}
\vutti{ဇနေ ဥတ္တရပဒေ ပုထဿ ဥ ဟောတိ။ အရိယေဟိ ပုထဂေဝါယံ ဇနောတိ ပုထုဇ္ဇနော။}

\sutta{379}{62}{သော ဆဿာဟာယတနေ ဝါ။}
\vutti{အဟေ အာယတနေ စုတ္တရပဒေ ဆဿ သော ဝါ ဟောတိ။ သာဟံ ဆာဟံ၊ သဠာယတနံ၊ ဆဠာယတနံ။}

\sutta{380}{63}{လ္တုပိတာဒီနမာရင်ရင်။}
\vutti{လ္တုပ္ပစ္စယန္တာနံ ပိတာဒီနဉ္စ ယထာက္ကမမာရင်ရင် ဝါ ဟောန္တုတ္တရပဒေ၊ သတ္ထာရဒဿနံ၊ ကတ္တာရနိဒ္ဒေသော၊ မာတရပိတရော၊ ဝါတွေဝ? သတ္ထုဒဿနံ၊ မာတာပိတရော။}

\sutta{381}{64}{ဝိဇ္ဇာယောနိသမ္ဗန္ဓာနမာ တတြ စတ္ထေ။}
\vutti{လ္တုပိတာဒီနံ ဝိဇ္ဇာသမ္ဗန္ဓီနံ ယောနိသမ္ဗန္ဓီနံ စ တေသွေဝ လ္တုပိတာဒီသု ဝိဇ္ဇာယောနိသမ္ဗန္ဓိသုတ္တရပဒေသု စတ္ထဝိသယေ အာ ဟောတိ။ ဟောတာပေါတရော မာတာပိတရော၊ လ္တုပိတာဒီနံ တွေဝ? ပုတ္တဘာတရော၊ တတြေတိ ကိံ? ပိတုပိတာမဟာ၊ စတ္ထေတိ ကိံ? မာတုဘာတာ၊ ဝိဇ္ဇာယောနိသမ္ဗန္ဓာနန္တိ ကိံ? ဒါတုဘတ္တာရော။}

\sutta{382}{65}{ပုတ္တေ။}
\vutti{ပုတ္တေ ဥတ္တရပဒေ စတ္ထဝိသယေ လ္တုပိတာဒီနံ ဝိဇ္ဇာယောနိ သမ္ဗန္ဓာနမာ ဟောတိ။ ပိတာပုတ္တာ၊ မာတာပုတ္တာ။}

\sutta{383}{66}{စိသ္မိံ။}
\vutti{စိပ္ပစ္စယန္တေ ဥတ္တရပဒေ အာ ဟောတိ။ ကေသာကေသိ၊ မုဋ္ဌာမုဋ္ဌိ။}

\sutta{384}{67}{ဣတ္ထိယမ္ဘာသိတပုမိတ္ထီ ပုမေဝေကတ္ထေ။}
\vutti{ဣတ္ထိယံ ဝတ္တမာနေ ဧကတ္ထေ သမာနာဓိကရဏေ ဥတ္တရပဒေ ပရေ ဘာသိတပုမာ ဣတ္ထီ ပုမေဝ ဟောတိ။ ကုမာရဘရိယော၊ ဒီဃဇင်္ဃော၊ ယုဝဇာယော၊ ဣတ္ထိယန္တိ ကိံ? ကလျာဏီ ပဓာနမေသံ ကလျာဏိပ္ပဓာနာ၊ ဘာသိတပုမေတိ ကိံ? ကညာဘရိယော၊ ဣတ္ထီတိ ကိံ? ဂါမဏိကုလံ ဒိဋ္ဌိ အဿ ဂါမဏိဒိဋ္ဌိ၊ ဧကတ္ထေတိ ကိံ? ကလျာဏိယာ မာတာ ကလျာဏိမာတာ။}

\sutta{385}{68}{ကွစိ ပစ္စယေ။}
\vutti{ဘာသိတပုမိတ္ထီ ပစ္စယေ ကွစိ ပုမေဝ မောတိ။ ဗျတ္တတရာ၊ ဗျတ္တတမာ။}

\sutta{386}{69}{သဗ္ဗာဒယော ဝုတ္တိမတ္ထေ။}
\vutti{ဣတ္ထိဝါစကာ သဗ္ဗာဒယော ဝုတ္တိမတ္ထေ ပုမေဝ ဟောန္တိ။ တဿာ မုခံ တမ္မုခံ၊ တဿံ တတြ၊ တာယ တတော၊ တဿံ ဝေလာယံ တဒါ။}

\sutta{387}{70}{ဇာယာယ ဇယံ ပတိမှိ။}
\vutti{ပတိမှိ ပရေ ဇာယာယ ဇယံ ဟောတိ၊ ဇယမ္ပတီ၊ ‘ဇာနိပတီ’တိ ပကတန္တရေန သိဒ္ဓံ၊ တထာ ‘ဒမ္ပတီ၊ ဇမ္ပတီ’တိ။}

\sutta{388}{71}{သညာယမုဒေါဒကဿ။}
\vutti{သညာယမုဒကဿုတ္တရပဒေ ဥဒါဒေသော ဟောတိ။ ဥဒဓိ၊ ဥဒပါနံ။}

\sutta{389}{72}{ကုမ္ဘာဒီသု ဝါ။}
\vutti{ကုမ္ဘာဒီသုတ္တရပဒေသု ဥဒကဿ ဥဒါဒေသော ဝါ ဟောတိ။ ဥဒကုမ္ဘော ဥဒကကုမ္ဘော၊ ဥဒပတ္တော ဥဒကပတ္တော၊ ဥဒဗိန္ဓု ဥဒကဗိန္ဓု၊ အာကတိဂဏော-ယံ။}

\sutta{390}{73}{သောတာဒီသူလောပေါ။}
\vutti{သောတာဒီသုတ္တရပဒေသု ဥဒကဿ ဥဿ လောပေါ ဟောတိ။ ဒကသောတံ၊ ဒကရက္ခသော။}

\sutta{391}{74}{ဋ နဉ်ဿ။}
\vutti{ဥတ္တရပဒေ နဉ်သဒ္ဒဿ ဋ ဟောတိ။ အဗြာဟ္မဏော၊ ဉကာရော ကိံ? ကေဝလဿ မာ ဟောတု ပါမနပုတ္တော။}

\sutta{392}{75}{အန် သရေ။}
\vutti{သရာဒေါ ဥတ္တရပဒေ နဉ သဒ္ဒဿ အန ဟောတိ။ အနက္ခာတံ။}

\sutta{393}{76}{နခါဒယော။}
\vutti{နခါဒယော သဒ္ဒါ အနန ဋာဒေသာ နိပစ္စန္တေ။ နာဿ ခမတ္ထီတိ နခေါ၊ အခမညံ၊ သညာသဒ္ဒေသု စ နိပ္ဖတ္တိမတ္တံ ယထာကထဉ္စိ ကတ္တဗ္ဗံနာဿ ကုလမထီတိ နကုလော၊ အကုလမညံ၊ နခ နကုလ နပုံသကနက္ခတ္တ နာက ဧဝမာဒိ။}

\sutta{394}{77}{နဂေါ ဝါပ္ပါဏိနိ။}
\vutti{နဂဣစ္စပ္ပာဏိနိ ဝါ နိပစ္စတေ။ နဂေါ ရုက္ခော၊ နဂေါ ပဗ္ဗတော၊ အဂေါ ရုက္ခော၊ အဂေါ ပဗ္ဗတော၊ အပ္ပာဏိနီတိ ကိံ? အဂေါ ဝသလော သီတေန။}

\sutta{395}{78}{သဟဿ သော-ညတ္ထေ။}
\vutti{အညပဒတ္ထဝုတ္ထိမှိ သမာသေ ဥတ္တရပဒေ ပရေ သဟဿ သော ဝါ ဟောတိ။ သပုတ္တော၊ သဟပုတ္တော၊ အညတ္ထေတိ ကိံ? သဟ ကတွာ၊ သဟ ယုဇ္ဈိတွာ။}

\sutta{396}{79}{သညာယံ။}
\vutti{သဟဿုတ္တရပဒေ သော ဟောတိ သညာယံ။ သာဿတ္ထံ၊ သပလာသံ။}

\sutta{397}{80}{အပ္ပစ္စက္ခေ။}
\vutti{အပ္ပစ္စက္ခေ ဂမျမာနေ သဟဿ သော ဟောတုတ္တရပဒေ၊ သာဂ္ဂိ ကပေါတော၊ သပိသာစာ ဝါတမဏ္ဍလိကာ။}

\sutta{398}{81}{အကာလေ သကတ္ထေ။}
\vutti{သကတ္ထပ္ပဓာနဿ သဟသဒ္ဒဿ အကာလေ ဥတ္တရပဒေ သော ဟောတိ။ သမ္ပန္နံ ဗြဟ္မံ သဗြဟ္မံ၊ သစက္ကံ နိဓေဟိ၊ သဓုရံ ပါဇေဟိ၊ အကာလေတိ ကိံ? သဟ ပုဗ္ဗဏှံ၊ သဟာပရဏှံ။}

\sutta{399}{82}{ဂန္ထန္တာဓိကျေ။}
\vutti{ဂန္ထန္တေ အာဓိကျေ စ ဝတ္တမာနဿ သဟဿ သော ဟောတုတ္တရပဒေ။ သကလံ ဇောတိမဓီတေ သမုဟုတ္တံ၊ ကာလတ္ထော အာရမ္ဘော၊ အာဓိကျေ-သဒေါဏာ ခါရီ၊ သမာသကော ကဟာပဏော၊ နိစ္စတ္ထောယမာရမ္ဘော။}

\sutta{400}{83}{သမာနဿ ပက္ခာဒီသု ဝါ။}
\vutti{ပက္ခာဒီသုတ္တရပဒေသု သမာနဿ သော ဟောတိ ဝါ။ သပက္ခော သမာနပက္ခော၊ သဇောတိ သမာနဇောတိ၊ ပက္ခာဒီသူတိ ကိံ? သမာနသီလော၊ ပက္ခ၊ ဇောတိ၊ ဇနပဒ၊ ရတ္တိ၊ ပတ္တိနီ၊ ပတ္တီ၊ နာဘိ၊ ဗန္ဓု ဗြဟ္မစာရီ၊ နာမ၊ ဂေါတ္တ၊ ရူပ၊ ဌာန၊ ဝဏ္ဏ၊ ဝယော၊ ဝစန၊ ဓမ္မ၊ ဇာတိယ၊ ဃစ္စ။}

\sutta{401}{84}{ဥဒရေ ဣယေ။}
\vutti{ဥဒရေ ဣယေ ပရေ ပရတော သမာနဿ သော ဝါ ဟောတိ။ သောဒရိယော၊ သမာနောဒရိယော၊ ဣယေတိ ကိံ? သမာနောဒရတာ။}

\sutta{402}{85}{ရီရိက္ခကေသု။}
\vutti{ဧတေသု သမာနဿ သော ဟောတိ။ သရီ၊ သရိက္ခော၊ သရိသော။}

\sutta{403}{86}{သဗ္ဗာဒီနမာ။}
\vutti{ရီရိက္ခကေသု သဗ္ဗာဒီနမာ ဟောတိ။ ယာဒီ၊ ယာဒိက္ခော၊ ယာဒိသော။}

\sutta{404}{87}{န္တကိမိမာနံ ဋာကီဋီ။}
\vutti{ရီရိက္ခကေသု န္တသဒ္ဒ ကိံသဒ္ဒ ဣမသဒ္ဒါနံ ဋာကီဋီ ဟောန္တိ ယထာက္ကမံ။ ဘဝါဒီ၊ ဘဝါဒိက္ခော၊ ဘဝါဒိသော၊ ကီဒီ၊ ကီဒိက္ခော၊ ကီဒိသော၊ ဤဒီ၊ ဤဒိက္ခော၊ ဤဒိသော။}

\sutta{405}{88}{တုမှာမှာနံ တာမေကသ္မိံ။}
\vutti{ရီရိက္ခကေသု တုမှာမှာနံ တာမာ ဟောန္တေကသ္မိံ ယထာက္ကမံ။ တာဒီ၊ တာဒိက္ခော၊ တာဒိသော၊ မာဒီ၊ မာဒိက္ခော၊ မာဒိသော။ ဧကသ္မိန္တိ ကိံ? တုမှာဒီ၊ အမှာဒီ၊ တုမှာဒိက္ခော၊ အမှာဒိက္ခော၊ တုမှာဒိသော၊ အမှာဒိသော။}

\sutta{406}{89}{တံ-မ-မညတြ။}
\vutti{ရီရိက္ခကန္တတော အညသ္မိံ ဥတ္တရပဒေ တုမှာမှာနမေကသ္မိံ တံမံ ဟောန္တိ ယထာက္ကမံ၊ တန္ဒီပါ၊ မန္ဒီပါ၊ တံသရဏာ၊ မံသရဏာ၊ တယျောဂေါ၊ မယျောဂေါတိ ဗိန္ဒုလောပေါ။}

\sutta{407}{90}{ဝေတဿေဋ်။}
\vutti{ရီရိက္ခကေသွေတဿေဋ် ဝါ ဟောတိ၊ ဧဒီ၊ ဧတာဒီ၊ ဧဒိက္ခော၊ ဧတာဒိက္ခော၊ ဧဒိသော၊ ဧတာဒိသော။}

\sutta{408}{91}{ဝိဓာဒီသု ဒွိဿ ဒု။}
\vutti{ဒွိဿ ဒု ဟောတိ ဝိဓာဒီသု၊ ဒုဝိဓော၊ ဒုပဋ္ပံ ဧဝမာဒိ။}

\sutta{409}{92}{ဒိ ဂုဏာဒီသု။}
\vutti{ဂုဏာဒီသု ဒွိဿ ဒိ ဟောတိ၊ ဒွိဂုဏံ၊ ဒိရတ္တိ၊ ဒိဂု ဧဝမာဒိ။}

\sutta{410}{93}{တီသွ။}
\vutti{တီသု ဒွိဿ အဟောတိ။ ဒွတ္တိက္ခတ္တုံ၊ ဒွိတ္တိပတ္တပုရာ။}

\sutta{411}{94}{အာ သင်္ချာယာသတာဒေါ-နညတ္ထေ။}
\vutti{သင်္ချာယမုတ္တရပဒေ ဒွိဿ အာ ဟောတိ အသတာဒေါ အနညဓတ္ထ။ ဒွါဒသ၊ ဒွါဝီသတိ ဒွတ္တိံသ၊ သင်္ချာယန္တိ ကိံ? ဒိရတ္တံ၊ အသတာဒေါတိ ကိံ? ဒိသတံ၊ ဒိသဟဿံ။ အနညတ္ထေတိ ကိံ? ဒွိဒသာ။}

\sutta{412}{95}{တိဿေ။}
\vutti{သင်္ချာယမုတ္တရပဒေ တိဿ ဧ ဟောတိ အသတာဒေါ အနညတ္ထေ၊ တေရသ၊ တေဝီသ၊ တေတ္တိံသ၊ သင်္ချာယံတွေဝ? တိရတ္တံ၊ အသတာဒေါတွေဝ? တိသတံ၊ အနညတ္ထေတွေဝ? တိစတုကာ။}

\sutta{413}{96}{စတ္တာလီသာဒေါ ဝါ။}
\vutti{တိဿေ ဝါ ဟောတိ စတ္တာလီသာဒေါ၊ တေစတ္တာလာသံ တိစတ္တာလီသံ၊ တေပညာသံ တိပညာသံ၊ တေသဋ္ဌိ တိသဋ္ဌိ၊ တေသတ္တတိ တိသတ္တတိ၊ တေအသီတိ တိယာသီတိ၊ တေနဝုတိ တိနဝုတိ၊ အသတာဒေါတွေဝ? တိသတံ။}

\sutta{414}{97}{ဒွိဿာ စ။}
\vutti{အသတာဒေါ-နညတ္ထေ စတ္တာလီသာဒေါ ဒွိဿေ ဝါ ဟောတိ အာ စ။ ဒွေစတ္တာလီသံ၊ ဒွါစတ္တာလီသံ ဒွိစတ္တာလီသံ၊ ဒွေပညာသံ၊ ဒွါပညာသံ ဒွိပညာသံ ဣစ္စာဒိ။}

\sutta{415}{98}{ဗာစတ္တာလီသာဒေါ။}
\vutti{ဒွိဿ ဗာ ဝါ ဟောတိ အစတ္တာလီသာဒေါ-နညတ္ထေ။ ဗာရသ ဒွါဒသ၊ ဗာဝီသတိ ဒွါဝီသတိ၊ ဗတ္တိံသ ဒွတ္တိံသ၊ အစတ္တာလီသာဒေါတိ ကိံ? ဒွိစတ္တာလီသံ။}

\sutta{416}{99}{ဝီသတိဒသေသု ပဉ္စဿ ပဏ္ဏပန္နာ။}
\vutti{ဝီသတိဒသေသု ပရေသု ပဉ္စဿ ပဏ္ဏပန္နာ ဟောန္တိ ဝါ ယထာက္ကမံ။ ပဏ္ဏဝီသတိ ပဉ္စဝီသတိ၊ ပန္နရသွ ပဉ္စဒသ။}

\sutta{417}{100}{စတုဿ စုစော ဒသေ။}
\vutti{စတုဿ စုစော ဟောန္တိ ဝါ ဒသသဒ္ဒေ ပရေ။ စုဒ္ဒသ၊ စောဒ္ဒသ၊ စတုဒ္ဒသ။}

\sutta{418}{101}{ဆဿ သော။}
\vutti{ဆဿ သောဣစ္စယမာဒေသော ဟောတိ ဒသသဒ္ဒေ ပရေ။ သောဠသ။}

\sutta{419}{102}{ဧကဋ္ဌာနမာ။}
\vutti{ဧကအဋ္ဌာနံ အာ ဟောတိ ဒသေ ပရေ။ ဧကာဒသ၊ အဋ္ဌာရသ။}

\sutta{420}{103}{ရ သင်္ချာတော ဝါ။}
\vutti{သင်္ချာတော ပရဿ ဒသဿ ရ ဟောတိ ဝိဘာသာ။ ဧကာရသ ဧကာဒသ၊ ဗာရသ ဒွါဒသ၊ ပန္နရသ ပဉ္စဒသ၊ သတ္တရသ သတ္တဒသ၊ အဋ္ဌာရသ အဋ္ဌာဒသ၊ ပန္နဗာဒေသေသု နိစ္စံ၊ ဣဓ န ဟောတိ စတုဒ္ဒသ။}

\sutta{421}{104}{ဆတီဟိ ဠော စ။}
\vutti{ဆတီဟိ ပရဿ ဒသဿ ဠော ဟောတိ ရော စ၊ သောဠသ သောရသ၊ တေဠသ တေရသ။}

\sutta{422}{105}{စတုတ္ထ-တတိယာန-မဍ္ဎု-ဍ္ဎတိယာ။}
\vutti{အဍ္ဎာ ပရေသံ စတုတ္ထတတိယာနံ ဥဍ္ဎတိယာ ဟောန္တိ ယထာက္ကမံ။ အဍ္ဎေန စတုတ္ထော အဍ္ဎုဍ္ဎော၊ အဍ္ဎေန တတိယော အဍ္ဎတိယော၊ ကထံ အဍ္ဎတေယျောတိ? သကတ္ထေ ဏျေ ဥတ္တရပဒဝုဍ္ဎိ။}

\sutta{423}{106}{ဒုတိယဿ သဟ ဒိယဍ္ဎဒိဝဍ္ဎာ။}
\vutti{အဍ္ဎာ ပရဿ ဒုတိယဿ သဟ အဍ္ဎသဒ္ဒေန ဒိယဍ္ဎဒိဝဍ္ဎာ ဟောန္တိ။ အဍ္ဎေန ဒုတိယော ဒိယဍ္ဎော၊ ဒိဝဍ္ဎော ဝါ။}

\sutta{424}{107}{သရေ ကဒ် ကုဿုတ္တရတ္ထေ။}
\vutti{ကုဿုတ္တရပဒတ္ထေ ဝတ္တမာနဿ သရာဒေါ ဥတ္တရပဒေ ကဒါဒေသော ဟောတိ။ ကဒန္နံ၊ ကဒသနံ၊ သရေတိ ကိံ? ကုပုတ္တော၊ ဥတ္တရတ္ထေတိ ကိံ? ကုဩဍ္ဎော ရာဇာ။}

\sutta{425}{108}{ကာပ္ပတ္ထေ။}
\vutti{အပ္ပတ္ထေ ဝတ္တမာနဿ ကုဿ ကာ ဟောတိ ဥတ္တရပဒတ္ထေ၊ အပ္ပကံ လဝဏံ ကာလဝဏံ။}

\sutta{426}{109}{ပုရိသေ ဝါ။}
\vutti{ကုဿ ပုရိသေ ကာ ဟောတိ ဝါ။ ကာပုရိသော ကုပုရိသော၊ အယမပ္ပတ္ထဝိဘာသာ၊ အပ္ပတ္ထေ တု ပုဗ္ဗေန နိစ္စံ ဟောတိ ဤသံ ပုရိသော ကာပုရိသော။}

\sutta{427}{110}{ပုဗ္ဗာပရဇ္ဇသာယမဇ္ဈေဟာဟဿ ဏှော။}
\vutti{ပုဗ္ဗာဒီဟုတ္တရပဒဿ အဟဿ ဏှာဒေသော ဟောတိ၊ ပုဗ္ဗဏှော၊ အပရဏှော၊ အဇ္ဇဏှော၊ သာယဏှော၊ မဇ္ဈဏှော (ပဏှော)။}

\begin{jieshu}
ဣတိ မောဂ္ဂလ္လာနေ ဗျာကရဏေ ဝုတ္တိယံ

သမာသကဏ္ဍော တတိယော။
\end{jieshu}
\chapter{ဏာဒိကဏ္ဍော စတုတ္ထော}
\markboth{မောဂ္ဂလ္လာနဗျာကရဏေ}{ဏာဒိကဏ္ဍော စတုတ္ထော}

\sutta{428}{1}{ဏော ဝါ ပစ္စေ။}
\vutti{ဆဋ္ဌိယန္တာ နာမသ္မာ ဝါ ဏပ္ပစ္စယော ဟောတိ အပစ္စေ-ဘိဓေယျေ၊ ဏကာရော ဝုဒ္ဓျတ္ထော၊ ဧဝမညတ္တာပိ၊ ဝသိဋ္ဌဿာပစ္စံ ဝါသိဋ္ဌော၊ ဝါသိဋ္ဌီ ဝါ၊ ဩပဂဝေါ၊ ဩပဂဝီ ဝါ၊ ဝေတိ ဝါကျသမာသဝိကပ္ပနတ္ထံ၊ တဿာဓိကာရော သကတ္ထာဝဓိ။}

\sutta{429}{2}{ဝစ္ဆာဒိတော ဏာနဏာယနာ။}
\vutti{ဝစ္ဆာဒီဟိ အပစ္စပ္ပစ္စယန္တေဟိ ဂေါတ္တာဒီဟိ စ သဒ္ဒေဟိ ဏာနဏာယနပ္ပစ္စယာ ဝါ ဟောန္တိ အပစ္စေ၊ ဝစ္ဆာနော ဝစ္ဆာယနော၊ ကစ္စာနော ကစ္စာယနော၊ ယာဂမေ ကာတိယာနော၊ မောဂ္ဂလ္လာနော မောဂ္ဂလ္လာယနော၊ သာကဋာနော သာကဋာယနော၊ ကဏှာနော ကဏှာယနော ဣစ္စာဒိ။}

\suttagananormal{430}{19}{ကတာ ဏိယောဝ။}

\suttagananormal{431}{20}{ကဏှော ဗြာဟ္မဏေ။}

\sutta{432}{3}{ကတ္တိကာဝိဓဝါဒီဟိ ဏေယျဏေရာ။}
\vutti{ကတ္တိကာဒီဟိ ဝိဓဝါဒီဟိ စ ဏေယျဏေရာ ဝါ ယထာက္ကမံ ဟောန္တိ အပစ္စေ၊ ကတ္တိကေယျော၊ ဝေနတေယျော၊ ဘာဂိနေယျော ဣစ္စာဒိ ၊ ဝေဓဝေရော၊ ဗန္ဓကေရော၊ နာလိကေရော၊ သာမဏေရော ဣစ္စာဒိ။}

\sutta{433}{4}{ဏျ ဒိစ္စာဒီဟိ။}
\vutti{ဒိတိပ္ပဘုတိဟိ ဏျော ဝါ ဟောတိ အပစ္စေ၊ ဒေစ္စော၊ အာဒိစ္စော၊ ကောဏ္ဍညော၊ ဂဂ္ဂျော၊ ဘာတဗ္ဗော ဣစ္စာဒိ။}

\sutta{434}{5}{အာ ဏိ။}
\vutti{အကာရန္တတော ဏိ ဝါ ဟောတိ အပစ္စေ၊ ဒက္ခိ၊ ဒတ္ထိ၊ ဒေါဏိ။ ဝါသဝိ၊ ဝါရုဏိ ဣစ္စာဒိ။}

\sutta{435}{6}{ရာဇတော ညော ဇာတိယံ။}
\vutti{ရာဇသဒ္ဒတော ညော ဝါ ဟောတိ အပစ္စေ ဇာတိယံ ဂမျမာနာယံ၊ ရာဇညော၊ ဇာတိယန္တိ ကိံ? ရာဇာပစ္စံ။}

\sutta{436}{7}{ခတ္တာ ယိယာ။}
\vutti{ခတ္တသဒ္ဒါ ယဣယာ ဟောန္တိ အပစ္စေ ဇာတိယံ၊ ခတျော၊ ခတ္တိယော၊ ဇာတိယံတွေဝ? ခတ္တိ။}

\sutta{437}{8}{မနုတော ဿသဏ။}
\vutti{မနုသဒ္ဒတော ဇာတိယံ ဿသဏ ဟောန္တိ အပစ္စေ၊ မနုဿော၊ မာနုသော၊ ဣတ္ထိယံ မနုဿာ၊ မာနုသီ၊ ဇာတိယံတွေဝ? မာနဝေါ။}

\sutta{438}{9}{ဇနပဒနာမသ္မာ ခတ္တိယာ ရညေ စ ဏော။}
\vutti{ဇနပဒဿ ယံ နာမံ၊ တန္နာမသ္မာ ခတ္တိယာ အပစ္စေ ရညေ စ ဏော ဟောတိ၊ ပဉ္စာလော၊ ကောသလော၊ မာဂဓော၊ ဩက္ကာကော၊ ဇနပဒနာမသ္မာတိ ကိံ? ဒါသရထိ၊ ခတ္တိယာတိ ကိံ? ပဉ္စာလဿ ဗြာဟ္မဏဿ အပစ္စံ ပဉ္စာလိ။}

\sutta{439}{10}{ဏျ ကုရုသိဝီဟိ။}
\vutti{ကုရုသိဝီဟိ အပစ္စေ ရညေ စ ဏျော ဟောတိ။ ကောရဗျော၊ သေဗျော။}

\sutta{440}{11}{ဏ ရာဂါ တေန ရတ္တံ။}
\vutti{ရာဂဝါစိတတိယန္တတော ရတ္တမိစ္ဆေတသ္မိံ အတ္ထေ ဏော ဟောတိ၊ ကသာဝေန ရတ္တံ ကာသာဝံ၊ ကောသုမ္ဘံ၊ ယာလိဒ္ဒံ၊ ရာဂါတိ ကိံ? ဒေဝဒတ္တေန ရတ္တံ ဝတ္ထံ၊ ဣဓ ကသ္မာ န ဟောတိ? ‘နီလံ ပီတ’န္တိ၊ ဂုဏဝစနတ္တာ ဝိနာပိ ဏေန ဏတ္ထဿာဘိဓာနတော။}

\sutta{441}{12}{နက္ခတ္တေနိန္ဒုယုတ္တေန ကာလေ။}
\vutti{တတိယန္တတော နက္ခတ္တာ တေန လက္ခိတေ ကာလေ ဏော ဟောတိ၊ တဉ္စေ နက္ခတ္တမိန္ဒုယုတ္တံ ဟောတိ၊ ဖုဿီ ရတ္တိ၊ ဖုဿံ အဟံ၊ နက္ခတ္တေနေတိ ကိံ? ဂုရုနာ လက္ခိတာ ရတ္တိ။ ဣန္ဒုယုတ္တေနေတိ ကိံ? ကတ္ထိကာယ လက္ခိတော မုဟုတ္တော၊ ကာလေတိ ကိံ? ဖုဿေန လက္ခိတာ အတ္ထသိဒ္ဓိ၊ အဇ္ဇကတ္တိကာတိ ကတ္တိကာယုတ္တေ စန္ဒေ ကတ္တိကာသဒ္ဒေါ ဝတ္တတေ။}

\sutta{442}{13}{သာ-ဿ ဒေဝတာ ပုဏ္ဏမာသီ။}
\vutti{သေတိ ပဌမန္တာ အဿာတိ ဆဋ္ဌျတ္ထေ ဏော ဘဝတိ၊ ယံ ပဌမန္တံ၊ သာ စေ ဒေဝတာ ပုဏ္ဏမာသီဝါ၊ သုဂတော ဒေဝတာ အဿာဘိ သောဂတော၊ မာဟိန္ဒော၊ ယာမော၊ ဝါရုဏော၊ ဖုဿီ ပုဏ္ဏမာသီ အဿ သမ္ဗန္ဓိနီတိ ဖုဿော၊ မာသော၊ မာယော၊ ဖဂ္ဂုနော၊ စိတ္တော၊ ဝေသာခေါ။ ဇေဋ္ဌမူလော၊ အာသဠှော၊ သာဝဏော၊ ပေါဋ္ဌပါဒေါ၊ အဿယုဇော၊ ကတ္တိကော၊ မာဂသိရော၊ ပုဏ္ဏမာသီတိ ကိံ? ဖုဿီ ပဉ္စမီ အဿ၊ ပုဏ္ဏမာသီ စ ဘတကမာသသမ္ဗန္ဓိနီ န ဟောတိ။ ပုဏ္ဏော မာ အဿန္တိ နိဗ္ဗစနာ၊ အတော ဧဝ နိပါတနာ ဏော သာဂမော စ၊ မာသသုတိယာဝ န ပဉ္စဒသ ရတ္တာဒေါ ဝိဓိ။}

\sutta{443}{14}{တမဓီတေ တံ ဇာနာတိ ကဏိကာ စ။}
\vutti{ဒုတိယန္တတော တမဓီတေ တံ ဇာနာတီတိ ဧတေသွတ္ထေသု ဏော ဟောတိ ဏော ဏိကော စ၊ ဗျာကရဏမဓီတေ ဇာနာတိ ဝါ ဝေယျာကရဏော၊ ဆန္ဒသော၊ ကမကော၊ ပဒကော၊ ဝေနယိကော၊ သုတ္တန္တိကော၊ ဒွိတဂ္ဂဟဏံ ပုထဂေဝ ဝိဓာနတ္ထံ ဇာနနဿ စ အဇ္ဈေနဝိသယဘာဝဒဿနတ္ထံ ပသိဒ္ဓူပသင်္ဂဟတ္ထံ စ။}

\sutta{444}{15}{တဿ ဝိသယေ ဒေသေ။}
\vutti{ဆဋ္ဌိယန္တာ ဝိသယေ ဒေသရူပေ ဏော ဟောတိ၊ ဝသာတီနံ ဝိသယော ဒေသော ဝါသာတော၊ ဒေသေတိ ကိံ? စက္ခုဿ ဝိသယော ရူပံ၊ ဒေဝဒတ္တဿ ဝိသယော-နုဝါကော။}

\sutta{445}{16}{နိဝါသေ တန္နာမေ။}
\vutti{ဆဋ္ဌိယန္တာ နိဝါသေ ဒေသေ တန္နာမေ ဏော ဟောတိ၊ သိဝီနံ နိဝါသော ဒေသော သေဗျော၊ ဝါသာတော။}

\sutta{446}{17}{အဒူရဘဝေ။}
\vutti{ဆဋ္ဌိယန္တာ အဒူရဘဝေ ဒေသေ တန္နာမေ ဏော ဟောတိ၊ ဝိဒိသာယ အဒူရဘဝံ ဝေဒိသံ။}

\sutta{447}{18}{တေန နိဗ္ဗတ္တေ။}
\vutti{တတိယန္တာ နိဗ္ဗတ္တေ ဒေသေ တန္နာမေ ဏော ဟောတိ၊ ကုသမ္ဗေန နိဗ္ဗတ္တာ ကောသမ္ဘီ နဂရီ၊ ကာကန္ဒီ၊ မာကန္ဒီ၊ သဟဿေန နိဗ္ဗတ္တာ သာဟဿီ ပရီခါ၊ ဟောတုမှိ ကတ္တရိ ကရဏေ စ ယထာယောဂံ တတိယာ။}

\sutta{448}{19}{တမီဓတ္ထိ။}
\vutti{တန္တိ ပဌမန္တာ ဣဓာတိ သတ္တမျတ္ထေ ဒေသေ တန္နာမေ ဏော ဟောတိ၊ ယန္တံ ပဌမန္တမတ္ထိ စေ၊ ဥဒုမ္ဗရာ အသ္မိံ ဒေသေ သန္တီတိ ဩဒုမ္ဗရော၊ ဗာဒရော၊ ဗဗ္ဗဇော။}

\sutta{449}{20}{တတြ ဘဝေ။}
\vutti{သတ္တမျန္တာ ဘဝတ္ထေ ဏော ဟောတိ၊ ဥဒကေ ဘဝေါ ဩဒကော၊ ဩရသော၊ ဇာနပဒေါ၊ မာဂဓော၊ ကာပိလဝတ္ထဝေါ၊ ကောသမ္ဗော။}

\sutta{450}{21}{အဇ္ဇာဒီဟိ တနော။}
\vutti{ဘဝတ္ထေ အဇ္ဇာဒီဟိ တနော ဟောတိ၊ အဇ္ဇ ဘဝေါ အဇ္ဇတနော၊ သွာတနော၊ ဟိယျတ္တနော။}

\sutta{451}{22}{ပုရာတော ဏော စ။}
\vutti{ပုရာဣစ္စသ္မာ ဘဝတ္ထေ ဏော ဟောတိ တနော စ၊ ပုရာဏော၊ ပုရာတနော။}

\sutta{452}{23}{အမာတွစ္စော။}
\vutti{အမာသဒ္ဒတော အစ္စော ဟောတိ ဘဝတ္ထေ၊ အမစ္စော။}

\sutta{453}{24}{မဇ္ဈာဒိတွိမော။}
\vutti{မဇ္ဈာဒီဟိ သတ္တမျန္တေဟိ ဘဝတ္တေ ဣမော ဟောတိ၊ မဇ္ဈိမော၊ အန္တိမော။ မဇ္ဈ၊ အန္တ၊ ဟေဋ္ဌာ၊ ဥပရိ၊ ဩရ၊ ပါရ၊ ပစ္ဆာ၊ အဗ္ဘန္တရ၊ ပစ္စန္တ (ပုရတ္ထာ၊ ဗာဟိရ)။}

\sutta{454}{25}{ကဏ ဏေယျ ဏေယျက ယိယာ။}
\vutti{သတ္တမျန္တာ ဧတေ ပစ္စယာ ဟောန္တိ ဘဝတ္ထေ၊ ကဏ-ကုသိနာ ရာယံ ဘဝေါ ကောသိနာရကော၊ မာဂဓကော၊ အာရညကော ဝိဟာရော။ ဏေယျ-ဂင်္ဂေယျော၊ ပဗ္ဗတေယျော၊ ဝါနေယျော။ ဏေယျက-ကောလေယျကော၊ ဗာရာဏသေယျကော၊ စမ္ပေယျကော၊ မိထိလေယျကောတိ ဧယျကော။ ယ-ဂမ္မော၊ ဒိဗ္ဗော။ ဣယဂါမိယော၊ ဥဒရိ-ယော၊ ဒိဝိယော၊ ပဉ္စာလိယော၊ ဗောဓိပက္ခိယော၊ လောကိယော။}

\sutta{455}{26}{ဏိကော။}
\vutti{သတ္တမျန္တာ ဘဝတ္ထေ ဏိကော ဟောတိ၊ သာရဒိကော ဒိဝသော၊ သာရဒိကာ ရတ္တိ။}

\sutta{456}{27}{တမဿ သိပ္ပံ သီလံ ပဏျံ ပဟရဏံ ပယောဇနံ။}
\vutti{ပဌမန္တာ သိပ္ပာဒိဝါစကာ အဿေတိဆဋ္ဌိယတ္ထေဏိကော ဟောတိ၊ ဝီဏာဝါဒနံ သိပ္ပမဿ ဝေဏိကော၊ မောဒင်္ဂိကော၊ ဝံသိကော၊ ပံသုကူ- လဓာရဏံ သီလမဿ ပံသုကူလိကော၊ တေစီဝရိကေဝ၊ ဂန္ဓောပဏျမဿ ဂန္ဓိကော၊ တေလိကော၊ ဂေါဠိကော၊ စာပေါ ပဟရဏမဿ စာပိကော၊ တောမရိကော၊ မုဂ္ဂရိကော၊ ဥပဓိပ္ပယောဇနမဿ ဩပဓိကံ၊ သာတိကံ၊ သာဟဿိကံ။}

\sutta{457}{28}{တံ ဟန္တရဟတိ ဂစ္ဆတုဉ္ဆတိ စရတိ။}
\vutti{ဒုတိယန္တာ ဟန္တီတိ ဧဝမာဒီသွတ္ထေသု ဏီကော ဟောတိ။ ပက္ခီဟိပက္ခိနော ဟန္တီတိ ပက္ခိကော၊ သာကုနိကော၊ မာယူရိကော။ မစ္ဆေဟိ-မစ္ဆိကော၊ မေနိကော။ မိဂေဟိ-မာဂဝိကော ဟာရိဏိကော၊ ‘သူကရိကော’တိ ဣကော။ သတမရဟတီတိ သာတိကံ၊ သန္ဒိဋ္ဌိကော၊ ဧဟိပဿဝိဓိံ အရဟတီတိ ဧဟိပဿိကော၊ သာဟဿိကော၊ ‘သဟဿိယော’တိ ဣယော။ ပရဒါရံ ဂစ္ဆတီတိ ပါရဒါရိကော၊ မဂ္ဂိကော၊ ပညာသယောဇနိကော။ ဗဒရေ ဥဉ္ဆတီတိ ဗာဒရိကော၊ သာမာကိကော။ ဓမ္မံ စရတီတိ ဓမ္မိကော၊ အဓမ္မိကော။}

\sutta{458}{29}{တေန ကတံ ကီတံ ဗဒ္ဓမဘိသင်္ခတံ သံသဋ္ဌံ ဟတံ ဟန္တိ ဇိတံ ဇယတိ ဒိဗ္ဗတိ ခဏတိ တရတိ စရတိ ဝါဟတိ ဇီဝတိ။}
\vutti{တတိယန္တာ ကတာဒိသွတ္ထေသု ဏိကော ဟောတိ။ ကာယေန ကတံ ကာယိကံ၊ ဝါစသိကံ၊ မာနသိကံ၊ ဝါတေန ကတော အာဗာဓော ဝါတိကော။ သတေန ကီတံ သာတိကံ၊ သာဟဿိကံ၊ မူလတောဝ၊ ဒေဝဒတေန ကီတန္တိ န ဟောတိ တဒတ္ထာပ္ပတီတိယာ။ ဝရတ္တာယ ဗဒ္ဓေါ ဝါရတ္တိကော၊ အာယသိကော၊ ပါသိကော။ ဃတေန အဘိသင်္ခတံ သံသဋ္ဌံ ဝါ ဃာတိကံ၊ ဂေါဠိကံ၊ ဒါဓိကံ၊ မာရိစိကံ။ ဇာလေန ဟတော ဟန္တီတိ ဝါ ဇာလိကော၊ ဗာလီသိကော။ အက္ခေဟိ ဇိတမက္ခိကံ၊ သာလာကိကံ။ အက္ခေဟိ ဇယတိ ဒိဗ္ဗတီတိ ဝါ အက္ခိကော။ ခဏိတ္တိယာ ခဏတီတိ ခါဏိတ္တိကော၊ ကုဒ္ဒါလိကော၊ ဒေဝဒတ္တေန ဇိတံ၊ အင်္ဂုလျာ ခဏတီတိ န ဟောတိ တဒတ္ထာနဝဂမာ။ ဥဠုမ္ပေန တရတီတိ ဩဠုမ္ပိကော၊ ‘ဥဠုမ္ပိကော’တိ ဣကော ဂေါပုစ္ဆိကော၊ နာဝိကော။ သကဋေန စရတီတိ သာကဋိကော၊ ‘ရထိကော၊ ပရပ္ပိကော’တိ ဣကော။ ခန္ဓေန ဝါဟတီတိ ခန္ဓိကော၊ အံသိကော၊ ‘သီသိကော’တိ ဣကော။ ဝေတနေန ဇီဝတီတိ ဝေတနိကော၊ ‘ဘတိကော၊ ကယိကော၊ ဝိက္ကယိကော၊ ကယဝိက္ကယိကော’တိ ဣကော။}

\sutta{459}{30}{တဿ သံဝတ္တတိ။}
\vutti{စတုတ္ထိယန္တာ သံဝတ္တတီတိ အသ္မိံ အတ္ထေ ဏိကော ဟောတိ၊ ပုနဗ္ဘဝါယ သံဝတ္တတီတိ ပေါနောဘဝိကော၊ ဣတ္ထိယံ ပေါနော ဘဝိကာ၊ လောကာယ သံဝတ္တတီတိ လောကိကော သုဋ္ဌု အဂ္ဂေါတိ သဂ္ဂေါ၊ သဂ္ဂါယ သံဝတ္တတီတိ သောဝဂ္ဂိကော၊ သဿောဝက တဒမိနာဒိပါဌာ၊ ဓနာယ သံဝတ္တတီတိ ဓညံ၊ ယော။}

\sutta{460}{31}{တတော သမ္ဘူတမာဂတံ။}
\vutti{ပဉ္စမျန္တာ သမ္ဘူတမာဂတန္တိ ဧတေသွတ္ထေသု ဏိကော ဟောတိ၊ မာတိကော သမ္ဘူတမာဂတံ ဝါ မတ္တိကံ၊ ပေတ္တိကံ၊ ဏျရိယဏ -ယာပိ ဒိဿန္တိ၊ သုရဘိတော သမ္ဘူတံ သောရဘျံ၊ ထနတော သမ္ဘူတံ ထညံ၊ ပိတိတော သမ္ဘူတော ပေတ္တိယော၊ မာတိယော၊ မတ္တိယော၊ မစ္စော ဝါ။}

\sutta{461}{32}{တတ္ထ ဝသတိ ဝိဒိတော ဘတ္တော နိယုတ္တော။}
\vutti{သတ္တမျန္တာ တတ္ထ ဝသတီတွေဝမာဒီသွတ္ထေသု ဏိကော ဟောတိ။ ရုက္ခမူလေ ဝသတီတိ ရုက္ခမူလိကော၊ အာရညိကော၊ သောသာနိကော။ လောကေ ဝိဒိတော လောကိကော။ စတုမဟာရာဇေသု ဘတ္တာ စာတုမ္မဟာရာဇိကာ ။ ဒွါရေ နိယုတ္တော ဒေါဝါရိကော ဒဿောက တဒမိနာဒိပါဌာ၊ ဘဏ္ဍာဂါရိကော၊ ဣကော-နဝကမ္မိကော၊ ကိယောဇာတိကိယော၊ အန္ဓကိယော။}

\sutta{462}{33}{တဿိဒံ။}
\vutti{ဆဋ္ဌိယန္တာ ဣဒံမစ္စသ္မိံ အတ္ထေ ဏိကော ဟောတိ၊ သင်္ဃဿ ဣဒံ သင်္ဃိကံ၊ ပုဂ္ဂလိကံ၊ သကျပုတ္တိကော၊ နာတိပုတ္တိကော၊ ဇေနဒတ္တိကော၊ ကိယေ-သကိယော၊ ပရကိယော၊ နိယေ-အတ္တနိယံ၊ ကေ- သကော ရာဇကံ ဘဏ္ဍံ။}

\sutta{463}{34}{ဏော။}
\vutti{ဆဋ္ဌိယန္တာ ဣဒံမစ္စသ္မိံ အတ္ထေ ဏော ဟောတိ၊ ကစ္စာယနဿ ဣဒံ ကစ္စာယနံ ဗျာကရဏံ၊ သောဂတံ သာသနံ၊ မာဟိသံ မံသာဒိ။}

\sutta{464}{35}{ဂဝါဒီဟိ ယော။}
\vutti{ဂဝါဒီဟိ ဆဋ္ဌိယန္တေဟိ ဣဒံမစ္စသ္မိံ အတ္ထေ ယော ဟောတိ၊ ဂုန္နံ ဣဒံ ဂဗျံ မံသာဒိ၊ ကဗျံ၊ ဒဗ္ဗံ။}

\sutta{465}{36}{ပိတိတော ဘာတရိ ရေယျဏ။}
\vutti{ပိတုသဒ္ဒါ တဿ ဘာတရိ ရေယျဏ ဟောတိ၊ ပိတု ဘာတာ ပေတ္တေယျော။}

\sutta{466}{37}{မာတိတော စ ဘဂိနိယံ ဆော။}
\vutti{မာတိတော ပိတိတော စ တေသံ ဘဂိနိယံ ဆော ဟောတိ၊ မာတု ဘဂိနီ မာတုစ္ဆာ၊ ပိတု ဘဂိနီ ပိတုစ္ဆာ၊ ကထံ ‘မာတုလော’တိ “မာတုလာဒိတွာနီ”တိ နိပါတနာ။}

\sutta{467}{38}{မာတာပိတူသွာမဟော။}
\vutti{မာတာပိတူဟိ တေသံ မာတာပိတူသု အာမဟော ဟောတိ၊ မာတု မာတာ မာတာမဟီ၊ မာတု ပိတာ မာတာမဟော၊ ပိတု မာတာ ပိတာမဟီ၊ ပိတု ပိတာ ပိတာမဟော၊ န ယထာသင်္ချံ၊ ပစ္စေကာဘိ သမ္ဗန္ဓာ။}

\sutta{468}{39}{ဟိတေ ရေယျဏ။}
\vutti{မာတာပိတူဟိ ဟိတေ ရေယျဏ ဟောတိ၊ မတ္တေယျော၊ ပေတ္တေယျော။}

\sutta{469}{40}{နိန္ဒာညာတပ္ပပဋိဘာဂရဿဒယာသညာသု ကော။}
\vutti{နိန္ဒာဒီသွတ္ထေယု နာမသ္မာ ကော ဟောတိ၊ နိန္ဒာယံ-မုဏ္ဍကော၊ သမဏကော။ အညာတေ-ကဿာယံ အဿောတိ အဿကော၊ ပယောဂသာမတ္ထိယာ သမ္ဗန္ဓိဝိသေသာနဝဂမောဝဂမျတေ။ အပ္ပတ္ထေတေလကံ၊ ဃတကံ။ ပဋိဘာဂတ္ထေ-ဟတ္ထီ ဝိယ ဟတ္ထိကော၊ အဿကော၊ ဗလီဗဒ္ဒကော။ ရဿေ-မာနုသကော၊ ရုက္ခကော၊ ပိလက္ခကော။ ဒယာယံ-ပုတ္တကော၊ ဝစ္ဆကော။ သညာယံ-မောရော ဝိယ မောရကော။}

\suttagananormal{470}{21}{ဝတ္ထိတော ဣဝတ္ထေ ဧယျော။}

\suttagananormal{471}{22}{သိလာယ ဏေယျော စ။}

\suttagananormal{472}{23}{သာခါဒီဟိ ဣယော။}

\suttagananormal{473}{24}{မုခါဒီဟိ ယော။}

\suttagananormal{474}{25}{အာကသ္မိကေ ဘိဓေယေ ဤယော။}

\suttagananormal{475}{26}{သက္ကရာဒီဟိ ဏော။}

\suttagananormal{476}{27}{အင်္ဂုလျာဒီဟိ ဏိကော။}

\sutta{477}{41}{တမဿ ပရိမာဏံ ဏိကော စ။}
\vutti{ပဌမန္တာ အဿေတိ အသ္မိံ အတ္ထေ ဏိကော ဟောတိ ကော စ တဉ္စေ ပဌမန္တံ ပရိမာဏံ ဘဝတိ၊ ပရိမီယတေ နေနေတိ ပရိမာဏံ၊ ဒေါဏော ပရိမာဏမဿ ဒေါဏိကော ဝီဟိ၊ ခါရသတိကော၊ ခါရသဟဿိကော အာသီတိကော ဝယော၊ ဥပဍ္ဎကာယိကံ ဗိမ္ဗောဟနံ၊ ပဉ္စကံ၊ ဆက္ကံ။}

\sutta{478}{42}{ယတေတေဟိ တ္တကော။}
\vutti{ယာဒီဟိ ပဌမန္တေဟီ အဿေတိ ဆဋ္ဌိယတ္ထေ တ္တကော ဟောတိ၊ တဉ္စေ ပဌမန္တံ ပရိမာဏံ ဘဝတိ၊ ယံ ပရိမာဏမဿ ယတ္တကံ၊ တတ္တကံ၊ ဧတ္တကံ၊ အာဝတကေ ယာဝတကော၊ တာဝတကော (ဧတာဝတကော)။}

\sutta{479}{43}{သဗ္ဗာ စာဝန္တု။}
\vutti{သဗ္ဗတော ပဌမန္တေဟိ ယာဒီဟိ စ အဿေတိ ဆဋ္ဌိယတ္ထေ အာဝန္တု ဟောတိ၊ တဉ္စေ ပဌမန္တံ ပရိမာဏံ ဘဝတိ။ သဗ္ဗံ ပရိမာဏမဿ သဗ္ဗာဝန္တံ၊ ယာဝန္တံ၊ တာဝန္တံ၊ ဧတာဝန္တံ။}

\sutta{480}{44}{ကိမှာ ရတိရီဝရီဝတကရိတ္တကာ။}
\vutti{ကိမှာ ပဌမန္တာ အဿေတိ ဆဋ္ဌိယတ္ထေ ရတိရီဝရီဝတကရိတ္တကာ ဟောန္တိ၊ တဉ္စေ ပဌမန္တံ ပရိမာဏံ ဘဝတိ၊ ကိံ သင်္ချာနံ ပရိမာဏမေသံ ကတိ ဧတေ၊ ကီဝ၊ ကီဝတကံ၊ ကိတ္တကံ။ ရီဝန္တော သဘာဝတော အသင်္ချော။}

\sutta{481}{45}{သဉ္ဇာတံ တာရကာဒိတွိတော။}
\vutti{တာရကာဒီဟိ ပဌမန္တေဟိ အဿေတိ ဆဋ္ဌိယတ္ထေ ဣတော ဟောတိ၊ တေ စေ သဉ္ဇာတာ ဟောန္တိ၊ တာရကာ သဉ္ဇာတာ အဿ တာရကိတံ ဂဂနံ၊ ပုပ္ဖိတော ရုက္ခော၊ ပလ္လဝိတာ လတာ။}

\sutta{482}{46}{မာနေ မတ္တော။}
\vutti{ပဌမန္တာ မာနဝုတ္တိတော အဿေတိ အသ္မိံအတ္ထေ မတ္တော ဟောတိ၊ ပလံ ဥမ္မာနမဿ ပလမတ္တံ၊ ဟတ္ထော ပမာဏမဿ ဟတ္ထမတ္တံ သတံ မာနမဿ သတမတ္တံ၊ ဒေါဏော ပရိမာဏမဿ ဒေါဏမတ္တံ၊ အဘေဒေါပစာရာ ဒေါဏောတိပိ ဟောတိ။}

\sutta{483}{47}{တဂ္ဃော စုဒ္ဓံ။}
\vutti{ဥဒ္ဓမာနဝုတ္တိဟော အဿေတိ ဆဋ္ဌိယတ္ထေ တဂ္ဃော ဟောတိ မတ္တော စ၊ ဇဏ္ဏုတဂ္ဃံ၊ ဇဏ္ဏုမတ္တံ။}

\sutta{484}{48}{ဏော စ ပုရိသာ။}
\vutti{ပုရိသာ ပဌမန္တာ ဥဒ္ဓမာနဝုတ္တိတော ဏော ဟောတိ မတ္တာဒယော စ၊ ပေါရိသံ၊ ပုရိသမတ္တံ၊ ပုရိသတဂ္ဃံ။}

\sutta{485}{49}{အယူဘဒွိတီဟံသေ။}
\vutti{ဥဘဒွိတီဟိ အဝယဝဝုတ္တီတိ ပဌမန္တေဟိ အဿေတိ ဆဋ္ဌိယတ္ထေ အယော ဟောတိ။ ဥဘော အံသာ အဿ ဥဘယံ၊ ဒွယံ၊ တယံ။}

\sutta{486}{50}{သင်္ချာယ သစ္စုတီသာသဒသန္တာယာဓိကာသ္မိံ သဝသဟဿေ ဍော။}
\vutti{သတျန္တာယ ဥတျန္တာယ ဤသန္တာယ အာသန္တာယ ဒသန္တာယ စ သင်္ချာယ ပဌမန္တာယ အသ္မိန္တိ သတ္တမျတ္ထေ ဍော ဟောတိ၊ သာ စေ သင်္ချာ အဓိကာ ဟောတိ၊ ယဒသ္မိန္တိ တဉ္စေ သတံ သဟဿံ သတသဟဿံ ဝါ ဟောတိ၊ ဝီသတိ အဓိကာ အသ္မိံ သတေတိ ဝီသံ သတံ၊ ဧကဝီသံ သတံ၊ သဟဿံ၊ သတသဟဿံ ဝါ၊ တိံသံ သတံ၊ ဧကတိံသံ သတံ။ ဥတျန္တာယ-နဝုတံ သတံ သဟဿံ သတသဟဿံ ဝါ။ ဤသန္တာယ စတ္တာလီသံ သတံ၊ သဟဿံ၊ သတသဟဿံ ဝါ။ အာသန္တာယ ပညာသံ သတံ၊ သဟဿံ၊ သတသဟဿံ ဝါ။ ဒသန္တာယဧကာဒသံ သတံ၊ သဟဿံ၊ သတသဟဿံ ဝါ။ သစ္စုတီသာသဒသန္တာယာတိ ကိံ? ဆာဓိကာ အသ္မိံသတေ။ အဓိကေတိ ကိံ? ပဉ္စဒသဟီနာ အသ္မိံသတေ၊ အသ္မိန္တိ ကိံ? ဝီသတျဓိကာ ဧတသ္မာ သတာ၊ သတသဟဿေတိ ကိံ? ဧကာဒသ အဓိကာ အဿံ ဝီသတိယံ။}

\sutta{487}{51}{တဿ ပူရဏေကာဒသာဒိတော ဝါ။}
\vutti{ဆဋ္ဌိယန္ထာယေကာဒသာဒိကာယ သင်္ချာယ ဍော ဟောတိ (တဿ) ပူရဏတ္ထေ ဝိဘာသာ၊ သာ သင်္ချာ ပူရီယတေ ယေန တံ ပူရဏံ၊ ဧကာဒသန္နံ ပူရဏော ဧကာဒသော။ ဧကာဒသမော၊ ဝီသော၊ ဝီသတိမော၊ တိံသော၊ တိံသတိမော၊ စတ္တာလီသော၊ ပညာသော။}

\sutta{488}{52}{မ ပဉ္စာဒိကတီဟိ။}
\vutti{ဆဋ္ဌိယန္တာယ ပဉ္စာဒိကာယ သင်္ချာယ ကတိသ္မာ စ မော ဟောတိ (တဿ) ပူရဏတ္ထေ၊ ပဉ္စမော၊ သတ္တမော၊ အဋ္ဌမော၊ ကတိမော၊ ကတိမီ။}

\sutta{489}{53}{သတာဒီနမိစ။}
\vutti{သတာဒိကာယ သင်္ချာယ ဆဋ္ဌိယန္တာယ (တဿ) ပူရဏတ္ထေ မောဟောတိ သတာဒီနမိစာန္တာဒေသော၊ သတိမော၊ သဟဿိမော။}

\sutta{490}{54}{ဆာ ဋ္ဌဋ္ဌမာ။}
\vutti{ဆသဒ္ဒါ ဋ္ဌဋ္ဌမာ ဟောန္တိ တဿ ပူရဏတ္ထေ၊ ဆဋ္ဌော၊ ဆဋ္ဌမော၊ ဣတ္ထိယံ ဆဋ္ဌီ၊ ဆဋ္ဌမီ၊ ကထံ ‘ဒုတိယံ စတုတ္ထ’န္တိ? “ဒုတိယဿ၊ စတုတ္ထ တတိယာန”န္တိ နိပါတနာ။}

\sutta{491}{55}{ဧကာ ကာကျသဟာယေ။}
\vutti{ဧကသ္မာ အသဟာယတ္ထေ ကအာကီ ဟောန္တိ ဝါ၊ ဧကကော၊ ဧကာကီ၊ ဧကော။}

\sutta{492}{56}{ဝစ္ဆာဒီဟိ တနုတ္တေ တရော။}
\vutti{ဝစ္ဆာဒီနံ သဘာဝဿ တနုတ္တေ ဂမျမာနေ တရော ဟောတိ၊ သုသုတ္တဿ တနုတ္တေ ဝစ္ဆတရော၊ ဣတ္ထိယံ ဝစ္ဆတရီ၊ ယောဗ္ဗနဿ တနုတ္တေ ဩက္ခတရော၊ အဿဘာဝဿ တနုတ္တေ အဿတရော၊ သာမတ္ထိယဿ တနုတ္တေ ဥသဘတရော။}

\sutta{493}{57}{ကိမှာ နိဒ္ဓါရဏေ ရတရ ရတမာ။}
\vutti{ကိံသဒ္ဒါ နိဒ္ဓါရဏေ ရတရ ရတမာ ဟောန္တိ၊ ကတရော ဘဝတံ ဒေဝဒတ္တော၊ ကတရော ဘဝတံ ကဌော၊ ကတမော ဘဝတံ ဒေဝဒတ္တော၊ ကတမော ဘဝတံ ကဌော၊ ဘာရဒွါဇာနံ ကတမောသိ ဗြဟ္မေ။}

\sutta{494}{58}{တေန ဒတ္တေ လိယာ။}
\vutti{တတိယန္တာ ဒတ္တေ-ဘိဓေယျေ လဣယာ ဟောန္တိ၊ ဒေဝေန ဒတ္တော ဒေဝလော၊ ဒေဝိယော၊ ဗြဟ္မလော၊ ဗြဟ္မိယော၊ သိဝါ သီဝလော၊ သီဝိယော၊ သိဿ ဒီဃော။}

\sutta{495}{59}{တဿ ဘာဝကမ္မေသု တ္တ တာ တ္တန ဏျ ဏေယျဏိယ ဏိယာ။}
\vutti{ဆဋ္ဌိယန္တာ ဘာဝေ ကမ္မေ စ တ္တာဒယော ဟောန္တိ ဗဟုလံ၊ န စ သဗ္ဗေ သဗ္ဗတော ဟောန္တိ အညတြ တ္တတာဟိ၊ ဘဝန္တိ ဧတသ္မာ ဗုဒ္ဓိသဒ္ဒါဘိ ဘာဝေါ သဒ္ဒဿ ပဝတ္တိနိမိတ္တံ၊ နီလဿ ပဋဿ ဘာဝေါ နီလတ္တံ နီလတာဘိ ဂုဏော ဘာဝေါ၊ နီလဿ ဂုဏဿ ဘာဝေါ နီလတ္တံ နီလတာဘိ နီလဂုဏဇာတိ၊ ဂေါတ္တံ ဂေါတာတိ ဂေါဇာတိ၊ ပါစကတ္တံ၊ ဒဏ္ဍိတ္တံ၊ ဝိသာဏိတ္တံ၊ ရာဇပုရိသတ္တန္တိ ကြိယာဒိသမ္ဗန္ဓိတ္တံ၊ ဒေဝဒတ္တတ္တံ၊ စန္ဒတ္တံ၊ သူရိယတ္တန္တိ တဒဝတ္ထာဝိသေသသာမညံ၊ အာကာသတ္တံ၊ အဘာဝတ္တန္တိ ဥပစရိတဘေဒသာမညံ။ တ္တန-ပုထုဇ္ဇနတ္တနံ၊ ဝေဒနတ္တနံ၊ ဇာယတ္တနံ၊ ဇာရတ္တနံ။ ဏျ-အာလသျံ၊ ဗြဟ္မညံ၊ စာပလျံ၊ နေပုညံ၊ ပေသုညံ၊ ရဇ္ဇံ၊ အာဓိပစ္စံ၊ ဒါယဇ္ဇံ၊ ဝေသမ္မံ ‘ဝေသမ’န္တိ၊ ကေစိ၊ သချံ၊ ဝါဏိဇ္ဇံ။ ဏေယျ-သောစေယျံ၊ အာဓိပတေယျံ။ ဏဂါရဝံ၊ ပါဋဝံ၊ အဇ္ဇဝံ၊ မဒ္ဒဝံ။ ဣယ-အဓိပတိယံ၊ ပဏ္ဍိတိယံ၊ ဗဟုဿုတိယံ၊ နဂ္ဂိယံ၊ သူရိယံ။ ဏိယ-အာလသိယံ၊ ကာဠုသိယံ၊ မန္ဒိယံဒက္ခိယံ၊ ပေါရောဟိတိယံ၊ ဝေယျတ္တိယံ။ ကထံ ‘ရာမဏီယက’န္တိ? သကတ္ထေ ကန္တာ ဏေန သိဒ္ဓံ။ ကမ္မံ ကိရိယာ၊ တတ္ထ အလသဿ ကမ္မံ အလသတ္တံ’ အလသတာ၊ အလသတ္တနံ၊ အာလသျံ၊ အာလသိယံ ဝါ၊ ‘သကတ္ထေ’ (၄.၁၂၂) တိ သကတ္ထေပိ၊ ယထာဘုစ္စံ၊ ကာရုညံ၊ ပတ္တကလ္လံ၊ အာကာသာနဉ္စံ၊ ကာယပါဂုညတာ။}

\sutta{496}{60}{ဗျ ဝဒ္ဓဒါသာ ဝါ။}
\vutti{ဆဋ္ဌိယန္တာ ဝဒ္ဓါ ဒါသာ စ ဗျော ဝါ ဟောတိ ဘာဝကမ္မေသု၊ ဝဒ္ဓဗျံ ဝဒ္ဓတာ၊ ဒါသဗျံ ဒါသတာ၊ ကထံ ‘ဝဒ္ဓဝ’န္တိ? ဏေ ဝါဂမော။}

\sutta{497}{61}{နဏ ယုဝါ ဗော စ ဝဿ။}
\vutti{ဆဋ္ဌိယန္တာ ယုဝသဒ္ဒါ ဘာဝကမ္မေသု နဏ ဝါ ဟောတိ တဿ ဗော စ၊ ယောဗ္ဗနံ၊ ဝါတွေဝ? ယုဝတ္တံ၊ ယုဝတာ။}

\sutta{498}{62}{အဏွာဒိတွိမော။}
\vutti{အဏုအာဒီဟိ ဆဋ္ဌိယန္တေဟိ ဘာဝေ ဝါ ဣမော ဟောတိ၊ အဏိမာ၊ လဃိမာ၊ မဟိမာ၊ (ဂရိမာ)၊ ကသိမာ၊ ဝါတွေဝ? အဏုတ္တံ အဏုတာ။}

\sutta{499}{63}{ဘာဝါ တေန နိဗ္ဗတ္တေ။}
\vutti{ဘာဝဝါစကာ သဒ္ဒါ တေန နိဗ္ဗတ္တေ-ဘိဓေယျေ ဣမော ဟောတိ၊ ပါကေန နိဗ္ဗတ္တ ပါဏိမံ၊ သေကိမံ။}

\sutta{500}{64}{တရတမိဿိကိယိဋ္ဌာတိသယေ။}
\vutti{အတိသယေ ဝတ္တမာနတော ဟောန္တေတေ ပစ္စယာ၊ အတိသယေန ပါပေါ ပါပတရော၊ ပါပတမော၊ ပါပိဿိကော၊ ပါပိယော၊ ပါပိဋ္ဌော၊ ဣတ္ထိယံ ပါပတရာ။ အတိသယန္တာပိ အတိသယပ္ပစ္စယော၊ အတိသယေန ပါပိဋ္ဌော ပါပိဋ္ဌတရော။}

\sutta{501}{65}{တန္နိဿိတေ လ္လော။}
\vutti{ဒုတိယန္တာ လ္လပ္ပစ္စယော ဟောတိ နိဿိတတ္ထေ၊ ဝေဒံ နိဿိတံ ဝေဒလ္လံ၊ ဒုဋ္ဌု နိဿိတံ ဒုဋ္ဌုလ္လံ။ ဣလ္လေ-သင်္ခါရိလ္လံ။}

\sutta{502}{66}{တဿ ဝိကာရာဝယဝေသု ဏ ဏိက ဏေယျမယာ။}
\vutti{ပကတိယာ ဥတ္တရမဝတ္ထန္တရံ ဝိကာရော၊ ဆဋ္ဌိယန္တာ နာမသ္မာ ဝိကာရေ-ဝယဝေ စ ဏာဒယော ဟောန္တိ ဗဟုလံ၊ ဏ-အာယသံ ဗန္ဓနံ၊ ဩဒုမ္ဗရံ၊ ပဏ္ဏံ၊ ဩဒုမ္ဗရံ ဘသ္မံ၊ ကာပေါတံ မံသံ၊ ကာပေါတံ သတ္ထိ။ ဏိက-ကပ္ပာသိကံ ဝတ္ထံ။ ဏေယျ-ဧဏေယျံ မံသံ၊ ဧဏေယျံ သတ္ထိ ။ ကောသေယျံ ဝတ္ထံ။ မယ-တိဏမယံ၊ ဒါရုမယံ၊ နဠမယံ၊ မတ္တိကာမယံ။ “အညသ္မိ”န္တိ (၄.၁၂၁) ဂုန္နံ ကရီသေပိ မယော၊ ဂေါမယံ။}

\sutta{503}{67}{ဇတုတော သဏ ဝါ။}
\vutti{ဆဋ္ဌိယန္တာ နာမသ္မာ ဇတုတော ဝိကာရာဝယဝေသု သဏ ဝါ ဟောတိ။ ဇတုနော ဝိကာရော ဇာတုသံ ဇတုမယံ။ “လောပေါ”တိ (၄.၁၂၃) ဗဟုလံ ပစ္စယလောပေါပိ ဖလပုပ္ဖမူလေသု ဝိကာရာဝယဝေသု၊ ပိယာလဿ ဖလာနိ ပိယာလာနိ၊ မလ္လိကာယ ပုပ္ဖာနိ မလ္လိကာ၊ ဥသိရဿ မူလံ ဥသီရံ၊ တံ သဒ္ဒေန ဝါ တဒဘိဓာနံ။}

\sutta{504}{68}{သမူဟေ ကဏ ဏ ဏိကာ။}
\vutti{ဆဋ္ဌိယန္တာ သမူဟေ ကဏ ဏ ဏိကာ ဟောန္တိ ဂေါတ္တပ္ပစ္စယန္တာ။ ကဏ-ရာဇညကံ၊ မာနုဿကံ၊ ဥက္ခာဒီဟိ-ဩက္ခကံ၊ ဩဋ္ဌကံ၊ ဩရဗ္ဘကံ၊ ရာဇကံ၊ ရာဇပုတ္တကံ၊ ဟတ္ထိကံ၊ ဓေနုကံ။ ဏ-ကာကံ၊ ဘိက္ခံ။ အစိတ္တာ ဏိက-အာပူပိကံ၊ သံကုလိကံ။}

\sutta{505}{69}{ဇနာဒီဟိ တာ။}
\vutti{ဇနာဒီဟိ ဆဋ္ဌိယန္တေဟိ သမူဟေ တာ ဟောတိ။ ဇနတာ၊ ဂဇတာ၊ ဗန္ဓုတာ၊ ဂါမတာ၊ သဟာယတာ၊ နာဂရတာ။ တာန္တာ သဘာဝတော ဣတ္ထိလိင်္ဂါ၊ ‘မဒနီယ’န္တိ ကရဏေ-ဓိကရဏေ ဝါ အနီယေန သိဒ္ဓံ။ ‘ဓူမာယိတတ္တ’န္တိ က္တာန္တာ နာမဓာတုတော က္တေန သိဒ္ဓံ။}

\sutta{506}{70}{ဣယော ဟိတေ။}
\vutti{ဆဋ္ဌိယန္တာ ဟိတေ ဣယော ဟောတိ။ ဥပါဒါနိယံ၊ အညတြာပိ သမာနောဒရေ သယိတော သောဒရိယော။}

\sutta{507}{71}{စက္ခွာဒိတော ဿော။}
\vutti{ဆဋ္ဌိယန္တေဟိ စက္ခုအာဒီဟိ ဟိတေ ဿော ဟောတိ၊ စက္ခုဿံ၊ အာယုဿံ။}

\sutta{508}{72}{ဏျော တတ္ထ သာဓု။}
\vutti{သတ္တမျန္တာ တတ္ထ သာဓူတိ အသ္မိံ အတ္ထေ ဏျော ဟောတိ။ သဗ္ဘော၊ ပါရိသဇ္ဇော။ သာဓူတိ ကုသလော ယောဂ္ဂေါ ဟိတော ဝါ။ အညတြာပိ ရထံ ဝဟတီတိ ရစ္ဆာ။}

\sutta{509}{73}{ကမ္မာ နိယ ညာ။}
\vutti{သတ္တမျန္တာ ကမ္မသဒ္ဒါ တတ္ထ သာဓူဘိ အသ္မိံ အတ္ထေ နိယ ညာ ဟောန္တိ။ ကမ္မေ သာဓု ကမ္မနိယံ၊ ကမ္မညံ။}

\sutta{510}{74}{ကထာဒိတွိကော။}
\vutti{ကထာဒီဟိ သတ္တမျန္တေဟိ တတ္ထ သာဓူတိ အသ္မိံ အတ္ထေ ဣကော ဟောတိ။ ကထိကော၊ ဓမ္မကထိကော၊ သင်္ဂါမိကော ပဝါသိကော၊ ဥပဝါသိကော။}

\sutta{511}{75}{ပထာဒီဟိ ဏေယျော။}
\vutti{ပထာဒီဟိ သတ္တမျန္တေဟိ တတ္ထ သာဓူတိ အသ္မိံ အတ္ထေ ဏေ-ယျော ဟောတိ၊ ပါထေယျံ သာပတေယျံ (အာတိထေယျံ)။}

\sutta{512}{76}{ဒက္ခိဏာယာရဟေ။}
\vutti{ဒက္ခိဏာသဒ္ဒတော အရဟတ္ထေ ဏေယျော ဟောတိ၊ ဒက္ခိဏံ အရဟတီတိ ဒက္ခိဏေယျော။}

\sutta{513}{77}{ရာယော တုမန္တာ။}
\vutti{တုမန္တတော အရဟတ္ထေ ရာယော ဟောတိ။ ဃာတေတာယံ ဝါ ဃာတေတုံ၊ ဇာပေတာယံ ဝါ ဇာပေတုံ၊ ပဗ္ဗာဇေတာယံ ဝါ ပဗ္ဗာဇေတုံ။}

\sutta{514}{78}{တမေတ္ထဿတ္ထီတိ မန္တု။}
\vutti{ပဌမန္တာ ဧတ္ထ အဿ အတ္ထီတိ ဧတေသွတ္ထေသု မန္တု ဟောတိ။ ဂါဝေါ ဧတ္ထ ဒေသေ၊ အဿ ဝါ ပုရိသဿ သန္တီတိ ဂေါမာ။ အတ္ထီတိ ဝတ္တမာနကာလောပါဒါနတော ဘူတာဟိ ဘဝိဿန္တီဟိ ဝါ ဂေါဟိ န ဂေါမာ။ ကထံ ‘ဂေါမာ အာသိ၊ ဂေါမာ ဘဝိဿတီ’တိ? တဒါပိ ဝတ္တမာနာဟိယေဝ ဂေါဟိ ဂေါမာ၊ အာသိ ဘဝိဿတီတိ ပဒန္တရာ ကာလန္တရံ၊ ဣတိကရဏတော ဝိသယနိယမော –
ပဟူတေ စ ပသံသာယံ၊ နိန္ဒာယဉ္စာတိသာယနေ။
နိစ္စယောဂေ စ သံသဂ္ဂေ၊ ဟောန္တိမေ မန္တုအာဒယော။
ဂေါ အဿောတိ ဇာတိသဒ္ဒါနံ ဒဗ္ဗာဘိဓာနသာမတ္ထိယာ မန္တွာဒယော န ဟောန္တိ၊ တထာ ဂုဏသဒ္ဒါနံ ‘သေတော ပဋော’တိ၊ ယေသန္တု ဂုဏသဒ္ဒါနံ ဒဗ္ဗာဘိဓာနသာမတ္ထိယံ နတ္ထိ၊ တေဟိ ဟောန္တေဝ ‘ဗုဒ္ဓိမာ၊ ရူပဝါ၊ ရသဝါ၊ ဂန္ဓဝါ၊ ဖဿဝါ၊ သဒ္ဒဝါ၊ ရသီ၊ ရသိကော၊ ရူပီ၊ ရူပိကော၊ ဂန္ဓီ၊ ဂန္ဓိကော’တိ။}

\sutta{515}{79}{ဝန္တွာဝဏ္ဏာ။}
\vutti{ပဌမန္တတော အဝဏ္ဏန္တာ မန္တွာတ္ထေ ဝန္တု ဟောတိ။ သီလဝါ၊ ပညဝါ၊ အဝဏ္ဏာတိ ကိံ? သတိမာ ဗန္ဓုမာ။}

\sutta{516}{80}{ဒဏ္ဍာဒိတွိကဤ ဝါ။}
\vutti{ဒဏ္ဍာဒီဟိ ဣက ဤ ဟောန္တိ ဝါ မန္တွာတ္ထေ။ ဗဟုလံ ဝိဓာနာ ကုတောစိ ဒွေ ဟောန္တိ၊ ကုတောစေကမေကံဝ ဒဏ္ဍိကော ဒဏ္ဍီ ဒဏ္ဍဝါ၊ ဂန္ဓိကော ဂန္ဓီ ဂန္ဓဝါ၊ ရူပိကော ရူပီ ရူပဝါ။ (၂၈) “\suttagananormal{517}{28}{ဥတ္တမီဏေ ဝ ဓနာ ဣကော။}”၊ ဓနိကော၊ ဓနီ ဓနဝါ အညော။ (၂၉) “\suttagananormal{518}{29}{အသန္နိဟိတေ အတ္ထာ။}”၊ အတ္ထိကော အတ္ထီ၊ အညတြ အတ္ထဝါ။ (၃၀) “\suttagananormal{519}{30}{တဒန္တာ စ။}”၊ ပုညတ္ထိကော၊ ပုညတ္ထီ၊ (၃၁) “\suttagananormal{520}{31}{ဝဏ္ဏန္တာ ဤယေဝ။}” ဗြဟ္မဝဏ္ဏီ၊ ဒေဝဝဏ္ဏီ၊ (၃၂) “\suttagananormal{521}{32}{ဟတ္ထဒန္တေဟိ ဇာတိယံ။}”၊ ဟတ္ထီ၊ ဒန္တီ၊ အညတြ ဟတ္ထဝါ ဒန္တဝါ။ (၃၃) “\suttagananormal{522}{33}{ဝဏ္ဏတော ဗြဟ္မစာရိမှိ။}”၊ ဝဏ္ဏီ ဗြဟ္မစာရီ၊ ဝဏ္ဏဝါ အညော။ (၃၄) “\suttagananormal{523}{34}{ပေါက္ခရာဒိတော ဒေသေ။}”၊ ပေါက္ခရဏီ၊ ဥပ္ပလိနီ၊ ကုမုဒိနီ၊ ဘိသိနီ၊ မုဠာလိနီ၊ သာလုကိနီ၊ ကွစာဒေသေပိ ပဒုမိပိ ပဒုမိနီ ပဏ္ဏံ။ အညတြ ပေါက္ခရဝါ ဟတ္ထီ၊ (၃၅) “\suttagananormal{524}{35}{နာဝါယိ-ကော။}”၊ နာဝိကော။ (၃၆) “\suttagananormal{525}{36}{သုခဒုက္ခာ ဤ။}”၊ သုခီ၊ ဒုက္ခီ။ (၃၇) “\suttagananormal{526}{37}{သိခါဒီဟိ ဝါ။}”၊ သိခီ၊ သိခါဝါ၊ မာလီ၊ မာလာဝါ၊ သီလီ၊ သီလဝါ၊ ဗလီ၊ ဗလဝါ။ (၃၈) “\suttagananormal{527}{38}{ဗလာ ဗာဟူရုပုဗ္ဗာ စ။}”၊ ဗာဟုဗလီ၊ ဦရုဗလီ။}

\sutta{528}{81}{တပါဒီဟိ ဿီ။}
\vutti{တပါဒိတော မန္တွတ္ထေ ဝါ ဿီ ဟောတိ။ တပဿီ၊ ယသဿီ၊ တေဇဿီ၊ မနဿီ၊ ပယဿီ။ ဝါတွေဝ? ယသဝါ။}

\sutta{529}{82}{မုခါဒိတော ရော။}
\vutti{မုခါဒီဟိ မန္တွတ္ထေရော ဟောတိ။ မုခရော၊ သုသိရော၊ ဦသရော၊ မဓုရော၊ ခရော၊ ကုဉ္ဇရော၊ နဂရံ၊ (၃၉) “\suttagananormal{530}{39}{ဒန္တဿု စ ဥန္နတဒန္တေ။}”၊ ဒန္တုရော။}

\sutta{531}{83}{တုန္ဒျာဒီဟိ ဘော။}
\vutti{တုန္ဒိအာဒီဟိ မန္တွတ္ထေ ဘော ဝါ ဟောတိ။ တုန္ဒိဘော၊ ဝဋိဘော၊ ဝလိဘော။ ဝါတွေဝ? တုန္ဒိမာ။}

\sutta{532}{84}{သဒ္ဓါဒိတွာ။}
\vutti{သဒ္ဓါဒီဟိ မန္တွတ္ထေအ ဟောတိ ဝါ။ သဒ္ဓေါ၊ ပညော၊ ဣတ္ထိယံ သဒ္ဓါ။ ဝါတွေဝ? ပညဝါ။}

\sutta{533}{85}{ဏော တပါ။}
\vutti{တပါ ဏော ဟောတိ မန္တွတ္ထေ။ တာပသော၊ ဣတ္ထိယံ တာပသီ။}

\sutta{534}{86}{အာလွဘိဇ္ဈာဒီဟိ။}
\vutti{အဘိဇ္ဈာဒီဟိ အာလု ဟောတိ မန္တွတ္ထေ၊ အဘိဇ္ဈာလု၊ သီတာလု၊ ဓဇာလု၊ ဒယာလု။ ဝါတွေဝ? ဒယာဝါ။}

\sutta{535}{87}{ပိစ္ဆာဒိတွိလော။}
\vutti{ပိစ္ဆာဒီဟိ ဣလော ဟောတိ ဝါ မန္တွတ္ထေ။ ပိစ္ဆိလော ပိစ္ဆဝါ၊ ဖေနိလော ဖေနဝါ၊ ဇဋိလော ဇဋာဝါ။ ကထံ ‘ဝါစာလော’တိ? နိန္ဒာယမိလဿာဒိလောပေ “ပရော ကွစီ”တိ (၁-၂၇)။}

\sutta{536}{88}{သီလာဒိတော ဝေါ။}
\vutti{သီလာဒီဟိ ဝေါ ဟောတိ ဝါ မန္တွတ္ထေ။ သီလဝေါ သီလဝါ၊ ကေသဝေါ ကေသဝါ၊ (၄၀) “\suttagananormal{537}{40}{အဏ္ဏာ နိစ္စံ။}” အဏ္ဏဝေါ။ (၄၁) \suttagananormal{538}{41}{ဂါဏ္ဍိရာဇီဟိ သညာယံ။}”ဂါဏ္ဍီဝံ ဓနု၊ ရာဇီဝံ ပင်္ကဇံ။}





\sutta{539}{89}{မာယာ မေဓာဟိ ဝီ။}
\vutti{ဧတေဟိ ဒွီဟိ ဝီ ဟောတိ မန္တွတ္ထေ။ မာယာဝီ၊ မေဓာဝီ။}

\sutta{540}{90}{သိဿရေ အာမျုဝါမီ။}
\vutti{သသဒ္ဒါ အာမျုဝါမီ ဟောန္တိ ဣဿရေ-ဘိဓာယျေ မန္တွတ္ထေ။ သမဿတ္ထီတိ သာမီ၊ သုဝါမီ။}

\sutta{541}{91}{လက္ချာ ဏော အ စ။}
\vutti{လက္ခီသဒ္ဒါ ဏော ဟောတိ မန္တွတ္ထေ အ စာန္တဿ။ ဏကာရောဝယဝေါ၊ လက္ခဏော။}

\sutta{542}{92}{အင်္ဂါ နော ကလျာဏေ။}
\vutti{ကလျာဏေ ဂမျမာနေ အင်္ဂသ္မာ နော ဟောတိ မန္တွတ္ထေ။ အင်္ဂနာ။}

\sutta{543}{93}{သော လောမာ။}
\vutti{လောမာ သော ဟောတိ မန္တွတ္ထေ။ လောမသော၊ ဣတ္ထိယံ၊ လောမသာ။}

\sutta{544}{94}{ဣမိယာ။}
\vutti{မန္တွတ္ထေ ဣမ ဣယာ ဟောန္တိ ဗဟုလံ။ ပုတ္တိမော၊ ကိတ္တိမော၊ ပုတ္တိယော၊ ကပ္ပိယော၊ ဇဋိယော၊ ဟာနဘာဂိယော၊ သေနိယော။}

\sutta{545}{95}{တော ပဉ္စမျာ။}
\vutti{ပဉ္စမျန္တာ ဗဟုလံ တော ဟောတိ ဝါ။ ဂါမတော အာဂစ္ဆတိ ဂါမသ္မာ အာဂစ္ဆတိ၊ စောရတော ဘာယတိ စောရေဟိ ဘာယတိ၊ သတ္ထတော ပရိဟီနော သတ္ထာ ပရိဟီနော။}

\sutta{546}{96}{ဣတော-တေတ္တော ကုတော။}
\vutti{တောမှိ ဣမဿ ဋိ နိပစ္စတေ၊ ဧတဿ ဋ ဧတ၊ ကိံသဒ္ဒဿ ကုတ္တဉ္စ။ ဣတော ဣမသ္မာ၊ အတော ဧတ္တော ဧတသ္မာ၊ ကုတော ကသ္မာ။}

\sutta{547}{97}{အဘျာဒီဟိ။}
\vutti{အဘိအာဒီဟိ တော ဟောတိ။ အဘိတော၊ ပရိတော၊ ပစ္ဆတော ဟေဋ္ဌတော။}

\sutta{548}{98}{အာဒျာဒီဟိ။}
\vutti{အာဒိပ္ပဘုတီဟိ တော ဝါ ဟောတိ၊ အာဒေါ အာဒိတော၊ မဇ္ဈတော အန္တတော၊ ပိဋ္ဌိတော၊ ပဿတော၊ မုခတော၊ ယတောဒကံ တဒါဒိတ္တံ၊ ယံ ဥဒကံ တဒေဝါဒိတ္တန္တိ အတ္ထော။}

\sutta{549}{99}{သဗ္ဗာဒိတော သတ္တမျာ တြတ္တာ။}
\vutti{သဗ္ဗာဒီဟိ သတ္တမျန္တေဟိ တြတ္ထာ ဝါ ဟောန္တိ။ သဗ္ဗတြ သဗ္ဗ။ သဗ္ဗသ္မိံ၊ ယတြ ယတ္ထ ယသ္မိံ။ ဗဟုလာဓိကာရာ န တုမှာမှေဟိ။}

\sutta{550}{100}{ကတ္ထေတ္ထ ကုတြာတြ ကွေဟိဓ။}
\vutti{ဧတေသဒ္ဒါ နိပစ္စန္တေ။ ကသ္မိံ ကတ္ထ၊ ကုတြ၊ ကွ၊ ဧတသ္မိံ၊ ဧတ္ထ၊ အတြ၊ အသ္မိံ ဣဟ၊ ဣဓ။}

\sutta{551}{101}{ဓိ သဗ္ဗာ ဝါ။}
\vutti{သတ္တမျန္တတော သဗ္ဗသ္မာ ဓိ ဝါ ဟောတိ။ သဗ္ဗဓိ၊ သဗ္ဗတ္ထ။}

\sutta{552}{102}{ယာ ဟိံ။}
\vutti{သတ္တမျန္တတော ယတော ဟိံ ဝါ ဟောတိ။ ယဟိံ ယတြ။}

\sutta{553}{103}{တာ ဟံ စ။}
\vutti{သတ္တမျန္တတော တတော ဝါ ဟံ ဟောတိ ဟိံ စ။ တဟံ၊ တဟိံ၊ တတြ။}

\sutta{554}{104}{ကုဟိံ ကဟံ။}
\vutti{ကိံသဒ္ဒါ သတ္တမျန္တာ ဟိံ ဟံ နိပစ္စန္တေ ကိဿ ကုကာ စ။ ကုဟိံ၊ ကဟံ။ ကထံ ‘ကုဟိဉ္စန’န္တိ? ‘စနံ’ ဣတိ နိပါတန္တရံ ‘ကုဟိဉ္စီ’တိ ဧတ္ထ စိသဒ္ဒေါ ဝိယ။}

\sutta{555}{105}{သဗ္ဗေကညယတေဟိ ကာလေ ဒါ။}
\vutti{ဧတေဟိ သတ္တမျန္တေဟိ ကာလေ ဒါ ဟောတိ။ သဗ္ဗသ္မိံ ကာလေ သဗ္ဗဒါ၊ ဧကဒါ၊ အညဒါ၊ ယဒါ၊ တဒါ။ ကာလေတိ ကိံ? သဗ္ဗတ္ထ ဒေသေ။}

\sutta{556}{106}{ကဒါ ကုဒါ သဒါ ဓုနေဒါနိ။}
\vutti{ဧတေ သဒ္ဒါ နိပစ္စန္တေ။ ကသ္မိံ ကာလေ ကဒါ၊ ကုဒါ၊ သဗ္ဗသ္မိံ ကာလေ သဒါ၊ ဣမသ္မိံ ကာလေ အဓုနာ၊ ဣဒါနိ။}

\sutta{557}{107}{အဇ္ဇ သဇ္ဇွပရဇ္ဇွေတရဟိ ကရဟာ။}
\vutti{ဧတေသဒ္ဒါ နိပစ္စန္တေ။ ပကတိပ္ပစ္စယော အာဒေသော ကာလဝိသေသောတိ သဗ္ဗမေတံ နိပါတနာ လဗ္ဘတိ၊ ဣမဿ ဋော ဇ္ဇော ဇာဟနိ နိပစ္စတေ၊ အသ္မိံ အဟနိ အဇ္ဇ။ သမာနဿ သဘာဝေါ ဇ္ဇု စာဟနိ၊ သမာနေ အဟနိ သဇ္ဇု။ အပရသ္မာ ဇ္ဇု၊ အပရသ္မိံ အဟနိ အပရဇ္ဇု။ ဣမဿေတော ကာလေ ရဟိ စ၊ ဣမသ္မိံ ကာလေ ဧတရဟိ။ ကိံသဒ္ဒဿ ကော ရဟ စာနဇ္ဇတနေ၊ ကသ္မိံ ကာလေ ကရဟ။}

\sutta{558}{108}{သဗ္ဗာဒီဟိ ပယာရေ ထာ။}
\vutti{သာမညဿ ဘေဒကော ဝိသေသော ပကာရော၊ တတ္ထ ဝတ္တမာနေဟိ သဗ္ဗာဒီဟိ ထာ ဟောတိ။ သဗ္ဗေန ပကာရေန သဗ္ဗထာ၊ ယထာ၊ တထာ။}

\sutta{559}{109}{ကထမိတ္ထံ။}
\vutti{ဧတေ သဒ္ဒါ နိပစ္စန္တေ ပကာရေ။ ကိမိမေဟိ ထံ ပစ္စယော၊ ကဣတ စ တေသံ ယထာက္ကမံ၊ ကထံ၊ ဣတ္ထံ။}

\sutta{560}{110}{ဓာ သင်္ချာဟိ။}
\vutti{သင်္ချာဝါစီဟိ ပကာရေ ဓာ ပရာ ဟောတိ။ ဒွီဟိ ပကာရေဟိ၊ ဒွေ ဝါ ပကာရေ ကရောတိ ဒွိဓာ ကရောတိ၊ ဗဟုဓာ ကရောတိ၊ ဧကံ ရာသိံ ပဉ္စပ္ပကာရံ ကရောတိ ပဉ္စဓာ ကရောတိ၊ ပဉ္စပ္ပကာရမေကပ္ပကာရံ ကရောတိ ဧကဓာ ကရောတိ။}

\sutta{561}{111}{ဝေကာ ဇ္ဈံ။}
\vutti{ဧကသ္မာ ပကရေ ဇ္ဈံ ဝါ ဟောတိ။ ဧကဇ္ဈံ ကရောတိ၊ ဧကဓာ၊ ကရောတိ။}

\sutta{562}{112}{ဒွိတီဟေဓာ။}
\vutti{ဒွိတီဟိ ပကာရေ ဧဓာ ဝါ ဟောတိ။ ဒွေဓာ၊ တေဓာ၊ ဒွိဓာ၊ တိဓာ။}

\sutta{563}{113}{တဗ္ဗတိ ဇာတိယော။}
\vutti{ပကာရဝတိ တံသာမညဝါစကာ သဒ္ဒါ ဇာတိယော ဟောတိ၊ ပဋုဇာတိယော၊ မုဒုဇာတိယော။}

\sutta{564}{114}{ဝါရသင်္ချာယ က္ခတ္တုံ။}
\vutti{ဝါရသမ္ဗန္ဓိနိယာ သင်္ချာယ က္ခတ္တုံ ဟောတိ။ ဒွေ ဝါရေ ဘုဉ္ဇတိ ဒွိက္ခတ္တုံ ဒိဝသဿ ဘုဉ္ဇတိ။ ဝါရဂ္ဂဟဏံ ကိံ? ပဉ္စ ဘုဉ္ဇတိ။ သင်္ချာယာတိ ကိံ? ပဟူတေ ဝါရေ ဘုဉ္ဇတိ။}

\sutta{565}{115}{ကတိမှာ။}
\vutti{ဝါရသမ္ဗန္ဓိနိယာ ကတိသင်္ချာယ က္ခတ္တုံ ဟောတိ၊ ကတိ ဝါရေ၊ ဘုဉ္ဇထိ၊ ကတိက္ခတ္တုံ ဘုဉ္ဇတိ။}

\sutta{566}{116}{ဗဟုမှာ ဓာ စ ပစ္စာသတ္တိယံ။}
\vutti{ဝါရသမ္ဗန္ဓိနိယာ ဗဟုသင်္ချာယ ဓာ ဟောတိ က္ခတ္ထုံ စ၊ ဝါရာနဉ္စေ ပစ္စာသတ္တိ ဟောတိ၊ ဗဟုဓာ ဒိဝသဿ ဘုဉ္ဇတိ ဗဟုက္ခတ္တုံ ဘုဉ္ဇတိ။ ပစ္စာသတ္တိယန္တိ ကိံ? ဗဟုဝါရေ မာသဿ ဘုဉ္ဇတိ။}

\sutta{567}{117}{သကိံ ဝါ။}
\vutti{ဧကံ ဝါရမိစ္စသ္မိံ အတ္ထေ သကိန္တိ ဝါ နိပစ္စတေ။ ဧကဝါရံ ဘုဉ္ဇတိ သကိံ ဘုဉ္ဇတိ၊ ဝါတိ ကိံ? ဧကက္ခတ္တုံ ဘုဉ္ဇတိ။}

\sutta{568}{118}{သော ဝီစ္ဆပ္ပကာရေသု။}
\vutti{ဝီစ္ဆာယံ ပကာရေ စ သော ဟောတိ ဗဟုလံ။ ဝီစ္ဆာယံ-ခဏ္ဍသော၊ ဗိလသော။ ပကာရေ-ပုထုသော၊ သဗ္ဗသော။}

\sutta{569}{119}{အဘူတတဗ္ဘာဝေ ကရာသဘူယောဂေ ဝိကာရာ စီ။}
\vutti{အဝတ္ထာဝတော-ဝတ္ထန္တရေနာဘူတဿ တာယာဝတ္ထာယ ဘာဝေကရာသဘူဟိ သမ္ဗန္ဓေ သတိ ဝိကာရဝါစကာ စီ ဟောတိ၊ အဓဝလံ ဓဝလံ ကရောတိ ဓဝလီ ကရောတိ၊ အဓဝလော ဓဝလော သိယာ ဓဝလီ သိယာ၊ အဓဝလော ဓဝလော ဘဝတိ ဓဝလီ ဘဝတိ။ အဘူတ တဗ္ဘာဝေတိ ကိံ? ဃဋံ ကရောတိ၊ ဒဓိ အတ္ထိ၊ ဃဋော ဘဝတိ။ ကရာသဘူယောဂေတိ ကိံ? အဓဝလောဓဝလော ဇာယတေ။ ဝိကာရာတိ ကိံ? ပကတိယာ မာဟောတု၊ သုဝဏ္ဏံ ကုဏ္ဍလံ ကရောတိ။}

\sutta{570}{120}{ဒိဿန္တညေပိ ပစ္စယာ။}
\vutti{ဝုတ္တတော-ညေပိ ပစ္စယာ ဒိဿန္တိ ဝုတ္တာဝုတ္တတ္ထေသု။ ဝိဝိဓာ မာတရော ဝိမာတရော၊ တာသံ ပုတ္တာ ဝေမာတိကာ-ရိကဏ။ ပထံ ဂစ္ဆတီတိ ပထာဝိနော-အာဝီ။ ဣသာ အဿ အတ္ထီတိ ဣဿုကီ-ဥကီ။ ဓုရံ ဝါဟတီတိ ဓောရယှော- ယှဏ။}

\sutta{571}{121}{အညသ္မိံ။}
\vutti{ဝုတ္တတော-ညသ္မိမ္ပိ အတ္ထေ ဝုတ္တပ္ပစ္စယာ ဒိဿန္တိ။ မဂဓာနံ ဣဿရော မာဂဓော-ဓဏာ။ ကာသီတိ သဟဿံ၊ တမဂ္ဃတီတိ ကာသျော ဣယော။}

\sutta{572}{122}{သကတ္ထေ။}
\vutti{သကတ္ထေပိ ပစ္စယာ ဒိဿန္တိ။ ဟီနကော၊ ပေါတကော၊ ကိစ္စယံ။}

\sutta{573}{123}{လောပေါ။}
\vutti{ပစ္စယာနံ လောပေါပိ ဒိဿတိ။ ဗုဒ္ဓေ ရတနံ ပဏီတံ၊ စက္ခုံ သုညံ အတ္တေန ဝါ အတ္တရိယေန ဝါတိ ဘာဝပ္ပစ္စယလောပေါ။}

\sutta{574}{124}{သရာနမာဒိဿာယုဝဏ္ဏဿာဧဩ ဏာနုဗန္ဓေ။}
\vutti{သရာနမာဒိဘူတာ ယေ အကာရိဝဏ္ဏုဝဏ္ဏာ၊ တေသံ အာဧဩ ဟောန္တိ ယထာက္ကမံဏာနုဗန္ဓေ။ ရာဃဝေါ၊ ဝေနတေယျော၊ မေနိကော၊ ဩဠုမ္ပိကော၊ ဒေါဘဂ္ဂံ။ ဏာနုဗန္ဓေတိ ကိံ? ပုရာတနော။}

\sutta{575}{125}{သံယောဂေ ကွစိ။}
\vutti{သရာနမာဒိဘူတာ ယေ အယုဝဏ္ဏာ၊ တေသံ အာဧဩ ဟောန္တိ ကွစိဒေဝ သံယောဂဝိသယေ ဏာနုဗန္ဓေ။ ဒေစ္စော၊ ကောဏ္ဍညော။ ကွစီတိ ကိံ? ကတ္တိကေယျော။}

\sutta{576}{126}{မဇ္ဈေ။}
\vutti{မဇ္ဈေ ဝတ္တမာနာနမ္ပိ အယုဝဏ္ဏာနံ အာ ဧ ဩ ဟောန္တိ ကွစိ။ အဍ္ဎတေယျော၊ ဝါသေဋ္ဌော။}

\sutta{577}{127}{ကောသဇ္ဇာဇ္ဇဝ ပါရိသဇ္ဇ သောဟဇ္ဇ မဒ္ဒဝါရိဿာသဘာဇညထေယျ ဗာဟုသစ္စာ။}
\vutti{ဧတေသဒ္ဒါ နိပစ္စန္တေဏာနုဗန္ဓေ။ ကုသီ တဿ ဘာဝေါကောသဇ္ဇံ၊ ဥဇုနော ဘာဝေါ အဇ္ဇဝံ၊ ပရိသာသု သာဓု ပါရိသဇ္ဇော၊ သုဟဒယောဝ သုဟဇ္ဇော၊ တဿ ပန ဘာဝေါ သောဟဇ္ဇံ၊ မုဒုနော ဘာဝေါ မဒ္ဒဝံ၊ ဣသိနော ဣဒံ ဘာဝေါ ဝါ အာရိဿံ၊ ဥသဘဿ ဣဒံ ဘာဝေါ ဝါ အာသဘံ၊ အာဇာနီယဿ ဘာဝေါ သော ဧဝ ဝါ အာဇညံ၊ ထေနဿ ဘာဝေါ ကမ္မံ ဝါ ထေယျံ၊ ဗဟုဿတဿ ဘာဝေါ ဗာဟုသစ္စံ၊ ဧတေသု ယမလက္ခဏိကံ၊ တံ နိပါတနာ။}

\sutta{578}{128}{မနာဒီနံ သက။}
\vutti{မနာဒီနံ သက ဟောတိ ဏာနုဗန္ဓေ။ မနသိ ဘဝံ မာနသံ၊ ဒုမ္မနသော ဘာဝေါ ဒေါမနဿံ၊ သောမနဿံ။}

\sutta{579}{129}{ဥဝဏ္ဏဿာဝင သရေ။}
\vutti{သရာဒေါ ဏာနုဗန္ဓေ ဥဝဏ္ဏဿာဝင ဟောတိ။ ရာဃဝေါ၊ ဇမ္ဗဝံ။}

\sutta{580}{130}{ယမှိ ဂေါဿ စ။}
\vutti{ယကာရာဒေါ ပစ္စယေ ဂေါဿုဝဏ္ဏဿ စ အဝင ဟောတိ။ ဂဗျံ၊ ဘာတဗျော။}

\sutta{581}{131}{လောပေါ-ဝဏ္ဏိဝဏ္ဏာနံ။}
\vutti{ယကာရာဒေါ ပစ္စယေ အဝဏ္ဏိဝဏ္ဏာနံ လောပေါ ဟောတိ။ ဒါယဇ္ဇံ၊ ကာရုညံ၊ အာဓိပစ္စံ၊ ဒေပ္ပံ။ ဗဟုလံဝိဓာနာ ကွစိ န ဟောတိ ကိစ္စယံ။}

\sutta{582}{132}{ရာနုဗန္ဓေ-န္တ သရာဒိဿ။}
\vutti{အန္တော သရော အာဒိမှိ ယဿာဝယဝဿ၊ တဿ လောပေါ ဟောတိ ရာနုဗန္ဓေ။ ကိတ္တကံ၊ ပေတ္တေယျံ။}

\sutta{583}{133}{ကိသမဟတမိမေ ကသ မဟာ။}
\vutti{ကိသဿ မဟတော ဣမေ ကသမဟာ ဟောန္တိ ယထာက္ကမံ၊ ကသိမာ၊ မဟိမာ။}

\sutta{584}{134}{အာယုဿာယသ မန္တုမှိ။}
\vutti{အာယုဿ အာယသာဒေသော ဟောတိ မန္တုမှိ။ အာယသ္မာ။}

\sutta{585}{135}{ဇော ဝုဒ္ဓဿိယိဋ္ဌေသု။}
\vutti{ဝုဒ္ဓဿ ဇော ဟောတိ ဣယဣဋ္ဌေသု၊ ဇေယျော၊ ဇေဋ္ဌော။}

\sutta{586}{136}{ဗာဠှန္တိကပသတ္ထာနံ သာဓ နေဒ သာ။}
\vutti{ဣယဣဋ္ဌေသု ဗာဠှန္တိကပသတ္ထာနံ သာဓ နေဒ သာ ဟောန္တိ ယထာက္ကမံ။ သာဓိယော၊ သာဓိဋ္ဌော၊ နေဒိယော၊ နေဒိဋ္ဌော၊ သေယျော၊ သေဋ္ဌော။}

\sutta{587}{137}{ကဏကနာပ္ပယုဝါနံ။}
\vutti{ဣယဣဋ္ဌေသု အပ္ပယုဝါနံ ကဏ ကနာ ဟောန္တိ ယထာက္ကမံ။ ကဏိယော ကဏိဋ္ဌော၊ ကနိယော ကနိဋ္ဌော။}

\sutta{588}{138}{လောပေါ ဝီ မန္တု ဝန္တူနံ။}
\vutti{ဝီ မန္တု ဝန္တူနံ လောပေါ ဟောတိ ဣယဣဋ္ဌေသု။ အတိသယေန မေဓာဝီ မေဓိယော၊ မေဓိဋ္ဌော၊ အတိသယေန သတိမာ သတိယော၊ သတိဋ္ဌော၊ အတိသယေန ဂုဏဝါ ဂုဏိယော၊ ဂုဏိဋ္ဌော။}

\sutta{589}{139}{ဍေ သတိဿ တိဿ။}
\vutti{ဍေပရေ သတျန္တဿ တိကာရဿ လောပေါ ဟောတိ၊ ဝီသံ သတံ၊ တိံသံ သတံ။}

\sutta{590}{140}{ဧတဿေဋ တ္တကေ။}
\vutti{တ္တကေ ပရေ ဧတဿ ဧဋ ဟောတိ။ ဧတ္တကံ။}

\sutta{591}{141}{ဏိကဿိယော ဝါ။}
\vutti{ဏိကဿ ဝါ ဣယော ဟောတိ၊ သကျပုတ္တိယော၊ သကျပုတ္တိကော။}

\sutta{592}{142}{အဓာတုဿ ကာ-သျာဒိတော ဃေ-ဿိ။}
\vutti{ဃေ ပရေ အဓာတုဿ ယော ကကာရော၊ တတော ပုဗ္ဗဿ အကာရဿ ဗဟုလံ ဣ ဟောတိ သစေ ဃော န သျာဒိတော ပရော ဟောတိ။ ဗာလိကာ၊ ကာရိကာ၊ အဓာတုဿာတိ ကိံ? သကာ၊ ကေတိ ကိံ? နန္ဒနာ၊ အသျာဒိတောတိ ကိံ? ဗဟုပရိဗ္ဗာဇကာ မထုရာ၊ ဗဟုစမ္မိကာတိ ကကာရေန သျာဒိနော ဗျဝဟိတတ္တာ သိဒ္ဓံ၊ ဃေတိ ကိံ? ဗာလကော၊ အဿာတိ ကိံ? ဗဟုကတ္တုကာ သာလာ။}

\begin{jieshu}
ဣတိ မောဂ္ဂလ္လာနေ ဗျာကရဏေ ဝုတ္တိယံ

ဏာဒိကဏ္ဍော စတုတ္ထော။
\end{jieshu}
\chapter{ခါဒိကဏ္ဍော ပဉ္စမော}
\markboth{မောဂ္ဂလ္လာနဗျာကရဏေ}{ခါဒိကဏ္ဍော ပဉ္စမော}


\sutta{593}{1}{တိဇ မာနေဟိ ခ သာ ခမာ ဝီမံသာသု။}
\vutti{ခန္တိယံ တိဇာ ဝီမံသာယံ မာနာ စ ခသပ္ပစ္စယာ ဟောန္တိ ယထာက္ကမံ၊ တိတိက္ခာ၊ ဝီမံသာ၊ တိတိက္ခတိ၊ ဝီမံသတိ။ ခမာဝီမံသာ၊ သူတိ ကိံ? တေဇနံ၊ တေဇော၊ တေဇယတိ၊ မာနနံ၊ မာနောမာနေတိ။}

\sutta{594}{2}{ကိတာ တိကိစ္ဆာသံသယေသု ဆော။}
\vutti{တိကိစ္ဆာယံ သံသယေ စ ဝတ္တမာနာ ကိတာ ဆော ဟောတိ။ တိကိစ္ဆာ၊ ဝိစိကိစ္ဆာ၊ တိကိစ္ဆတိ၊ ဝိစိကိစ္ဆတိ။ အညတြ နိကေတော၊ သံကေတော၊ ကေတနံ၊ ကေတော၊ ကေတယတိ။}

\sutta{595}{3}{နိန္ဒာယံ ဂုပ ဗဓာ ဗဿ ဘော စ။}
\vutti{နိန္ဒာယံ ဝတ္တမာနေဟိ ဂုပ ဗဓေဟိ ဆော ဟောတိ ဗဿ ဘော စ။ ဇိဂုစ္ဆာ၊ ဗီဘစ္ဆာ၊ ဇိဂုစ္ဆတိ၊ ဗီဘစ္ဆတိ၊ အညတြ ဂေါပနံ၊ ဂေါပေါ၊ ဂေါပေတိ၊ ဗဓကော။}

\sutta{596}{4}{တုံသ္မာ လောပေါ စိစ္ဆာယံ တေ။}
\vutti{တုမန္တတော ဣစ္ဆာယမတ္ထေ တေ ခသဆာ ဟောန္တိ ဗဟုလံ၊ လောပေါ စ တုံပစ္စယဿ ဟောတိ သုတတ္တာ၊ ဗုဘုက္ခာ၊ ဇိဂီသာံ၊ ဇိဃစ္ဆာ၊ ဗုဘုက္ခတိ၊ ဇိဂီသတိ ဇိဃစ္ဆတိ။ ဣဓ ကသ္မာ န ဟောတိ ‘ဘောတ္တုမိစ္ဆတီ ’တိ? ပဒန္တရေနာဘိဓာနာ။ တုံသ္မာတိ ကိံ? ဘောဇနမိစ္ဆတိ။ ဣစ္ဆာယန္တိ ကိံ? ဘုဉ္ဇိတုံ ဂစ္ဆတိ။ ကထံ ‘ကူလံ ဝိပတိ သတီ ’တိ? ယထာ ကူလံ ပတိတု မိစ္ဆတီတိ ဝါကျံ ဟောတိ၊ ဧဝံ ဝုတ္တိပိ ဟောဿတိ။ ဝါကျမေဝ စရဟိ ကထံ ဟောတိ? လောကဿ တထာ ဝစနိစ္ဆာယ။}

\sutta{597}{5}{ဤယော ကမ္မာ။}
\vutti{ဣစ္ဆာကမ္မတော ဣစ္ဆာယမတ္ထေ ဤယပ္ပစ္စယော ဟောတိ။ ပုတ္တမိစ္ဆတိ ပုတ္ထီယတိ။ ကမ္မာတိ ကိံ? အသိနေစ္ဆတိ။ ဣဓ ကသ္မာ န ဟောတိ ‘ရညောပုတ္တမိစ္ဆတီ ’တိ? သာပေက္ခတ္တာ၊ န ဟိ အညမပေက္ခမာနော အညေန သဟေကတ္ထိဘာဝမနုဘဝိတုံ သက္ကောတိ။ ဣဓာပိ စရဟိ န သိယာ ‘အတ္တနော ပုတ္တ မိစ္ဆတီ ’တိ? နေဝေတ္ထ ဘဝိတဗ္ဗံ၊ န ဟိ ဘဝတိ ‘အတ္တနော ပုတ္တီယတီ ’တိ၊ ကထံ စရဟိ ပုတ္တဿ အတ္တနိယတာ-ဝဂမျတေ ? အညဿာသုတတ္တာ ဣစ္ဆာယ စ တဗ္ဗိသယတ္တာ။}

\sutta{598}{6}{ဥပမာနာစာရေ။}
\vutti{ကမ္မတော ဥပမာနာ အာစာရတ္ထေ ဤယော ဟောတိ။ ပုတ္တမိ-ဝါ-စရတိ ပုတ္တီယတိ မာဏဝကံ၊ ဥပမာနာတိ ကိံ? ပုတ္တမာစရတိ။}

\sutta{599}{7}{အာဓာရာ။}
\vutti{အာဓာရတူ-ပမာနာ အာစာရတ္ထေ ဤယော ယောတိ။ ကုဋိယမိဝါ-စရတိ ကုဋီယတီ ပါသာဒေ၊ ပါသာဒီယတိ ကုဋိယံ ဘိက္ခု။}

\sutta{600}{8}{ကတ္တုတာယော။}
\vutti{ကတ္တုတူ-ပမာနာ အာစာရတ္ထေ အာယော ဟောတိ။ ပဗ္ဗတော ဣဝါစရတိ ပဗ္ဗတာယတိ။}

\sutta{601}{9}{ယ္စတ္ထေ။}
\vutti{ကတ္တုတော အဘူတတဗ္ဘာဝေ အာယော ဟောတိ ဗဟုလံ။ ဘုသာယတိ၊ ပဋပဋာယတိ၊ လောဟိတာယတိ၊ ကတ္တုတောတွေဝ? (အဘုသံ) ဘုသံ ကရောတီဟိ၊ ဣဟ ကသ္မာ န ဟောတိ ‘ဘုသီ ဘဝတီ ’တိ? ဝုတ္တတ္ထတာယ။}

\sutta{602}{10}{သဒ္ဒါဒီနိ ကရောတိ။}
\vutti{သဒ္ဒါဒီဟိ ဒုတိယန္တေဟိ ကရောတီတိ အသ္မိံ အတ္ထေ အာယော ဟောတိ။ သဒ္ဒါယတိ၊ ဝေရာယတိ၊ ကလဟာယတိ၊ ဓူပါယတိ။}

\sutta{603}{11}{နုမောတွ-ဿော။}
\vutti{နမောဣစ္စသ္မာ ကရောတီတိ အသ္မိံ အတ္ထေ အဿော ဟောတိ။ နမဿတိ တထာဂ္တံ။}

\sutta{604}{12}{ဓာတွတ္ထေ နာမသ္မိ။}
\vutti{နာမသ္မာ ဓာတွတ္ထေ ဗဟုလမိဟောတိ။ ဟတ္ထိနာ အတိက္ကမတိ အတိဟတ္ထယတိ၊ ဝီဏာယ ဥပဂါယတိ ဥပဝီဏယတိ၊ ဒဠှံ ကရောတိ ဝိနယံ ဒဠှယတိ၊ ဝိသုဒ္ဓါ ဟောတိ ရတ္တိ ဝိသုဒ္ဓယတိ၊ ကုသလံ ပုစ္ဆတိ ကုသလယတိ။}

\sutta{605}{13}{သစ္စာဒီဟာပိ။}
\vutti{သစ္စာဒီဟိ ဓာတွတ္ထေ အာပိ ဟောတိ။ သစ္စာပေတိ၊ အတ္ထာပေတိ၊ ဝေဒါပေတိ၊ သုက္ခာပေတိ၊ သုခါပေတိ၊ ဒုက္ခာပေတိ။}

\sutta{606}{14}{ကြိယတ္ထာ။}
\vutti{အယမဓိကာရော အာသတ္ထပရိသမတ္တိယာ။ ကြိယာ အတ္ထော ယဿ သော ကြိယတ္ထော ဓာတု။}

\sutta{607}{15}{စုရာဒိတော ဏိ။}
\vutti{စုရာဒီဟိ ကြိယတ္ထေဟိ သကတ္ထေ ဏိ ပရော ဟောတိ ဗဟုလံ။ ဏကာရော ဝုဒ္ဓျတ္ထော၊ ဧဝမညတြာပိ၊ စောရယတိ၊ လာဠယတိ၊ ကထံ ‘ရဇ္ဇံ ကာရေတီ ’တိ? ယောဂဝိဘာဂတော။}

\sutta{608}{16}{ပယောဇကဗျာပါရေ ဏာပိ စ။}
\vutti{ကတ္တာရံ ယော ပယောဇယတိ၊ တဿ ဗျာပါရေ ကြိယတ္ထာ ဏိဏာပီ ဟောန္တိ ဗဟုလံ၊ ကာရေတိ၊ ကာရာပေတိ။ နနြ စ ကတ္တာပိ ကရဏာဒီနံ ပယောဇကောတိ တံဗျာပါရေပိ ဏိဏာပီ ပါပုဏန္တိ? ပယောဇကဂ္ဂဟဏသာမတ္ထိယာ န ဘဝိဿန္တိ စုရာဒီဟိ ဝိသုံ ဝစနသာမတ္ထိယာ စ။ အတော ဘိယျော ဏာပိယေဝ၊ ဏိယေဝုဝဏ္ဏတော၊ ဒွယမေဝညေဟိ။}

\sutta{609}{17}{ကျော ဘာဝကမ္မေသွ-ပရောက္ခေသု မာန န္တ တျာဒီသု။}
\vutti{ဘာဝကမ္မဝိဟိတေသု ပရောက္ခာဝဇ္ဇိတေသု မာနန္တတျာဒီသု ပရေသု ကျော ဟောတိ ကြိယတ္ထာ။ န္တဂ္ဂဟဏမုတ္တရတ္ထံ၊ ကကာရော အဝုဒ္ဓျတ္ထော ဧဝမုတ္တရတြာပိ။ ဌီယမာနံ၊ ဌီယတေ၊ သူယမာနံ၊ သူယတေ၊ အပရောက္ခေသု မာနန္တတျာဒီသူတိ ကိံ? ဗဘူဝ ဒေဝဒတ္တေန၊ ဗိဘိဒ ကုသုလော။ ဘိဇ္ဇတေ ကုသုလော သယမေဝါတိ ‘ဘိဇ္ဇတေ ’တိ သဝနာ ကမ္မတာ-ဝဂမျတေ၊ ‘သယမေဝါ ’တိ သဝနတော ကတ္တုတာ၊ ကတ္တုတာဝစနိစ္ဆာယန္တု ‘ဘိန္ဒတိ ကုသုလော အတ္တာန ’န္တိ ဘဝတိ၊ ဧဝမညမ္ပိ ယထာဂမမနုဂန္တဗ္ဗံ။ ‘အပရောက္ခေသု မာနန္တတျာဒီသူ ’တိ အယမဓိကာရော အာ ‘တနာဒိတွော ’တိ ၅.၂၆။ အပိစ ဧတေ ကျာဒယော တျာဒီသု ပရဘူတေသု ကတ္တုကမ္မဘာဝ ဝိဟိတေသု ကျလာဒီနံ ဝိဓာနတော တေသွေဝ ဝိညာယန္တီတိ အကမ္မကေဟိ ဓာတူဟိ ကတ္တုဘာဝေသု၊ သကမ္မကေဟိ ကတ္တုကမ္မေသု၊ ကမ္မာဝစနိစ္ဆာယံ ဘာဝေ စ ဘဝန္တီတိ ဝေဒိတဗ္ဗာ။ ယဿ ပန ဓာတုဿ ကိရိယာ ကမ္မမပေက္ခတေ၊ သော သကမ္မကော၊ ယဿ တု ကိရိယာ ကတ္တုမတ္တမပေက္ခတေ၊ သွာကမ္မကောတိ ဉာတဗ္ဗံ။}

\sutta{610}{18}{ကတ္တရိ လော။}
\vutti{ကြိယတ္ထတော အပရောက္ခေသု ကတ္တုဝိဟိတမာန န္တတျာဒီသု လော ဟောတိ။ လကာရော၊ “ဉိလဿေ ”တိ ၅-၁၆၃ ဝိသေသနတ္ထော။ ပစမာနော၊ ပစန္တော၊ ပစတိ။}

\sutta{611}{19}{မံ စ ရုဓာဒီနံ။}
\vutti{ရုဓာဒိတော ကတ္တုဝိဟိတမာနန္တ တျာဒီသု လော ဟောတိ မံ စ အန္တသရာ ပရော။ မကာရော-နုဗန္ဓော၊ အကာရော ဥစ္စာရဏတ္ထော။ ရုန္ဓမာနော၊ ရုန္ဓန္တော၊ ရုန္ဓတိ။}

\sutta{612}{20}{ဏိဏာပျာပီဟိ ဝါ။}
\vutti{ဏိဏာပျာပီဟိ ကတ္တုဝိဟိတမာနန္တ တျာဒီသု လော ဟောတိ ဝိဘာသာ၊ ဩရယန္တော၊ စောရေန္တော၊ ကာရယန္တော၊ ကာရေန္တော၊ ကာရာပယန္တော၊ ကာရာပေန္တော၊ သစ္စာပယန္တော၊ သစ္စာပေန္တော၊ စောရယတိ၊ စောရေတိ၊ ကာရယတိ၊ ကာရေတိ၊ ကာရာပယတိ၊ ကာရာပေတိ၊ သစ္စာပယတိ၊ သစ္စာပေတိ။ ဝဝတ္ထိတဝိဘာသတ္ထော-ယံ ဝါသဒ္ဒေါ၊ တေန မာနေ နိစ္စံ၊ စောရယမာနော၊ ကာရယမာနော၊ ကာရာပယမာနော၊ သစ္စာပယမာနော။}

\sutta{613}{21}{ဒိဝါဒီဟိ ယက။}
\vutti{ဒိဝါဒီဟိ လဝိသယေ ယက ဟောတိ။ ဒိဗ္ဗန္တော၊ ဒိဗ္ဗတိ။}

\sutta{614}{22}{တုဒါဒီဟိ ကော။}
\vutti{တုဒါဒီဟိ လဝိသယေ ကော ဟောတိ။ တုဒမာနော၊ တုဒန္တော၊ တုဒတိ။}

\sutta{615}{23}{ဇျာဒီဟိ က္နာ။}
\vutti{ဇိအာဒီဟိ လဝိသယေ က္နာ ဟောတိ။ ဇိနန္တော၊ ဇိနာတိ။ ကထံ ‘ဇယန္တော ’ ဇယတီ၊ တိ? ဘူဝါဒိပါဌာ။}

\sutta{616}{24}{ကျာဒီဟိ က္ဏာ။}
\vutti{ကီအာဒီဟိ လဝိသယေ က္ဏာ ဟောတိ။ ကိဏန္တော၊ ကိဏာတိ။}

\sutta{617}{25}{သွာဒီဟိ က္ဏော။}
\vutti{သုအာဒီဟိ လဝိသယေ က္ဏော ဟောတိ။ သုဏမာနော၊ သုဏန္တော၊ သုဏောတိ။ ကထံ သုဏာတီတိ? ကျာဒိပါဌာ။}

\sutta{618}{26}{တနာဒိတွော။}
\vutti{တနာဒိတော လဝိသယေ ဩ ဟောတိ။ တနောတိ။}

\sutta{619}{27}{ဘာဝကမ္မေသု တဗ္ဗာ-နီယာ။}
\vutti{တဗ္ဗအနီယာ ကြိယတ္ထာ ပရေ ဘာဝကမ္မေသု ဗဟုလံ ဘဝန္တိ။ ကတ္တဗ္ဗံ၊ ကရဏီယံ၊ ကတ္တဗ္ဗော ကဋော၊ ကရဏီယော။ ဗဟုလာဓိကာရာ ကရဏာဒီသုပိ ဘဝန္တိ၊ သိနာနီယံ စုဏ္ဏံ၊ ဒါနီယော ဗြာဟ္မဏော၊ သမ္မာဝတ္တနီယော ဂုရု၊ ပဝစနီယော ဥပဇ္ဈာယော၊ ဥပဋ္ဌာနီယော သိဿော။}

\sutta{620}{28}{ဃျဏ။}
\vutti{ဘာဝကမ္မေသု ကြိယတ္ထာ ပရော ဃျဏ ဟောတိ ဗဟုလံ။ ဝါကျံ၊ ကာရိယံ၊ စေယျံ၊ ဇေယျံ။}

\sutta{621}{29}{အာဿေ စ။}
\vutti{အာတောဃျဏ ဟောတိ ဘာဝကမ္မေသု၊ အာဿ ဧ စ။ ဒေယျံ။}

\sutta{622}{30}{ဝဒါဒီဟိ ယော။}
\vutti{ဝဒါဒီဟိ ကြိယတ္ထေဟိ ယော ဟောတိ ဗဟုလံ ဘာဝကမ္မေသု။ ဝဇ္ဇံ၊ မဇ္ဇံ၊ ဂမ္မံ။ (၄၂) “\suttagananormal{623}{42}{ဘုဇာန္နေ။} ”၊ ဘောဇ္ဇော ဩဒနော၊ ဘောဇ္ဇာ ယာဂု၊ ဘောဂ္ဂမညံ။}

\sutta{624}{31}{ကိစ္စ ဃစ္စ ဘစ္စ ဘဗ္ဗ လေယျာ။}
\vutti{ဧတေ သဒ္ဒါ ယပ္ပစ္စယန္တာ နိပစ္စန္တေ။}

\suttagana{625}{43}{သညာယံ ဘရာ။}

\sutta{626}{32}{ဂုဟာဒီဟိ ယက။}
\vutti{ဂုတာဒီဟိ ကြိယတ္တေဟိ ဘာဝကမ္မေသု ယက ဟောတိ။ ဂုယှံ၊ ဒုယှံ၊ သိဿော။ သိဒ္ဓါ ဧဝေတေ တဗ္ဗာဒယော ပေသာတိသဂ္ဂပတ္တကာလေသုပိ ဂမျမာနေသု သာမညေနဝိဓာနတော၊ တွယာ ခလု ကဋော ကတ္တဗ္ဗော၊ ကရဏီယော၊ ကာရိယော၊ ကိစ္စော၊ ဧဝံ တွယာ ကဋော ကတ္တဗ္ဗော၊ ဘောတာ ကဋော ကတ္တဗ္ဗော၊ ဘောတော ဟိ ပတ္တော ကာလော ကဋကရဏေ။ ဧဝံ ဥဒ္ဓမောဟုတ္တိကေပိ ဝတ္တမာနတော ပေသာဒီသု သိဒ္ဓါ ဧဝ။ တထာ အရဟေ ကတ္တရိ သတ္တိဝိသိဋ္ဌေ စ ပတီယမာနေ အာဝဿကာဓမီဏတာဝိသိဋ္ဌေ စ ဘာဝါဒေါ သိဒ္ဓါ၊ ဥဒ္ဓံမုဟုတ္တတော-ဘောတာ ကဋော ကတ္တဗ္ဗော၊ ဘောတာ ရဇ္ဇံ ကတ္တဗ္ဗံ။ ဘဝံ အရဟော၊ ဘောတာ သာရော ဝဟိတဗ္ဗော၊ ဘဝံ သက္ကော၊ ဘောတာ အဝဿံ ကဋော ကတ္တဗ္ဗော၊ ဘောတာ နိက္ခော ဒါတဗ္ဗော။}

\sutta{627}{33}{ကတ္တရိ လ္တုဏကာ။}
\vutti{ကတ္တရိ ကာရကေ ကြိယတ္ထာ လ္တုဏကာ ဟောန္တိ ဗဟုလံ။ ပဌိတာ၊ ပါဌကော။ ဗဟုလမိတွေဝ? ပါဒေဟိ ဟရီယတီတိ ပါဒဟာ-ရကော ၊ ဂလေ စုပ္ပတေတိ ဂလေစောပကော။ သိဒ္ဓေါဝ လ္တု၊ အရဟေ သီလသာဓုဓမ္မေသု စ သာမညဝိဟိတတ္တာ၊ ဘဝံ ခလု ကညာယ ပရိဂ္ဂဟိတာ၊ ဘဝမေတံ အရဟတိ။ သီလာဒီသု-ခလွပိ ဥပါဒါတာ ကုမာရကေ၊ ဂန္တာ ခေလံ၊ မုဏ္ဍယိတာရော သာဝိဋ္ဌာယနာ ဝဓုံ ကတပရိဂ္ဂဟံ။}

\sutta{628}{34}{အာဝီ။}
\vutti{ကြိယတ္ထာ အာဝီ ဟောတိ ဗဟုလံ ကတ္တရိ။ ဘယဒဿာဝီ။ အပ္ပဝိသယတညာပနတ္ထံ ဘိန္နယောဂကရဏံ၊ သာမညဝိဟိတတ္တာ သီလာ ဒီသု စ ဟောတေဝ။}

\sutta{629}{35}{အာသိံသာယမကော။}
\vutti{အာသိံသာယံ ဂမျမာနာယံ ကြိယတ္ထာ အကော ဟောတိ ကတ္တရိ။ ဇီဝတူတိ ဇီဝကော၊ နန္ဒတူတိ နန္ဒကော၊ ဘဝတူတိ ဘဝကော။}

\sutta{630}{36}{ကရာ ဏနော။}
\vutti{ကရတော ကတ္တရိ ဏနော ဟောတိ။ ကရောတီတိ ကာရဏံ၊ ကတ္တရီတိ ကိံ? ကရဏံ။}

\sutta{631}{37}{ဟာတော ဝီဟိကာလေသု။}
\vutti{ဟာတော ဝီဟိသ္မိံ ကာလေ စ ဏနော ဟောတိ ကတ္တရိ။ ဟာယနာ နာမ ဝိဟယော၊ ဟာယနော သံဝစ္ဆရော၊ ဝီဟိကာလေသူတိ ကိံ? ဟာတာ။}

\sutta{632}{38}{ဝိဒါ ကူ။}
\vutti{ဝိဒသ္မာ ကူ ဟောတိ ကတ္တရိ။ ဝိဒူ၊ လောကဝိဒူ။}

\sutta{633}{39}{ဝိတော ဉာတော။}
\vutti{ဝိပုဗ္ဗာ ဉာဣစ္စသ္မာ ကူ ဟောတိ ကတ္တရိ။ ဝိညူ။ ဝိတောတိ ကိံ? ပညာ။}

\sutta{634}{40}{ကမ္မာ။}
\vutti{ကမ္မတော ပရာ ဉာဣစ္စသ္မာ ကူ ဟောတိ ကတ္တရိ။ သဗ္ဗညူ၊ ကာလညူ။}

\sutta{635}{41}{ကွစဏ။}
\vutti{ကမ္မတော ပရာ ကြိယတ္ထာ ကွစိ အဏ ဟောတိ ကတ္တရိ။ ကုမ္ဘကာရော၊ သရလာဝေါ၊ မန္တဇ္ဈာယော၊ ဗဟုလာဓိကာရာ ဣဓ န ဟောတိ အာဒိစ္စံ ပဿတိ၊ ဟိမဝန္တံ သုဏောတိ၊ ဂါမံ ဂစ္ဆတိ။ ကွစီတိ ကိံ? ကမ္မကရော။}

\sutta{636}{42}{ဂမာ ရူ။}
\vutti{ကမ္မတော ပရာ ဂမာ ရူ ဟောတိ ကတ္တရိ။ ဝေဒဂူ ပါရဂူ။}

\sutta{637}{43}{သမာနည ဘဝန္တ ယာဒိတူ-ပမာနာ ဒိသာ ကမ္မေ ရီရိက္ခကာ။}
\vutti{သမာနာဒီဟိ ယာဒီဟိ စောပမာနေဟိ ပရာ ဒိသာ ကမ္မကာရကေ ရီရိက္ခကာ ဟောန္တိ။ သမာနော ဝိယ ဒိဿတီတိ သဒီ သဒိက္ခော သဒိသော။ အညာဒီ အညာဒိက္ခော အညာဒိသော။ ဘဝါဒီ ဘဝါဒိက္ခော ဘဝါဒိသော။ ယာဒီ ယာဒိက္ခော ယာဒိသော။ တျာဒီ တျာဒိက္ခော တျာဒိသော။ သမာနာဒီဟီတိ ကိံ? ရုက္ခော ဝိယ ဒိဿတိ။ ဥပမာနာတိ ကိံ? သော ဒိဿတိ။ ကမ္မေတိ ကိံ? သော ဝိယ ပဿတိ။ ရကာရာ အန္တသရာဒိလောပတ္ထာ၊ ကကာရော ဧကာရာဘာဝတ္ထော။}

\sutta{638}{44}{ဘာဝကာရကေသွ-ဃဏ ဃကာ။}
\vutti{ဘာဝေ ကာရကေ စ ကြိယတ္ထာ အ ဃဏ ဃ ကာ ဟောန္တိ ဗဟုလံ။ အ-ပဂ္ဂဟော၊ နိဂ္ဂဟော၊ ကရော၊ ဂရော၊ စယော၊ ဇယော၊ ရဝေါ၊ ဘဝေါ၊ ပစော၊ ဝစော၊ အန္နဒေါ၊ ပုရိန္ဒဒေါ၊ ဤသက္ကရော၊ ဒုက္ကရော၊ သုကရော။ ဃဏ-ဘာဝေ ပါကော၊ စာဂေါ၊ ဘာဝေါ၊ ကာရကေပိ သညာယံ တာဝ ပဇ္ဇတေနေနာတိ ပါဒေါ၊ ရုဇတီတိ ရောဂေါ၊ ဝိသတီတိ ဝေသော၊ သရတိ ကာလန္တရန္တိ သာရော ထိရံသော၊ ဒရီယန္တေ ဧတေဟီတိ ဒါရာ၊ ဇီရယတိ ဧတေနာတိ ဇာရော၊ အသညာယမ္ပိ ဒါယော ဒတ္တော၊ လာဘော လဒ္ဓေါ၊ ဃ-ဝကော၊ နိပကော၊ က-ပိယော၊ ခိပေါ၊ ဘုဇော၊ အာယုဓံ။}

\sutta{639}{45}{ဒါဓာတွိ။}
\vutti{ဒါဓာဟိ ဗဟုလမိ ဟောတိ ဘာဝကာရကေသု။ အာဒိ၊ နိဓိ၊ ဝါလမိ။}

\sutta{640}{46}{ဝမာဒီယျထု။}
\vutti{ဝမာဒီဟိ ဘာဝကာရကေသွထု ဟောတိ။ ဝမထု၊ ဝေပထု၊ (အဝထု၊ သယထု)။}

\sutta{641}{47}{ကွိ။}
\vutti{ကြိယတ္ထာ ကွိ ဟောတိ ဗဟုလံ ဘာဝကာရကေသု။ ကကာရော ကာနုဗန္ဓကာရိယတ္ထော၊ အဘိဘူ၊ သယမ္ဘူ၊ ဘတ္တဂ္ဂံ၊ (ဒါနဂ္ဂံ) သလာကဂ္ဂံ၊ သဘာ၊ ပဘာ။}

\sutta{642}{48}{အနော။}
\vutti{ကြိယတ္တာ ဘာဝကာရကေသွနော ဟောတိ။ ဂမနံ၊ ဒါနံ၊ သမ္ပဒါနံ၊ အပါဒါနံ၊ အဓိကရဏံ၊ စလနော၊ ဇလနော၊ ကောဓနော၊ ကောပနော၊ မဏ္ဍနော၊ ဘူသနော။}

\sutta{643}{49}{ဣတ္ထိယမဏ တ္တိ က ယက ယာ စ။}
\vutti{ဣတ္ထိလိင်္ဂေ ဘာဝေ ကာရကေ စ ကြိယတ္ထာ အအာဒယော ဟောန္တိ အနော စ ဗဟုလံ။ အ တိတိက္ခာ၊ ဝီမံသာ၊ ဇိဂုစ္ဆာ၊ ပိပါသာ၊ ပုတ္တိယာ ဤဟာ ဘိက္ခာ၊ အာပဒါ၊ မေဓာ၊ ဂေါဓာ၊ ဏကာရာ၊ ဟာရာ၊ တာရာ၊ ဓာရာ၊ အာရာ၊ က္တိ-ဣဋ္ဌိ၊ သိဋ္ဌိ၊ ဘိတ္တိ၊ ဘတ္တိ၊ တန္တိ ဘူတိ၊ က-ဂုဟာ၊ ရုဇာ၊ မုဒါ၊ ယက-ဝိဇ္ဇာ၊ ဣဇ္ဇာ၊ ယ-သေယျာ၊ သမဇ္ဇာ၊ ပဗ္ဗဇ္ဇာ၊ ပရိစရိယာ၊ ဇာဂရိယာ၊ အနကာရဏာ၊ ဟာရဏာ၊ ဝေဒနာ၊ ဝန္ဒနာ၊ ဥပါသနာ။}

\sutta{644}{50}{ဇာဟာဟိ နိ။}
\vutti{ဇာဟာဣစ္စေတေဟိ နိ ဟောတိတ္ထိယံ။ ဇာနိ၊ ဟာနိ။}

\sutta{645}{51}{ကရာ ရိရိယော။}
\vutti{ကရတော ရိရိယော ဟောတိတ္ထိယံ။ ကရဏံ ကိရိယာ။ ကထံ ‘ကြိယာ ’တိ? “ကြိယာယံ ”တိ နိပါတနာ။}

\sutta{646}{52}{ဣ ကိ တီ သရူပေ။}
\vutti{ကြိယတ္ထဿ သရူပေ-ဘိဓေယျေ ကြိယတ္ထာ ပရေ ဣကိတီ ဟောန္တိ၊ ဝစိ၊ ယုထိ၊ ပစတိ၊ ‘အကာရော ကကာရော ’တိ အာဒီသု ကာရသဒ္ဒေန သမာသော၊ ယထာ ဧဝကာရောတိ။}

\sutta{647}{53}{သီလာဘိက္ခညာ-ဝဿကေသု ဏီ။}
\vutti{ကြိယတ္ထာ ဏီ ဟောတိ သီလာဒီသု ပတီယမာနေသု၊ ဥဏှဘောဇီ၊ ခီရပါယီ၊ အဝဿကာရီ၊ သတန္ဒာယီ။}

\sutta{648}{54}{ထာဝရိတ္တရ ဘင်္ဂုရ ဘိဒုရ ဘာသုရ ဘဿရာ။}
\vutti{ဧတေ သဒ္ဒါ နိပစ္စန္တေ သီလေ ဂမျမာနေ။}

\sutta{649}{55}{ကတ္တရိ ဘူတေ က္တဝန္တု က္တာဝီ။}
\vutti{ဘူတေ-တ္ထေ ဝတ္တမာနတော ကြိယတ္ထာ က္တဝန္တုတ္တာဝီ ဟောန္တိ ကတ္တရိ။ ဝိဇိတဝါ၊ ဝိဇိတာဝီ၊ ဘူတေတိ အဓိကာရော ယာဝ “အာဟာရတ္ထာ ”တိ (၅-၆၀)။}

\sutta{650}{56}{က္တောဘာဝကမ္မေသု။}
\vutti{ဘာဝေ ကမ္မေ စ ဘူတေ က္တော ဟောတိ။ အာသိတံ ဘဝတာ။ ကတော ကဋော ဘဝတာ။}

\sutta{651}{57}{ကတ္တရိ စာရမ္ဘေ။}
\vutti{ကြိယာရမ္ဘေ ကတ္တရိ က္တော ဟောတိ ယထာပတ္တဉ္စ။ ပကတော ဘဝံ ကဋံ၊ ပကတော ကဋော ဘဝတာ၊ ပသုတ္တော ဘဝံ၊ ပသုတ္တံ ဘဝတာ။}

\sutta{652}{58}{ဌာ-သ ဝသ သိလိသ သီ ရုဟ ဇရ ဇနီဟိ။}
\vutti{ဌာဒီဟိ ကတ္တရိ က္တော ဟောတိ ယထာပတ္တဉ္စ။ ဥပဋ္ဌိတော ဂုရုံ ဘဝံ၊ ဥပဋ္ဌိတော ဂုရု ဘောတာ၊ ဥပါသိတော ဂုရုံ ဘဝံ၊ ဥပါသိတော ဂုရု ဘောတာ၊ အနုဝုသိတော ဂုရုံ ဘဝံ၊ အနုဝုသိတော ဂုရု ဘောတာ၊ အာသိလိဋ္ဌော ဂုရုံ ဘဝံ၊ အာသိလိဋ္ဌော ဂုရု ဘောတာ၊ အဓိဿိတော ခဋောပိကံ ဘဝံ၊ အဓိဿိတာ ခဋောပိကာ ဘောတာ၊ အာရဠှော ရုက္ခံ ဘဝံ၊ အာရုဠှော ရုက္ခော ဘောတာ၊ အနုဇိဏ္ဏော ဝသလိံ ဒေဝဒတ္တော၊ အနုဇိဏ္ဏာ ဝသလီ ဒေဝဒတ္တေန၊ အနုဇာတော မာဏဝကော မာဏဝိကံ၊ အနုဇာတာ မာဏဝိကာ �မာဏဝကေန။}

\sutta{653}{59}{ဂမနတ္ထာကမ္မကာဓာရေ စ။}
\vutti{ဂမနတ္ထတော အကမ္မကတော စ ကြိယတ္ထာ အာဓာရေ က္တော ဟောတိ ကတ္တရိ စ ယထာပတ္တဉ္စ၊ ဣဒမေသံ ယာတံ၊ ဣဟ တေ ယာတာ၊ ဣဟ တေဟိ ယာတံ၊ အယံ တေဟိ ယာတော ပထော၊ ဣဒမေသမာသိတံ၊ ဣဟ တေ အာသိတာ၊ ဣဟတေဟိ အာသိတံ၊ ‘ဒေါဝေါ စေ ဝုဋ္ဌော သမ္ပန္နာ သာလယော ’တိ ကာရဏသာမဂ္ဂီသမ္ပတ္တိ ဧတ္ထာဘိမတာ။}

\sutta{654}{60}{အာဟာရတ္ထာ။}
\vutti{အဇ္ဈောဟာရတ္ထာ အာဓာရေ က္တော ဟောတိ ယထာပတ္တဉ္စ၊ ဣဒမေသံ ဘုတ္တံ၊ ဣဒမေသံ ပီတံ၊ ဣဟ တေဟိ ဘုတ္တံ၊ ဣဟ တေဟိ ပီတံ၊ ဩဒနော တေဟိ ဘုတ္တော ပီတမုဒကံ၊ အကတ္တတ္ထော ယောဂဝိဘာဂေါ၊ ကထံ ‘ပီတာ ဂါဝေါ ’တိ? ပီတမေသံ ဝိဇ္ဇတီတိ ပီတာ၊ ဗာဟုလကာ ဝါ၊ ‘ပဿိန္နော ’တိ ယာ ဧတ္ထ ဘူတကာလတာ၊ တတြ တ္တော၊ ဧဝံ ရညံ မတော ရညံ ဣဋ္ဌော၊ ရညံ ဗုဒ္ဓေါ၊ ရညံ ပူဇိတော၊ ဧဝံ သီလိတော၊ ရက္ခိတော၊ ခန္တော၊ အာကုဋ္ဌော၊ ရုဋ္ဌော၊ ရုသိတော၊ အဘိဗျာဟဋော၊ ဒယိယော၊ ဟဋ္ဌော၊ ကန္တာ၊ သံယတော၊ အမတော၊ ‘ကဋ္ဌ ’န္တိ ဘူတတာယမေဝ ဟေတုနော၊ ဖလံ တွတြ ဘာဝိ။}

\sutta{655}{61}{တုံ တာယေ တဝေ ဘာဝေ ဘဝိဿတိ ကြိယာယံ တဒတ္ထာယ။}
\vutti{ဘဝိဿတိ အတ္ထေ ဝတ္တမာနတော ကြိယတ္ထာ ဘာဝေ တုံ တာယေ တဝေ ဟောန္တိ ကြိယာယံ တဒတ္ထာယံ ပတီယမာနာယံ။ ကာတုံ ဂစ္ဆတိ၊ ကတ္တာယေ ဂစ္ဆတိ၊ ကာတဝေ ဂစ္ဆတိ၊ ဣစ္ဆတိ ဘောတ္တုံ ကာမေတိ ဘောတ္တုန္တိ ဣမီနာဝ သိဒ္ဓံ၊ ပုနဗ္ဗိဓာနေ တွိဟာပိ သိယာ ‘ဣစ္ဆန္တော ကရောတီ ’တိ၊ ဧဝံ သက္ကောတိ ဘောတ္တုံ၊ ဇာနာတိ ဘောတ္တုံ၊ ဂိလာယတိ ဘောတ္တုံ၊ ဃဋတေ ဘောတ္တုံ၊ အာရဘတေ ဘောတ္တုံ၊ လဘတေ ဘောတ္တုံ၊ ပက္ကမတိ ဘောတ္တုံ၊ ဥဿဟတိ ဘောတ္တုံ၊ အရဟတိ ဘောတ္တုံ၊ အတ္ထိ ဘောတ္တုံ၊ ဝိဇ္ဇတိ ဘောတ္တုံ၊ ဝဋ္ဋ တိဘောတ္တုံ၊ ကပ္ပတိ ဘောတ္တုန္တိ။ တထာ ပါရယတိ ဘောတ္တုံ၊ ပဟု ဘောတ္တုံ၊ သမတ္ထော ဘောတ္တုံ၊ ပရိယတ္တော ဘောတ္တုံ၊ အလံ ဘောတ္တုန္တိ ဘဝတိဿ သဗ္ဗတ္ထ သမ္ဘဝါ။ တထာ ကာလော ဘောတ္တုံ။ သမယော ဘောတုံ၊ ဝေလာ ဘောတုန္တိ၊ ယထာ ဘောတ္တုံမနော၊ သောတ္တုံ သောတော၊ ဒဋ္ဌုံ စက္ခု၊ ယုဇ္ဈိတုံ ဓနု၊ ဝတ္တုံ ဇဠော၊ ဂန္တုမနော၊ ကတ္တုမလသောတိ၊ ဥစ္စာရဏန္တု ဝတ္တာယတ္တံ။ ဘာဝေတိ ကိံ? ကရိဿာမီတိ ဂစ္ဆတိ၊ ကြိယာယန္တိ ကိံ? ဘိက္ခိဿံ ဣစ္စဿ ဇဋာ၊ တဒတ္ထာယန္တိ ကိံ? ဂစ္ဆိဿတော တေ ဘဝိဿတိ ဘတ္တံ ဘောဇနာယ။}

\sutta{656}{62}{ပဋိသေဓေ-လံခလူနံ တုန တွာန တွာ ဝါ။}
\vutti{အလံ ခလုသဒ္ဒါနံ ပဋိသေဓတ္ထာနံ ပယောဂေ တုနာဒယော ဝါ ဟောန္တိ ဘာဝေ။ အလံ သောတုန၊ ခလု သောတုန၊ အလံ သုတွာန၊ ဓလု သုတွာန၊ အလံ သုတွာ၊ ခလု သုတွာ၊ အလံ သုတေန၊ ခလု သောတေန၊ အလံ ခလူနန္တိ ကိံ? မာ ဟောတု၊ ပဋိသေဓေတိ ကိံ? အလင်္ကာရော။}

\sutta{657}{63}{ပုဗ္ဗေကကတ္ထုကာနံ။}
\vutti{ဧကော ကတ္တာ ယေသံ ဗျာပါရာနံ၊ တေသု ယော ပုဗ္ဗော၊ တဒတ္ထတော ကြိယတ္ထာ တုနာဒယော ဟောန္တိ ဘာဝေ၊ သောတုန ယာတိ၊ သုတွာန၊ သုတွာ ဝါ၊ ဧကကတ္တုကာနန္တိကိံ? ဘုတ္တသ္မိံ ဒေဝဒတ္တေ ယညဒတ္တော ဝဇတိ၊ ပုဗ္ဗာတိ ကိံ? ဘုဉ္ဇဘိ စ ပစတိ စ။ ‘အပ္ပတွာ နဒိံ ပဗ္ဗတော အတိက္ကမ္မ ပဗ္ဗတံ နဒီ ’တိ ဘူဓာတုဿ သဗ္ဗတ္ထ သမ္ဘဝါ ဧကကတ္တုကတာ ပုဗ္ဗကာလတာ စ ဂမျတေ။ ‘ဘုတွာ ဘုတွာ ဂစ္ဆတီတိ ’ ဣမိနာဝ သိဒ္ဓံ အာဘိက္ခညန္တု ဒွိဗ္ဗစနာဝဂမျတေ။ ကထံ ‘ဇီဝဂ္ဂါဟံ အဂါဟယိ၊ ကာယပ္ပစာလကံ ဂစ္ဆန္တီ ’တိ အာဒိ? ဃဏန္တေန ကြိယာဝိသေသနေန သိဒ္ဓံ ယထာ ‘ဩဒနပါကံ သယတီ ’တိ။}

\sutta{658}{64}{န္တော ကတ္ထရိ ဝတ္တမာနေ။}
\vutti{ဝတ္တမာနတ္ထေ ဝတ္တမာနတော ကြိယတ္တာ န္တော ဟော တိ ကတ္တရိ၊ တိဋ္ဌန္တော။}

\sutta{659}{65}{မာနော။}
\vutti{ဝတ္တမာနတ္ထေ ဝတ္တမာနတော ကြိယတ္ထာ မာနော ဟောတိ၊ ကတ္တရိ။ တိဋ္ဌမာနော။}

\sutta{660}{66}{ဘာဝကမ္မေသု။}
\vutti{ဝတ္တမာနတ္ထေ �ဝတ္တမာနတော ကြိယတ္ထာ ဘာဝေ ကမ္မေစ မာနော ဟောတိ။ ဌီယမာနံ၊ ပစ္စမာနော ဩဒနော။}

\sutta{661}{67}{တေ ဿပုဗ္ဗာနာဂတေ။}
\vutti{အနာဂတတ္ထေ ဝတ္တမာနတော ကြိယတ္ထာ တေန္တမာနာ ဿပုဗ္ဗာ ဟောန္တိ။ ဌဿန္တော၊ ဌဿမာနော၊ ဌီယိဿမာနံ၊ ပစ္စိဿမာနော ဩဒနော။}

\sutta{662}{68}{ဏွာဒယော။}
\vutti{ကြိယတ္ထာ ပရေ ဗဟုလံ ဏွာဒယော ဟောန္တိ။ စာရု၊ ဒါရု။}

\sutta{663}{69}{ခဆသာန မေကဿရောဒိ ဒွေ။}
\vutti{ခဆသပ္ပစ္စယန္တာနံ ကြိယတ္ထာနံ ပဌမေကဿရံ သဒ္ဒရူပံ ဒွေ ဘဝတိ။ တိတိက္ခာ၊ ဇိဂုစ္ဆာ၊ ဝီမံသာ။}

\sutta{664}{70}{ပရောက္ခာယဉ္စ။}
\vutti{ပရောက္ခာယံ ပဌမေကဿရံ သဒ္ဒရူပံ ဒွေ ဘဝတိ။ ဇဂါမ၊ စကာရော အနုတ္တသမုစ္စယတ္ထော၊ တေနညတြာပိ ယထာဂမံ၊ ဇဟာတိ၊ ဇဟိတဗ္ဗံ၊ ဇဟိတုံ၊ ဒဒ္ဒလ္လတိ၊ စင်္ကမတိ။ ‘လောလုပေါ၊ မောမူဟောတိ ဩတ္တံ တဒမိနာဒိပါဌာ။}

\sutta{665}{71}{အာဒိသ္မာ သရာ။}
\vutti{အာဒိဘူတာ သရာ ပရမေကဿရံ ဒွေ ဟောတိ၊ အသိသိသတိ၊ အာဒိသ္မာတိ ကိံ? ဇဇဂါရ၊ သရာတိ ကိံ? ပပါစ။}

\sutta{666}{72}{န ပုန။}
\vutti{ယံ ဒွိဘူတံ၊ န တံ ပုန ဒွိတ္တမာပဇ္ဇတေ၊ တိတိက္ခိသတိ၊ ဇိဂုစ္ဆိသတိ။}

\sutta{667}{73}{ယထိဋ္ဌံ သျာဒိနော။}
\vutti{သျာဒျန္တဿ ယထိဋ္ဌမေကဿရမာဒိဘူတမညံ ဝါ ယထာဂမံ ဒွိတ္တပဇ္ဇတေ၊ ပုပုတ္တီယိသတိ၊ ပုတိတ္တီယိသတိ၊ ပုတ္တီယိယိသတိ။}

\sutta{668}{74}{ရဿော ပုဗ္ဗဿ။}
\vutti{ဒွိတ္တေ ပုဗ္ဗဿ သရော ရဿော ဟောတိ။ ဒဒါတိ။}

\sutta{669}{75}{လောပေါ-နာဒိဗျဉ္ဇနဿ။}
\vutti{ဒွိတ္တေ ပုဗ္ဗဿာဒိတော-ညဿ ဗျဉ္ဇနဿ လောပေါ ဟောတိ။ အသိသိသတိ။}

\sutta{670}{76}{ခဆသေသွဿိ။}
\vutti{ဒွိတ္တေ ပုဗ္ဗဿ အဿ ဣ ဟောတိ ခဆသေသု။ ပိပါသတိ၊ ဇိဃံသတိ၊ ခဆသေသူတိ ကိံ? ဇဟာတိ၊ အဿာတိ ကိံ? ဗုဘုက္ခတိ။}

\sutta{671}{77}{ဂုပိဿုဿ။}
\vutti{ဒွိတ္တေ ပုဗ္ဗဿ ဂုပိဿ ဥဿ ဣ ဟောတိ ခဆသေသု၊ ဇိဂုစ္ဆတိ။}

\sutta{672}{78}{စတုတ္ထဒုတိယာနံ တတိယပဌမာ။}
\vutti{ဒွိတ္တေ ပုဗေသံ စတုတ္ထဒုတိယာနံ တတိယပဌမာ ဟောန္တိ။ ဗုဘုက္ခတိ၊ စိစ္ဆေဒ။}

\sutta{673}{79}{ကဝဂ္ဂဟာနံ စဝဂ္ဂဇာ။}
\vutti{ဒွိတ္တေ ပုဗ္ဗေသံ ကဝဂ္ဂဟာနံ စဝဂ္ဂဇာ ဟောန္တိ ယထာက္ကမံ။ စုကောပ၊ ဇဟာတိ။}

\sutta{674}{80}{မာနဿ ဝီ ပရဿ စ မံ။}
\vutti{ဒွိတ္တေ ပုဗ္ဗဿ မာနဿ ဝီ ဟောတိ ပရဿ စ မံ၊ ဝီမံသတိ။}

\sutta{675}{81}{ကိတဿာသံသယေ တိ ဝါ။}
\vutti{သံသယသော-ညသ္မိံ ဝတ္တမာနဿ ဒွိတ္တေ ပုဗ္ဗဿ ကိတဿ ဝါ တိ ဟောတိ။ တိကိစ္ဆတိ၊ စိကိစ္ဆတိ၊ အသံသယေတိ ကိံ? ဝိစိကိစ္ဆတိ။}

\sutta{676}{82}{ယုဝဏ္ဏာနမေဩ ပစ္စယေ။}
\vutti{ဣဝဏ္ဏုဝဏ္ဏန္တာနံ ကြိယတ္ထာနံ ဧဩဟောန္တိ ယထာက္ကမံ ပစ္စယေ။ စေတဗ္ဗံ၊ နေတဗ္ဗံ၊ သောတဗ္ဗံ၊ ဘဝိတဗ္ဗံ။}

\sutta{677}{83}{လဟုဿုပန္တဿ။}
\vutti{လဟုဘူတဿ ဥပန္တဿ ယုဝဏ္ဏဿ ဧဩ ဟောန္တိ ယထာက္ကမံ။ ဧသိတဗ္ဗံ၊ ကောသိတဗ္ဗံ၊ လဟုဿာတိ ကိံ? ဓူပိတာ၊ ဥပန္တဿာတိ ကိံ? ရုန္ဓတိ။}

\sutta{678}{84}{အဿာ ဏာနုဗန္ဓေ။}
\vutti{ဏကာရာနုဗန္ဓေ ပစ္စယေ ပရေ ဥပန္တဿ အကာရဿ အာ ဟောတိ။ ကာရကော။}

\sutta{679}{85}{န တေ ကာနုဗန္ဓနာဂမေသု။}
\vutti{တေ ဧဩအာ ကာနုဗန္ဓေ နာဂမေ စ န ဟောန္တိ။ စိတော၊ သုတော၊ ဒိဋ္ဌော၊ ပုဋ္ဌော၊ နာဂမေ ‘ဝနာ ’ဒိနာ (၁.၄၅) စိနိတဗ္ဗံ၊ စိနိတုံ၊ သုဏိတဗ္ဗံ၊ သုဏိတုံ ပါပုဏိတဗ္ဗံ၊ ပါပုဏိတုံ၊ ဓုနိတဗ္ဗံ၊ ဓုနိတုံ၊ ဓုနနံ၊ ဓုနယိတဗ္ဗံ၊ ဓုနာပေတဗ္ဗံ၊ ဓုနယိတုံ ဓုနာပေတုံ၊ ဓုနယနံ၊ ဓုနာပနံ၊ ဓုနယတိ၊ ဓုနာပေတိ၊ ပီနေတဗ္ဗံ၊ ပီနယိတုံ၊ ပီနနံ၊ ပီနိတုံ၊ ပီနယတိ၊ သုနောတိ၊ သိနောတိ၊ ဒုနောတိ၊ ဟိနောတိ၊ ပဟိဏိထဗ္ဗံ၊ ပဟိဏိတုံ၊ ပဟိဏနံ။}

\sutta{680}{86}{ဝါ ကွစိ။}
\vutti{တေ ကွစိ ဝါ န ဟောန္တိ ကာနုဗန္ဓနာဂမေသု။ မုဒိတော၊ ရုဒိတံ၊ ရောဒိတံ။}

\sutta{681}{87}{အညတြာပိ။}
\vutti{ကာနုဗန္ဓနာဂမတော-ညသ္မိမ္ပိ တေ ကွစိ၊ နု ဟောန္တိ။ ခိပကော၊ ပနူဒနံ၊ ဝဓကော။}

\sutta{682}{88}{ပျေ သိဿာ။}
\vutti{သိဿ အာ ဟောတိ ပျာဒေသေ၊ နိဿာယ။}

\sutta{683}{89}{ဧဩနမယဝါ သရေ။}
\vutti{သရေ ပရေ ဧဩနံ အယအဝါ ဟောန္တိ။ ဇယော၊ ဘဝေါ၊ သရတိ ကိံ? ဇေတိ၊ အနုဘောတိ။}

\sutta{684}{90}{အာယာဝါ ဏာနုဗန္ဓေ။}
\vutti{ဧဩနံ အာယာဝါ ဟောန္တိ သရာဒေါ ဏာနုဗန္ဓေ။ နာယယတိ၊ ဘာဝယတိ၊ ‘သယာပေတွာ ’တိအာဒီသု ရဿတ္တံ။}

\sutta{685}{91}{အာဿာဏာပိမှိ ယုက။}
\vutti{အာကာရန္တဿ ကြိယတ္ထဿ ယုက ဟောတိ ဏာပိတော-ညသ္မိံ ဏာနုဗန္ဓေ။ ဒါယကော၊ ဏာနုဗန္ဓေတွေဝ? ဒါနံ၊ အဏာပိမှီတိ ကိံ? ဒါပယတိ။}

\sutta{686}{92}{ပဒါဒီနံ ကွစိ။}
\vutti{ပဒါဒီနံ ယုက ဟောတိ ကွစိ။ နိပဇ္ဇိတဗ္ဗံ၊ နိပဇ္ဇိတုံ နိပဇ္ဇနံ၊ ပမဇ္ဇိတဗ္ဗံ၊ ပမဇ္ဇိတုံ၊ ပမဇ္ဇနံ၊ ကွစီတိ ကိံ? ပါဒေါ။}

\sutta{687}{93}{မံ ဝါ ရုဓာဒီနံ။}
\vutti{ရုဓာဒီနံ ကွစိ မံ ဝါ ဟောတိ။ ရုန္ဓိတုံ၊ ရုဇ္ဈိတုံ၊ ကွစိတွေဝ? နိရောဓော။}

\sutta{688}{94}{ကွိမှိ လောပေါ-န္တဗျဉ္ဇနဿ။}
\vutti{အန္တဗျဉ္ဇနဿ လောပေါ ဟောတိ ကွိမှိ။ ဘတ္တံ ဂသန္တိ ဂဏှန္တိ ဝါ ဧတ္ထာတိ ဘတ္တဂ္ဂံ။}

\sutta{689}{95}{ပရရူပမယကာရေ ဗျဉ္ဇနေ။}
\vutti{ကြိယတ္ထာနမန္တဗျဉ္ဇနဿ ပရရူပံ ဟောတိ ယကာရတော-ညသ္မိံ ဗျဉ္ဇနေ။ ဘေတ္တဗ္ဗံ၊ ဗျဉ္ဇနေတိ ကိံ? ဘိန္ဒိတဗ္ဗံ၊ အယကာရေတိ ကိံ? ဘိဇ္ဇတိ။}

\sutta{690}{96}{မနာနံ နိဂ္ဂဟီတံ။}
\vutti{မကာရနကာရန္တာနံ ကြိယတ္ထာနံ နိဂ္ဂဟီတံ ဟောတိ အယကာရေ ဗျဉ္ဇနေ။ ဂန္တဗ္ဗံ၊ ဇင်္ဃာ၊ ဗျဉ္ဇနေတွေဝ? ဂမနံ၊ အယကာရေတွေဝ? ဂမျတေ။}

\sutta{691}{97}{န ဗြူဿော။}
\vutti{ဗြူဿ ဩ န ဟောတိ ဗျဉ္ဇနေ။ ဗြူမိ၊ ဗျဉ္ဇနေတွေဝ? အဗြဝိ။}

\sutta{692}{98}{ကဂါ စဇာနံ ဃာနုဗန္ဓေ။}
\vutti{ဃာနုဗန္ဓေ စကာရဇကာရန္တာနံ ကြိယတ္ထာနံ ကဂါ ဟောန္တိ ယထာက္ကမံ။ ဝါကျံ၊ ဘာဂျံ။}

\sutta{693}{99}{ဟနဿ ဃာတော ဏာနုဗန္ဓေ။}
\vutti{ဟနဿ ဃာတော ဟောတိ ဏာနုဗန္ဓေ။ အာဃာတော။}

\sutta{694}{100}{ကွိမှိ ဃော ပရိပစ္စ-သမောဟိ။}
\vutti{ပယျာဒီဟိ ပရဿ ဟနဿ ဃော ဟောတိ ကွိမှိ။ ပလိဃော၊ ပဋိဃော၊ အဃံ ရဿတ္တံ နိပါတနာ၊ သင်္ဃော၊ ဩဃော။}

\sutta{695}{101}{ပရဿ ဃံ သေ။}
\vutti{ဒွိတ္တေ ပရဿ ဟနဿ ဃံ ဟောတိ သေ။ ဇိဃံသာ။}

\sutta{696}{102}{ဇိဟရာနံ ဂီ။}
\vutti{ဒွိတ္တေ ပရေသံ ဇိတရာနံ ဂီ ဟောတိ သေ။ ဝိဇိဂီသာ၊ ဇိဂီသာ။}

\sutta{697}{103}{ဓာဿ ဟော။}
\vutti{ဒွိတ္တေ ပရဿ ဓာဿ ဟ ဟောတိ။ ဒဟတိ။}

\sutta{698}{104}{ဏိမှိ ဒီဃော ဒုသဿ။}
\vutti{ဒုသဿ ဒီဃော ဟောတိ ဏိမှိ။ ဒုသိတော။ ဏိမှီတိ ကိံ? ဒုဋ္ဌော။}

\sutta{699}{105}{ဂုဟိဿ သရေ။}
\vutti{ဂုဟိဿ ဒီဃော ဟောတိ သရေ။ နိဂူဟနံ သရေတိ ကိံ? ဂုယှံ။}

\sutta{700}{106}{မုဟဗဟာနဉ္စ တေ ကာနုဗန္ဓေ-တွေ။}
\vutti{မုဟဗဟာနံ ဂုဟိဿ စ ဒီဃော ဟောတိ တကာရာဒေါ ကာနုဗန္ဓေ တွာနတွာဝဇ္ဇိတေ၊ မူဠှော၊ ဗာဠှော၊ ဂူဠှော၊ တေတိ ကိံ? မုယှတိ၊ ကာနုဗန္ဓေတိ-ကိံ? မုယှိတဗ္ဗံ၊ အတွေတိ ကိံ? မုယှိတွာန၊ မုယှိတွာ၊ ‘တေ ကာနုဗန္ဓေ-တွေ ’တိ အယမဓိကာရော ယာဝ “သာသဿ သိသွေ ”တိ -၁၁၇။}

\sutta{701}{107}{ဝဟဿုဿ။}
\vutti{ဝဟဿ ဥဿ ဒီဃော ဟောတိ တေ ကာနုဗန္ဓေ တွာနတွာဝဇ္ဇိတေ။ ဝုဋ္ဌော။}

\sutta{702}{108}{ဓာဿ ဟိ။}
\vutti{ဓာ=ဓာရဏေတီမဿ ဟိ ဟောတိ တေ ကာနုဗန္ဓေ တွာနတွာဝဇ္ဇိတေ။ နိဟိတော၊ နိဟိတဝါ။}

\sutta{703}{109}{ဂမာဒိရာနံ လောပေါ-န္တဿ။}
\vutti{ဂမာဒီနံ ရကာရန္တာနံ စ အန္တဿ လောပေါ ဟောတိ တေ ကာနုဗန္ဓေ တွာနတွာဝဇ္ဇိတေ။ ဂတော၊ ခတော၊ ဟတော၊ မတော၊ တတော၊ သညတော၊ ရတော၊ ကတော၊ တေတွေဝ? ဂမျတေ၊ ကာနုဗန္ဓေတွေဝ? ဂန္တဗ္ဗံ၊ အတွေတွေဝ? ဂန္တွာန၊ ဂန္တွာ။}

\sutta{704}{110}{ဝစာဒီနံ ဝဿုဋ ဝါ။}
\vutti{ဝစာဒီနံ ဝဿ ဝါ ဥဋ ဟောတိ ကာနုဗန္ဓေ-တွေ။ ဥတ္တံ၊ ဝုတ္တံ၊ ဥဋ္ဌံ၊ ဝုဋ္ဌံ၊ ‘အတွေတွေဝ? ဝတွာန၊ ဝတွာ။}

\sutta{705}{111}{အဿု။}
\vutti{ဝစာဒီနမဿ ဥ ဟောတိ ကာနုဗန္ဓေ-သွေ။ ဝုတ္တံ၊ ဝုဋ္ဌံ။}

\sutta{706}{112}{ဝဒ္ဓဿ ဝါ။}
\vutti{ဝဒ္ဓဿ အဿ ဝါ ဥ ဟောတိ ကာနုဗန္ဓေ တွေ။ ဝုဒ္ဓေါ။ ဝဒ္ဓေါ။ အတွေတွေဝ? ဝဒ္ဓိတွာန၊ ဝဒ္ဓိတွာ၊ ကထံ ‘ဝုတ္တီ ’တိ? “ဝုတ္တီမတ္တေ ”တိ ၃-၆၉. နိပါတနာ၊ ‘ဝတ္တီ ’တိ ဟောတေဝ ယထာလက္ခဏံ။}

\sutta{707}{113}{ယဇဿ ယဿ ဋိယီ။}
\vutti{ယဇဿ ယဿ ဋိယီ ဟောန္တိ ကာနုဗန္ဓေ-တွေ။ ဣဋ္ဌံ၊ ယိဋ္ဌံ၊ အတွေတွေဝ? ယဇိတွာန၊ ယဇိတွာ။}

\sutta{708}{114}{ဌာဿိ။}
\vutti{ဌာဿိ ဟောတိ ကာနုဗန္ဓေ-တွေ။ ဌိတော၊ အတွေတွေဝ? ဌတွာန၊ ဌတွာ။}

\sutta{709}{115}{ဂါပါနမီ။}
\vutti{ဂါပါနမီ ဟောတိ ကာနဗန္ဓေ-တွေ။ ဂီတံ၊ ပီတံ၊ အတွေတွေဝ? ဂါယိတွာ နိစ္စံ ယာဂမော၊ ပါဿ တု ပီတွာတိ ဗဟုလာဓိကာရာ။}

\sutta{710}{116}{ဇနိဿာ။}
\vutti{ဇနိဿ အာ ဟောတိ ကာနုဗန္ဓေ-တွေ။ ဇာတော။ အတွေတွေဝ? ဇနိတွာ။}

\sutta{711}{117}{သာသဿ သိသ ဝါ။}
\vutti{သာသဿ သိသ ဝါ ဟောတိ ကာနုဗန္ဓေ-တွေ။ သိဋ္ဌံ၊ သတ္ထံ၊ သိဿော၊ သာသိယော အတွေတွေဝ? အနုသာသိတွာန။}

\sutta{712}{118}{ကရဿာ တဝေ။}
\vutti{ကရဿ အာ ဟောတိ တဝေ။ ကာတဝေ။}

\sutta{713}{119}{တုံတုနတဗ္ဗေသု ဝါ။}
\vutti{တုမာဒီသု ဝါ ကရဿာ ဟောတိ။ ကာတုံ ကတ္တုံ၊ ကာတုန ကတ္တုန၊ ကာတဗ္ဗံ ကတ္တဗ္ဗံ။}

\sutta{714}{120}{ဉာဿ နေ ဇာ။}
\vutti{ဉာဓာတုဿ ဇာ ဟောတိ နကာရေ။ ဇာနိတုံ၊ ဇာနန္တော၊ နေတိ ကိံ? ဉာတော။}

\sutta{715}{121}{သကာပါနံ ကုကကူ ဏေ။}
\vutti{သကအာပါနံ ကုကကုဣစ္စေတေ အာဂမာ ဟောန္တိ ဏကာရေ။ သက္ကုဏန္တော၊ ပါပုဏန္တော၊ သက္ကုဏောတိ၊ ပါပုဏောတိ၊ ဏေတိ ကိံ? သက္ကောတိ၊ ပါပေတိ။}

\sutta{716}{122}{နိတော စိဿ ဆော။}
\vutti{နိသ္မာ ပရဿ စိဿ ဆော ဟောတိ။ နိစ္ဆယော။}

\sutta{717}{123}{ဇရသဒါနမီမ ဝါ။}
\vutti{ဇရသဒါနမန္တသရာ ပရော ဤမ ဟောတိ ဝိဘာသာ။ ဇီရဏံ၊ ဇီရတိ၊ ဇီရာပေတိ၊ နိသီဒိတဗ္ဗံ၊ နိသီဒနံ၊ နိသီဒိတုံ၊ နိသီဒတိ၊ ဝါတိ ကိံ? ဇရာ၊ နိသဇ္ဇာ၊ ‘ဤမ ဝါ ’တိ ယောဂဝိဘာဂါ အညေသမ္ပိ၊ အဟီရထ၊ သံယောဂါဒိ လောပေါတ္ထဿ။}

\sutta{718}{124}{ဒိသဿ ပဿ ဒဿ ဒသ ဒ ဒက္ခာ။}
\vutti{ဒိသဿ ပဿာဒယော ဟောန္တိ ဝိဘာသာ။ ဝိပဿနာ၊ ဝိပဿိတုံ၊ ဝိပဿတိ၊ သုဒဿီ၊ ပိယဒဿီ၊ ဓမ္မဒဿီ၊ သုဒဿံ၊ ဒဿနံ၊ ဒဿေတိ၊ ဒဋ္ဌဗ္ဗံ။ ဒဋ္ဌာ၊ ဒဋ္ဌုံ၊ ဒုဒ္ဒသော၊ အဒ္ဒသ၊ အဒ္ဒါ၊ အဒ္ဒံ၊ အဒ္ဒက္ခိ၊ ဒက္ခိဿတိ၊ ဝါတွေဝ? ဒိဿန္တိ ဗာလာ။}

\sutta{719}{125}{သမာနာ ရော ရီရိက္ခကေသု။}
\vutti{သမာနသဒ္ဒတော ပရဿ ဒိသဿ ရ ဟောတိ ဝါ ရီဝိက္ခကေသု။ သရီ၊ သဒီ၊ သရိက္ခော၊ သဒိက္ခော၊ သရိသော၊ သဒိသော။}

\sutta{720}{126}{ဒဟဿ ဒဿ ဍော။}
\vutti{ဒဟဿ ဒဿ ဍော ဟောတိ ဝါ။ ဍာဟော၊ ဒါဟော၊ ဍဟတိ၊ ဒဟတိ။}

\sutta{721}{127}{အနဃဏ သွာပရီဟိ ဠော။}
\vutti{အာပရီဟိ ပရဿ ဒဟဿ ဒဿ ဠော ဟောတိ အနဃဏသု။ အာဠဟနံ၊ ပရိဠာဟော။}

\sutta{722}{128}{အတျာဒိန္တေသွတ္ထိဿ ဘူ။}
\vutti{တျာဒိန္တဝဇ္ဇိတေသု ပစ္စယေသု ‘အသ=ဘုဝိ ’ဣစ္စဿ ဘူ ဟောတိ။ ဘဝိတဗ္ဗံ။ အာဒေသဝိဓာနမသဿာပ္ပယောဂတ္ထမေတသ္မိံ ဝိသယေ၊ ဧတေန ကတ္ထစိ ကဿစိ ဓာတုဿ အပ္ပယောဂါပိ ဉာပိတော ဟောတိ။ အတျာဒိန္တေသူတိ ကိံ? အတ္ထိ၊ သန္တော၊ အတ္ထိဿာတိ ကိံ? အဿတိဿ မာ ဟောတု။}

\sutta{723}{129}{အအာဿာအာဒီသု။}
\vutti{အအာဒေါ၊ အာအာဒေါ၊ ဿာဒေါ စ အတ္ထိဿ ဘူ ဟောတိ။ ဗဘူဝ၊ အဘဝါ၊ အဘဝိဿာ၊ ဘဝိဿတိ။}

\sutta{724}{130}{န္တမာနန္တိယိယုံသွာဒိလောပေါ။}
\vutti{န္တာဒိသူတ္ထိဿာဒိလောပေါ ဟောတိ။ သန္တော၊ သမာနော၊ သန္တိ၊ သန္တု၊ သိယာ၊ သိယုံ၊ ဧတေသွီတိ ကိံ? အတ္ထိ။}

\sutta{725}{131}{ပါဒိတော ဌာဿ ဝါ ဌဟော ကွစိ။}
\vutti{ပါဒီဟိ ကိရိယာဝိသေသဇောတကေဟိ သဒ္ဒေဟိ ပရဿ ဌာဿ ကွစိ ဌဟော ဝါ ဟောတိ။ သဏ္ဌဟန္တော သန္တိဋ္ဌန္တော။ သဏ္ဌဟတိ၊ သန္တိဋ္ဌတိ။ ပ ပရာ အပ သံ အနု အဝ ဩ နိ ဒု ဝိ အဓိ အပိ အတိသု ဥ အဘိ ပတိ ပရိ ဥပ အာ ပါဒီ။ ကွစီတိ ကိံ? သဏ္ဌိတိ။}

\sutta{726}{132}{ဒါဿိယင။}
\vutti{ပါဒိတော ပရဿ ဒါဿ ဣယင ဟောတိ ကွစိ။ အနာဒိယိတွာ၊ သမာဒိယတိ၊ ကွစိတွေဝ? အာဒါယ။}

\sutta{727}{133}{ကရောတိဿ ခေါ။}
\vutti{ပါဒိတော ပရဿ ကရဿ ကွစိ ခ ဟောတိ။ သင်္ခါရော၊ သင်္ခရီယတိ၊ ကရဿာတိ အဝတွာ ကရောတိဿာတိ ဝစနံ တိမှိ စ ဝိကရဏုပ္ပတ္တိဉာပေတုံ။}

\sutta{728}{134}{ပုရာသ္မာ။}
\vutti{ပုရာ ဣစ္စသ္မာ နိပါတာ ပရဿ ကရဿ ခ ဟောတိ။ ပုရက္ခတွာ၊ ပုရေက္ခာရော-ဧတ္တံ တဒမိနာဒိပါဌာ။}

\sutta{729}{135}{နိတော ကမဿ။}
\vutti{နိသ္မာ ပရဿ ကမဿ ကွစိ ခ ဟောတိ၊ နိက္ခမတိ၊ ကွစိတွေဝ? နိက္ခမော။}

\sutta{730}{136}{ယုဝဏ္ဏာနမိယငုဝင သရေ။}
\vutti{ဣဝဏ္ဏုဝဏ္ဏတ္တာနံ ကြိယတ္ထာနမိယငုဝင ဟောတိ သရေ ကွစိ။ ဝေဒိယတိ၊ ဗြုဝန္တိ၊ သရေတိ ကိံ? နိဝေဒေတိ၊ ဗြူတိ၊ ကွစိတွေဝ? ဇယတိ၊ ဘဝတိ။}

\sutta{731}{137}{အညာဒိဿာဿီ ကျေ။}
\vutti{ဉာဒိတော-ညဿ အာကာရန္တဿ ကြိယတ္ထဿ ဤ ဟောတိ ကျေ။ ဒီယတိ၊ အညာဒိဿာတိ ကိံ? ဉာယတိ၊ တာယတိံ။}

\sutta{732}{138}{တနဿာ ဝါ။}
\vutti{တနဿ အာ ဟောတိ ဝါ ကျေ။ တာယတေ၊ တညတေ။}

\sutta{733}{139}{ဒီဃော သရဿ။}
\vutti{သရန္တဿ ကြိယတ္ထဿ ဒီဃော ဟောတိ ကျေ၊ စီယတေ၊ သူယတေ။}

\sutta{734}{140}{သာနန္တရဿ တဿ ဌော။}
\vutti{သကာရန္တတော ကြိယတ္ထာ ပရဿာ-နန္တရဿ တကာရဿ ဌ ဟောတိ။ တုဋ္ဌော၊ တုဋ္ဌဝါ၊ တုဋ္ဌဗ္ဗံ၊ တုဋ္ဌိ၊ အနန္တရဿာတိ ကိံ? တုဿိတွာ။}

\sutta{735}{141}{ကသဿိမ စ ဝါ။}
\vutti{ကသသ္မာ ပရဿာနန္တရဿ တဿ ဌ ဟောတိ ကသဿ ဝါ ဣမ စ။ ကိဋ္ဌံ၊ ကဋ္ဌံ၊ အနန္တရဿာတွေဝ? ကသိတဗ္ဗံ။}

\sutta{736}{142}{ဓသ္တောတြသ္တာ။}
\vutti{ဧတေ သဒ္ဒါ နိပစ္စန္တေ။}

\sutta{737}{143}{ပုစ္ဆာဒိတော။}
\vutti{ပုစ္ဆာဒီဟိ ကြိယတ္ထေဟိ ပရဿာနန္တရဿ တကာရဿ ဌ ဟောတိ။ ပုဋ္ဌော၊ ဘဋ္ဌော၊ ယိဋ္ဌော၊ အနန္တရဿာတွေဝ? ပုစ္ဆိတွာ။}

\sutta{738}{144}{သာသ ဝသ သံသ သသာ ထော။}
\vutti{ဧတေဟိ ပရဿာနန္တရဿ တဿ ထ ဟောတိ၊ သတ္ထံ၊ ဝတ္ထံ၊ ပသတ္ထံ၊ သတ္ထံ။ ကထမနုသိဋ္ဌော (ဝုဋ္ဌော) တိ? ‘တထနရာနံ ဋဋ္ဌဏလာ ’ ၁-၅၂ တိ ဋ္ဌော၊ အနန္တရဿာတွေဝ? သာသိတုံ။}

\sutta{739}{145}{ဓော ဓဟဘေဟိ။}
\vutti{ဓကာရဟကာရဘကာရန္တေဟိ ကြိယတ္ထေဟိ ပရဿာနန္တရဿ တဿ ဓ ဟောတိ။ ဝုဒ္ဓေါ၊ ဒုဒ္ဓံ၊ လဒ္ဓံ။}

\sutta{740}{146}{ဒဟာ ဎော။}
\vutti{ဒဟာ ပရဿာနန္တရဿ တဿ ဎ ဟောတိ။ ဒဍ္ဎော။}

\sutta{741}{147}{ဗဟဿုမ စ။}
\vutti{ဗဟာ ပရဿာနန္တရဿ တဿ ဎော ဟောတိ၊ ဗဟဿုမ စ ဎသန္နိယောဂေန။ ဗုဍ္ဎော။}

\sutta{742}{148}{ရုဟာဒီဟိ ဟော ဠ စ။}
\vutti{ရဟာဒီဟိ ပရဿာနန္တရဿ တဿ ဟ ဟောတိ ဠော စာန္တဿ။ အာရုဠှော၊ ဂူဠှော၊ ဝူဠှော၊ ဗာဠှော၊ (ဩဂါဠှော)၊ အနန္တရဿာတွေဝ? အာရောဟိတုံ။}

\sutta{743}{149}{မုဟာ ဝါ။}
\vutti{မုဟာ ပရဿာနန္တရဿ တဿ ဟ ဟောတိ ဝါ ဠော စာန္တဿ ဟသန္နိယောဂေန။ မူဠှော၊ မုဒ္ဓေါ။}

\sutta{744}{150}{ဘိဒါဒိတော နော က္တက္တဝန္တူနံ။}
\vutti{ဘိဒါဒိတော ပရေသံ က္တက္တဝန္တူနံ တဿ နော ဟောတိ။ ဘိန္နော ဘိန္နဝါ၊ ဆိန္နော ဆိန္နဝါ၊ ဆန္နော ဆန္နဝါ၊ ဆိန္နော ခိန္နဝါ၊ ဥပ္ပန္နော ဥပ္ပန္နဝါ၊ သိန္နော၊ သိန္နဝါ၊ သန္နော သန္နဝါ၊ ပီနော ပီနဝါ၊ သူနော သူနဝါ၊ ဒီနော ဒီနဝါ၊ ဍီနော ဍီနဝါ၊ လီနော လီနဝါ၊ လူနော လူနဝါ၊ က္တက္တဝန္တူနန္တိ ကိံ? ဘိတ္တိ၊ ဆိတ္တိ၊ ဘေတ္တုံ၊ ဆေတ္တုံ။}

\sutta{745}{151}{ဒါတွိန္နော။}
\vutti{ဒါတော ပရေသံ က္တက္တဝန္တူနံ တဿ ဣန္နော ဟောတိ။ ဒိန္နော၊ ဒိန္နဝါ။}

\sutta{746}{152}{ကိရာဒီဟိ ဏော။}
\vutti{ကိရာဒီဟိ ပရေသံ က္တက္တဝန္တူနံ တဿာနန္တရဿ ဏ ဟောတိ၊ ကိဏ္ဏော ကိဏ္ဏဝါ၊ ပုဏ္ဏော ပုဏ္ဏဝါ၊ ခီဏော ခီဏဝါ။}

\sutta{747}{153}{တရာဒီဟိ ရိဏ္ဏော။}
\vutti{တရာဒီဟိ ပရေသံ က္တက္တဝန္တူနံ တဿ ရိဏ္ဏော ဟောတိ။ တိဏ္ဏော တိဏ္ဏဝါ၊ ဇိဏ္ဏော ဇိဏ္ဏဝါ၊ စိဏ္ဏော စိဏ္ဏဝါ။}

\sutta{748}{154}{ဂေါ ဘန္ဇာဒီဟိ။}
\vutti{ဘန္ဇာဒီဟိ ပရေသံ က္တက္တဝန္တူနံ တဿာနန္တရဿ ဂ ဟောတိ။ ဘဂ္ဂေါ ဘဂ္ဂဝါ၊ လဂ္ဂေါ လဂ္ဂဝါ၊ နိမုဂ္ဂေါ နိမုဂ္ဂဝါ၊ သံဝိဂ္ဂေါ သံဝိဂ္ဂဝါ။}

\sutta{749}{155}{သုသာ ခေါ။}
\vutti{သုသာ ပရေသံ က္တက္တဝန္တူနံ တဿ ခေါ ဟောတိ။ သုက္ခော သုက္ခဝါ။}

\sutta{750}{156}{ပစာ ကော။}
\vutti{ပစာ ပရေသံ က္တက္တဝန္တူနံ တဿ ကော ဟောတိ။ ပက္ကော ပက္ကဝါ။}

\sutta{751}{157}{မုစာ ဝါ။}
\vutti{မုစာ ပရေသံ က္တက္တဝန္တူနံ တဿ ကော ဝါ ဟောတိ။ မုက္ကော မုတ္တော၊ မုက္ကဝါ မုတ္တဝါ။ ‘သက္ကော’တိ ဏွာဒီသု သိဒ္ဓံ၊ က္တက္တဝန္တူသု သတ္တော၊ သတ္တာဝါတွေဝ ဟောတိ။}

\sutta{752}{158}{လောပေါ ဝဍ္ဎာ က္တိဿ။}
\vutti{ဝဍ္ဎာ ပရဿ က္တိဿ တဿ လောပေါ ဟောတိ။ ဝဍ္ဎိ။}

\sutta{753}{159}{ကွိဿ။}
\vutti{ကြိယတ္ထာ ပရဿ ကွိဿ လောပေါ ဟောတိ၊ အဘိဘူ။}

\sutta{754}{160}{ဏိဏာပီနံ တေသု။}
\vutti{ဏိဏာပီနံ လောပေါ ဟောတိ တေသု ဏိဏာပီသု။ ကာရေန္တံ ပယောဇယတိ ကာရေတိ ကာရာပယတိ။}

\sutta{755}{161}{ကွစိ ဝိကရဏာနံ။}
\vutti{ဝိကရဏာနံ ကွစိ လောပေါ ဟောတိ။ ဥဒပါဒိ၊ ဟန္တိ။}

\sutta{756}{162}{မာနဿ မဿ။}
\vutti{ကြိယတ္ထာ ပရဿ မာနဿ မကာရဿ လောပေါ ဟောတိ ကွစိ။ ကရာဏော၊ ကွစီတိ ကိံ? ကုရုမာနော။}

\sutta{757}{163}{ဉိ လဿေ။}
\vutti{ဉိလာနမေ ဟောတိ ကွစိ။ ဂဟေတွာ၊ အဒေန္တိ၊ ကွစိတွေဝ? ဝပိတွာ။}

\sutta{758}{164}{ပျော ဝါ တွာဿ သမာသေ။}
\vutti{တွာဿ ဝါ ပျော ဟောတိ သမာသေ။ ပကာရော “ပျေ သိဿာ ” တိ ၅-၈၈ ဝိသေသနတ္ထော။ အဘိဘူယ၊ အဘိဘဝိတွာ၊ သမာသေတိ ကိံ? ပတွာ၊ ကွစာသမာသေပိ ဗဟုလာဓိကာရာ ‘လတံ ဒန္တေဟိ ဆိန္ဒိယ ’။}

\sutta{759}{165}{တုံယာနာ။}
\vutti{က္တွာဿ ဝါ တုံယာနာ ဟောန္တိ သမာသေ ကွစ္စိ။ အဘိဟဋ္ဌုံ အဘိဟရိတွာ၊ အနုမောဒိယာန အနုမောဒိတွာ၊ အသမာသေပိ ဗဟုလာဓိ ကာရာ၊ ဒဋ္ဌုံ ဒိသွာ၊ ဧသမပ္ပဝိသယတာဉာပနတ္ထော ယောဂဝိဘာဂေါ။}

\sutta{760}{166}{ဟနာ ရစ္စော။}
\vutti{ဟနသ္မာ ပရဿ က္တွာဿ ရစ္စော ဝါ ဟောတိ သမာသေ။ အာဟစ္စ၊ အာဟနိတွာ။}

\sutta{761}{167}{သာသာဓိကရာ စ စရိစ္စာ။}
\vutti{သာသာဓီဟိ ပရာ ကရာ ပရဿ က္တွဿ စစရိစ္စာ ဟောန္တိ ယထာက္ကမံ။ သက္ကစ္စ သက္ကရိတွာ၊ အသက္ကစ္စ အသက္ကရိတွာ၊ အဓိကိစ္စ အဓိကရိတွာ။}

\sutta{762}{168}{ဣတော စ္စော။}
\vutti{ဣဣစ္စသ္မာ ပရဿ က္တွာဿ စ္စော ဝါ ဟောတိ။ အဓိစ္စ အဓီယိတွာ၊ သမေစ္စ သမေတွာ။}

\sutta{763}{169}{ဒိသာ ဝါနဝါသ စ။}
\vutti{ဒိသတော က္တွာဿ ဝါနဝါ ဟောန္တိ ဝါ ဒိသဿ စ သ ကာရော တံသန္နိယောဂေန။ သဿ သဝိဓာနံ ပရရူပဗာဓနတ္ထံ။ ဒိသွာန၊ ဒိသွာ ပဿိတွာ၊ ကထံ ‘နာဒဋ္ဌာ ပရတော ဒေါသ ’န္တိ? ဉာပကာ တွာဿ ဝလောပေါ၊ ဧဝံ ‘လဒ္ဓါ ဓန ’န္တိ အာဒီသု။}

\sutta{764}{170}{ဉိ ဗျဉ္ဇနဿ။}
\vutti{ကြိယတ္ထာ ပရဿ ဗျဉ္ဇနာဒိပ္ပစ္စယဿ ဉိ ဝါ ဟောတိ။ ဘုဉ္ဇိတုံ ဘောတ္တုံ၊ ဗျဉ္ဇနဿာတိ ကိံ? ပါစကော။}

\sutta{765}{171}{ရာ နဿ ဏော။}
\vutti{ရန္တတော ကြိယတ္ထာ ပရဿ ပစ္စယနကာရဿ ဏော ဟောတိ။ အရဏံ၊ သရဏံ။}

\sutta{766}{172}{န န္တမာနတျာဒီနံ။}
\vutti{ရန္တတော ပရေသံ န္တမာနတျာဒီနံ နဿ ဏော န ဟောတိ၊ ကရောန္တော၊ ကုရုမာနော၊ ကရောန္တိ။}

\sutta{767}{173}{ဂမယမိသာသဒိသာနံ ဝါ စ္ဆင။}
\vutti{ဧတေသံ ဝါ စ္ဆင ဟောတိ န္တမာနတျာဒီသု။ ဂစ္ဆန္တော ဂစ္ဆမာနော ဂစ္ဆတိ၊ ယစ္ဆန္နော ယစ္ဆမာနော ယစ္ဆတိ၊ ဣစ္ဆန္တော ဣစ္ဆမာနော ဣစ္ဆတိ အစ္ဆန္တော အစ္ဆမာနော အစ္ဆတိ၊ ဒိစ္ဆန္တော ဒိစ္ဆမာနော ဒိစ္ဆတိ၊ ဝါတိ ကိံ? ဂမိဿတိ၊ ဝဝတ္ထိတဝိဘာသာ-ယံ၊ တေနာညေသု စ ကွစိ-ဣစ္ဆိတဗ္ဗံ ဣစ္ဆာ ဣစ္ဆိတုံ၊ အစ္ဆိတဗ္ဗံ အစ္ဆတုံ၊ အညေသဉ္စ ယောဂဝိဘာဂါ-ပဝေစ္ဆတိ။}

\sutta{768}{174}{ဇရမရာနမီယင။}
\vutti{ဧတေသမီယင ဝါ ဟောတိ န္တမာနတျာဒီသု။ ဇီယန္တော ဇီရန္တော၊ ဇီယမာနော ဇီရမာနော၊ ဇီယတိ ဇီရတိ၊ မီယန္တော မရန္တော၊ မီယမာနော မရမာနော၊ မီယတိ မရတိ။}

\sutta{769}{175}{ဌာပါနံ တိဋ္ဌ ပိဝါ။}
\vutti{ဌာပါနံ တိဋ္ဌပိဝါ ဟောန္တိ န္တမာနတျာဒီသု။ တိဋ္ဌန္တော၊ တိဋ္ဌမာနော၊ တိဋ္ဌတိ၊ ပိဝန္တော၊ ပိဝမာနော၊ ပိဝတိ၊ ဝါတွေဝိ? ဌာတိ၊ ပါတိ။}

\sutta{770}{176}{ဂမဝဒဒါနံ ဃမ္မဝဇ္ဇဒဇ္ဇာ။}
\vutti{ဂမာဒီနံ ဃမ္မာဒယော ဝါ ဟောန္တိ န္တမာနတျာဒီသု။ ဃမ္မန္တော၊ ဂစ္ဆန္တော၊ ဝဇ္ဇန္တော ဝဒန္တော၊ ဒဇ္ဇန္တော ဒဒန္တော။}

\sutta{771}{177}{ကရဿ သောဿ ကုဗ္ဗကုရုကယိရာ။}
\vutti{ကရဿ သဩကာရဿ ကုဗ္ဗာဒယော ဝါ ဟောန္တိ န္တမာနတျာဒီသု။ ကုဗ္ဗန္တော ကယိရန္တော ကရောန္တော၊ ကုဗ္ဗမာနော ကုရုမာနော ကယိရမာနော၊ ကရာဏော၊ ကုဗ္ဗတိ ကယိရတိ ကရောတိ၊ ကုဗ္ဗတေ ကုရုတေ၊ ကယိရတေ၊ ဝဝတ္ထိတဝိဘာသတ္တာ ဝါဓိကာရဿ ဘိယျော မာနပရစ္ဆက္ကေသု ကုရု၊ ကွစိဒေဝ ပုဗ္ဗဆက္ကေ ‘အဂ္ဃံ ကုရုတု၊ နော ဘဝံ၊ သောဿာတိ ဝုတ္တတ္တာ ကတ္တရိယေဝိမေ။}

\sutta{772}{178}{ဂဟဿ ဃေပ္ပော။}
\vutti{ဂဟဿ ဝါ ဃေပ္ပော ဟောတိ န္တမာနတျာဒီသု။ ဃေပ္ပန္တော၊ ဃေပ္ပမာနော၊ ဃေပ္ပတိ၊ ဝါတွေဝ? ဂဏှတိ။}

\sutta{773}{179}{ဏော နိဂ္ဂဟီတဿ။}
\vutti{ဂဟဿ နိဂ္ဂဟီတဿ ဏော ဟောတိ။ ဂဏှိတဗ္ဗံ၊ ဂဏှိတုံ၊ ဂဏှန္တော။}

\begin{jieshu}
ဣတိမောဂ္ဂလ္လာနေ ဗျာကရဏေ ဝုတ္တိယံ

ခါဒိကဏ္ဍော ပဉ္စမော။
\end{jieshu}
\chapter{တျာဒိကဏ္ဍော ဆဋ္ဌော}
\markboth{မောဂ္ဂလ္လာနဗျာကရဏေ}{တျာဒိကဏ္ဍော ဆဋ္ဌော}

\sutta{774}{1}{ဝတ္တမာနေ တိ အန္တိ သိ ထ မိ မ တေ အန္တေ သေ ဝှေ ဧ မှေ။}
\vutti{ဝတ္တမာနေ အာရဒ္ဓါပရိသမတ္တေ အတ္ထေ ဝတ္တမာနတော ကြိယတ္ထာ တျာဒယော ဟောန္တိ။ ဂစ္ဆတိ၊ ဂစ္ဆန္တိ၊ ဂစ္ဆသိ ဂစ္ဆထ၊ ဂစ္ဆာမိ ဂစ္ဆာမ၊ ဂစ္ဆတေ ဂစ္ဆန္တေ၊ ဂစ္ဆသေ ဂစ္ဆဝှေ၊ ဂစ္ဆေ ဂစ္ဆာမှေ။ ကထံ‘ပုရေ အဓမ္မော ဒိပ္ပတိ၊ ပုရာ မရာမီ’တိ? ဝတ္တမာနဿေဝတ္တုမိဋ္ဌတ္တာ တံသမီပဿ တဂ္ဂဟဏေန ဂဟဏာ၊ ပုရေပုရာသဒ္ဒေဟိ ဝါ အနာဂတတ္တာဝဂမေ တဒါ တဿ ဝတ္တမာနတ္တာ၊ ကာလဗျတ္တယော ဝါ ဧသော၊ ဘဝန္တေဝ ဟိ ကာလန္တရေပိ တျာဒယော ဗာဟုလကာ‘သန္တေသု ပရိဂူဟာမိ၊ မာ စ ကိဉ္စ ဣတော အဒံ’‘ကာယဿ ဘေဒါ အဘိသမ္ပရာယံ၊ သဟဗျတံ ဂစ္ဆတိ ဝါသဝဿ၊‘အနေကဇာတိသံသာရံ သန္ဓာဝိဿံ’ အတိဝေလံ န မဿိဿ’န္တိ။}

\sutta{775}{2}{ဘဝိဿတိ ဿတိ ဿန္တိ ဿသိ ဿထ ဿာမိ ဿာမ ဿတေ ဿန္တေ ဿသေ ဿဝှေ ဿံ ဿာမှေ။}
\vutti{ဘဝိဿတိ အနာရဒ္ဓေ အတ္ထေ ဝတ္တမာနတော ကြိယတ္ထာ ဿတျာဒယော ဟောန္တိ။ ဂမိဿတိ ဂမိဿန္တိ၊ ဂမိဿသိ ဂမိဿထ၊ ဂမိဿာမိ ဂမိဿာမ၊ ဂမိဿတေ ဂမိဿန္တေ၊ ဂမိဿသေ ဂမိဿဝှေ၊ ဂမိဿံ ဂမိဿာမှေ။}

\sutta{776}{3}{နာမေ ဂရဟာဝိမှယေသု။}
\vutti{နာမသဒ္ဒေ နိပါတေ သတိ ဂရဟာယံ ဝိမှယေ စ ဂမျမာနေ ဿတျာဒယော ဟောန္တိ။ ဣမေ ဟိ နာမ ကလျာဏဓမ္မာ ပဋိဇာနိဿန္တိ၊ န ဟိ နာမ ဘိက္ခဝေ တဿ မောဃပုရိသဿ ပါဏေသု အနုဒ္ဒယာ ဘဝိဿတိ၊ ကထံ ဟိ နာမ သော ဘိက္ခဝေ မောဃပုရိသော သဗ္ဗမတ္တိကာမယံ ကုဋိကံ ကရိဿတိ? တတ္ထ နာမ တွံ မောဃပုရိသ မယာ ဝိရာဂါယ ဓမ္မေ ဒေသိတေ သရာဂါယ စေတေဿသိ? အတ္ထိ နာမ တာတ သုဒိန္န အာဘိဒေါသိကံ ကုမ္မာသံ ပရိဘုဉ္ဇိဿသိ၊ အတ္ထိယေဝိဟာပိ နိန္ဒာဝဂမော။ ဝိမုယေ-အစ္ဆရိယံ ဝတ ဘော အဗ္ဘုတံ ဝတ ဘော သန္တေန ဝတ ဘော ပဗ္ဗဇိတာ ဝိဟာရေန ဝိဟရန္တိ၊ ယတြ ဟိ နာမ သညီ သမာနော ဇာဂရော ပဉ္စမတ္တာနိ သကဋသတာနိ နိဿာယ နိဿာယ အဘိက္ကန္တာနိ နေဝ ဒက္ခတိ န ပန သဒ္ဒံ သောဿတိ၊ အစ္ဆရိယံ အန္ဓော နာမ ပဗ္ဗတမာရောဟိဿတိ၊ ဗဓိရော နာမ သဒ္ဒံ သောဿတိ။}

\sutta{777}{4}{ဘူတေ ဤဥံ ဩတ္ထ ဣံ မှာ အာ ဦ သေ ဝှံ အ မှေ။}
\vutti{ဘူတေ ပရိသမတ္တေ အတ္ထေ ဝတ္တမာနတော ကြိယတ္ထာ ဤ အာဒယော ဟောန္တိ။ အဂမီ အဂမုံ၊ အဂမော အဂမိတ္ထ၊ အဂမိံ အဂမိမှာ ၊ အဂမာ အဂမူ၊ အဂမိသေ အဂမိဝှံ၊ အဂမ အဂမိမှေ။ ဘူတသာမညဝစနိစ္ဆာယမနဇ္ဇတနေပိ‘သုဝေါ အဟောသိ အာနန္ဒော’။}

\sutta{778}{5}{အနဇ္ဇတနေ အာ ဦ ဩ တ္ထ အ မှာ တ္ထ တ္ထုံ သေ ဝှံ ဣံ မှသေ။}
\vutti{အဝိဇ္ဇမာနဇ္ဇတနေ ဘူတေ-တ္ထေ ဝတ္တမာနတော ကြိယတ္ထာ အာအာဒယော ဟောန္တိ။
အာဉာယျာ စ ဥဋ္ဌာနာ၊ အာဉာယျာ စ သံဝေသနာ။
ဧသဇ္ဇတနော ကာလော၊ အဟရုဘတဍ္ဎရတ္တံ ဝါ။
အဂမာ အဂမူ၊ အဂမော၊ အဂမတ္ထ၊ အဂမ အဂမမှာ၊ အဂမတ္ထ အဂမတ္ထုံ၊ အဂမသေ အဂမဝှံ၊ အဂမိံ အဂမမှသေ။ အညပဒတ္ထော ကိံ? အဇ္ဇ ဟိယျော ဝါ အဂမာသိ။}

\sutta{779}{6}{ပရောက္ခေ အ ဥ ဧ တ္ထ အ မှ တ္ထ ရေ တ္ထော ဝှေါ ဣ မှေ။}
\vutti{အပစ္စက္ခေ ဘူတာနဇ္ဇတနေ-တ္ထေ ဝုတ္တမာနတော ကြိယာတ္ထာ အ အာဒယော ဟောန္တိ။ ဇဂါမ ဇဂမု၊ ဇဂမေ ဇဂပိတ္ထ၊ ဇဂမ ဇဂမိမှ၊ ဇဂမိတ္ထ ဇဂမိရေ၊ ဇဂမိတ္ထော ဇဂမိဝှေါ၊ ဇဂမိ ဇဂမိမှေ။ မူဠှဝိက္ခိတ္တဗျာသတ္တစိတ္တေန အတ္တနာပိ ကြိယာကတာဘိနိဗ္ဗတ္တိတကာလေ-နုပလဒ္ဓါ သမာနာ ဖလေနာ-နုမီယမာနာ ပရောက္ခာဝ ဝတ္ထုတော၊ တေနုတ္တမဝိသယေပိ ပယောဂသမဝေါ။}

\sutta{780}{7}{ဧယျာဒေါ ဝါတိပတ္တိယံ ဿာ ဿံ သု ဿေ ဿထ ဿံ ဿာမှာ ဿထ ဿိံသု သဿေ ဿဝှေ ဿိံ ဿာမှသေ။}
\vutti{ဧယျာဒေါ ဝိသယေ ကြိယာတိပတ္တိယံ ဿာဒယော ဟောန္တိ ဝိဘာသာ။ ဝိဓုရပ္ပစ္စယောပနိပါတတော ကာရဏဝေကလ္လတော ဝါ ကြိယာယာတိပတနမနိပ္ဖတ္တိ ကြိယာတိပတ္တိ၊ ဧတေ စ ဿာဒယော သာမတ္ထိယာတီတာနာဂတေသွေဝ ဟောန္တိ န ဝတ္တမာနေ တတြ။ ကြိယာတိပတျသမ္ဘဝါ၊ သစေ ပဌမဝယေ ပဗ္ဗဇ္ဇံ အလဘိဿာ အရဟာ အဘဝိဿာ၊ ဒက္ခိဏေန စေ အဂမိဿာ န သကဋံ ပရိယာ ဘဝိဿာ၊ ဒက္ခိဏေန စေ အဂမိဿံသု၊ အဂမိဿေ အဂမိဿထ၊ အဂမိဿံ အဂမိဿာမှာ၊ အဂမိဿထ အဂမိဿိံသု၊ အဂမိဿသေ အဂမိဿဝှေ၊ အဂမိဿိံ အဂမိဿာမှသေ၊ န သကဋံ ပရိယာဘဝိဿာ၊ ဝါတိကိံ? ဒက္ခိဏေ န စေ ဂမိဿတိ န သကဋံ ပရိယာ ဘဝိဿတိ။}

\sutta{781}{8}{ဟေတုဖလေသွေယျ ဧယျုံ ဧယျာသိ ဧယျာထ ဧယျာမိ ဧယျာမ ဧထ ဧရံ ဧထော ဧယျဝှေါ ဧယျံ ဧယျာမှေ။}
\vutti{ဟေတုဘူတာယံ ဧလဘူတာယဉ္စ ကြိယာယံ ဝတ္တမာနတော ကြိယတ္ထာ ဧယျာဒယော ဝါ ဟောန္တိ၊ သစေ သင်္ခါရာ နိစ္စာ ဘဝေယျုံ န နိရုဇ္ဈေယျုံ၊ ဒက္ခိဏေန စေ ဂစ္ဆေယျ န သကဋံ ပရိယာဘဝေယျ၊ ဒက္ခိဏေန စေ ဂစ္ဆေယျုံ၊ ဂစ္ဆေယျာသိ ဂစ္ဆေယျာထ၊ ဂစ္ဆေယျာမိ ဂစ္ဆေယျာမ၊ ဂစ္ဆေထ ဂစ္ဆေရံ၊ ဂစ္ဆေထော ဂစ္ဆေယျဝှေါ၊ ဂစ္ဆေယျံ ဂစ္ဆေယျာမှေ၊ န သကဋံ ပရိယျာဘဝေယျ၊ ဘဝနံ ဂမနံ စ ဟေတု၊ အနိရုဇ္ဈနံ အပရိယာဘဝနံ စ ဖလံ၊ ဣဟ ကသ္မာ န ဟောတိ‘ဟန္တီဟိ ပလာယတိ၊ ဝဿတီတိ ဓာဝတိ၊ ဟနိဿတီတိ ပလာယိဿတီ’တိ? ဣတိ သဒ္ဒေနေဝ ဟေတုဟေတုမန္တတာယ ဇောတိတတ္တာ၊ ဝါတိ ကိံ? ဒက္ခိဏေန စေ ဂမိဿတိ န သကဋံ ပရိယာဘဝိဿတိ။}

\sutta{782}{9}{ပဉှပတ္ထနာဝိဓီသု။}
\vutti{ပဉှော=သမ္ပုစ္ဆနံ သမ္ပဓာရဏံ နိရူပဏံ ကာရိယာနိစ္ဆယနံ ပတ္ထနာ=ယာစနံ ဣဋ္ဌာသိံ ဘနဉ္စ၊ ဝိဓိ=ဝိဓာနံ နိယောဇနံ ကြိယာသု ဗျာပါရဏာ၊ သာ စ ဒုဝိဓာဝ သာဒရာနာဒရဝသေန။ ဝိသယဘေဒေန ဘိန္နာယပိ တဒုဘယာနတိဝတ္တနတော၊ ဧတေသု ပဉှာဒီသု ကြိယတ္ထတော ဧယျောဒယော ဟောန္တိ၊ ပဉှေ - ကိမာယသ္မာ ဝိနယံ ပရိယာပုဏေယျ? ဥဒါဟု ဓမ္မံ၊ ဂစ္ဆေယျံ ဝါဟံ ဥပေါသထံ န ဝါ ဂစ္ဆေယျံ၊ ပတ္ထနာယံ-လဘေယျာ-ဟံ ဘန္တေ ဘဂဝတော သန္တိကေ ပဗ္ဗဇ္ဇံ လဘေယျံ ဥပသမ္ပဒံ၊ ပဿေယျံ တံ ဝဿသတံ အရောဂံ၊ ဝိဓိမှိ-ဘဝံ ပတ္တံ ပစေယျ၊ ဘဝံ ပုညံ ကရေယျ၊ ဣဟ ဘဝံ ဘုဉ္ဇေယျ၊ ဣဟ ဘဝံ နိသီဒေယျ၊ မာဏဝကံ ဘဝံ အဇ္ဈာပေယျ၊ အနုညာပတ္တကာလေသုပိ သိဒ္ဓါဝ။ တတ္ထာပိ ဝိဓိပ္ပတီတိတော၊ အနုညာယံ-ဧဝံ ကရေယျာသိ၊ ပတ္တကာလေ-ကဋံ ကရေယျာသိ၊ ပတ္တော တေ ကာလော ကဋကရဏေ၊ ယဒိ သင်္ဃဿ ပတ္တကလ္လံ သင်္ဃော ဥပေါသထံ ကရေယျ၊ ဧတဿ ဘဂဝါ ကာလော ဧတဿ သုဂတ ကာလော ယံ ဘဂဝါ သာဝကာနံ သိက္ခာပဒံ ပညပေယျ၊ ပေသနေပိစ္ဆန္တိ‘ဂါမံ တွံ ဘဏေ ဂစ္ဆေယျာသိ’။}

\sutta{783}{10}{တု အန္တု ဟိ ထ မိ မ တံ အန္တံ ဿု ဝှေါ ဧ အာမသေ။}
\vutti{ပဉှာဒီသွေတေ ဟောန္တိ ကြိယတ္ထတော။ ဂစ္ဆတု ဂစ္ဆန္တု၊ ဂစ္ဆာဟိ ဂစ္ဆထ၊ ဂစ္ဆာမိ ဂစ္ဆာမ၊ ဂစ္ဆတံ ဂစ္ဆန္တံ၊ ဂစ္ဆဿု ဂစ္ဆဝှေါ၊ ဂစ္ဆေ၊ ဂစ္ဆာမသေ၊ ပဉှေ-ကိန္နု ခလု ဘော ဗျာကရဏမဓီယဿု၊ ပတ္ထနာယံ-ဒဒါဟိ မေ၊ ဇီဝတု ဘဝံ၊ ဝိဓိမှိ-ကဋံ ကရောတု ဘဝံ၊ ပုညံ ကရောတု ဘဝံ၊ ဣဟ ဘဝံ ဘုဉ္ဇတု၊ ဣဟ ဘဝံ နိသီဒတု၊ ဥဒ္ဒိသတု ဘန္တေ ဘဂဝါ ဘိက္ခူနံ ပါတိမောက္ခံ၊ ပေသနေ-ဂစ္ဆ ဘဏေ ဂါမံ၊ အနုမတိယံ - ဧဝံ ကရောဟိ၊ ပတ္တကာလေ-ကာလော-ယံ တေ မဟာဝီရ ဥပ္ပဇ္ဇ မာတုကုစ္ဆိယံ။}

\sutta{784}{11}{သတျရဟေသွေယျာဒီ။}
\vutti{သတ္တိယံ အရဟတ္ထေ စ ကြိယတ္ထာ ဧယျာဒယော ဟောန္တိ။ ဘဝံ ခလု ရဇ္ဇံ ကရေယျ၊ ဘဝံ သတ္တော အရဟော။}

\sutta{785}{12}{သမ္ဘာဝနေ ဝါ။}
\vutti{သမ္ဘာဝနေ ဂမျမာနေ ဓာတုနာ ဝုစ္စမာနေ စ ဧယျာဒယော ဟောန္တိ ဝိဘာသာ။ အပိ ပဗ္ဗတံ သိရသာ ဘိန္ဒေယျ၊ ကြိယာတိပတ္တိယန္တု ဿာဒီ-အသနိယာပိ ဟတော နာပတိဿာ၊ သမ္ဘာဝေမိ သဒ္ဒဟာမိ အဝကပ္ပေမိ ဘုဉ္ဇေယျ ဘဝံ ဘုဉ္ဇိဿတိ ဘဝံ အဘုဉ္ဇိ ဘဝံ၊ ကြိယာတိပတ္တိယန္တု ဿာဒီ-သမ္ဘာဝေမိ နာဘုဉ္ဇိဿာ ဘဝံ။}

\sutta{786}{13}{မာယောဂေ ဤအာအာဒီ။}
\vutti{မာ ယောဂေ သတိ ဤအာဒယော အာအာဒယော စ ဝါ ဟောန္တိ။ မာ သု ပုနပိ ဧဝရူပမကာသိ၊ မာ ဘဝံ အဂမာ ဝနံ၊ ဝါတွေဝ? မာ တေ ကာမဂုဏေ ဘမဿု စိတ္တံ၊ မာ တွံ ကရိဿသိ၊ မာ တွံ ကရေယျာသိ၊ အသကကာလတ္ထောယမာရမ္ဘော၊ ဗုဒ္ဓေါ ဘဝိဿတီတိ ပဒန္တရသမ္ဗန္ဓေနာနာဂတကာလတာ ပတီယတေ၊ ဧဝံ ကတော ကဋော သွေ ဘဝိဿတိ၊ ဘာဝိ ကိစ္စမာသီတိ။
လုနာဟိ လုနာဟိတွေဝါယံ လုနာတိ၊ လုနဿု လုနဿုတွေဝါယံ လုနာ-တီတိ တွာဒီနမေဝေတံ မဇ္ဈိမပုရိသေကဝစနာနမာဘိက္ခညေ ဒွိဗ္ဗဝစနံ၊ ဣဒံ ဝုတ္တံ ဟောတိ‘ဧဝ မေသ တုရိတော အညေပိ နိယောဇေန္တောဝိယ ကိရိယံ ကရောတီ’တိ၊ ဧဝံ လုနာထ လုနာထတွေဝါယံ လုနာတိ၊ လုနဝှေါ လုနဝှေါတွေဝါယံ လုနာတိ၊ တထာ ကာလန္တရေသုပိ လုနာဟိ လုနာဟိတွေဝါယံ အလုနိ၊ အလုနာ၊ လုလာဝ၊ လုနိဿတီတိ၊ ဧဝံ ဿုမှိ စ ယောဇနီယံ၊ တထာ သမုစ္စယေပိ မဌမဋ၊ ဝိဟာရမဋေတွေဝါယမဋတိ၊ မဌမဋဿု၊ ဝိဟာရမဋဿုတွေဝါယမဋတိ၊ ဗျာပါရဘေဒေ တုသာမညဝစနဿေဝ ဗျာပကတ္တာ အနုပ္ပယောဂေါ ဘဝတိ၊ ဩဒနံ ဘုဉ္ဇ၊ ယာဂုံ ပိဝ၊ ဓာနာ ခါဒေတွေဝါယ-မဇ္ဈောဟရတိ။}

\sutta{787}{14}{ပုဗ္ဗပရစ္ဆက္ကာနမေကာနေကေသု တုမှာမှသေသေသု ဒွေ ဒွေ မဇ္ဈိမုတ္တမပဌမာ။}
\vutti{ဧကာနေကေသု တုမှာမှသဒ္ဒဝစနီယေသု တဒညသဒ္ဒဝစနီယေသု စ ကာရကေသု ပုဗ္ဗစ္ဆက္ကာနံ ပရစ္ဆက္ကာနံ မဇ္ဈိမုတ္တမပဌမာ ဒွေ ဒွေ ဟောန္တိ ယထာက္ကမံ ကြိယတ္ထာ၊ ဥတ္တမသဒ္ဒေါ-ယံ သဘာဝတော တတိယဒုကေ ရုဠှော၊ တွံ ဂစ္ဆသိ၊ တုမှေ ဂစ္ဆထ၊ တွံ ဂစ္ဆသေ၊ တုမှေ ဂစ္ဆဝှေ၊ အဟံ ဂစ္ဆာမိ၊ မယံ ဂစ္ဆာမ၊ အဟံ ဂစ္ဆေ၊ မယံ ဂစ္ဆာမှေ၊ သော ဂစ္ဆတိ၊ တေ ဂစ္ဆန္တိ၊ သော ဂစ္ဆတေ၊ တေ ဂစ္ဆန္တေ၊ သာမတ္ထိယာ လဒ္ဓတ္တာ အပ္ပယုဇ္ဇမာနေသုပိ တုမှာမှသေသေသု ဘဝန္တိ။ ဂစ္ဆသိ၊ ဂစ္ဆထ၊ ဂစ္ဆသေ၊ ဂစ္ဆဝှေ၊ ဂစ္ဆာမိ၊ ဂစ္ဆာမ၊ ဂစ္ဆေ၊ ဂစ္ဆာမှေ၊ ဂစ္ဆတိ၊ ဂစ္ဆန္တိ၊ ဂစ္ဆတေ၊ ဂစ္ဆန္တေ။}

\sutta{788}{15}{အာဤဿာဒီသွဉ် ဝါ။}
\vutti{အာအာဒေါ ဤအာဒေါ ဿာ အာဒေါ စ ကြိယတ္ထဿ ဝါ အဉ် ဟောတိ။ ဉကာရော-နုဗန္ဓော၊ အဂမာ၊ ဂမာ၊ အဂမီ၊ ဂမီ၊ အဂမိဿာ၊ ဂမိဿာ။}

\sutta{789}{16}{အအာဒီသွာဟော ဗြူဿ။}
\vutti{ဗြူဿ အာဟော ဟောတိ အအာဒီသု။ အာဟ၊ အာဟု။}

\sutta{790}{17}{ဘူဿ ဝုက်။}
\vutti{အအာဒီသု ဘူဿ ဝုက် ဟောတိ။ ကကာရော-နုဗန္ဓော၊ ဥကာရော ဥစ္စာရဏတ္ထော၊ ဗဘူဝ။}

\sutta{791}{18}{ပုဗ္ဗဿ အ။}
\vutti{အအာဒီသု ဒွိတ္တေ ပုဗ္ဗဿ ဘူဿ အ ဟောတိ၊ ဗဘူဝ။}

\sutta{792}{19}{ဥဿံသွာဟာ ဝါ။}
\vutti{အာဟာဒေသာ ပရဿ ဥဿ အံသု ဝါ ဟောတိ။ အဟံသု၊ အာဟု။}

\sutta{793}{20}{တျန္တီနံ ဋဋူ။}
\vutti{အာဟာ ပရေသံ တိအန္တီနံ ဋဋူ ဟောန္တိ။ ဋကာရာ သဗ္ဗာဒေသတ္ထာ၊ အာဟ၊ အာဟု၊ အတောယေဝ စ ဉာပကာ တိအန္တီသု စ ဗြူဿာ-ဟော။}

\sutta{794}{21}{ဤအာဒေါ ဝစဿောမ်။}
\vutti{ဤအာဒီသု ဝစဿ ဩမ် ဟောတိ။ မကာရော-နုဗန္ဓော၊ အဝေါစ၊ ဤအာဒေါတိ ကိံ? အဝစာ။}

\sutta{795}{22}{ဒါဿ ဒံ ဝါ မိမေသွဒွိတ္တေ။}
\vutti{အဒွိတေ ဝတ္တမာနဿ ဒါဿ ဒံ ဝါ ဟောတိ မိမေသု။ ဒမ္မိ ဒေမိ၊ ဒမ္မ ဒေမ၊ အဒွိတ္တေတိ ကိံ? ဒဒါဓိ ဒဒါမ။}

\sutta{796}{23}{ကရဿ သောဿ ကုံ။}
\vutti{ကရဿ သဩကာရဿ ကုံ ဝါ ဟောတိ မိမေသု။ ကုမ္မိ ကုမ္မ၊ ကရောမိ ကရောမ။}

\sutta{797}{24}{ကာ ဤအာဒီသု။}
\vutti{ကရဿ သဩကာရဿ ကာ ဟောတိ ဝါ ဤအာဒီသု။ အကာသိ အကရိ၊ အကံသု အကရိံသု၊ အကာ အကရာ။}

\sutta{798}{25}{ဟာဿ စာဟင် ဿေန။}
\vutti{ကရဿ သောဿ ဟာဿ စ အာဟင် ဝါ ဟောတိ ဿေန သဟ။ ကာဟတိ ကရိဿတိ၊ အကာဟာ အကရိဿာ၊ ဟာဟတိ ဟာယိဿတိ၊ အဟာဟာ အဟာယိဿာ။}

\sutta{799}{26}{လဘဝသစ္ဆိဒဘိဒရုဒါနံ စ္ဆင်။}
\vutti{လဘာဒီနံ စ္ဆင် ဝါ ဟောတိ ဿေန သဟ။ အလစ္ဆာ အလဘိဿာ၊ လစ္ဆတိ လဘိဿာတိ၊ အဝစ္ဆာ အဝသိဿာ၊ ဝစ္ဆတိ ဝသိဿတိ၊ အစ္ဆေစ္ဆာ အစ္ဆိန္ဒိဿာ၊ ဆေစ္ဆတိ ဆိန္ဒိဿတိ၊ အဘေစ္ဆာ အဘိန္ဒိဿာ၊ ဘေစ္ဆတိ ဘိန္ဒိဿတိ၊ အရုစ္ဆာ အရောဒိဿာ၊ ရုစ္ဆတိ ရောဒိဿတီ ၊ အညသ္မိမ္ပိ ဆိဒဿ ဝါ စ္ဆင ယောဂဝိဘာဂါ၊ အစ္ဆေစ္ဆုံ အစ္ဆန္တိံသု၊ အညေသဉ္စ ဂစ္ဆံ ဂစ္ဆိဿံ။}

\sutta{800}{27}{ဘုဇ မုစ ဝစ ဝိသာနံ က္ခင်။}
\vutti{ဘုဇာဒီနံ က္ခင် ဝါ ဟောတိ ဿေန သဟ။ အဘောက္ခာ အဘုဉ္ဇိဿာ၊ ဘောက္ခတိ ဘုဉ္ဇိဿတိ၊ အမောက္ခာ အမုဉ္စိဿာ၊ မောက္ခတိမုဉ္စိဿတိ၊ အဝက္ခာ အဝစိဿာ၊ ဝက္ခတိ ဝစ္စိဿတိ၊ ပါဝေက္ခာ ပါဝိသိဿာ၊ ပဝေက္ခတိ ပဝိသိဿတိ၊ ဝိသဿာ-ညသ္မိမ္ပိ ဝါ က္ခင ယောဂဝိဘာဂါ ပါဝေက္ခိ၊ ပါဝိသိ။}

\sutta{801}{28}{အာဤအာဒီသု ဟရဿာ။}
\vutti{အာအာဒေါ ဤအာဒေါ အ ဟရဿ အာ ဟောတိ ဝါ။ အဟာ အဟရာ၊ အဟာသိ အဟရိ။}

\sutta{802}{29}{ဂမိဿာ။}
\vutti{အာအာဒေါ ဤအာဒေါ အ ဂမိဿ အာ ဟောတိ ဝါ။ အဂါ အဂမာ၊ အဂါ အဂမီ။}

\sutta{803}{30}{ဍံသဿ စ ဆင်။}
\vutti{ဍံသဿ ဂမိဿ စ ဆင် ဝါ ဟောတိ အာဤအာဒီသု။ အဍဉ္ဆာ အဍံသာ၊ အဍဉ္ဆိ အဍံသီ၊ အဂဉ္ဆာ အဂစ္ဆာ၊ အဂဉ္ဆိ အဂစ္ဆီ။}

\sutta{804}{31}{ဟူဿ ဟေဟေဟိဟောဟိ ဿတျာဒေါ။}
\vutti{ဟူဿ ဟေအာဒယော ဟောန္တိ ဿတျာဒေါ။ ဟေဿတိ၊ ဟေဟိဿတိ၊ ဟောဟိဿတိ။}

\sutta{805}{32}{ဏာနာသု ရဿော။}
\vutti{က္ဏာက္နာသု ကြိယတ္ထဿ ရဿော ဟောတိ။ ကိဏာတိ၊ ဓုနာတိ။}

\sutta{806}{33}{အာ ဤ ဦ မှာ ဿာ ဿမှာနံ ဝါ။}
\vutti{ဧသံ ဝါ ရဿော ဟောတိ။ ဂမ ဂမာ၊ ဂမိ ဂမီ၊ ဂမု ဂမူ၊ ဂမိမှ ဂမိမှာ၊ ဂမိဿ ဂမိဿာ၊ ဂမိဿမှ ဂမိဿမှာ။}

\sutta{807}{34}{ကုသရုဟေဟီဿ ဆိ။}
\vutti{ကုသာ ရုဟာ စ ပရဿ ဤဿ ဆိ ဝါ ဟောတိ။ အက္ကောစ္ဆိ အက္ကောသိ၊ အဘိရုစ္ဆိ အဘိရုဟိ။}

\sutta{808}{35}{အဤဿာအာဒီနံ ဗျဉ္ဇနဿိဉ်။}
\vutti{ကြိယတ္ထာ ပရေသံ အအာဒီနံ ဤအာဒီနံ ဿအာဒီနဉ္စ ဗျဉ္ဇနဿ ဣဉ် ဟောတိ ဝိဘာသာ။ ဗဘုဝိတ္ထ၊ အဘဝိတ္ထာ၊ အနုဘဝိဿာ၊ အနုဘဝိဿတိ အနုဘောဿတိ ဟရိသတိ ဟဿတိ၊ ဧတေသန္တိ ကိံ? ဘဝတိ၊ ဗျဉ္ဇနဿာတိ ကိံ? ဗဘူဝ။}

\sutta{809}{36}{ဗြူတော တိဿီဉ်။}
\vutti{ဗြူတော ပရဿ တိဿ ဤဉ် ဝါ ဟောတိ။ ဗြဝီတိ၊ ဗြူတိ။}

\sutta{810}{37}{ကျဿ။}
\vutti{ကြိယတ္ထာ ပရဿ ကျဿ ဤဉ ဝါ ဟောတိ။ ပစီယတိ၊ ပစ္စတိ။}

\sutta{811}{38}{ဧယျာထ ဿေ အ အာ ဤထာနံ ဩ၊ အ၊ အံ၊ တ္ထ၊ တ္ထော၊ ဝှေါက်။}
\vutti{ဧယျာထာဒီနံ ဩအာဒယော ဝါ ဟောန္တိ ယထာက္ကမံ။ တုမှေ ဘဝေယျာထော ဘဝေယျာထ၊ တွံ အဘဝိဿ အဘဝိဿေ၊ အဟံ အဘဝံ အဘဝ၊ သော အဘဝိတ္ထ အဘဝါ၊ သော အဘဝိတ္ထော၊ အဘဝီ၊ တုမှေ ဘဝထဝှေါ ဘဝထ၊ အာသဟစရိတောဝ အကာရော ဂယှတေ၊ ထော ပန-န္တေ နိဒ္ဒေသာ တွာဒိသမ္ဗန္ဓီယေဝ၊ တဿေဝ ဝါ နိဿိတတ္တာ၊ နိဿယကရဏမ္ပိ ဟိ သုတ္တကာရာစိဏ္ဏံ။}

\sutta{812}{39}{ဥံဿိံသွံသု။}
\vutti{ဥမီစ္စဿ ဣံသု အံသု ဝါ ဟောန္တိ။ အဂမိံသု၊ အဂမံသု၊ အဂမုံ။}

\sutta{813}{40}{ဧဩတ္တာ သုံ။}
\vutti{ဧအာဒေသတော ဩအာဒေသတော စ ပရဿ ဥမိစ္စဿ သုံ ဝါ ဟောတိ။ နေသုံ၊ နယိံသု၊ အဿောသုံ၊ အဿုံ၊ အာဒေသတ္တာချာပနတ္ထံတ္တဂ္ဂဟဏံ။}

\sutta{814}{41}{ဟူတော ရေသုံ။}
\vutti{ဟူတော ပရဿ ဥမိစ္စဿ ရေသုံ ဝါ ဟောတိ။ အဟေသုံ၊ အဟဝုံ။}

\sutta{815}{42}{ဩဿ အ ဣ တ္ထ တ္ထော။}
\vutti{ဩဿ အအာဒယော ဝါ ဟောန္တိ။ တွံ အဘဝ၊ အဘဝိ၊ အဘဝိတ္ထ၊ အဘဝိတ္ထော အဘဝေါ။}

\sutta{816}{43}{သိ။}
\vutti{ဩဿ သိ ဝါ ဟောတိ။ အဟောသိ တွံ အဟုဝေါ။}

\sutta{817}{44}{ဒီဃာ ဤဿ။}
\vutti{ဒီဃတော ပရဿ ဤဿ သိ ဝါ ဟောတိ။ အကာသိ အကာ၊ အဒါသိ အဒါ။}

\sutta{818}{45}{မှာတ္ထာနမုဉ်။}
\vutti{မှာတ္ထာနမုဉ် ဝါ ဟောတိ။ အဂမုမှာ အဂမိမှာ၊ အဂမုတ္ထ အဂမိတ္ထ။}

\sutta{819}{46}{ဣံဿ စ သိဉ်။}
\vutti{ဣမိစ္စဿ သိဉ် ဝါ ဟောတိ မှာတ္ထာနဉ္စ ဗဟုလံ။ အကာသိံ အကရိံ၊ အကာသိမှာ အကရိမှာ အကာသိတ္ထ အကရိတ္ထ။}

\sutta{820}{47}{ဧယျုံဿုံ။}
\vutti{ဧယျုမိစ္စဿ ဉံ ဝါ ဟောတိ။ ဂစ္ဆုံ ဂစ္ဆေယျုံ။}

\sutta{821}{48}{ဟိဿ-တော လောပေါ။}
\vutti{အတော ပရဿ ဟိဿ လောပေါ ဝါ ဟောတိ။ ဂစ္ဆ ဂစ္ဆာဟိ၊ အတောတိ ကိံ? ကရောဟိ။}

\sutta{822}{49}{ကျဿ ဿေ။}
\vutti{ကျဿ ဝါ လောပေါ ဟောတိ ဿေ။ အနွဘဝိဿာ အနွဘူယိဿာ၊ အနုဘဝိဿတိ အနုဆူယိဿတိ။}

\sutta{823}{50}{အတ္ထိတေယျာဒိစ္ဆန္နံ သ သု သ သထ သံ သာမ။}
\vutti{အသ=ဘုဝိစ္စသ္မာ ပရေသံ ဧယျာဒိစ္ဆန္နံ သာဒယော ဟောန္တိ ယထာက္ကမံ။ အဿ၊ အဿု၊ အဿ၊ အဿထ၊ အဿံ၊ အဿာမ။}

\sutta{824}{51}{အာဒိဒွိန္နမိယာဣယုံ။}
\vutti{အတ္ထိတေယျာဒိစ္ဆန္နံ အာဒိဘူတာနံ ဒွိန္နံ ဣယာ ဣယုံ ဟောန္တိ ယထာက္ကမံ။ သိယာ၊ သိယုံ။}

\sutta{825}{52}{တဿ ထော။}
\vutti{အတ္ထိတော ပရဿ တကာရဿ ထော ဟောတိ။ အတ္ထိ၊ အတ္ထု။}

\sutta{826}{53}{သိဟိသွဋ်။}
\vutti{အတ္ထိဿ အဋ် ဟောတိ သိဟိသု၊ ဋော သဗ္ဗာဒေသတ္ထော။ အသိ အဟိ။}

\sutta{827}{54}{မိမာနံ ဝါ မှိမှာ စ။}
\vutti{အတ္ထိသ္မာ ပရေသံ မိမာနံ မှိမှာ ဝါ ဟောန္တိ၊ တံသန္နိယောဂေနအတ္ထိဿ အဋ် စ။ အမှိ အသ္မိ၊ အမှ အသ္မ။}

\sutta{828}{55}{ဧသု သ်။}
\vutti{ဧသု မိမေသု အတ္ထိဿ သကာရော ဟောတိ။ အသ္မိ အသ္မ၊ ပရရူပဗာဓနတ္ထံ။}

\sutta{829}{56}{ဤအာဒေါ ဒီဃော။}
\vutti{အတ္ထိဿ ဒီဃော ဟောတိ ဤအာဒိမှိ။ အာသိ၊ အာသုံ၊ အာသိ၊ အာသိတ္ထ၊ အာသိံ၊ အာသိံ မှာ။}

\sutta{830}{57}{ဟိမိမေသွဿ။}
\vutti{အကာရဿ ဒီဃော ဟောတိ ဟိမိမေသု။ ပစာဟိ၊ ပစာမိ၊ ပစာမ၊ မုယှာမိ။}

\sutta{831}{58}{သကာ ဏာဿ ခ ဤအာဒေါ။}
\vutti{သကသ္မာ က္ဏာဿခေါဟောတိ ဤအာဒီသု။ အသက္ခိ၊ အသက္ခိံသု။}

\sutta{832}{59}{ဿေ ဝါ။}
\vutti{သကသ္မာ က္ဏဿခေါ ဝါ ဟောတိ ဿေ။ သက္ခိဿာ သက္ကုဏိဿာ၊ သက္ခိဿတိ၊ သက္ကုဏိဿတိ။}

\sutta{833}{60}{တေသု သုတော က္ဏောက္ဏာနံ ရောဋ်။}
\vutti{တေသု ဤအာဒိဿေသု သုတော ပရေသံ က္ဏောက္ဏာနံ ရောဋ် ဝါ ဟောတိ။ အဿောသိ အသုဏိ၊ အဿောဿာ အသုဏိဿာ၊ သောဿတိ သုဏိဿတိ။}

\sutta{834}{61}{ဉာဿ သနာဿ နာယော တိမှိ။}
\vutti{သနာဿ ဉာဿ နာယော ဝါ ဟောတိ တိမှိ။ နာယတိ၊ ဇာနာတိ။}

\sutta{835}{62}{ဉာမှိ ဇံ။}
\vutti{ဉာဒေသေ သနာဿ ဉာဿ ဇံ (ဝါ) ဟောတိ။ ဇညာ (ဇာနေယျ)။}

\sutta{836}{63}{ဧယျာဿိယာဉာ ဝါ။}
\vutti{ဉာတော ဧယျာဿ ဣယာဉာ ဟောန္တိ ဝါ။ ဇာနိယာ၊ ဇညာ ဇာနေယျ။}

\sutta{837}{64}{ဤဿတျာဒီသု က္နာလောပေါ။}
\vutti{ဤအာဒေါ ဿတျာဒေါ စ ဉာတော က္နာလောပေါ ဝါ ဟောတိ။ အညာသိ အဇာနိ၊ ဉဿတိ ဇာနိဿတိ။}

\sutta{838}{65}{ဿဿ ဟိ ကမ္မေ။}
\vutti{ဉာတော ပရဿ ဿဿ ဟိ ဝါ ဟောတိ ကမ္မေ။ ပညာယိဟိတိ ပညာယိဿတိ။}

\sutta{839}{66}{ဧတိသ္မာ။}
\vutti{ဧတိသ္မာ ပရဿ ဿဿ ဟိ ဟောတိ ဝါ။ ဧဟိတိ ဧဿတိ။}

\sutta{840}{67}{ဟနာ ဆခါ။}
\vutti{ဟနာ ဿဿ ဆခါ ဝါ ဟောန္တိ။ ဟဉ္ဆာမိ ဟနိဿာမိ၊ ပဋိဟင်္ခါမိ ပဋိဟနိဿာမိ။}

\sutta{841}{68}{ဟာတော ဟ။}
\vutti{ဟာတော ပရဿ ဿဿ ဟ ဟောတိ ဝါ။ ဟာဟတိ ဇဟိဿတိ။}

\sutta{842}{69}{ဒက္ခ ခ ဟေဟိ ဟောဟီဟိ လောပေါ။}
\vutti{ဒက္ခာဒီဟိ အာဒေသေဟိ ပရဿ ဿဿ လောပေါ ဝါ ဟောတိ။ ဒက္ခတိ ဒက္ခိဿတိ၊ သက္ခတိ သက္ခိဿတိ၊ ဟေဟိတိ ဟေဟိဿတိဟောဟိတိ ဟောဟိဿတိ။}

\sutta{843}{70}{ကယိရေယျဿေယျုမာဒီနံ။}
\vutti{ကယိရာ ပရဿ ဧယျုမာဒီနံ ဧယျဿ လောပေါ ဟောတိ။ ကယိရုံ၊ ကယိရာသိ၊ ကယိရာထ၊ ကဟိရာမိ၊ ကယိရာမ။}

\sutta{844}{71}{ဋာ။}
\vutti{ကယိရာ ပရဿ ဧယျဿ ဋာ ဟောတိ။ သော ကယိရာ။}

\sutta{845}{72}{ဧထဿာ။}
\vutti{ကယိရာ ပရဿ ဧထဿ အာ ဟောတိ။ ကယိရာထ။}

\sutta{846}{73}{လဘာ ဣံ ဤနံ ထံ ထာ ဝါ။}
\vutti{လဘသ္မာ ဣံဤဣစ္စေသံ ထံထာ ဟောန္တိ ဝါ။ အလတ္ထံ အလဘိံ၊ အလတ္ထ အလဘိ။}

\sutta{847}{74}{ဂုရုပုဗ္ဗာ ရဿာ ရေ န္တေန္တီနံ။}
\vutti{ဂုရုပုဗ္ဗသ္မာ ရဿာ ပရေသံ န္တေန္တီနံ ရေ ဝါ ဟောတိ။ ဂစ္ဆရေ ဂစ္ဆန္တိ၊ ဂစ္ဆရေ ဂစ္ဆန္တေ၊ ဂမိဿရေ ဂမိဿန္တိ၊ ဂမိဿရေ ဂမိဿန္တေ၊ ဂုရုပုဗ္ဗာတိ ကိံ? ပစ၊ ရဿတိ ကိံ? ဟောန္တိ။}

\sutta{848}{75}{ဧယျေယျာသေယျန္နံ ဋေ။}
\vutti{ဧယျာဒီနံ ဋေ ဝါ ဟောတိ။ သော ကရေ ကရေယျ၊ တွံ ကရေ ကရေယျာသိ၊ အဟံ ကရေ ကရေယျံ။}

\sutta{849}{76}{ဩဝိကရဏဿု ပရစ္ဆက္ကေ။}
\vutti{ဩဝိကရဏဿ ဥ ဟောတိ ပရစ္ဆက္ကေ ဝိသယေ။ တနုတေ။}

\sutta{850}{77}{ပုဗ္ဗစ္ဆက္ကေ ဝါ ကွစိ။}
\vutti{ဩဝိကရဏဿ ဥ ဟောတိ ဝါ ကွစိ ပုဗ္ဗစ္ဆက္ကေ။ ဝနုတိ ဝနောတိ။}

\sutta{851}{78}{ဧယျာမဿေမု စ။}
\vutti{ဧယျာမဿေမု ဝါ ဟောတိ ဥ စ။ ဘဝေမု၊ ဘဝေယျာမု ဘဝေယျာမ။}

\begin{jieshu}
ဣတိ မောဂ္ဂလ္လာနေ ဗျာကရဏေ ဝုတ္တိယံ

တျာဒိကဏ္ဍော ဆဋ္ဌော။
\end{jieshu}
\chapter{ဏွာဒိကဏ္ဍော သတ္တမော}
\markboth{မောဂ္ဂလ္လာနဗျာကရဏေ}{ဏွာဒိကဏ္ဍော သတ္တမော}

“ဗဟုလံ ” (၁.၅၈) “ကြိယတ္ထာ ”တိ (၅.၁၄) စ သဗ္ဗတ္ထ ဝတ္ထတေ။

\sutta{851}{1}{စရ ဒရ ကရ ရဟ ဇန သန တလ သာဒ သာဓ ကသာသ စဋာယ ဝါဟိ ဏု။}
\vutti{စရ-ဂတိဘက္ခဏေသု၊ ဒရ-ဒရဏေ၊ ကရ-ကရဏေ၊ ရဟ-စာဂေ၊ ဇန-ဇနနေ၊ သန-သမ္ဘတ္တိယံ၊ တလ-ပတိဋ္ဌာယံ၊ သာဒ-အဿာဒနေ၊ သာဓ-သံသိဒ္ဓိယံ၊ ကသ-ဝိလေခနေ၊ အသ-ခေပနေ၊ စဋ-ဘေဒနေ၊ အယ-ဣတိ ဂမနတ္ထော ဒဏ္ဍကော ဓာတု၊ ဝါ-ဂတိဂန္ဓနေသု၊ ဧတေဟိ ကြိယတ္ထေဟိ ဗဟုလံ ဏု ဟောတိ။ “အဿာ ဏာနုဗန္ဓေ ”တိ (၅.၈၄) ဥပန္တဿ အဿ အာ၊ စရတိ ဟဒယေ မနုညဘာဝေနာတိ စာရု=သောဘနံ။ ဒရီယတီတိ ဒါရ=ကဋ္ဌံ။ ကရောတီတိ ကာရု=သိပ္ပီ၊ မဃ ဝါ၊ ဝိသုကမ္မော စ။ ရဟတိ စန္ဒာဒီနံ သောဘာဝိသေသံ နာသေတီတိ ရာဟု=အသုရိန္ဒော။ ဇာယတိ ဂမနာဂမနံ အနေနာတိဇာနု=ဇင်္ဃောရူနံ သန္ဓိ။ သနေတိ အတ္တနိ ဘတ္တိံ ဥပ္ပာဒေတီတိ သာနု=ဂိရိပ္ပဒေသော။ တလန္တိ ပတိဋ္ဌဟန္တိ ဧတ္ထ ဒန္တာတိ တာလု=ဝဒနေကဒေသော။ သာ ဒီယတိ အဿာဒီယတီတိ သာဒု=မဓုရံ။ သာဓေတိ အတ္တပရဟိတန္တိ သာဓု=သဇ္ဇနော။ ကသီယတီတိ ကာသု=အာဝါဋော၊ အသတိ သီဃဘာဝေန ပဝတ္တတီတိ အာသု=သီဃံ။ စဋတိ ဘိန္ဒတိ အမနုညဘာဝန္တိ စာဋု=မနုညော။ အယန္တိ ပဝတ္တန္တိ သတ္တာ ဧတေနာတိ အာယု=ဇီဝိတံ။ “အဿာ ဏာပိမှိ ယုက ” ဣတိ (၅.၉၁) ယုက- ဝါတိ ဂစ္ဆတီတိ ဝါယု-ဝါတော။}

\sutta{852}{2}{ဘရ မရ စရ တရ အရ ဂရ ဟန တန မန ဘမ ကိတ ဓန ဗံဟ ကမ္ဗမ္ဗ စက္ခ ဘိက္ခ သံကိန္ဒန္ဒ ယဇ ပဋာဏာသ ဝသ ပသ ပံသ ဗန္ဓာ ဥ။}
\vutti{ဘရ-ဘရဏေ၊ မရ-ပါဏစာဂေ၊ စရ-ဂတိဘက္ခဏေသု၊ တရ တရဏေ၊ အရ-ဂမနေ၊ ဂရ ဃရ-သေစနေ၊ ဂိရာတိ ဝါ နိပါတနာ အကာရော၊ ဟန-ဟိံ သာယံ၊ တန-ဝိတ္တာရေ၊ မန-ဉာဏေ၊ ဘမ-အနဝဋ္ဌာနေ၊ ကိတ-နိဝါသေ၊ ဓန-သဒ္ဒေ၊ ဗံဟ ဗြဟ ဗြူဟ-ဝုဒ္ဓိယံ၊ ကမ္ဗ-သံဝရဏေ၊ အမ္ဗ-သဒ္ဒေ၊ စက္ခ ဣက္ခ ဒဿနေ၊ ဘိက္ခ-ယာစနေ၊ သံကသင်္ကာယံ၊ ဣန္ဒ-ပရမိဿရိယေ၊ အန္ဒ-ဗန္ဓနေ၊ ယဇ-ဒေဝပူဇာယံ၊ အဋ ပဋ-ဂမနထာ၊ အဏ-သဒ္ဒတ္ထော၊ အသ-ခေပနေ၊ ဝသ-နိဝါသေ၊ ပသဗာဓနေ၊ ပံသ-နာသနေ၊ ဗန္ဓ-ဗန္ဓနေ၊ ဧတေဟိ ကြိယတ္ထေဟိ ဥ ဟောတိ။ ဘရတီတိ ဘရ=ဘတ္တာ။ မရတိ ရူပကာယေန သဟေဝါတိ မရု=ဒေဝေါ၊ နိဇ္ဇလဒေသော စ။ စရီယတိ ဘက္ခီယတီတိ စရု=ဟဗျပါ ကော။ တရန္တိ အနေနာတိ တရု=ရုက္ခော။ အရတိ သူနဘာဝေန ဥဒ္ဓံ ဂစ္ဆတီတိ အရု=ဝဏော။ ဂရတိ သိဉ္စတိ၊ ဂိရတိ ဝမတိ ဝါ သိဿေသု သိနေဟန္တိ ဂရု=အာစရိယော။ ဟနတိ ဩဒနာဒီသု ဝဏ္ဏဝိသေသံ နာသေတီတိ ဟနု=ဝဒနေကဒေသော။ တနောတိ သံသာရ ဒုက္ခန္တိ တနု-သရီရံ။ မညတိ သတ္တာနံ ဟိတာဟိတန္တိ မနု=ပဇာပတိ။ ဘမတိ စလတီတိ ဘမု=နယနော ပတိဋ္ဌာနံ။ ကေတတိ ဥဒ္ဓံ ဂစ္ဆတိ၊ ဥပရိ နိဝသတီတိ ဝါ ကေတု=ဓဇော။ ဓနတိ သဒ္ဒံ ကရောတီတိ ဓနု=စာပေါ။ ဗံဟ ဣတိ နိဒ္ဒေသာ ဥမှိ နိစ္စံ နိဂ္ဂဟီတလောပေါ၊ ဗံဟတိ ဝုဒ္ဓိံ ဂစ္ဆတီတိ ဗဟု=အနပ္ပကံ။ ကမ္ဗတိ သံဝရံ ကရောတီထိ ကမ္ဗု=ဝလယော၊ သင်္ခေါ စ။ အမ္ဗတိ နာဒံကရောတီတိ အမ္ဗု=ဝါရိ။ စက္ခတိ ရူပန္တိ စက္ခု=နယနံ။ ဘိက္ခတီတိ ဘိက္ခု=သမဏော။ သံကိရတီတိ သံကု=သူလံ။ ဣန္ဒတိ နက္ခတ္တာနံ ပရမိဿရိယံ ပဝတ္တေတီတိ ဣန္ဒု=စန္ဒော။ အန္ဒန္တိ ဗန္ဓန္တိ သတ္တာ ဧတာယာတိ အန္ဒု=သင်္ခလိကာ။ ယဇန္တိ အနေနာတိ ယဇု=ဝေဒေါ။ ပဋတိ ဗျတ္တဘာဝံ ဂစ္ဆတီတိ ပဋု=ဝိစက္ခဏော။ အဏတိ သုခုမဘာဝေန ပဝတ္တတီတိ အဏု=သုခုမော၊ ဝီဟိဘေဒေါ စ။ အသန္တိ ပဝတ္တန္တိ သတ္တာ ဧတေဟီတိ အသဝေါ=ပါဏာ။ သုခံ ဝသန္တျနေနာတိ ဝသု=ဓနံ။ ပသီယတိ ဗာဓီယတိ သာမိကေဟီတိ ပသု။ စတုပ္ပဒေါ။ ပံသတိ သောဘာဝိသေသံ နာသေတီတိ ပံသု=ရေဏု။ ဗန္ဓီယတိ သိနေဟဘာဝေနာတိ ဗန္ဓု=ဉာတိ။}

\sutta{853}{3}{ဗန္ဓာ ဦ ဝဓော စ။}
\vutti{ဗန္ဓ-ဗန္ဓနေ တိမသ္မာ ဦ ဟောတိ၊ ဗန္ဓဿ ဝဓာဒေသော စ။ ပဉ္စဟိ ကာမဂုဏေဟိ အတ္တနိ သတ္ထေ ဗန္ဓတီတိ ဝဓူ-သုဏိသာ၊ ဣတ္ထီစ။}

\sutta{854}{4}{ဇမ္ဗာဒယော။}
\vutti{ဇမ္ဗူအာဒယော သဒ္ဒါဦပစ္စယန္တာ နိပစ္စန္တေ၊ နိပါတနံ အပ္ပတ္တဿ ပါပနံ ပတ္တဿ ပဋိသေဓော စ။ ဇနိသ္မာ ဦ၊ ဗုကာဂမော၊ “မနာနံ နိဂ္ဂဟီတ ”န္တိ (၅.၉၆) နဿ နိဂ္ဂဟီတံ၊ “ဝဂ္ဂေ ဝဂ္ဂန္တော ”တိ (၁.၄၁) နိဂ္ဂဟီတဿ မော၊ ဇာယတိ၊ ဇနီယတီ ဝါ ဇမ္ဗူ=ရုက္ခော။ ဘမိဿ အမလောပေါ၊ ဘမတိ ကမ္ပတီတိ ဘူ=ဘမု။ ကရောတိသ္မာ ဦ၊ တဿ ကန္ဓုဉ စ၊ “ပရရူပမယကာရေ ဗျဉ္ဇနေ ”တိ (၅.၉၅) ဓာတွန္တဿ ဗျဉ္ဇနဿ ပရရူပတ္တံ၊ ရုဓိရုပ္ပာဒံ ကရောတီတိ ကက္ကန္ဓူ=ဗဒရီ။ လမ္ဗ-အဝသံသနေ ၊ အာပုဗ္ဗော၊ သံယောဂါဒိလောပဒီဃရဿာ၊ အာလမ္ဗတိ အဝသံသတီတိ အလာဗူ=တုမ္ဗိ။ သရ-ဂတိ ဟိံသာစိန္တာသု၊ ဦမှိ အဘုက အဗုက စ၊ သရတိ ဂစ္ဆတီတိ သရဘူ=ဧကာ မဟာနဒီ။ သရတိ ပါဏေ ဟိံသတီတိ သရဗူ=ခုဒ္ဒဇန္တုကဝိသေသော။ စမအဒနေ၊ စမတိ ဘက္ခတိ နိဝါပန္တိ စမူ=သေနာ။ တန-ဝိတ္ထာရေ၊ တနောတိ သံသာရဒုက္ခန္တိ တနု=သရီရံ၊ ဧဝမညေပိ ပယောဂတော ဒဋ္ဌဗ္ဗာ။}

\sutta{855}{5}{တပုသ ဝိ ဓ ကုရ ပုထ မုဒါ ကု။}
\vutti{တပ-သန္တာပေ၊ ဥသ-ဒါဟေ၊ ဝိဓ-ဝေဓနေ၊ ကုရ-သဒ္ဒေ၊ ပုထ ပထ-ဝိတ္ထာရေ၊ မုဒ-တောသေ၊ ဧတေဟိ ကု၊ ဟောတိ။ ကကာရောနုဗန္ဓော “န တေ ကာနုဗန္ဓနာဂမေသူ ”တိ (၅.၈၅) ဧဩနမဘာဝတ္ထော။ တပ ဣတိ နိဒ္ဒေသတောဝ အဿ ဣတ္တံ၊ တာပီယတီတိ တိပု=လောဟဝိသေသော။ ဥသတိ ဒါဟံ ကရောတီတိ ဥသု=သရော။ ဝေဓတိ ရံသီဟိ တိမိရန္တိ ဝိဓု=စန္ဒော။ ကုရတိ ကိစ္စာကိစ္စံ ဝဒတီတိ ကုရု=ရာဇာ၊ ကုရဝေါ=ဇနပဒေါ။ ပုထတိ မဟန္ထဘာဝေန ပတ္ထိရတီတိ ပုထု=ဝိတ္ထိဏ္ဏော၊ မောဒနံ၊ မုဒီယတီတိ ဝါ မုဒု=အထဒ္ဓေါ။}

\sutta{856}{6}{သိန္ဓာဒယော။}
\vutti{သိန္ဓု အာဒယော ကုပစ္စယန္တာ နိပစ္စန္တေ။ သန္ဒ-ပဿဝနေ၊ အဿ ဣတ္တံ၊ ဓောစာန္တာ ဒေသော နိပါတနာ၊ သန္ဒတိ ပဿဝတီတိ သိန္ဓု=နဒိ၊ ဝဟိသုပန္တဿ ဒီဃာဒိဗတ္တေ၊ ဗာဓိဿ ဝါန္တဟတ္တေ စ၊ ဝဟန္တျနေနာတိ ဗာဟု၊ ဗာဓတိ ဥပဒ္ဒဝေ ဝါရေတီ တိ ဝါ ဗာဟု=ဘုဇော။ ရံဃ-ဂမနေ၊ နိဂ္ဂဟီတလောပေါ၊ ရင်္ဃတိ ပဝတ္တဟိ ရာဇဓမ္မေတိ ရဃု=ရာဇာ။ ဝိဒ-လာဘေ၊ နကာရာဂမော၊ ဝဿ ဗော၊ ဝိန္ဒတျနေန နန္ဒနန္တိ ဗိန္ဒု=ကဏိကာ။ မန-ဉာဏေ၊ နဿ ဓော၊ မညတိ ဉာယတိ မဓုရန္တိ မဓု=မဓုကရီဟိ ကတမဓု၊ ရပ လပ ဇပ ဇပ္ပ-ဝစနေ၊ အဿဣတ္တံ၊ ရပတိ ဇပ္ပတိ မန္တန္တီ ရိပု=ပစ္စာမိတ္တော။ သသ-ဂတိ ဟိံ သာပါဏနေသု၊ အဿ ဥတ္တံ၊ သသတိ ဇီဝတီတိ သုသု=ယုဝါ။ အရ ဂမနေ၊ အဿဥတ္တ မူတ္တဉ္စ၊ အရတိ မဟန္တဘာဝံ ဂစ္ဆတီတိ ဥရု=မဟာ၊ အရတိနေနာတိ ဦရု-သတ္ထိ။ ခဏ-အဝဒါရဏေ၊ အာ ပုဗ္ဗော အနလောပေါ၊ အာခနတီတိ အာခု=ဥန္ဒူရော။ တရတရဏေ၊ တဿ ထော၊ တရတီတိ ထရု=ခဂ္ဂါဝယဝေါ။ လံဃ-ဂတိ သောသနေသု ဃဿ ဝါ ဟတ္တံ၊ နိဂ္ဂဟီတလောပေါ စ၊ လင်္ဃတိ ပဝတ္တတိ လဃုဘာဝေနာတိ လဃု လဟူ ဧကတ္ထာ၊ (ဘံဇ-ဩမဒ္ဒနေ၊ ပ-ပုဗ္ဗော၊ ဇဿ ဂတ္တံ၊ ပဘဉ္ဇတိ ဝိသေသေနာတိ ပဘင်္ဂု=ဘင်္ဂုရော။) ဌာ-ဂတိနိဝတ္တိယံ၊ သုပုဗ္ဗော၊ ဌာတိ ပဝတ္တတိ သုန္ဒရဘာဝေနာတိ သုဋ္ဌု=သောဘနံ၊ ဒုပုဗ္ဗော ဌာတိ ပဝတ္တတိ အသုန္ဒရဘာဝေနာတိ ဒုဋ္ဌု=အသောဘနံ။ ဧဝမညေပိ ဝိညေယျာ။}

\sutta{857}{7}{ဣ။}
\vutti{ကြိယတ္ထာ ဗဟုလံ ဣ ဟောတိ၊ အဘ-ခေပနေ၊ အသီယတိ ခိပီယတီတိ အသိ=ခဂ္ဂေါ။ ကသ-ဝိလေခနေ၊ ကသီယတေ ကသိ=ကသနံ။ မသ-အာမသနေ၊ မသီယတီတိ မသိ=မေဠာ။ ကု-သဒ္ဒေ ဩ အဝါဒေသာ၊ ကဝတိ ကထေဘီတိ ကဝိ=ကဗ္ဗကာရော။ ရုသဒ္ဒေ၊ ရဝတိ ဂဇ္ဇတီတိ ရဝိ=အာဒိစ္စော။ သပ္ပ-ဂမနေ၊ သပ္ပတိ ပဝတ္တတီတိ သပ္ပိ=ဃတံ။ ဂန္ထ-ဂန္ထနေ၊ “တ ထ န ရာနံ ဋ ဌ ဏလာ ”တိ (၁.၂၅) နထာနံ ဏဌာ၊ ဂန္ထေတီတိ ဂဏ္ဌိ=ပဗ္ဗော၊ ဂဏ္ဌိစ။ ရာဇ-ဒိတ္တိယံ၊ ရာဇတိ ပဝတ္တတီတိ ရာဇိ=ပါဠိ။ ကလ-သင်္ချာနေ၊ ကလီယတိ ပရိမီယတီတိ ကလိ=ပါပံ။ ဗလ-ပါဏနေ၊ ဗလန္တိ ဇီဝန္တိ အနေနာတိ ဗလိ=ကရော။ ဓန-သဒ္ဒေ၊ ဓနတိ နဒတီတိ ဓနိ=သဒ္ဒေါ။ အစ္စ အဉ္စ-ပူဇာယံ၊ အစ္စီယတိ ပူဇီယတီတိ အစ္စိ=ဇာလာ။ ဝလ ဝလ္လ-သံဝရဏေ၊ ဝလနံ သံကောစနံ ဝလိ=ဥဒရာဒီသု ပလိ၊ ဝလ္လီယန္တိ သံဝရီယန္တိ သတ္တာ ဧတာယာတိ ဝလ္လိ=လတာ။ ဝိမှိ ဝလီ ဝလ္လီတိပိ ဟောတိ၊ သောယေဝတ္ထော၊ ဧဝမညေပိ။}

\sutta{858}{8}{ဒဈာဒယော။}
\vutti{ဒဓိအာဒယော သဒ္ဒါ ဣပစ္စယန္တာ နိပစ္စန္တေ။ ဓာ-ဓာရဏေ၊ ဒွိဘာဝေါ နိပါတနာ၊ ဃတမာဒဓာတီတိ ဒဓိ=ဂေါရ သဝိသေသော။ အံဟ-ဂမနေ၊ နိဂ္ဂဟီတလောပေါ၊ အံဟတိ ဂစ္ဆတီတိ အဟိ=သပ္ပော။ ကမ္ပ စလနေ၊ သံယောဂါဒိလောပေါ၊ ကမ္ပတိ စလတီတိ ကပိ=ဝါနရော။ မန-ဉာဏေ၊ အဿ ဥတ္တံ၊ မနတိ ဇာနာတီတိ မုနိ=သမဏော။ နဿ ဏတ္တံ၊ မနတိ မဟဂ္ဃဘာဝံ ဂစ္ဆတီတီ မဏိ=ရတနံ။ ဣက္ခ စက္ခ-ဒဿနေ၊ ဣဿ အတ္တံ ဣက္ခတိ အနေနာတိ အက္ခိ=နယနံ။ ကမပဝိက္ခေပေ၊ အဿ ဣတ္တံ၊ ကမတိ ယာတီတိ ကိမိ=ခုဒ္ဒဇန္တုကဝိသေသော။ တရ တရဏေ၊ တိတ္တိရာဒေသော၊ တုရိတော တရတိ ယာတီတိ တိတ္တိရိ=ပက္ခိဝိသေသော။ ကီဠ-ဝိဟာရေ၊ ဤဿ ဧတ္တံ၊ ကီဠနံ ကေဠိ=ကီဠာ။ ဥသိသ္မာ ဣဿ ခလုဉ၊ ဥသတိ ဒဟတီ ဥက္ခလိ=ဘာဇနံ၊ ဧဝမညေပိ။}

\sutta{859}{9}{ယုဝဏ္ဏုပန္တာ ကိ။}
\vutti{ဣဝဏ္ဏုပန္တေဟိ စ ဥဝဏ္ဏုပန္တေဟိ စ ကြိယတ္ထေဟိ ဗဟုလံ ကိ ဟောတိ။ ကကာရော-နုဗန္ဓော၊ ဣသ သိသ-ဣစ္ဆာယံ၊ သိဝံ ဣစ္ဆတီတိ ဣသိ=တပဿီ။ ဂိရ-နိဂိရဏေ၊ ဂိရတိ ပသဝတိ ဆဝိမံသသာရဘူတံ ဘေသဇ္ဇာဒိန္တိ ဂိရိ=သေလော။ သုစ-သောစနေ။ သောစနံ သုစိ=သောစေယျံ၊ ရုစ-အဘိလာသေ ရုစန္တိ ဧတာယာတိ ရုစိ=အာဘိလာသော၊ ဧဝမညေပိ။}

\sutta{860}{10}{ဝပ ဝရ ဝသ ရသ နဘ ဟရ ဟန ပဏာ ဣဏ။}
\vutti{ဝပ-ဗီဇနိက္ခေပေ၊ ဝရ-ဝရဏသမ္ဘတ္တီသု၊ ဝသ-နိဝါသေ၊ ရသ-အဿာဒနေ၊ နဘ-ဟိံသာယံ၊ ဟရ ဟရဏေ၊ ဟန ဟိံသာယံ၊ ပဏ-ဗျဝဟာရထုတိသု၊ ဧတေဟိ ဣဏ ဟောတိ။ ဝပန္တိ ဧတာယာတိ ဝါပိ=ဇလာသယော။ ဝါရေန္တိ ဧတေနာတိ ဝါရိ=ဇလံ။ ဝသန္တိ ဧတာယာတိ ဝါသိ=တစ္ဆက ဘဏ္ဍံ။ ရသီယတိ အဿာဒနဝသေန သမော သရီယတီတိ ရာသိ=သမူဟော။ နဘတိ ဟိံသတီတိ နာဘိ=သရီရာဝယဝေါ။ ဟာရေတီတိ ဟာရိ=မနုညံ။ “ဟနဿ ဃာတော ဏာနုဗန္ဓေ ” (၅.၉၉) တိ ဟနဿ ဃာတော၊ ဟနန္တိ ဧတေနာတိ ဃာတိ=ပဟရဏံ၊ ပဏတိ ဝေါဟရတီတိ ပါဏိ၊ ပဏတိ ဝေါဟရတိ ဧတေနာတိ ဝါ ပါဏိ=ကရော။}

\sutta{861}{11}{ဘူဂမာ ဤဏ။}
\vutti{ဘူ-သတ္တာယံ၊ အမ ဂမ-ဂမနေ ဧတေဟိ ဤဏ ဟောတိ ဘဝိဿတိ ကာလေ။ ဘဝိဿတီတိ ဘာဝီ=ဘဝိဿမာနော။ ဂမိဿတီတိ ဂါမီ=ဂမိဿာမာနော။}

\sutta{862}{12}{တန္ဒလက္ခာ ဤ။}
\vutti{တန္ဒ-အာလသိယေ၊ လက္ခ-ဒဿနင်္ကေသု၊ ဧတေဟိ ဤ ဟောတိ။ တန္ဒနံ တန္ဒီ=အာလသျံ၊ လက္ခီယန္တိ သတ္တာ ဧတာယာတိ လက္ခီ=သိရီ။}

\sutta{863}{13}{ဂမာ ရော။}
\vutti{ဂမိသ္မာ ရော ဟောတိ။ “ရာနုဗန္ဓေန္တသရာဒိဿာ ”တိ (၄.၁၃၂) အမလောပေါ၊ ဂစ္ဆတီတိ ဂေါ=ပသု။}

\begin{jieshu}
ဣတိ သရပစ္စယဝိဓာနံ။
\end{jieshu}


\sutta{864}{14}{ဣ ဘီ ကာ ကရာရ ဝက သက ဝါဟိ ကော။}
\vutti{ဣ-အဇ္ဈေန ဂတီသု၊ ဘီ-ဘယေ၊ ကာ ဂါ-သဒ္ဒေ၊ ကရ-ကရဏေ၊ အရ-ဂမနေ၊ ကုက ဝက-အာဒါနေ၊ သက-သတ္တိယံ၊ ဝါ-ဂတိဗန္ဓနေသု၊ ဧတေဟိ ကပစ္စယော ဟောတိ။ ဧတိ ပဝတ္တတီတိ ဧတော=အသဟာယော။ ဘာယန္တိ ဧတသ္မာတိ ဘေကော=မဏ္ဍူကော။ ကာယတိ သဒ္ဒံ ကရောတီတိ ကာကော=ဝါယသော။ ကရောတိ ဝဏ္ဏကန္တိ ကက္ကော=ဝဏ္ဏဝိသေသော၊ ပိသိတဒဗ္ဗဉ္စ။ အရတိ ယာတီတိ အက္ကော=သူရော၊ ဝိဋပိဝိသေသော စ။ ဝကတိ ဩဒနံ အာဒဒါတီတိ ဝက္ကံ=ဒေဟကောဋ္ဌာသဝိသေသော၊ သက္ကတီတိ သက္ကော-ဒေဝိန္ဒော၊ သမတ္ထော စ။ ဝါတိ ဗန္ဓတိ ဧတေနာတိ ဝါကော=ဝက္ကလံ။}

\sutta{865}{15}{ဉုကာဒယော။}
\vutti{ဉုကာဒယော ကပစ္စယန္တာ နိပစ္စန္တေ။ ဦဟ-ဝိတက္ကေ၊ ဟလောပေါ နိပါတနာ ဦဟီယတိ ဝိစိနီယတီတိ ဦကာ-ဩကောဒနီ။ ဥန္ဒ ကိလေဒနေ၊ သံယောဂါဒိ လောပေါ၊ အက စ။ ဥန္ဒတိ ဒြဝံ ကရောတီတိ ဥဒကံ=ဇလံ။ ဘီ-ဘယေ၊ ဧတ္တာဘာဝေါ၊ ဘာယန္တိ ဧတသ္မာတိ ဘီကော=ဘီရု။ သက-သတ္တိယံ၊ ဥပန္တဿိ၊ သက္ကောတိ ဓာရေတုန္တိ သိက္ကာ=ဥပကရဏဝိသေသော။ ဟာ-စာဂေ၊ ဟီယတိ သာဓူဟိ ဇဟီယတီတိ ဟာကော=ကောဓော။ သမ္ဗ-မဏ္ဍနေ၊ ကဿ ဥဉ ၊ သမ္ဗတိ ဥဒကံ မဏ္ဍေတီတိ သမ္ဗုကော=ဇလဇန္တုဝိသေဘော။ ပုထ ပထ-ဝိတ္ထာရေ၊ ဩတ္တာဘာဝေါ၊ ကဿ ဥဉ၊ ပုထတိ ပတ္ထရတိ အတ္တနော ဗာလဘာဝန္တိ ပုထုကော=ဗာလော။ သုစ-သောကေ၊ သောစန္တိ ဧတေနာတိ သုက္ကံ=သမ္ဘဝေါ၊ သေတဉ္စ။ စိ-စယေ၊ ဥပပုဗ္ဗော၊ ဧတ္တာဘာဝေါ၊ ဥပစိနန္တီတိ ဥပစိကာ=ဝမ္မိက ကာရာ။ ကမ္ပ-စလနေ၊ ကမ္ပိဿ ပံ၊ ကမ္ပတိ စလတီတိ ပင်္ကော=ကဒ္ဒမော။ ဥသ-ဒါဟေ၊ ဥသတီတိ ဥက္ကာ=ဇာလာ။ ကဿ မုဉ၊ ဥသတိ ဒဟတီဘိ ဥမ္မုကံ=အလာတံ။ ဝမ-ဥဂ္ဂိရဏေ၊ ကဿ မိဉ၊ ဝမီယတီတိ ဝမ္မိကော=ဥပစိကာကတော စယော၊ မသ-အာမသနေ၊ သဿ တ္ထင၊ မသီယတိ ပေမေနာတိ မတ္ထကံ=သီသံ၊ ဧဝမညေပိ။}

\sutta{866}{16}{ဘီတွာ နကော။}
\vutti{ဘီ-ဘယေ တီမသ္မာ အာနကော ဟောတိ။ ဘာယန္တိ ဧတသ္မာကိ ဘယာနကော=ဘယဇနကော။}

\sutta{867}{17}{သိင်္ဃာ အာဏိကာဋကာ။}
\vutti{သိင်္ဃ-ဃာယနေတီမသ္မာ အာဏိက အာဋကာ ဟောန္တိ။ “ဣတ္ထိယမတွာ ”တိ ၉၃.၂၆၀ အာ၊ သိင်္ဃယတိ ပဿဝတီတိ သိင်္ဃာဏိကာ=နာသာဿဝေါ။ သိင်္ဃတိ ဧကီဘာဝံ ယာတီတိ သိင်္ဃာဋကံ=ဝီထိစတုက္ကံ။}

\sutta{868}{18}{ကရာဒိတွကော။}
\vutti{ကရ-ကရဏေ၊ သရ-ဂတိဟိံသာစိန္တာသု၊ နရ-နယေ၊ တရတရဏေ၊ ဝရ-ဝရဏ သမ္ဘတ္တီသု၊ ဇန-ဇနနေ၊ ကရ-ဒိတ္တိ ဂတိကန္တီသု၊ ကဋ-မန္ဒနေ၊ ကုရ-သဒ္ဒေ၊ ထု-အဘိတ္ထဝေ၊ ဧဝမာဒီဟိ အကော ဟောတိ။ ကရီယတီတိ ကရကော=ကမဏ္ဍလု။ ကရောတီတိ ကရကာ=ဝဿောပလာ။ သရတိ ဥဒကမေတ္ထာတိ သရကော=ပါနဘာဇနံ။ နရန္တိ ပါပုဏန္တိ သတ္တာ ဧတ္ထာတိ နရကော=နိရယော။ တရန္တျနေနာတိ တရကော=တရဏံ။ ဝါရေတီတိဝရကော=ဝရဏော၊ ဓညဝိသေသော စ။ ဇနေတီတိ ဇနကော=ပိတာ။ ကနတိ ဒိဗ္ဗတီတိ ကနကံ=သုဝဏ္ဏံ။ ကဋတိ မဒ္ဒတိ ရိပဝေါတိ ကဋကံ=နဂရံ။ ကုရတီတိ ကောရကော=ကလိကာ။ ထဝီယတီတိ ထဝကော-ဂုစ္ဆော။}

\sutta{869}{19}{ဗလ ပတေဟျာကော။}
\vutti{ဗလ-ပါဏနေ၊ ပတ ပထ-ဂမနေ၊ ဧတေဟိ အာကော ဟောတိ။ ဗလတိ ဇီဝတီတိ ဗလာကာ=ပက္ခိဝိသေသော။ ပတတိ ယာတီတိ ပတာကာ=ဓဇော။}

\sutta{870}{20}{သာမာကာဒယော။}
\vutti{သာမာကအာဒယော အာကန္တာ နိပစ္စန္တေ။ သာ-တနု ကရဏာဝသာနေသု၊ သာဿ မုက၊ သာတိ ဒေဟံ တ နုံ ကရောတီတိ သာမာကော=တိဏဓညံ။ ပါ-ပါနေ၊ ဣနင၊ ပိဝတိ ရတ္တန္တိ ပိနာကော=မဟိဿရဓနု။ ဂု-သဒ္ဒေ၊ “ယုဝဏ္ဏာန မိယငုဝင သရေ ”တိ (၅.၁၃၆) ဥဝင၊ ဂဝတိ နဒတိ ဧတေနာတိ ဂုဝကော=ပူဂဖလံ။ အဋ ပဋ-ဂမနတ္ထာ၊ ပဋတိ ယာတီတိ ပဋာကာ=ဝေဇယန္တီ။ သလ ပိလ ပလ ဟုလ-ဂမနတ္ထာ၊ သလတိ ယာတီတိ သလာကာ=ဝေဇ္ဇောပကရဏဒဗ္ဗံ။ ဝိဒ-ဉာဏေ၊ ဝိဒတိ ဇာနာတီတိ ဝိဒါကော=ဝိဒွါ။ ပဏဗျဝဟာရထုတီသု အဿ ဣတ္တံ၊ ဦက စ၊ ပဏီယတိ ဝေါဟရီယတီတိ ပိညာကော=တိလကက္ကော၊ ဧဝမညေပိ။}

\sutta{871}{21}{ဝိစ္ဆာလဂမမုသာ ကိကော။}
\vutti{ဝိစ္ဆာ-ဂမနေ ၊ အလ=ဗန္ဓနေ၊ အမ ဂမ-ဂမနေ၊ မုသ-ထေယျေ၊ ဧတေဟိ ကိကော ဟောတိ။ ဝိစ္ဆတိ ယာတီတိ ဝိစ္ဆိကော=ကီဋော။ အလတိ ဗန္ဓတိ ဧတေနာတိ အလိကံ=အသစ္စံ။ ဂစ္ဆတီတိ ဂမိကော=ဂန္တာ။ မုသာတိ နိဒ္ဒေသာ ဒီဃော၊ မုသတိ ထေနေတီတိ မူသိကော=ဥန္ဒူရော။}

\sutta{872}{22}{ကိံကဏိကာဒယော။}
\vutti{ကိံကဏိကာဒယော သဒ္ဒါ ကိကန္တာ နိပစ္စန္တေ။ ကဏဣတိ ဒဏ္ဍကော ဓာတု သဒ္ဒတ္ထော၊ ကဿ ဒွိတ္တံ၊ အဿ ဣတ္တံ၊ နိဂ္ဂဟီတာဂမော စ၊ ကဏတိ သဒ္ဒံ ကရောတီ ကိံကဏိကာ=ဃဏ္ဍိကာ။ မုဒ-တောသေ၊ ဒဿ ဒွိတ္တံ၊ မုဒန္တိ ဧတမယာတိ မုဒ္ဒိကာ=အင်္ဂုလိယာဝေဋ္ဌနံ၊ ဖလဝိသေသော အ။ မဟီယတိ ပူဇီယတီတိ မဟိကာ=ဟိမံ။ ကလ-သင်္ချာနေ၊ ကလီယတိ ပရိမီတယတီတိ ကလိကာ=ကောရကော။ သပ္ပ-ဂမနေ၊ အဿ ဣတ္တံ။ သပ္ပတိ ဂစ္ဆတီတိ သိပ္ပိကာ=ဇလဇန္တုဝိသေသော၊ ဧဝမညေပိ။}

\sutta{873}{23}{ဣသာ ကီကော။}
\vutti{ဣသ သိံသ-ဣစ္ဆာယံ၊ ဣစ္စသ္မာ ကီကော ဟောတိ။ ဣစ္ဆီယတီတိ ဣသီကာ=တူလနိဿယော။}

\sutta{874}{24}{ကမပဒါ ဏုကော။}
\vutti{ကမ-ဣစ္ဆာယံ၊ ပဒ-ဂမနေ၊ ဧတေဟိ ဏုကော ဟောတိ။ “ဏိဏာပီန ”န္တိ (၅.၁၆၀) ယောဂဝိဘာဂါ ဏိလောပေါ။ ကာမေတီတိ ကာမုကော=ကာမယိတာ ။ ပဇ္ဇတိ ယာတိ ဧတာယာတိ ပါဒုကာ=ပါဒေါပကရဏံ။}

\sutta{875}{25}{မဏ္ဍသလာ ဏူကော။}
\vutti{မဏ္ဍ-ဘူသနေ၊ သလ-ဂမနတ္ထော ဒဏ္ဍကော ဓာတု၊ ဧတေဟိ ဏူကော ဟောတိ။ မဏ္ဍေတိ ဇလံ ဘူသေတီတိ မဏ္ဍူကော=ဘေကော၊ သလတိ ဂေါစရတ္တ မုပယာတီတိ သာလူကံ=ဥပ္ပလကဏ္ဍော။}

\sutta{876}{26}{ဥလူကာဒယော။}
\vutti{ဥလူကာဒယော သဒ္ဒါ ဏူကန္တာ နိပစ္စန္တေ။ ဥလ-ဂဝေသနေ၊ ဩတ္တာဘာဝေါ နိပါတနာ၊ ဥလတိ ဂဝေသတီတိ ဥလူကော=ကောသိယော။ မန=ဉာဏေ၊ နဿ ဓတ္တံ၊ မညတီတိ မဓူကော=ရုက္ခော။ ဇလ-ဒိတ္တိယံ၊ ဇလတီတိ ဇလူကာ=လောဟိတပေါ၊ ဧဝမညေပိ။}

\sutta{877}{27}{ကသာ သကော။}
\vutti{ကသ-ဝိလေခနေတီမသ္မာ သကော ဟောတိ။ ကသတီတိ ကဿကော=ကသိကမ္မကာရော။}

\sutta{878}{28}{ကရာ တိကော။}
\vutti{ကရောတိသ္မာ တိကော ဟောတိ။ ကရောန္တိ ကီဠံ ဧတ္ထာတိ ကတ္တိကော=ဗာဟုလော။}

\sutta{879}{29}{ဣသာ ဌကန။}
\vutti{ဣသ သိံသ-ဣစ္ဆာယံ၊ ဣစ္စသ္မာ ဌကန ဟောတိ။ ဣစ္ဆီယတီတိ ဣဋ္ဌကာ=မတ္တိကာဝိကာရော။}

\sutta{880}{30}{သမာ ခေါ။}
\vutti{သမ-ဥပသမခေဒေသု ၊ ဧတသ္မာ ခေါ ဟောတိ။ သမေတိ ဥပသမေတီတိ သင်္ခေါ=ကမ္ဗု။}

\sutta{881}{31}{မုခါ ဒယော။}
\vutti{မုခ အာဒယော ခန္တာ နိပစ္စန္တေ။ မု-ဗန္ဓနေ၊ ဩတ္တာဘာဝေါ နိပါတနာ၊ မုနန္တိ ဗန္ဓန္တိ ဧတေနာတိ မုခံ=လပနံ။ သိ-သေဝါယံ၊ ဧတ္တာ ဘာဝေါ နိပါတနာ၊ သယန္တိ ဧတ္ထ ဦကာ ကုသုမ-ဒယော စာတိ သိခါ=စူဠာ။ ဝိပုဗ္ဗဿ သိဿ၊ ဝိသတိဿ ဝါ၊ ဝိသေသေန သယန္တိ ဧတ္ထ၊ ပဝိသန္တီတိ ဝိသိခါ=ရစ္ဆာ။ ကန-ဒီတ္တိ ဂတိ ကန္တီသု၊ နိပုဗ္ဗော၊ အနဘာဂလောပေါ၊ ကနတိ ဒိဗ္ဗတီတိ နိက္ခော=သုဝဏ္ဏဝိကာရော။ မယဣတိ ဂမနတ္ထော ဒဏ္ဍကော ဓာတု၊ ခဿ ဦဉ၊ မယတိ ယာတီတိ မယူခေါ-ကိရဏော။ လူ-ဆေဒနေ၊ ဩတ္ထာဘာဝေါ၊ လုနာတိ ဆိန္ဒတိ သောဘန္တိ လူခေါ=အသိနိဒ္ဓေါ။ အရ-ဂမနေ၊ အရန္တိ ယန္တိ ဧတေနာတိ အက္ခော=သကဋာဝယဝါ၊ ပါသကော စ။ ယသ-ပယတနေ၊ ယဿတိ ပယတတိ ဗလိမာဟရဏတ္ထာယာတိ ယက္ခော=အမနုဿော၊ ရုဟ-ဇနနေ၊ ရုဟတိ ဇာယတီတိ ရုက္ခော=ပါဒပေါ၊ ဥသ-ဒါဟေ၊ ဥသတိ ဒဟတိ ကာမဂ္ဂိနာတိ ဥက္ခော=ဗလီဗဒ္ဒေါ။ သဟ-မရိသနေ ’ဟလောပေါ၊ သဟတိ အတ္တနိ ကတာပရာဓံ ခမတီတိ သခါ=သဟာယော၊ ဧဝမညေပိ။}

\sutta{882}{32}{အဇ ဝဇ မုဒ ဂဒ ဂမာ ဂက။}
\vutti{အဇ ဝဇ-ဂမနေ၊ မုဒ-တောသေ၊ ဂဒ-ဝစနေ၊ အမ ဂမ-ဂမနေ၊ ဧတေဟိ ဂက ဟောတိ။ အဇတိ ဂစ္ဆတိ သေဋ္ဌဘာဝန္တိ အဂ္ဂေါ-သေဋ္ဌော ။ ဝဇတိ သမူဟတ္တံ ဂစ္ဆတီတိ ဝဂ္ဂေါ=သမူဟော။ မုဒန္တိ ဧတေနာတိ မုဂ္ဂေါ=ဓညဝိသေသော။ ဂဒတီတိ ဂဂ္ဂေါ=ဣသိ။ “မနာနံ နိဂ္ဂဟီတ ”န္တိ (၅.၉၆) မဿ နိဂ္ဂဟီတံ၊ ဂစ္ဆတီတိ ဂင်္ဂါ=သုရာပဂါ။}

\sutta{883}{33}{သိင်္ဂါဒယော။}
\vutti{သိင်္ဂအာဒယော သဒ္ဒါ ဂကအန္တာ နိပစ္စန္တေ။ သီ-သယေ၊ နိဂ္ဂဟီတာဂမော၊ ရဿတ္တဉ္စ၊ သယတိ ပဝတ္တတိ မတ္ထကေတိ သိင်္ဂံ=ဝိသာဏံ။ ဖုရ-စလနေ၊ လိမုင၊ ဖုရတိ စလတီတိ ပုလိင်္ဂေါ=ဇလိတင်္ဂါ ရာဝယဝေါ။ စလ-ကမ္ပနေ၊ ဥပုဗ္ဗော၊ စလဿ စာလိံ၊ ဥစ္စလတိ ကမ္ပတီတိ ဥစ္စာလိင်္ဂေါ=သုက္ကကီဋော၊ ကလ-သဒ္ဒေ၊ ဣမုက၊ ကလတိ အဘိနာဒံ ကရောတိ ဗဟုရဇ္ဇတာယာ တိ ကလိင်္ဂေါ=ဒက္ခိဏာပထော။ ဘမ-အနဝဋ္ဌာနေ၊ အဿ ဣတ္တံ၊ ဘမတီတိ ဘိင်္ဂေါ=ဘမရော။ ပဋ အဋ-ဂမနေ၊ ပဋိဿ အမုက၊ အက စ။ ပဋတိ ပတန္တော ဂစ္ဆတီတိ ပဋင်္ဂေါ၊ ပဋဂေါ=သလဘော၊ ဧဝမညေပိ။}

\sutta{884}{34}{အဂါ ဂိ။}
\vutti{အဂ-ကုဋိလဂမနေတီမသ္မာ ဂိ ဟောတိ။ အဂတိ ကုဋိလော ဟုတွာ ဂစ္ဆတီတိ အဂ္ဂိ=ပါဝကော။}

\sutta{885}{35}{ယာဝလာ ဂု။}
\vutti{ယာ-ပါပုဏနေ၊ ဝလ ဝလ္လ-သံဝရဏေ၊ ဧတေဟိ ဂု ဟောတိ။ ယာတီတိ ယာဂု=ပေယျာ။ ဝလီယတိ သံဝရီယတီတိ ဝဂ္ဂု=မနုညော။}

\sutta{886}{36}{ဖေဂ္ဂါဒယော။}
\vutti{ဖေဂ္ဂု အာဒယော ဂုအန္တာ နိပစ္စန္တေ။ ဖလ-နိပ္ဖတ္တိယံ၊ အဿ ဧတ္တံ၊ ဖလတိ နိဋ္ဌာနံ ဂစ္ဆတီတိ ဖေဂ္ဂု=အသာရော။ ဘရ-ဘရဏေ၊ ရလောပေါ။ ဘရတီတိ ဘဂု=ဣသိ။ ဟိ-ဂတီယံ၊ နိဂ္ဂဟီတာဂမော၊ ဟိနော တိ ပဝတ္တတီတိ ဟိင်္ဂု=ရာမဌဇံ။ ကမ-ဣစ္ဆာယံ၊ ကာမီယတီတိ ကင်္ဂု=ဓညဝိသေသော၊ ဧဝမညေပိ။}

\sutta{887}{37}{ဇနာ ဃော။}
\vutti{ဇန-ဇနနေတီမသ္မာ ဃော ဟောတိ။ “မနာနံ နိဂ္ဂဟီတ ”န္တိ (၅.၉၆) နဿ နိဂ္ဂဟီတံ၊ ဇာယတိ ဂမနမေတာယာတိ ဇင်္ဃာ=ပါဏျင်္ဂဝိသေသော။}

\sutta{888}{38}{မေဃာဒယော။}
\vutti{မေဃအာဒယော ဃန္တာ နိပစ္စန္တေ။ မိဟ-သေစနေ၊ ဟလောပေါ၊ မေဟတိ သိဉ္စတီတိ မေဃော=အမ္ဗုဒေါ။ မုဟ-မုစ္ဆာယံ၊ ဟလောပေါ၊ မုယှန္တိ သတ္တာ ဧတ္ထာတိ မောဃော=တုစ္ဆော။ သီ-သယေ၊ ဧတ္တာဘာဝေါ၊ သေတိ လဟု ဟုတွာ ပဝတ္တတီတိ သီဃံ=အာသု။ အဟ-ဘသ္မီကရဏေ၊ နိပုဗ္ဗော၊ ဟလောပဒီဃာ၊ နိဒဟတီတိ နိဒါဃော=ဂိမှော။ မဟီဿ ဟလောပေါ၊ မဟီယတိ ပူဇီယတီတိ မဃာ=နက္ခတ္တံ၊ ဧဝမညေပိ။}

\begin{jieshu}
ဣတိ ကဝဂ္ဂပစ္စယဝိဓာနံ။
\end{jieshu}


\sutta{889}{39}{စုသရဝရာ စော။}
\vutti{စု-စဝနေ၊ သရ-ဂတိ ဟိံသာ စိန္တာသု၊ ဝရ-ဝရဏ သမ္ဘတ္တီသု၊ ဧယေဟိ စော ဟောတိ။ စဝတိ ရုက္ခာတိ စောစံ=ဥပဘုတ္တဖလ- ဝိသေသော ။ သရတိ အာယတိံ ဒုက္ခံ ဟိံသတီတိ သစ္စံ=အဝိတထံ။ ဝါရေတိ သုခန္တိ ဝစ္စံ=ဂူထော။}

\sutta{890}{40}{မရာ စုဤစီ စ။}
\vutti{မရ-ပါဏစာဂေ တီမသ္မာ စုဤစီ ဟောန္တိ စော စ။ မရဏံ မစ္စု=မရဏံ။ မာရေတိ အန္ဓကာရံ ဝိနာသေတီတိ မရီစိ=ရံသိ၊ မိဂတဏှိကာ စ။ မရတီတိ မစ္စော=သတ္တော။}

\sutta{891}{41}{ကုသ ပသာ ဆိက။}
\vutti{ကုသ-အက္ကောသေ၊ ပသ-ဗာဓနေ၊ ဧတေဟိ ဆိက ဟောတိ။ ကုသီယတိ အက္ကောသီယတီတိ ကုစ္ဆိ=ဥဒရံ။ ပသီယတိ ဗာဓီယတိ ဧတ္ထာတိ ပစ္ဆိ=ဘာဇနဝိသေသော။}

\sutta{892}{42}{ကသ ဥသာ ဆုက။}
\vutti{ကသ-ဝိလေခနေ၊ ဥသ-ဒါဟေ၊ ဧတေဟိ ဆုက ဟောတိ။ ကသန္တိ ဝိလေခန္တိ ဧတ္ထာတိ ကစ္ဆု=ပါမံ။ ဥသတိ ဒဟတိ သန္တာပန္တိ ဥစ္ဆု=ရသာလော။}

\sutta{893}{43}{အသ မသ ဝဒ ကုစ ကစာ ဆော။}
\vutti{အသ-ခိပနေ၊ မသ-အာမသနေ၊ ဝဒ-ဝစနေ၊ ကုစ-သံကောစနေ၊ ကစ-ဗန္ဓနေ၊ ဧတေဟိ ဆော ဟောတိ။ အသတိ ဆိပတီတိ အစ္ဆော=ဘလ္လူကော။ မသတိ ဇလန္တိ မစ္ဆော=မီနော။ ဝဒတီတိ ဝစ္ဆော=နေလကော။ ကုစီယတိ သံကောစီယတီတိ ကောစ္ဆော=ဘဒ္ဒပီဋ္ဌံ။ ကစီယတိ ဗန္ဓီယတီတိ ကစ္ဆော=တနုပဒေသော၊ အနူပေါ စ။}

\sutta{894}{44}{ဂုစ္ဆာဒယော။}
\vutti{ဂုစ္ဆအာဒယော ဆန္တာ နိပစ္စန္တေ။ ဂုပ-ဂေါပနေ၊ ဩတ္တာဘာဝေါ၊ ဂေါပီယတီတိ ဂုစ္ဆော=ထဝကော။ တုသ-တုဋ္ဌိမှီ၊ တုသန္တိ ဧတေနာတိ တုစ္ဆံ=မုသာ။ ပုသ-ပေါသနေ၊ ပေါသန္တိ တနုမနေနာတိ ပုစ္ဆော=ဝါလဓိ၊ ဧဝမညေပိ။}

\sutta{895}{45}{အရာ ဇု ဉဋ စ။}
\vutti{အရ-ဂမနေတီမသ္မာ ဇု ဟောတိ၊ အရိဿ ဥဋ စ။ ဋကာရော သဗ္ဗာဒေသတ္ထော၊ အရတိ အကုဋိလဘာဝေန ပဝတ္တတီတိ ဥဇု=အဝင်္ကော။}

\sutta{896}{46}{ရဇ္ဇာဒယော။}
\vutti{ရဇ္ဇုအာဒယော ဇုအန္တာ နိပစ္စန္တေ။ ရုဓ-အာဝရဏေ၊ ဥဿ အတ္တံ၊ ရုန္ဓန္တိ ဧတေနာတိ ရုဇ္ဇု-ယောတ္တံ။ မန-ဉာဏေ၊ အမညိတ္ထာတိ မဉ္ဇု=မဉ္ဇုလံ၊ ဧဝမညေပိ။}

\sutta{897}{47}{ဂိဓာ ဈက။}
\vutti{ဂိဓ-အဘိကင်္ခါယမိစ္စသ္မာ ဈက ဟောတိ။ ဂေဓတီတိ ဂိဇ္ဈော=ပက္ခိဝိသေသော။}

\sutta{898}{48}{ဝဉ္ဈာဒယော။}
\vutti{ဝဉ္စျအာဒယော ဈက အန္တာ နိပစ္စန္တေ။ ဝန-ယာစနေ၊ ဝနောတိ အတ္တာနံ အနုဘဝိတုံ ယာစတီတိ ဝဉ္ဈော=အဖလရုက္ခော၊ ဝဉ္ဈာ=အပသဝါ ဣတ္ထီ၊ အဿ ဣတ္တံ ဝိဉ္ဈော=ပဗ္ဗတော၊ သံဇ-သင်္ဂေ၊ နိဂ္ဂဟီတလောပေါ၊ သဉ္ဇိယတီတိ သဇ္ဈံ=ရဇတံ၊ ဧဝမညေပိ။}

\sutta{899}{49}{ကမယဇာ ဉော။}
\vutti{ကမ-ဣစ္ဆာယံ ၊ ယဇ-ဒေဝပူဇာသင်္ဂတိကရဏဒါနေသု၊ ဧတေဟိ ဉော ဟောတိ၊ မဿ နိဂ္ဂဟီတဝဂ္ဂန္တာ၊ ကာမီယတီတိ ကညာ=ကုမာရီ။ ဇဿ ပရရူပတ္တံ၊ ယဇန္တျနေနာတိ ယညော=ယာဂေါ။}

\sutta{900}{50}{ပုညံ။}
\vutti{ပုနာတိ ပုဏတိသ္မာ ဝါ ဉောဩတ္တာဘာဝေါ စ နိပစ္စန္တေ။ ပုနာတိ၊ ပုဏတိ သုန္ဒရတ္တံ ကရောတီတိ ဝါ ပုညံ=ကုသလံ။}

\sutta{901}{51}{အရ ဟာညော ဟာဿ ဟိရ စ။}
\vutti{အရ-ဂမနေ၊ ဝါ-စာဂေ၊ ဧတေဟိ အညော ဟောတိ၊ ဟာဿ ဟိရဉ္စာဒေသော။ အရီယတေ ဂမျတေတိ အရညံ=ဝနံ။ ဇဟာတိ သတ္တာနံ ဟီနတ္တန္တိ ဟိရညံ=ဓနံ၊ သုဝဏ္ဏဉ္စ။}

\begin{jieshu}
ဣတိ စဝဂ္ဂပစ္စယဝိဓာနံ။
\end{jieshu}


\sutta{902}{52}{ကိရ တရာ ကီဋော။}
\vutti{ကိရ-ဝိကိရဏေ၊ တရ-တရဏေ၊ ဧတေဟိ၊ ကီဋော ဟောတိ။ သောဘေတုမေတ္ထ ရတနာနိ ဝိကိရီယန္တီတိ ကိရီဋံ=မကုဋံ။ တရဣတိ နိဒ္ဒေသာ အဿ ဣတ္တံ၊ တရန္တိ ယန္တိ သုရူပတ္တ မနေနာတိ တိရီဋံ=ဝေဋ္ဌနံ။}

\sutta{903}{53}{သကာဒီဟျဋော။}
\vutti{သက-သတ္တိယံ၊ ကသ-ဂမနေ၊ ကရ-ကရဏေ၊ မက္က ဣတိ သုတ္တိယော ဓာတု၊ ဒေဝ-ဒေဝနေ၊ ကမ-ဣစ္ဆာယံ၊ ဧဝမာဒိဟိ အဋော ဟောတိ။ သဏ္ဏောတိ ဘာရံ ဝဟိတုန္တိ သကဋော=ယာနံ။ အကသိ နိရောဇတ္တံ အဂမီတိ ကသဋံ=နိရောဇံ။ ကရောတိ အမနာပန္တိ ကရဋော=ကာကော ။ မက္ကတိ စလတီတိ မက္ကဋော=ဝါနရော။ ဒေဝီယတိ ပူဇီယတီတိ ဒေဝဋော=ဣသိ။ ကမတိ ဣစ္ဆတိ အာရောဟတ္ထန္တိ ကမဋော=ဝါမနော။}

\sutta{904}{54}{မကုဋာဝါဋ ကဝါဋ ကုက္ကုဋာ။}
\vutti{ဧတေ သဒ္ဒါ နိပစ္စန္တေ။ မံကိသ္မာ ဥဋော၊ နိဂ္ဂဟီတလောပေါ စ၊ မံကေတိ သောဘေတီတိ မကုဋံ=ကိရီဋံ။ အဝတိသ္မာ အာဋဏ၊ အဝျတေ ခညတေတိ အာဝါဋော=ကာသု။ ကု-သဒ္ဒေတီမသ္မာ အာဋော၊ ဩအဝါဒေသာ ယထာယောဂံ၊ ကဝတိ ရဝတီတိ ကဝါဋံ=ဒွါရပိဓာနံ။ ကုက ဝက-အာဒါနေတီမသ္မာ ကုဋက၊ ကုကတိ ဂေါစရမာဒဒါတီတိ ကုက္ကုဋော=တမ္ဗစူဠော။}

\sutta{905}{55}{ကမုသ ကုသ ကသာ ဌော။}
\vutti{ကမ-ဣစ္ဆာယံ၊ ဥသ-ဒါဟေ၊ ကုသ-အက္ကောသေ၊ ကသ-ဂမနေ၊ ဧတေဟိဌော ဟောတိ၊ နိဂ္ဂဟီတဝဂ္ဂန္တာ၊ ဩဒနာဒီနိ ကာမေတီတိ ကဏ္ဌော=ဂလော။ ဩက္ကပရရူပါဒီနိ၊ ဩဒနာဒီသု ဥဏှေန ဥသီယတီ တိ ဩဋ္ဌော=ဒန္တစ္ဆဒေါ၊ ကရဘော စ။ ကုသီယတိ အက္ကောသီယတီတိ ကောဋ္ဌော=ဓညနိလယော။ ကသတိ ယာတိ ဝိနာသန္တိ ကဋ္ဌံ=ဒါရု။}

\sutta{906}{56}{ကုဋ္ဌာဒယော။}
\vutti{ကုဋ္ဌအာဒယော သဒ္ဒါ ဌန္တာ နိပစ္စန္တေ။ ကုသိသ္မာ ဌော၊ ဩတ္တာဘာဝေါ စ၊ ကုသီယတိ အက္ကောသီယတီတိ ကုဋ္ဌံ=ဆဝိရောဂေါ။ ကုဏ-သဒ္ဒေ၊ ပရရူပါဘာဝေါ ဩတ္တာဘာဝေါ စ၊ ကုဏတိ နဒတီတိ ကုဏ္ဌော=အတိခိဏော၊ ကုဏီယတိ အက္ကောသီယတီတိ ကုဏ္ဌော= ဆိန္နဟတ္ထပါဒါဒိကော ။ ဒံသိဿ ဒါ၊ ဒံသတိ ဧတာယာတိ ဒါဌာ=ဒန္တဝိသေသော။ ကမိဿ အက စ။ ကာမီယာတိ ဒီနေတီတိ ကမဌော=ဘိက္ခာဘာဇနံ၊ ဝါမနော၊ ကုမ္မော စ။ ဖဿိဿ ဖုဋော၊ ဖဿီယတီတိ ဖုဋ္ဌော=ဖဿော၊ ဧဝမညေပိ။}

\sutta{907}{57}{ဝရ ကရာ အဏ္ဍော။}
\vutti{ဝရ-ဝရဏေ၊ ကရ-ကရဏေ၊ ဧတေဟိ အဏ္ဍော ဟောတိ။ အတ္တနိ ပေဖံ ဝါရယတီတိ ဝရဏ္ဍော=မုခရောဂေါ။ ကရီယတီတိ ကရဏ္ဍော=ဘဏ္ဍဝိသေသော။}

\sutta{908}{58}{မနန္တာ ဍော။}
\vutti{မကာရ နကာရန္တေဟိ ကြိယတ္ထေဟိ ဗဟုလံ ဍပ္ပစ္စယော ဟောတိ။ သမ-ဥပသမေ၊ သမနံ သဏ္ဍံ=သမူဟော။ ကမ-ပဒဝိက္ခေပေ၊ ကမတိ ယာတီတိ ကဏ္ဍော=သရော၊ ပရိစ္ဆေဒေါ စ၊ ဒမ-ဒမနေ၊ ဒမန္တွျနေနာတိ ဒဏ္ဍော=နိဂ္ဂဟော။ အမ ဂမ-ဂမနေ၊ အမန္တိ ဥပ္ပဇ္ဇန္တိ ဧတ္ထာတိ အဏ္ဍော=ပက္ခိပသဝေါ၊ ကောသော စ။ ဂစ္ဆတိ သူနဘာဝန္တိ ဂဏ္ဍော=ဗျာဓိ၊ ဝဒနေကဒေသော စ။ ရမု-ကီဠာယံ၊ ရမန္တိ ဧတ္ထာတိ ရုဏ္ဍာ=ဝိဓဝါ။ မန-ဉာဏေ၊ မညန္တိ ဧတေနာတိ မဏ္ဍော=ဩဒနာဒိနိဿာဝေါ။ ခန ခဏ=အဝဒါရဏေ၊ ခညတီတိ ခဏ္ဍော=ဥစ္ဆုဝိကာရ ဝိသေသော။ လမ-ဟိံသာယံ၊ လမတိ ဟိံသတိ သုစိဘာဝန္တိ လဏ္ဍော=ဝစ္စံ၊ ဧဝမညေပိ။}

\sutta{909}{59}{ကုဏ္ဍာ ဒယော။}
\vutti{ကုဏ္ဍအာဒယော ဍန္တာ နိပစ္စန္တေ။ ကမ မန တနာနံ အဿ ဥတ္တံ၊ ကာမီယတီတိ ကုဏ္ဍံ=ဘာဇနံ။ မညတိ ဟိတာဟိတန္တိ မုဏ္ဍော= ဆိန္နကေသော ။ တနောတိ ဧတေနာတိ တုဏ္ဍံ=လပနံ။ ဤရ-ခေပေ၊ ဧရံအာဒေသော။ ဧရတိ ကမ္ပတီတိ ဧရဏ္ဍော=ဗျဂ္ဃပုစ္ဆော၊ သိ-သေဝါယံ၊ သိဿ ခမုက၊ သုဂန္ဓံ သေဝတီတိ သိခဏ္ဍော=စူဠာ၊ ဧဝမညေပိ။}

\sutta{910}{60}{တိဇ ကသ တသ ဒက္ခာ ကိဏော ဇဿ ခေါ စ။}
\vutti{တိဇ-နိသာနေ၊ ကသ-ဂမနေ၊ တသ-ပိပါသာယံ၊ ဒက္ခ-ဝုဒ္ဓိယံ၊ ဧတေဟိ ကိဏော ဟောတိ၊ ဇဿ ခေါ စ၊ တေဇယိတ္ထာတိ တိခိဏံ=နိသိတံ။ ကသတိ ပဝတ္တတီတိ ကသိဏံ=အသေသံ။ တသနံ တသိဏာ=တဏှာ။ ဒက္ခတိ ဝုဒ္ဓိံ ဂစ္ဆတိ ဧတာယာတိ ဒက္ခီဏာ=ကုသလံ။}

\sutta{911}{61}{ဝီအာဒိတော ဏိ။}
\vutti{ဝီ-တန္တသန္တာနေ၊ သိ-သေဝါသံ၊ သူ-ပဿဝနေ၊ ဒု-ဂမနေ၊ ကီ-ဒဗ္ဗဝိနီမယေ၊ သာ-တနုကရဏာဝသာနေသု၊ ဧဝမာဒီဟိ ဏိ ဟောတိ။ ဝီယတီတိ ဝေဏိ=ကေသကလာပေါ။ သေဝနံ သေဏိ=သဇာတီနံ ကာရူနံ သမူဟော။ နိပုဗ္ဗော၊ နိသေဝီယတီတိ နိသေဏိ=သောပါနံ။ သဝတိ ပဿဝတီတိ သောဏိ=ကဋိ။ ဒဝတိ ဝါတီတိ ဒေါဏိ=ကဋ္ဌမ္ဗုဝါဟနီ၊ နာဝါ စ။ နဒါဒိပါဌာ ဝိမှိ သောဏီ ဒေါဏီ တိပိ ဟောတိ။ ကယနံ၊ ကီယတေ ဧတာယာတိ ဝါ ကေဏိ=ကယော သာတိ ဒုက္ခံ တနုံ ကရောတီတိ သာဏိ=တိရောကရဏီ၊ ဧဝမညေပိ။}

\sutta{912}{62}{ဂဟာဒီဟျဏိ။}
\vutti{ဂဟ-ဥပါဒါနေ၊ အရ-ဂမနေ၊ ဓရ-ဓာရဏေ၊ သရ-ဂတိယိံသာ စိန္တာသု၊ တရ-တရဏေ၊ ဧဝမာဒီဟိ အဏိပ္ပစ္စယော ဟောတိ။ ဂဏှာတီတိ ဂဟဏိ=အသိတာဒိပါစကော အဂ္ဂိ။ အရီယတိ ဂမီယတီတိ အရဏိ=အဂ္ဂိမန္ထနကဋ္ဌံ ။ ဓာရေတီတိ ဓရဏိ=မဟီ။ သရီယတိ ဂမီယတီတိ ဘရဏိ=မဂ္ဂေါ။ တရန္တျနေနာတိ တရဏိ=နာဝါ၊ သူရိယော စ။}

\sutta{913}{63}{ရီဝီဘာဟိ ဏု။}
\vutti{ရီ-ပဿဝနေ၊ ဝီ ဝါ-ဂမနေ၊ ဘာ-ဒိတ္တိယံ၊ ဧတေဟိဏု ဟောတိ။ ရီယတိ ပဿဝတီတိ ရေဏု=ရဇော။ ဝေတိ ပဝတ္တတီတိ ဝေဏု=ဝေဠု။ ဘာတိ ဒိဗ္ဗတီတိ ဘာဏု=ရံသိ။}

\sutta{914}{64}{ခါဏွာ ဒယော။}
\vutti{ခါဏုအာဒယော ဏုအန္တာ နိပစ္စန္တေ။ ခဏ ခန-အဝဒါရဏေ၊ ဏဿ အာ၊ ခညတိ အဝဒါရီယတီတိ ခါဏု=ဆိန္နသာခေါ ရုက္ခော။ ဇန-ဇနနေ၊ နဿ ဝါ အာတ္တံ၊ ဇာယတိ ဂမန မနေနာတိ ဇာဏု၊ ဇဏ္ဏု=ဇင်္ဃောရူနံ သန္ဓိ။ ဟရ-ဟရဏေ၊ ဧက၊ ဟရီယတီတိ ဟရေဏု=ဂန္ဓဒဗ္ဗ၊ ဧဝမညေပိ။}

\sutta{915}{65}{ကွာဒိတော ဏော။}
\vutti{ကု-သဒ္ဒေ၊ သု-သဝနေ၊ ဒု-ဂမနေ၊ ဝရ-ဝရဏေ၊ ကရ-ကရဏေ၊ ပဏ-ဗျဝဟာရထုတိသု၊ တာ-ပါလနေ၊ လီ-နိလီယနေ၊ ဧဝမာဒီဟိ ဏော ဟောတိ။ ကဝတိ နဒတိ ဧတ္ထာတိ ကောဏော=အဿိ၊ ဝီဏာဒိဝါဒနဒဏ္ဍော စ။ သုဏောတီတိ သောဏော=သုနခေါ၊ နရော စ။ ဒဝတိ ပဝတ္တတီတိ ဒေါဏော=ပရိမာဏဝိသေသော။ ဝိရူပတ္တ ဝါရေတီတိ ဝဏ္ဏော=နီလာဒိ။ သဝနံ ကရောတီတိ ကဏ္ဏော=သဝနံ။ ပဏီယတိ ဝေါဟရီယတီတိ ပဏ္ဏော=ပလာသော။ လာယတီတိ တာဏံ=ရက္ခာ။ နိလီယန္တိ ဧတ္ထာတိ လေဏံ=နိလီယနဋ္ဌာနံ။}

\sutta{916}{66}{သုဝီဟိ ဏက။}
\vutti{သု-သဝနေ ၊ ဝီ-တန္တသန္တာနေ၊ ဧတေဟိ ဏက ဟောတိ။ သုဏောတီတိ သုဏော=သုနခေါ။ ဝီယတီတိ ဝီဏာ=ဝလ္လကီ။}

\sutta{917}{67}{တိဏာဒယော။}
\vutti{တိဏအာဒယော ဏကန္တာ နိပစ္စန္တေ။ တိဇ-တေဇနေ၊ ဇလောပေါ၊ တေဇေတိ ဧတေနာတိ တိဏံ=ဗီရဏာဒိ။ လီ လိဟ သာဒ က္လေဒါနံ လော လဝါ၊ လီယတိ ရသတော သဗ္ဗတ္ထ အလ္လီယတီတိ လောဏံ လဝဏံ၊ လေဟီယတီတိ လောဏံ လဝဏံ၊ သာဒီယတီတိ လောဏံ လဝဏံ၊ က္လေဒယတီတိ လောဏံ လဝဏံ၊ ဂမိဿ ဩ၊ ဂစ္ဆတီတိ ဂေါဏော=ဂေါ။ ဟရ-ဟရဏေ၊ ဏဿဉ္ဇိ၊ ဟရီယတီတိ ဟရိဏော=မိဂေါ။ ဤရ-ကမ္ပနေ၊ ရဿတ္တံ၊ ဏဿ ဉိစ၊ အတ္တနော လူခဘာဝေ သမ္ပတ္တေ ဤရတိ ကမ္ပတီတိ ဣရိဏံ=ဦသရံ။ ထု-အဘိတ္ထဝေ၊ ဒီဃော၊ အဘိတ္ထဝီယတီတိ ထူဏံ=နဂရံ၊ ထူဏော=ဃရတ္ထမ္ဘော၊ ဧဝမညေပိ။}

\sutta{918}{68}{ရဝဏ ဝရဏ ပူရဏာ ဒယော။}
\vutti{ရဝဏ ဝရဏ ပူရဏာ ဒယော အဏပ္ပစ္စယေန သိဒ္ဓါ။ ရဝတီတိ ရဝဏော=ကောကိလော။ ဝါရေတီတိ ဝရဏော=ပါကာရော။ ပူရီယတေ အနေနာတိ ပူရဏော=ပရိပူရီ။}

\begin{jieshu}
ဣတိ ဋဝဂ္ဂပစ္စယဝိဓာနံ။
\end{jieshu}


\sutta{919}{69}{ပါဝသာ အတိ။}
\vutti{ပါ-ရက္ခဏေ၊ ဝသ-နိဝါသေ၊ ဧတေတိ အတိ ဟောတိ။ ပုဗ္ဗသရလောပေါ၊ ပါတိ ရက္ခတီတိ ပတိ=သာမီ၊ ဝသန္တိ ဧတ္ထာတိ ဝသတိ=ဂေဟံ။}

\sutta{920}{70}{ဓာ ဟိ သိ တန ဇန ဇရ ဂမ သစာ တု။}
\vutti{ဓာ-ဓာရဏေ ၊ ဟိ-ဂတိယံ၊ တန-ဝိတ္ထာရေ၊ ဇန-ဇနနေ၊ ဇရ-ဝယော-ဟာနိယံ၊ အမ ဂမ-ဂမနေ၊ သစ-သမဝါယေ၊ ဧတေဟိ တု ဟောတိ။ ဓာရေတီတိ ဓာတု=ဂေရုကာဒိ။ ဟိနောတိ ပဝတ္တာတိ ဖလံ ဧတေနာတိ ဟေတု=ကာရဏံ၊ သေဝီယတိ ဇနေဟီတိ သေတု=ဗန္ဓတိ (ပဒ္ဓတိ)။ တညတေတိ တန္တု=သုတ္တံ။ ဇနီယတေ ကမ္မကိလေသေဟီတိ ဇန္တု၊ ဇာယတိ ကမ္မကိလေဟီတိ ဝါ ဇန္တု=သတ္တော။ ဇီရတီတိ ဇတ္တု=အံသသန္ဓိ။ ဂစ္ဆတီတိ ဂန္တု=ဂမိကော၊ သစတိ သမေတီတိ သတ္တု=ယဝါဒိစုဏ္ဏံ။}

\sutta{921}{71}{အရိဿုဋ စ။}
\vutti{အရ-ဂမနေတီမသ္မာ တု ဟောတိ၊ အရိဿ ဥဋ အာဒေသောစ။ အရတိ ပဝတ္တတီတိ ဥတု=ဟေမန္တာဒိ၊}

\sutta{922}{72}{ပိတွာဒယော။}
\vutti{ပိတုအာဒယော သဒ္ဒါ တုအန္တာ နိပစ္စန္တေ။ ပါ-ရက္ခဏေ၊ အာဿ ဣတ္တံ၊ ပါတိ ရက္ခတီတိ ပိတာ=ဇနကော။ ပါတိဿေဝါဒိဿ မော၊ ပါယေတီတိ မာတာ=ဇနနီ။ ဘာ-ဒိတ္တိယံ၊ ဘာတီတိ ဘာတာ=သောဒရိယော။ ဓာ-ဓာရဏေ၊ အာဿ ဤတ္တံ၊ ဓာရီယတီတိ ဓီတာ=ပုတ္တီ။ ဒုဟ-ပပူရဏေ၊ ဩတ္တာဘာဝေါ၊ တုဿ ဉ္ဆိစ။ ဒုဟတိ ပသဝေ ပပူရေတီတိ ဒုဟိတာ=ပုတ္တီ။ ဇန-ဇနနေ၊ အဿ အာတ္တံ၊ မာ စာန္တာဒေသော။ ပပုတ္တေ ဇနေတီတိ ဇာမာတာ=ဒုဟိတုပတိ။ နဟ-ဗန္ဓနေ၊ နဟီယတိ ဗန္ဓီယတိ ပေမေနာတိ နတ္တာ-ပပုတ္တော။ ဟု-ဟဝနေ၊ ဟဝတိ ပူဇေတီတိ ဟောတာ=ယညကော။ ပူ-ပဝနေ၊ ပုနာတိ အာယတိံ ဘဝံ ပဝိတ္တံ ကရောတီတိ ပေါတာ=သောယေဝ။}

\sutta{923}{73}{ဇန ကရာ ရတု။}
\vutti{ဇန-ဇနနေ၊ ကရ-ကရဏေ၊ ဧတေဟိ ရတု ဟောတိ၊ ရကာရော အန္တသရာဒိလောပတ္ထော။ ဇာယတီတိ ဇတု=လာခါ။ ကရီယတီတိ ကတု=သယူပေါ ယညော။}

\sutta{924}{74}{သကာ ဥန္တော။}
\vutti{သက-သတ္တိယမိစ္စသ္မာ ဥန္တော ဟောတိ။ သက္ကောတီတိ သကုန္တော=ပက္ခီ။}

\sutta{925}{75}{ကပါ ဩတော။}
\vutti{ကပ-အစ္ဆာဒနေ ဣစ္စသ္မာ ဩတော ဟောတိ။ ကပတီတိ ကပေါတော=ပါရေဝတော။ ဋော တဿ ဝါ ဟောတိ ကပေါဋော=သောယေဝ။}

\sutta{926}{76}{ဝသာဒီဟျန္တော။}
\vutti{ဝသ-နိဝါသေ၊ ရုဟ-ဇနနေ၊ ဘဒ္ဒ-ကလျာဏေ၊ နန္ဒ-သမိဒ္ဓိယံ၊ ဇီဝ-ပါဏဓာရဏေ၊ ဧဝမာဒီတိ အန္တော ဟောဟိ။ ဝသန္တိ ဧတသ္မိံ ကာလေ ကီဠာပသုတာတိ ဝသန္တော=ဥတု။ ရုဟတိ ဇာယတီတိ ရုဟန္တော=ရုက္ခော၊ ဧဝံနာမကော မိဂရာဇာ စ။ ဘဒ္ဒိဿ သံယောဂါဒိလောပေါ၊ ဘဇတိ ကလျာဏဓမ္မန္တိ ဘဒန္တော=ပဗ္ဗဇိတော။ နန္ဒတိ ဧတာယာတိ နန္ဒန္တီ=သခီ၊ နဒါဒိပါဌာ ဝီ၊ ဧဝမုပရိ စ။ ဇီဝန္တိ ဧတာ- ယာတိ ဇီဝန္တီ=ဩသဓိ။ သဝတီတိ သဝန္တီ=နဒီ။ ရောဒါပေတီတိ ရောဒန္တီ=ဩသဓိ။ အဝတိ ရက္ခတီတိ အဝန္တီ=ဇနပဒေါ။}

\sutta{927}{77}{ဟိသီနံ မုက စ။}
\vutti{ဟိ-ဂတိယ၊ သီ-သယေ၊ ဧတေဟိ အန္တော ဟောတိ မုက စ။ ကကာရော အန္တာဝယဝတ္ထော၊ ဟိမံ ဟိနောတိ ပဝတ္တတိ ဧတသ္မိန္တိ ဟေမန္တော=ဥတု၊ သယန္တိ ဧတ္ထ ဦကာ ကုသုမာဒယော စာတိ သီမန္တော=ကေသမဂ္ဂေါ။}

\sutta{928}{78}{ဟရ ရုဟ ကုလာ ဣတော။}
\vutti{ဟရ-ဟရဏေ၊ ရုဟ-ဇနနေ၊ ကုလ-ပတ္ထာရေ၊ ဧတေတိ ဣတော ဟောတိ။ အတ္တနော သိနေဟံ ဟရတီတိ ဟရိတော=ဝဏ္ဏဝိသေသော။ ရုဟတီတိ ရောဟိတော=မစ္ဆဝိသေသော။ ရုဟတိ သရီရေ ဗျာပနဝသေနာတိ ရောဟိတံ၊ ရဿ လတ္တေ လောဟိတံ=ရုဓိရံ။ အတ္တနော ဂုဏံ ကုလတိ ပတ္ထရတီတိ ကောလိတော=ဒုတိယဂ္ဂသာဝကော၊ ဧဝံ နာမကော မရု စ။}

\sutta{929}{79}{ဘရာဒီဟျတော။}
\vutti{ဘရ ရံဇ ယဇ ပစ ဧဝမာဒီဟိ အတော ဟောတိ။ ဘရတီတိ ဘရတော=နဋော။ နိဂ္ဂဟီတလောပေါ၊ ရဉ္ဇန္တိ ဧတ္ထာဟိ ရဇတံ=သဇ္ဈံ။ ယဇိတဗ္ဗောတိ ယဇတော=အဂ္ဂိ။ ပစတီတိ ပစတော=သူပကာရော။}

\sutta{930}{80}{ကိရာ ဒီဟျာတက။}
\vutti{ကိရ-ဝိကိရဏေ၊ အလ-ဗန္ဓနေ၊ စိလ-ဝသနေ၊ ဧဝမာဒီဟိ အာတက ဟောတိ၊ ကိရတီတိ ကိရာတော=သဝရော၊ ရဿ လက္ကေ ကိ- လာတော=သောဣ ။ အလတီတိ အလာတံ=ဥမ္မုကံ။ စိလတီတိ စိလာတော=မလစ္ဆဇာတိ။}

\sutta{931}{81}{အမာဒီဟျတ္တော။}
\vutti{အမ မာ ဝရ ကလာဒီဟိ အတ္တော ဟောတိ။ အမတိ ကာလန္တရံ ပဝတ္တတီတိ အမတ္တံ=ဘာဇနံ။ ပုဗ္ဗသရလောပေါ၊ မာနံ မတ္တံ=ပမာဏံ ပရိစ္ဆေဒေါ စ။ ဝရန္တုနေနာတိ ဝရတ္တာ=ယောတ္တံ။ ကလတိ ပရိစ္ဆိန္ဒတီတိ ကလတ္တံ=ဘရိယာ။}

\sutta{932}{82}{ဝါဒီဟိ တော။}
\vutti{ဝါ-ဂမနေ၊ တာ-ပါလနေ၊ တန-ဝိတ္ထာရေ၊ ဒမ-ဥပသမေ၊ အ-မဂမနေ၊ သိ-သေဝါယံ၊ သု-သဝနေ၊ ပူ ပဝနေ၊ ဂုပ-ဂေါဝနေ၊ ယုဇ-သံယမေ၊ ဂဟ-ဥပါဒါနေ၊ အတ-သာတစ္စဂမာန၊ ခိပ-ပေရဏေ၊ ဧဝမာဒီဟိ တော ဟောတိ။ ဝါယတီတိ ဝါတော=ဝါယု။ တာယတီတိ တာတော=ပိတာ။ တနုတေတိ တန္တံ=တန္တဝေါ။ ဒမတီတိ အန္တော=ဒသနော။ အမတိ ယာတီတိ အန္တော=ဩသာနံ၊ ကောဋ္ဌာသသမီပါဝယဝါ စ။ သေဝီယတီတိ သေတော=ဓဝလော။ သုဏန္တုနေနာတိ သောတံ=သဝနံ။ သဝတီတိ သောတော=ဇလပ္ပဝါဟော။ ပုနီယတီတိ ပေါတော=ဗာလော။ ဂေါပီယတီတိ ဂေါတ္ထံ=ကလာဒိ။ ယောဇန္တျနေနာတိ ယောတ္တံ=ရဇ္ဇု။ မမာယန္တေဟိ ဂယှတီတိ ဂတ္တံ=သရီရံ။ အဗာဓံ နိရန္တရံ အတတိ ပဝတ္တတီတိ အတ္တာ=မနာဒိ။ ခိပီယတိ ဧတ္ထာတိ ခေတ္တံ=ကေဒါရံ။}

\sutta{933}{83}{ဃရာဒီဟိ တက။}
\vutti{ဂရ ဃရ-သေစနေ၊ သိ-သေဝါယံ၊ ဒူ-ပရိတာပနေ၊ မိဒ-သိနေဟေ၊ စိတ-သဉ္စေတနေ၊ ပုသ-ပေါသနေ၊ ဝိဒ-လာဘေ၊ ဧဝမာဒီဟိ တက ၄ ဟောတိ၊ ဃရတိ သိဉ္စတီတိ ဃတံ=သပ္ပိ။ သေဝီယတီတိ သိတော=သေတော။ ဒုဗ္ဗစတ္တာ ဒူယတိ ပရိတာပေတီတိ ဒူတော=ပေသရကာရော။ မိဇ္ဇတိ သိနေဟတီတိ မိတ္တော=သုဟဒယော။ စိန္တေတီတိ စိတ္တံ=ဝိညာဏံ စိတ္တကမ္မာဒိ စ။ ပေါသီယတီတိ ပုတ္တော=အတ္တဇော။ ဝိန္ဒန္တိ ပီတိ မနေနာတိ ဝိတ္တံ=ဓနံ။ ဝရ-ဝရဏသမ္ဘတ္တီသု၊ ဝရဏံ ဝတ္တံ=ဗြဟ္မစရိယာဒိ။}

\sutta{934}{84}{နေတ္တာဒယော။}
\vutti{နေတ္တအာဒယော တက ပရာ နိပစ္စန္တေ။ နီ-ပါပနေ၊ ဧတ္တံ၊ တုက စ နိပါတနာ။ နယတိ ပါပေတီတိ နေတ္တံ=နယနံ၊ နေတာ စ။ ကရ-ကရ-ဏေ၊ အဿု၊ ကရဏံ ကုတ္တံ=ကိရိယာ။ ကမိဿ အဿု၊ ကမတိ ယာတီတိ ကုန္တော=အာဝုဓဝိသေသော။ ရမ-ကီဠာယံ၊ သုပုဗ္ဗော၊ သုဿနိစ္စံ ဒီဃော။ သုဋ္ဌု ရမဏံ၊ သုဋ္ဌု ရမတီတိ ဝါ သူရတော=သုခသံဝါသော။ မိဟိဿဣဿု၊ မိဟတိ သိဉ္စတီတိ မုတ္တံ=ပဿာဝေါ။ ပါလ=ရက္ခဏေ အာဿ ရဿတ္တံ၊ ဉိစ။ ပါလီယတီတိ ပလိတံ=ကေသလောမာနံ ဇရာယ ကတံ သေတတ္တံ၊ သဒ္ဓါဒိတ္တာ အကာရေ တံ ယဿ အတ္ထိ သော ပလိတော=ပုမာ၊ ပလိတာ=ဣတ္ထီ။ မှိဿ သိ၊ မိဟိ စ၊ မှယနံ သိတံ=မန္ဒဟသိတံ၊ မှယနံ မိဟိတံ=တဒေဝ၊ ကုသ-အက္ကောသေ တဿ ဤဉ၊ ကုသီယတိ အက္ကောသီယတီတိ ကုသီတော=အလသော၊ သိ=ဗန္ဓနေ၊ ဒီဃော၊ သေန္တိ ဗန္ဓန္တိ ဃရာဝါသံ ဧတာယာတိ သီတာ=နင်္ဂလလေခါ၊ ဧဝမညေပိ။}

\sutta{935}{85}{သမာဒီဟျထော။}
\vutti{သမ-ဥပသမေ ၊ ဒရ-ဒရဏေ၊ ဒမ-ဥပသမေ၊ ကိလမ က္လမ-ဂေလညေ၊ သပ-အက္ကောသေ၊ ဝသ-နိဝါသေ၊ အာပုဗ္ဗော၊ ဧဝမာဒီဟိ အထော ဟောတိ။ သမေတီတိ သမထော=သမာဓိ။ ဒရဏံ ဒရထော=ပီဠာ။ ဒမနံ ဒမထော=ဒမော။ ကိလမနံ ကိလမထော=ပရိဿမော။ သပနံ သပထော=သစ္စကရဏံ။ အာဝသန္တိ ဧတ္ထာတိ အာဝသထော=ဃရံ။}

\sutta{936}{86}{ဥပဝသာ ဝဿောဋ စ။}
\vutti{ဥပပုဗ္ဗာ ဝသတိသ္မာ အထော ဟောတိ၊ ဝဿ ဩဋ စာဒေသော။ ဥပဝသန္တိ ဧတ္ထာတိ ဥပေါသထော=တိထိဝိသေသော၊ နဝမဟတ္ထိ ကုလဉ္စ။}

\sutta{937}{87}{ရမာ ထက။}
\vutti{ရမတိသ္မာ ထက ဟောတိ၊ ကာနုဗန္ဓကရဏသာမတ္ထိယာ အတ ကာရာဒေါပိ မလောပေါ။ ရမန္တိ ကီဠန္တိ ဧတေနာတိ ရထော=သန္ဒနော။}

\sutta{938}{88}{တိတ္ထာဒယော။}
\vutti{တိတ္ထအာဒယော ထကပရာ နိပစ္စန္တေ။ တရ-တရဏေ၊ အဿ ဣတ္တံ၊ ပရရူပါဒိ၊ တရန္တျနေနာတိ တိတ္ထံ=နဇ္ဇာဒိံ ယေနာဝတရန္တိ တံ။ သိစ-ရက္ခဏေ၊ သေစတီတိ သိတ္တံ=မဓုစ္ဆိဋ္ဌံ။ ဟသ-ဟသနေ၊ ဟသန္တျနေနာတိ ဟတ္ထော=ကရော၊ နက္ခတ္တာဉ္စ။ ဂါယတီတိ ဂါထာ=ပဇ္ဇဝိသေသော။ အရန္တိ ပဝတ္တန္တျနေနာတိ အတ္ထော=ဓနံ။ ရောဂံ တုဒတိ ပီဠေတီတိ တုတ္ထံ=ဩသဓံ။ ယု-မိဿနေ၊ ဒီဃော၊ ယဝတီတိ ယူ- ထော=သဇာတိကာနံ တိရစ္ဆာနာနံသမူဟော။ ဂုပ-ဂေါပနေ၊ ဒီဃော၊ ပလောပေါ၊ ပဋိကူလတ္တာ ဂေါပီယတီတိ ဂူထော=ဝစ္စံ၊ ဧဝမညေပိ။}

\sutta{939}{89}{ဝသ မသ ကုသာ ထု။}
\vutti{ဝသ-နိဝါသေ၊ မသ-အာမသနေ၊ ကုသ-အက္ကောသေ၊ ဧတေဟိ ထု ဟောတိ။ ဝသန္တိ ဧတ္ထာတိ ဝတ္ထု=ပဒတ္ထော။ ဒဓိံ အာမသတီတိ မတ္ထု=ဒဓိမဏ္ဍော။ ကုသီယတိ အက္ကောသီယတိ ဘေရ ဝနာဒတ္တာတိ ကောတ္ထု=သိဂါလော။}

\sutta{940}{90}{သက ဝသာ ထိ။}
\vutti{သက-သတ္တိယံ၊ ဝသ-နိဝါသေ၊ ဧတေဟိ ထိ ဟောတိ။ သက္ကောတိ ဂန္တုမနေနာတိ သတ္ထိ=ဦရု။ ဝသီယတိ အစ္ဆာဒီယတီတိ ဝတ္ထိ=နာဘိယာ အဓော။}

\sutta{941}{91}{ဝီတော ထိက။}
\vutti{ဝီ ဝါ-ဂမနေတီမသ္မာ ထိက ဟောတိ။ ဝီယန္တိ ဂစ္ဆန္တိ ဧတာယာတိ ဝီထိ=အာဝလိ။}

\sutta{942}{92}{သာရိသ္မာ ရတိ။}
\vutti{သာရိသ္မာ ဏျန္တာ ရထိ ဟောတိ။ သာရေတီတိ သာရထိ=ရထ-ဝါဟော။}

\sutta{943}{93}{တာ-တာ ဣထိ။}
\vutti{တာ-ပါလနေ၊ အတ-သာတစ္စဂမနေ၊ ဧတေဟိ ဣထိ ဟောတိ။ တာယတိ ပါလေတီတိ တိထိ=ပဋိပဒါဒိ၊ အတတိ ဂစ္ဆတီတိ အတိထိ=အဗ္ဘာဂတော။}

\sutta{944}{94}{ဣသာ ထီ။}
\vutti{ဣသတိသ္မာ ထီ ဟောတိ။ ဣစ္ဆတိ ဣစ္ဆီယတီတိ ဝါ ဣတ္ထီ=နာရီ။}

\sutta{945}{95}{ရုဒ ခုဒ မုဒ မဒ ဆိဒ သူဒ သပ ကမာ ဒက။}
\vutti{ဧတေဟိ ဒက ဟောတိ။ ရုဒတီတိ ရုဒ္ဒေါ=ဥမာပတိ။ ရဿ လတ္တေ လုဒ္ဒေါ=နေသာဒေါ။ ခုဒတိ အသဟတီတိ ခုဒ္ဒေါ=နီစော။ မောဒန္တိ ဧတာယာတိ မုဒ္ဒါ=သက္ခရမင်္ဂုလိယံ။ မဇ္ဇန္တိ အသ္မိန္တိ မဒ္ဒေါ=ဇနပဒေါ။ ဆိဇ္ဇတီတိ ဆိဒ္ဒံ=ရန္ဓံ။ ဦဿ ရဿတ္တံ၊ သူဒတိ သာမိကေဟိ ဘတ္ထိံ ပက္ခရတီတိ သုဒ္ဒေါ=ဝသလော။ သပန္တုနေနာတိ သဒ္ဒေါ=သောတဝိသယော။ ကာမီယတီတိ ကန္ဒော=မူလဝိသေသော။}

\sutta{946}{96}{ကုန္ဒာဒယော။}
\vutti{ကုန္ဒအာဒယော ဒကအန္တာ နိပစ္စန္တေ။ ကမိဿ အဿု၊ ကာမီယတီတိ ကုန္ဒော=ပုပ္ဖဝိဓသသော၊ မဏိဿ မန၊ မညတေတိ မန္ဒော=ဇဠော။ ဝုဏာတိဿ ဗုန၊ ဝုဏီယတိ သံဝရီယတီတိ ဗုန္ဒော=မူလပ္ပဒေသော။ နိန္ဒ-ဂရဟာယံ၊ နလောပေါ၊ နိန္ဒီယတီတိ နိဒ္ဒါ=သောပ္ပံ။ ဥန္ဒ-ကိလေဒနေ၊ နလောပေါ၊ ဥန္ဒတိ ကိလေဒတီတိ ဥဒ္ဒေါ=ဇလဗိဠာလော။ သံပုဗ္ဗဿ ဥန္ဒိဿ စ၊ သမ္မာ ဥန္ဒတိ ကိလေဒတီတိ သမုဒ္ဒေါ=သာဂရော။ ပုလ-မဟတ္တဟိံ သာဉာဏေသု၊ ဣမုဉ၊ ပုလတိ ဟိံသတီတိ ပုလိန္ဒော=သဝရော။ ဧဝံ-မညေပိ။}

\sutta{947}{97}{ဒဒါ ဒု။}
\vutti{ဒဒ-ဒါနေတီမသ္မာ ဒု ဟောတိ။ ဒုက္ခံ ဒဒါတီဟိ ဒဒ္ဒု=ကုဋ္ဌဝိသေသော။}

\sutta{948}{98}{ခနာန ဒမ ရမာ ဓော။}
\vutti{ခန ခဏ-အဝဒါရဏေ၊ အန-ပါဏနေ၊ ဒမ-ဥပသမေ၊ ရမ-ကီဠာယံ၊ ဧတေဟိ ဓော ဟောတိ။ ဉာဏေန ဓညတေတိ ခန္ဓော=ရာသိ။ အနတိ ဇီဝတိ ဧတေနာတိ အန္ဓော=အစက္ခုကော။ ဒမေတဗ္ဗောတိ ဒန္ဓော=ဇဠော၊ ရမန္တိ ဧတ္ထ သပ္ပာဒယောတိ ရန္ဓံ=ဆိဒ္ဒံ။}

\sutta{949}{99}{မုဒ္ဓါ ဒယော။}
\vutti{မုဒ္ဓအာဒယော ဓန္တာ နိပစ္စန္တေ။ မုဒ-တောသေ၊ ဩတ္တာဘာဝေါ၊ မောဒန္တိ ဧတ္ထ ဦကာတိ မုဒ္ဓါ=မတ္ထကော။ အရ-ဂမနေ၊ အရန္တိ ယန္တိ ဧတ္ထာတိ အဒ္ဓါ=မဂ္ဂေါ၊ ကာလော စ၊ အဒ္ဓံ=ဥပဒ္ဓံ။ ဂိဓ-အဘိ ကင်္ခါယံ၊ ဣဿ အတ္တံ၊ ဂေဓတီတိ ဂဒ္ဓေါ=ဂိဇ္ဈော။ ဝိဓ-ဝေဓနေ၊ ဧတ္တာဘာဝေါ၊ ပရိဝဇ္ဈတီတိ ဝိဒ္ဓံ=ဝိမလံ၊ ဧဝမညေပိ။}

\sutta{950}{100}{သီတော ဓုက။}
\vutti{သေတိသ္မာ ဓုက ဟောတိ။ သယန္တိ ဧတာယာတိ သီဓု=သုရာဝိသေသော။}

\sutta{951}{101}{ဝရာရကရတရဒရယမအဇ္ဇမိထသကာ ကုနော။}
\vutti{ဝရ-ဝရဏသမ္ဘတ္တီသု၊ အရ-ဂမနေ၊ ကရ-ကရဏေ၊ တရ-တရ-ဏေ ဒရ-ဝိဒါရဏေ၊ ယမ-ဥပရမေ၊ အဇ္ဇ သဇ္ဇ-အဇ္ဇနေ၊ မိထ-သင်္ဂမေ၊ သက-သတ္တိယံ၊ ဧတေဟိ ကုနော ဟောတိ။ “ရာ နဿ ဏော ”တိ (၅-၁၇၁) နဿ ဏတ္တံ၊ ဝါရေတီတိ ဝရုဏော=ဧဝံနာမကော ဣသိ၊ ဒေဝရာဇာ၊ ပါဒပေါ စ။ အရတိ ဂစ္ဆတီတိ အရုဏော=သူရိယော၊ တဿ သာရထိ စ။ ပရဒုက္ခေ သတိ သာဓူနံ ဟဒယကမ္မနံ ကရောတီတိ ကရုဏာ=ဒယာ။ ဗာလဘာဝံ အတရီတိ တရုဏော=ယုဝါ။ “ဏိဏာပိန ”န္တိ (၅.၁၆၀) ယောဂဝိဘာဂါ ဏိလောပေါ၊ ဝိဒါရေတီတိ ဒါရုဏော=ကက္ခဠော။ ယမေတိ ပါဝံ နာသေတီတိ ယမုနာ=ဧကာ မဟာနဒီ။ အဇ္ဇတိ ဓနသဉ္စယံ ကရောတီတိ အဇ္ဇုနော=ရာဇာ၊ ရုက္ခဝိသေသော စ။ မိထော သင်္ဂမော မိထုနံ=ပုမိတ္ထိယုဂဠံ။ သက္ကောတီတိ သကုနော=ပက္ခီ။ နဒါဒိပါဌာ ဝီမှိ=သကုနီ။ “တထနရာနံဋဋ္ဌဏလာ ”တိ (၁-၂၇) ဝါ ဏတ္တေ=သကုဏော၊ သကုဏိ။}

\sutta{952}{102}{အဇာ ဣနော။}
\vutti{အဇ ဝဇ-ဂမနေ တီမသ္မာ ဣနော ဟောတိ။ အဇတိ ဝိက္ကယံ ယာတီတိ အဇိနံ=စမ္မံ။}

\sutta{953}{103}{ဝိပိနာဒယော။}
\vutti{ဝိပိနအာဒယော ဣနန္တာ နိပစ္စန္တေ။ ဝပ-ဗီဇနိက္ခေပေ၊ အဿ ဣတ္တံ၊ ဝပန္တိ ဧတ္ထာတိ ဝိပိနံ=ဝနံ။ သုပ-သယေ၊ သုပန္တိ ဧတေနာတိ သုပိနံ=နိဒ္ဒါ၊ သုပန္တေန ဒိဋ္ဌဉ္စ။ တုဒ-ဗျထနေ၊ ဒဿ ဟော၊ တုဒတိ သတ္တေ ပီဠေတီတိ တုဟိနံ=ဟိမံ။ ကပ္ပ-သာမတ္ထိယေ၊ ကပ္ပတိ ရိပဝေါ ဝိဇေတုံ သမတ္ထေတီတိ ကပ္ပိနော=ရာဇာ။ ကမ-ပဒဝိက္ခေပေ၊ အဿ ဥတ္တံ၊ ကမန္တိ ဧတ္ထ မီနာဒယော ပဝိသန္တီတိ ကုမိနံ=မစ္ဆဗန္ဓနောပကရဏဝိသေသော။ ဒါ-ဒါနေ၊ ဒေန္တိ ဧတသ္မိန္တိ ဒိနံ=ဒိဝသော၊ ဧဝမညေပိ။}

\sutta{954}{104}{ကိရာ ကနော။}
\vutti{ကိရတိသ္မာ ကနော ဟောတိ။ နဿ ဏော၊ ကိရန္တိ ဝိကိရန္တီတိ ကိရဏာ=ရံသိယော။}

\sutta{955}{105}{ဒီ ဇိဣမီဟိ နက။}
\vutti{ဒီ-ခယေ၊ ဇိ-ဇယေ၊ ဣ-အဇ္ဈေနဂတီသု၊ မီ-ဟိံသာယံ၊ ဧတေဟိ နက ဟောတိ။ အဒေသိ ခယမဂမာသီတိ ဒီနော=နိဒ္ဓနော။ ပဉ္စမာရေ အဇိနီတိ ဇိနော=ဗုဒ္ဓေါ။ ဧသိ ဣဿရတ္တမဂမာသီတိ ဣနော=သာမီ။ မီယတေ ဟိံသီယတေတိ မီနော=မစ္ဆော။}

\sutta{956}{106}{သိဓာဝီဝါဟိ နော။}
\vutti{သိ-ဗန္ဓနေ၊ ဓာ-ဓာရဏေ၊ ဝီ ဝါ-ဂမနေ၊ ဧတေဟိ နော ဟောတိ။ သေတိ ဗန္ဓတီတိ သေနော=သသာဒနော သေနာ=စမူ။ ဓာရေတီတိ ဓာနာ=ဘဇ္ဇိတယဝေါ။ ဝေတိ ပဝတ္တတီတိ ဝေနော=ဟီနဇာတိ။ သတ္တေသု ဝါတိ ပဝတ္တတီတိ ဝါနံ=တဏှာ။}

\sutta{957}{107}{ဦနာဒယော။}
\vutti{ဦနအာဒယော နန္တာ နီပစ္စန္တေ။ ဦဟ-ဝိတက္ကေ၊ ဟလောပေါ၊ ဦဟနံ ဦနော=အပုဏ္ဏော။ ဟိ-ဂတိယံ၊ ဒီဃတ္တံ၊ ဟေသိ ဟီနတ္တမဂမီတိ ဟီနော=နိဟီနော။ စိ-စယေ၊ ဒီဃတ္တံ၊ စယန္တိ ဧတ္ထ ရတနာနီတိ စီနော=ဇနပဒေါ။ ဟနိဿ ဇဃော၊ ဟညတီတိ ဇဃနံ=ကဋိ။ ဌာဿ ထေ ဌာတိ ပဝတ္တတီတိ ထေနော=စောရော။ ဥန္ဒိဿ ဩဒေါ၊ ဥန္ဒီယတီတိ ဩဒနော=အန္နံ။ အန္နံ။ ရံဇိဿ နိဂ္ဂဟီတလောပေါ၊ အက စ၊ ရံဇတေ အနေနာတိ ရဇနံ=ရာဂေါ။ ရဉ္ဇန္တိ ဧတ္ထာတိ ရဇနီ=ရတ္တိ။ ပဒိဿ ဇုနုက ၊ ပဇ္ဇတိ ဂစ္ဆတီတိ ပဇ္ဇုန္နော=ဣန္ဒော၊ မေဃော စ။ ဂမိဿ ဂင၊ ဂစ္ဆန္တိ ဧတ္ထ ဝိဟင်္ဂါဒယောတိ ဂဂနံ=အန္တလိက္ခံ၊ ဧဝမညေပိ။}

\sutta{958}{108}{ဝီ ပတာ တနော။}
\vutti{ဝီ ဝါ-ဂမနေ၊ ပတ ပထ-ဂမနေ၊ ဧတေဟိ တနော ဟောတိ။ ဝေတိ ပဝတ္တတိ ဧတေနာတိ ဝေတနံ=ဘတိ။ ပတန္တိ ဧတ္ထာတိ ပတ္တနံ=နဂရံ။}

\sutta{959}{109}{ရမာ တနက။}
\vutti{ရမ ကီဠာယမိစသ္မာ တနက ဟောတိ။ “ဂမာဒိရာနံ လောပေါန္တဿာ ”တိ (၅.၁၀၉) မလောပေါ၊ ရမန္တိ ဧတ္ထာတိ ရတနံ=မဏိ အာဒီ၊ ဟတ္ထမတ္တဉ္စ။}

\sutta{960}{110}{၊ သူ ဘာဟိ နုက၊}
\vutti{သူ-ပသဝေ၊ ဘာ-ဒိတ္တိယံ၊ ဧတေဟိ နုက ဟောတိ။ ပသဝီယတီတိ သူနု=ပုတ္တော။ ဘာတိ ဒိဗ္ဗတီတိ ဘာနု=သူရိယော။}

\sutta{961}{111}{ဓာဿေ စ။}
\vutti{ဓာ-ဓာရဏေတီမသ္မာ နုက ဟောတိ၊ ဓာဿ ဧ စ။ ဓာရေတီတိ ဓေနု=ဂါဝီ။}

\sutta{962}{112}{ဝတ္တာ ဋာဝ ဓမာသေဟျနိ။}
\vutti{ဝတ္တ-ဝတ္တနေ၊ အဋ-ဂမနတ္ထော၊ အဝ-ရက္ခဏေ၊ ဓမ-သဒ္ဒေ၊ အသ-ခေပနေ၊ ဧတေဟိ အနိ ဟောတိ။ ဝတ္တန္တိ ဧတေနာတိ ဝတ္တနိ=တသရဒဏ္ဍံ။ ဝီမှိ ဝတ္တနီ=ပန္ထော။ အဋတေ ဂမ္မတေတိ အဋနိ= မဉ္စင်္ဂေါ ။ သတ္တေ အဝတိ ရက္ခတီတိ အဝနိ=မဟီ။ ဓမန္တိ ဧတေန ဝီဏာဒယောတိ ဓမနိ=သီရာ။ ဒဏ္ဍတ္ထာယ အသီယတေ ခိပီယတေတိ အသနိ=ကုလိသံ။}

\sutta{963}{113}{ယုတော နိ။}
\vutti{ယု-မိဿနေတီမသ္မာ နိ ဟောတိ။ ယဝန္တိ သတ္တာ အနေန ဧတီဘာဝံ ဂစ္ဆန္တီတိ ယောနိ=ဘဂံ၊ အဏ္ဍဇာဒိယောနိ စ။}

\begin{jieshu}
ဣတိ တဝဂ္ဂပစ္စယဝိဓာနံ။
\end{jieshu}


\sutta{964}{114}{စမာပ ပါ ဝပါ ပေါ။}
\vutti{စမ-အဒနေ၊ အပ-ပါပုဏနေ၊ ပါ-ရက္ခဏေ၊ ဝပ-ဗီဇနိက္ခေပေ၊ ဧတေဟိ ပေါ ဟောတိ။ စမန္တိ အဒန္တိ ဧတ္ထာတိ စမ္ပာ=နဂရံ၊ အပေသိ ဤသကမတ္တမဂမာသီတိ အပ္ပံ=အဗဟု။ အပါယံ ပါတိ ရက္ခတီတိ ပါပံ-ကိဗ္ဗိသံ။ ဝပန္တိ ဧတ္ထာတိ ဝပ္ပော=ကေဒါရံ။}

\sutta{965}{115}{ယု ထု ကူနံ ဒီဃော စ။}
\vutti{ယု-မိဿနေ၊ ထု-အဘိတ္ထဝေ၊ ကု-သဒ္ဒေ၊ ဧတေဟိ ပေါ ဟောတိ၊ ဧတေသံ ဒီဃော စ။ ဒီဃဝိဓာနသာမတ္ထိယာ ဩတ္တာဘာဝေါ။ ယဝန္တိ သဟ ဝတ္တန္တိ ဧတ္ထာတိ ယူပေါ=ယညယဋ္ဌိ၊ ပါသာဒေါ စ။ ထဝီယတီတိ ထူပေါ=စေတိယံ။ ကဝန္တိ နဒန္တိ ဧတ္ထာတိ ကူပေါ=ဥဒပါနော။}

\sutta{966}{116}{ခိပ သုပ နီသူ ပူဟိ ပက။}
\vutti{ခိပ-ပေရဏေ၊ သုပ-သယေ၊ နီ-နယေ၊ သူ-ပသဝေ၊ ပူ-ပဝနေ၊ ဧတေဟိ ပက ဟောတိ။ ခပတိ ခယံ ဂစ္ဆတီတိ ခိပ္ပံ=သီဃံ။ သုပန္တိ ဧတ္ထ သုနခါဒယောတိ သုပ္ပံ=ပပ္ဖောဋနံ။ နယန္တိ ဧတသ္မာ ဖလန္တိ နိပေါ=ရက္ခော။ သဝတိ ရုစိံ ဇနေတီတိ သူပေါ=ဗျဉ္ဇနဝိသေသော။ ပဝီယတိ မရိစဇီရကာဒီဟိ ပဝိတ္တံ ကရီယတီတိ ပူပံ=ခဇ္ဇကံ။}

\sutta{967}{117}{သိပ္ပာဒယော။}
\vutti{သိပ္ပအာဒယော ပကအန္တာ နိပစ္စန္တေ။ သပိဿ အဿဣတ္တံ၊ သပတိ အနေနာတိ သိပ္ပံ=ကလာ။ ဝပိဿ အဿိ၊ ဝိဇ္ဇံ ဝပတီတိ ဝိပ္ပော=ဗြာဟ္မဏော။ ဝဿ ဗော၊ ဝပတိ ဗဟိ နိက္ခမတိ ဟဒယင်္ဂတသောကေနာတိ ဗပ္ပံ=အဿု။ ဆုပ-သမ္ဖဿေ၊ ဥဿေ၊ ဆုပတိ အနေနာတိ ဆေပ္ပံ=နင်္ဂုဋ္ဌံ။ ရုပ=ရုပ္ပနေ၊ ပလောပ ဒီဃာ၊ ရုပ္ပတိ ဝိကာရမာပဇ္ဇတီတိ ရူပံ-ဘူတဘူတိကံ၊ ဧဝမညေပိ။}

\sutta{968}{118}{သာသာ အပေါ။}
\vutti{သာသ အနုသိဋ္ဌိယမိစ္စသ္မာ အပေါ ဟောတိ။ သာသီယန္တိ ဧတေနာတိ သာသပေါ=ဝီဟိဝိသေသော။}

\sutta{969}{119}{ဝိဋပါဒယော။}
\vutti{ဝိဋပအာဒယော အပန္တာ နိပစ္စန္တေ။ ဝဋ-ဝေဋ္ဌနေ၊ အဿ ဣတ္တံ၊ ဝဋတိ ဝေဋ္ဌတိ ဧတေနာတိဝိဋပေါ=ဂုမ္ဗဝိသေသော၊ ကုထ-ပူတိဘာဝေ၊ ထဿ ဏော၊ အကုထိ ပူတိဘာဝမဂမီတိ ကုဏပေါ=မတကော။ မဏ္ဍ=ဘူသနေ၊ မဏ္ဍေတိ ဇနံ၊ မဏ္ဍီယတိ ဇနေဟီတိ ဝါ မဏ္ဍပေါ=ဇနာလယော၊ ဧဝမညေပိ။}

\sutta{970}{120}{ဂုပါ ဖော။}
\vutti{ဂုပိသ္မာ ဖော ဟောတိ။ ဂေါပီယတီတိ ဂေါပ္ဖော=စရဏဂဏ္ဌိ။}

\sutta{971}{121}{ဂရသရာဒီဟိ ဗော။}
\vutti{ဂရသရာဒီဟိ ဗော ဟောတိ။ ဂရတိ အညေ အနေန ပီဠေတီတိ ဂဗ္ဗော=အဘိမာနော။ သရတိ ပဝတ္တတီတိ သဗ္ဗော=သကလော။ ဖလကာမေဟိ ဇနေဟိ အမီယတိ ဂမီယတီတိ အမ္ဗော=စူတော။ ပုတ္တေန အမီယတိ ဂမီယတီတိ အမ္ဗာ=မာတာ။}

\sutta{972}{122}{နိမ္ဗာဒယော။}
\vutti{နိမ္ဗအာဒယော ဗန္တာ နိပစ္စန္တေ။ နမိဿ အဿိ၊ နမတိ ဖလဘာရေနာတိ နိမ္ဗော=အရိဋ္ဌော။ ဝမိဿ ဝဿ ဗိတ္တံ။ ပိတ္တာဒယော ဝမတိ ဥဂ္ဂိရတီတိ ဗိမ္ဗံ=သရီရံ။ ကုသိဿ အမုက၊ တိတ္တေန ကုသီယတိ အက္ကောသီယတီတိ ကောသမ္ဗော=ရုက္ခော။ ကဒတိဿ အမုက၊ ကဒန္တိ ဧတေန ဒွါရာဒီနီတိ ကဒမ္ဗော=ရုက္ခော။ ကုဋိဿ ဥမုက၊ ဇနေဟိ ကောဋီယတိ ပဝတ္တီယတီတိ ကုဋုမ္ဗံ=စတုပ္ပဒေါ၊ ခေတ္တံ၊ ဃရံ၊ ကလတ္တံ၊ ဒါသာ စ။ ကဏ္ဍိဿ ကုဍု။ တဏ္ဍုလာဒယော အနေန ကဏ္ဍန္တိ ပရိစ္ဆိန္ဒန္တီတိ ကုဍုဗော=မာနံ၊ ဧဝမညေပိ။}

\sutta{973}{123}{ဒရာ ဗိ။}
\vutti{ဒရ ဝိဒါရဏေတီမသ္မာ ဗိ ဟောတိ။ ဩဒနာဒီနိ ဒါရေန္တိ ဇတာယာတိ ဒဗ္ဗိ=ကဋစ္ဆု၊ ဝီမှိ ဒဗ္ဗီ။}

\sutta{974}{124}{ကရ သရ သလ ကလ ဝလ္လ ဝသာ အဘော။}
\vutti{ကရ-ကရဏေ၊ သရ-ဂတိ ဟိံသာစိန္တာသု၊ သလ-ဂမနတ္ထော၊ ကလ-သင်္ချာနေ၊ ဝလ ဝလ္လ-သံဝရဏေ၊ ဝသ-နိဝါသေ၊ ဧတေဟိ အဘော ဟောတိ။ ကရောတီတိ ကရဘော=ဩဋ္ဌာ၊ ပါဏိပ္ပဒေသော စ။ သရတိ ဂစ္ဆတီတိ သရဘော=မိဂဝိသေသော။ သလတိ ဂစ္ဆတီတိ သလဘော=ပဋင်္ဂေါ ။ ကလီယတိ ပရိမီယတိ ဝယသာတိ ကလဘော=ဟတ္ထိပေါတကော။ တဒမိနာဒိပါဌာ ဠတ္တေ ကဠဘော=သောဣ။ ဝလ္လေတိ သံဝရဏံ ကရောတီတိ ဝလ္လဘော=ပိယော။ ဝသန္တျနေနာတိ ဝသဘော=ပုင်္ဂဝေါ။}

\sutta{975}{125}{ဂဒါ ရဘော။}
\vutti{ဂဒတိသ္မာ ရဘော ဟောတိ။ ဂဒတီတိ ဂဒြဘော=ခရော။}

\sutta{976}{126}{ဥသရာသာ ကဘော။}
\vutti{ဥသ-ဒါဟေ၊ ရာသ-သဒ္ဒေ၊ ဧတေဟိ ကဘော ဟောတိ။ ဥသတိ ပဋိပက္ခေ ဒဟတီတိ ဥသဘော=သေဋ္ဌော။ ရာသတိ နဒတီတိ ရာသဘော=ဂဒြဘော။}

\sutta{977}{127}{ဣတော ဘက။}
\vutti{ဣ ဣတိသ္မာ ဘက ဟောတိ။ ဧတိ ဂစ္ဆတီတိ ဣဘော=ဟတ္ထီ။}

\sutta{978}{128}{ဂရာဝါ ဘော။}
\vutti{ဂရ ဃရ-သေစနေ၊ အဝ-ရက္ခနေ၊ ဧတေဟိ ဘော ဟောတိ။ ဂရတိ ဗဟိ နိက္ခမနဝသေန သိဉ္စတီတိ ဂဗ္ဘော=ပသဝေါ၊ ဩဝရကော စ။ အဝတိ သတ္တေ ရက္ခတီတိ အဗ္ဘံ=မေဃော။}

\sutta{979}{129}{သောဗ္ဘာဒယော။}
\vutti{သောဗ္ဘအာဒယော ဘန္တာ နိပစ္စန္တေ။ သဒတိဿ အဿ ဩတ္တံ၊ သီဒန္တိ ဧတ္ထာတိ သောဗ္ဘံ=ဝိဝရံ၊ သောဗ္ဘော=ဇလာသယဝိသေသော။ ကမိဿ အဿု၊ ကာမီယတီတိ ကုမ္ဘော=ဒသမ္ဗဏမတ္တော၊ ဃဋော စ။ (ကေန ဇလေန ဥမ္ဘီယတိ ပူရီယတီတိ ဝါ ကုမ္ဘော=ဃဋော။) ကုသိဿ ဥမုက၊ ကုသတိ အဝှာယတီတိ ကုသုမ္ဘံ=မဟာရဇနံ။ ကုသုမ္ဘော=ကနကံ၊ ဧဝမညေပိ။}

\sutta{980}{130}{ဥသ ကုသ ပဒ သုခါ ကုမော။}
\vutti{ဥသာဒီဟိ ကုမော ဟောတိ။ ဥသတိ ဒဟတီတိ ဥသုမံ=ဥဏှံ။ ကုသတိ အဝှာယတီတိ သုကုမံ=ပုပ္ဖံ။ ပဇ္ဇတိ ဒေဝပူဇာဒိံ ယာတီတိ ပဒုမံ=ပင်္ကဇံ။ သုခယတီတိ သုခုမ=အဏု။}

\sutta{981}{131}{ဝဋုမာဒယော။}
\vutti{ဝဋုမအာဒယော ကုမန္တာ နိပစ္စန္ထေ။ ဝဇိဿ-န္တဿ ဋော၊ ဝဇန္တိ ဧတ္ထာတိ ဝဋုမံ=ပထော။ သိလိသဿ လိဿေ၊ သိလိဿတီတိ သိလေသုပံ=သေမှံ။ ကမိဿ ကုင်္ကာဒေသော၊ ကာမီယတီတိ ကုင်္ကမံ=ကသ္မီရဇံ၊ ဧဝမညေပိ။}

\sutta{982}{132}{ဂုဓာ ဥမော။}
\vutti{ဂုဓ ပရိဝေဋ္ဌနေတီမသ္မာ ဥမော ဟောတိ။ ဂုဓတိ ပရိဝေဋ္ဌတီတိ ဂေါဓုမော=ဓညဝိသေသော။}

\sutta{983}{133}{ပဌ စရာ အမိမာ။}
\vutti{ပဌတိစရတိသ္မာ အမ ဣမာ ဟောန္တိ ယထာက္ကမံ။ ပဋ္ဌီယတိ ဥစ္စာရီယတိ ဥတ္တမဘာဝေနာတိ ပဌမံ=သေဋ္ဌံ။ စရတိ ဟီနတ္တံ ယာတီတိ စရိမံ=ပစ္ဆိမံ။}

\sutta{984}{134}{ဟိဓူဟိ မက။}
\vutti{ဟိ-ဂတိယံ ၊ ဓူ-ကမ္ပနေ၊ ဧတေဟိ မက ဟောတိ။ ဟိနောတိ ပဝတ္တဟီတိ ဟိမံ=တုဟိနံ။ ဓုနာတိ ကမ္ပတီတိ ဓူမော=အဂ္ဂိပသဝေါ။}

\sutta{985}{135}{ဘီတော ရီသနော စ။}
\vutti{ဘီ-ဘယေတီမသ္မာ ရီသနော ဟောတိ မကစ။ အန္တသရာဒိလောပေါ၊ ဘာယန္တိ ဧတသ္မာတိ ဘီသနော=ဘယာနကော။ ဘီမော=သောဣ။}

\sutta{986}{136}{ခီ သု ဝီ ယာ ဂါဟိ သာ လူ ခု ဟု မရ ဓရ ဃရ ဇမာမ သမာ မော။}
\vutti{ခီ-ခယေ။ သု-သဝနေ၊ ဝီ-တန္တသန္တာနေ၊ ယာ-ပါပုဏနေ၊ ဂါ-သဒ္ဒေ၊ ဟိ-ဂတိယံ၊ သာ-တနုကရဏာဝသာနေသု၊ လူ-ဆေဒနေ၊ ခု-သဒ္ဒေ၊ ဟု-ဟဝနေ၊ မရ-ပါဏစာဂေ၊ ဓရ-ဓာရဏေ၊ ကရ-ကရဏေ၊ ဃရ-သေစနေ၊ ဇမ-အဒနေ၊ အမ ဂမ-ဂမနေ၊ သမ-ဥပမေ၊ ဧတေဟိ ဓာတူဟိ မော ဟောတိ။ ခေပနံ နိရုပဒ္ဒဝကရဏတာယ ခေမော=နိရုပဒ္ဒဝေါ။ သုနောတီတိ သောမော=စန္ဒော။ ဝါယန္တိ ဧတေနာတိ ဝေမော=တန္တဝါယောပကရဏံ။ ယာတီတိ ယာမော=ဒိနဿ ဆဋ္ဌော ဘာဂေါ၊ အဋ္ဌမော ဝါ။ ဂါယန္တိ ဧတ္ထာတိ ဂါမော=သံဝသထော။ ဟိနောတိ ပဝတ္တတီတိ ဟေမံ=သုဝဏ္ဏံ။ သာတိ သုန္ဒရတ္တံ တနုံ ကရောတီတိ သာမော=ကာဠော။ လူယတေတိ လောမံ=တနုရုဟံ။ ခူယတေ ဥတ္တမဘာဝေနာတိ ခေါမံ=အတသိ။ ဟဝနံ၊ ဟူယတေ ဝါ ဟောမံ=ဟုတိ။ မရန္တျနေနာတိ မမ္ဗံ=ယသ္မိံ တာဠိတေ န ဇီဝတိ တံ။ အတ္တာနံ ဓာရေန္တေ အပါယေ ဝဋ္ဋဒုက္ခေ စ အပတမာနေ ကတွာ ဓာရေတီတိ ဓမ္မော=ပရိယတျာဒိ ။ ကရဏံ၊ ကရီယတီတိ ဝါ ကမ္မံ=သုခဒုက္ခဖလာဒိ။ သေဒေါ ပဂ္ဃရတိ အနေနာတိ ဃမ္မော=နိဒါဃော။ ဇမေတိ အဘက္ခိ တဗ္ဗံ အဒတီတိ ဇမ္မော=နိဟီနော၊ အနိသမ္မကာရီ စ။ အမေတိ ပေမေန ပုတ္တကေသု ပဝတ္တတီတိ အမ္မာ=မာတာ။ သမေန္တိ အနေနာတိ သမ္မာ=ပိယသမုဒါစာရော။}

\sutta{987}{137}{အသ္မာဒယော။}
\vutti{အသ္မအာဒယော မန္တာ နိပစ္စန္တေ၊ ပရရူပါဒီနမဘာဝေါ နိပါတနာ။ အသ-ခေပနေ၊ အဿတေတိ အသ္မာ=ပါသာဏော။ ဘသ-ဘသ္မီကရဏေ၊ ဘသတိ ပက္ခရတီတိ ဘသ္မာ=ဆာရိကာ။ ဥသ-ဒါဟေ၊ ဥသတိ ဒဟတီတိ ဥသ္မာ=တေဇောဓာတု။ ဝိသ-ပဝိသနေ၊ ပဝိသန္တိ ဧတ္ထာတိ ဝေသ္မံ=ဃရံ။ ဘီ-ဘယေ၊ မဿ သုဉ၊ ဘာယန္တိ ဧတသ္မာတိ ဘေသ္မာ=ဘယာနကော။ အသတိဿ ဓင၊ အဿတိ ဇနေဟိ စဇီယတီတိ အဓမော=နိဟီနော။ ကရောတိဿ အဿ ဥတ္တံ၊ ကရောတီတိ ကုမ္မော=ကစ္ဆပေါ။ ဧဝမညေပိ။}

\sutta{988}{138}{နီတော မိ။}
\vutti{နယတိသ္မာ မိ ဟောတိ။ နယတီတိ နေမိ=စက္ကန္တံ။}

\sutta{989}{139}{ဦမိ ဘူမိနိမိ ရသ္မိ။}
\vutti{ဧတေသဒ္ဒါ မိအန္တာ နိပစ္စန္တေ။ ဦဟ-ဝိတက္ကေ၊ ဟလောပေါ၊ ဦဟန္တိ ဝိတက္ကေန္တိ ဧတေနာတိ ဦမိ=တရင်္ဂေါ။ ဘူ-သတ္တာယံ ဩတ္တာဘာဝေါ၊ ဘဝန္တိ ဧတ္ထာတိ ဘူမိ=မဟီ။ နိ-ပါပနေ၊ ဧတ္တာဘာဝေါ၊ သုဂတိံ နေတိ ပါပေတီတိ နိမိ=ရာဇာ။ ရသ-အဿာဒနေ၊ ပရရူပါဘာဝေါ၊ ရသန္တိ သတ္တာ ဧတာယာတိ ရသ္မိ=ရဇ္ဇု။}

\begin{jieshu}
ဣတိ ပဝဂ္ဂပစ္စယဝိဓာနံ။
\end{jieshu}


\sutta{990}{140}{မာဆာဟိ ယော။}
\vutti{မာ-မာနေ-ဆာ-ဆာဒနေ ၊ ဧတေဟိ ယော ဟောတိ။ မေတိ ပရိမေတိ အညေန ဥတ္တမေန ဂုနေန အတ္တနော အဂုနန္တိ မာယာ=သန္တဒေါသ ပဋိစ္ဆာဒန လက္ခဏာ။ ဆေတိ ဆိန္ဒတိ သံသယန္တိ ဆာယာ=ပဋိဗိမ္ဗံ။}

\sutta{991}{141}{ဇနိဿ ဇာ စ။}
\vutti{ဇနိသ္မာ ယော ဟောတိ၊ ဇနိဿ ဇာ စ၊ ဇနေတီတိ ဇာယာ=ဘရိယာ။}

\sutta{992}{142}{ဟဒယာ ဒယော။}
\vutti{ဟဒယအာဒယော ယန္တာ နိပစ္စန္တေ။ ဟရိဿ ဒင၊ ဟရတီတိ ဟဒယံ=စိတ္တံ၊ မနောဓာတု မနောဝိညာဏဓာတု နိဿယော စ၊ တနိဿအက၊ အတ္တနိ ပေမံ တနောတီတိ တနယော=ပုတ္တော။ သရတိဿ သုရိ၊ သရတိ ဂစ္ဆတီတိ သူရိယော=အာဒိစ္စော။ ဟရတိဿ မ္မိင၊ သုခမာဟရတီတိ ဟမ္မိယံ=မုဏ္ဍစ္ဆဒနပါသာဒေါ။ ကသ-ဂမနေ၊ ကသဿ အလက၊ အဿ ဣ စ၊ ကသတိ ဝုဒ္ဓိံ ယာတီတိ ကိသလယံ=ပလ္လဝံ၊ ဧဝမညေပိ။}

\sutta{993}{143}{ခီ သိ သိ နီ သီ သု ဝီ ကုသူဟိ ရက။}
\vutti{ခီ-ခယေ၊ သိ-သေဝါယံ၊ သိ-ဗန္ဓနေ၊ နီ-ပါပနေ၊ သီ-သယေ၊ သု-သဝနေ၊ ဝီ ဝါ-ဂမနေ၊ ကု-သဒ္ဒေ၊ သူ-ပသဝေ၊ ဧတေဟိ ရက ဟောတိ။ ခယတိ ဒုဟနေနာတိ ခီရံ=ပယော။ ကုသုမာဒီဟိ သေဝီယတီတိ သိရော=မုဒ္ဓါ။ သေတိ သရီရံ ဗန္ဓတီတိ သိရာ=ဒေဟဗန္ဓနီ။ နေတိ၊ ပရေဟိဝါနီယတီတိ နီရံ=ဇလံ။ သယတီတိ သီရော=ဟလံ။ အနိဋ္ဌဖလ- ဒါယကတ္တံ သဝတီတိ သုရာ=မဒိရာ။ သုဏောတိ ဥတ္တမဂီတာဒိန္တိ သုရော=ဒေဝေါ။ ဝေတိ ဥတ္တမဘာဝံ ယာတီတိ ဝီရော=ဝိက္ကန္တော။ ကဝတိ နဒတီတိ ကုရံ=ဘတ္တံ။ ဘယဋ္ဋိတာနံ ပဌမကပ္ပိယာနံ သူရတ္တံ ပသဝတီတိ သူရော=သူရိယော၊ ဝိက္ကန္တော စ။}

\sutta{994}{144}{ဟိ စိ ဒု မိနံ ဒီဃော စ။}
\vutti{ဟိ-ဂတိယံ၊ စိ-စယေ၊ ဒု-ဂတိယံ၊ မိ-ပက္ခေပနေ၊ ဧတေဟိ ရက ဟောတိ၊ ဒီဃော စာန္တဿ။ ဟိနောတိ ပဝတ္တတီတိ ဟီရံ=တာလဟီရာဒိ။ စယတီတိ စီရံ=ဝက္ကလံ။ ဒူယတိ ဒုက္ခေန ဂမီယတီတိ ဒူရံ=အနာသန္နံ။ မီယတေ ပက္ခီပီယတေတိ မီရော=သမုဒ္ဒေါ။}

\sutta{995}{145}{ဓာ တာနမီ စ။}
\vutti{ဓာ-ဓာရဏေ၊ တာ-ပါလနေ၊ ဧတေဟိ ရက ဟောတိ၊ ဤ စာန္တာဒေသော။ ဓာရေတိဓီရော=ဓိတိမာ။ ဇလံ တာယတီတိ တီရံ=တဋံ။}

\sutta{996}{146}{ဘဒြာယော။}
\vutti{ဘဒြအာဒယော ရကအန္တာ နိပစ္စန္တေ။ ဘဒ္ဒ-ကလျာဏေ၊ ဒလောပေါ ပရရူပါဘာဝေါ၊ ဘဇီယတီတိ ဘဒြံ=ကလျာဏံ။ ဘီ-ဘယေ၊ နဒါဒိပါဌာ ဝီ၊ ဘာယန္တိ ဧတာယာတိ ဘေရီ=ဒုန္ဒုဘိ။ စိတ-သဉ္စေတနေ၊ ဝိပုဗ္ဗော၊ ဝိစိန္တိတဗ္ဗန္တိ ဝိစိတြံ=နာနာကာရံ၊ ယာ-ပါပုဏနေ၊ ရဿ တုဉ၊ ဂမနံ ယာတြာ=ယာနံ။ ဂုပ-ဂေါပနေ၊ ဥဿ ဩ၊ ပဿ တဉ္စ ဂေါပီယတီတိ ဂေါတြံ=ကုလာဒိ၊ ဘသ-ဘသ္မီကရဏေ၊ ရဿ တုဉ၊ ဘသတိ ဘက္ခံ ကရောတိ တောယာတိ ဘသ္တာ-ကမ္မာရဂဂ္ဂရီ။ ဥသ=ဒါဟေ၊ သလောပေါ၊ သောကေန တာဠိတေ ဥသတိ ဒဟတီတိ ဥရော=သရီရေကဒေသော၊ ဧဝမညေပိ။}

\sutta{997}{147}{မန္ဒင်္က သသာသ မထ စတာ ဥရော။}
\vutti{မန္ဒ-ဇဠတ္တေ ၊ အင်္က-လက္ခဏေ၊ သသ-ဂတိဟိံသာပါဏနေသု၊ အသ-ခေပနေ၊ မထ မန္ထ-ဝိလောဠနေ၊ စတ-ယာစနေ၊ ဧတေဟိ ဥရော ဟောတိ။ အမန္ဒိ အသုန္ဒရတ္တာ ဇဠတ္ထမဂမီတိ မန္ဒုရာ=ဝါဇိသာလာ။ အင်္ကီယတိ လက္ခီယတီတိ အင်္ကုရော=ဗီဇပသဝေါ။ သသတိ ဟိံသတီတိ သသုရော=ဇယမ္ပတီနံ ပိတာ။ အသီယိတ္ထာတိ အသုရော=ဒါနဝေါ၊ အရီဟိ မထီယတိ အာလောဠီယတီတိ မထုရာ=နဂရံ။ စတီယတီတိ စတုရော=ဒက္ခော။}

\sutta{998}{148}{ဝိဓုရာဒယော။}
\vutti{ဝိဓုရအာဒယော ဥရန္တာ နိပစ္စန္တေ။ ဝိဓ-ဝေဓနေ၊ ဧတ္တာဘာဝေါ၊ ဝေဓတိ ဟိံသတီတိ ဝိဓုရော=ဝိရုဒ္ဓေါ။ ဥန္ဒ-ကိလေဒနေ၊ ဥန္ဒတိ ကိလေဒတီတိ ဥန္ဒုရော=အာခု။ မံက-မဏ္ဍနေ၊ နိဂ္ဂဟီတလောပေါ၊ မင်္ကတိ အနေန အတ္တာနံ အလင်္ကရောတီတိ မကုရော=အာဒါသော၊ ရထော၊ ကက္ကော၊ မစ္ဆော စ။ ကုက ဝက-အာဒါနေ၊ ကဿ ဒွိတ္တံ။ ကုကတိ သလာဒယော အာဒဒါတီတိ ကုက္ကုရော=သာ။ မင်္ဂ-မင်္ဂလျေ၊ အမင်္ဂိ ပသတ္ထမဂမီတိ မင်္ဂုရော=မစ္ဆဝိသေသော၊ ဧဝမညေပိ။}

\sutta{999}{149}{တိမ ရုဟ ရုဓ ဗဓ မဒ မန္ဒ ဝဇာဇ ရုစ ကသာ ကိရော။}
\vutti{တိမ-တေမနေ၊ ရုဟ-ဇနနေ၊ ရုဓ-အာဝရဏေ၊ ဗဓ-ဗာဓနေ၊ မဒဥမ္မာဒေ၊ မန္ဒ-မောဒနထုတိဇဠတ္တေသု၊ ဝဇ အဇ-ဂမနေ၊ ရုစ-ဒိတ္တိယံ၊ ကသ-ဂမနေ၊ ဧတေဟိ ကိရော ဟောတိ။ တေမေတီတိ တိမိရံ=အန္ဓကာရံ၊ အာပေါ စ။ ရုဟတိ ပဝတ္တတီတိ ရုဟိရံ=လောဟိတံ။ ဇီဝိတံ ရုန္ဓတီတိ ရုဓိရံ=တဒေဝ။ ဗာဓီယတီတိ ဗဓိရော=သောတဝိကလော။ ဇနာ မဇ္ဇန္တိ ဧတာယာတိ မဒိရာ=သုရာ။ မောဒန္တိ ဧတ္ထာတိ မန္ဒိရံ= ဃရံ ။ ဝဇတီတိ ဝဇိရံ=ကုလိသုံ။ အဇန္တိ ဂစ္ဆန္တီ ဧတ္ထာတိ အဇိရံ=အင်္ဂဏံ ဃရဝိသယောကာသော စ။ ရောစတီတိ ရုစိရံ=မနုညံ။ ကသီယတိ ဒုက္ခေန ဂမီယတီတိ ကသိရံ=ကိစ္ဆံ။}

\sutta{1000}{150}{ထိရာဒယော။}
\vutti{ထိရအာဒယော ကိရန္တာ နိပစ္စန္တေ။ ဌာ-ဂတိနိဝတ္တိယံ၊ ဌဿ ထတ္တံ၊ ဌာတိ ပဝတ္တတီတိ ထိရံ=စိရဋ္ဌာယီ။ ဣသ သိံသ-ဣစ္ဆာယံ၊ နိဂ္ဂဟီတလောပေါ၊ ဣစ္ဆီယတီတိ သိသိရော=ဥတုဝိသေသော။ အဒ ခါဒ-ဘက္ခနေ၊ အာဿ ရဿတ္ထံ၊ ခါဒီယတိ ပါဏကေဟီတိ ခဒိရော=ဒန္တဓာဝနော၊ ဧဝမညေပိ။}

\sutta{1001}{151}{ဒဒ ဂရေဟိ ဒုရဘရာ။}
\vutti{ဒဒ-ဒါနေ၊ ဂရ ဃရ-သေစနေ၊ ဧတေဟိ ယထာက္ကမံ ဒုရဘရာ ဟောန္တိ။ အန္တာနံ ဒဒါတီတိ ဒဒ္ဒုရော=ဘေကော။ ဂရတိ သိဉ္စတီတိ ဂဗ္ဘရံ=ဂုဟာ။}

\sutta{1002}{152}{စရ ဒရ ဇရ ဂရ မရေဟိ တေ။}
\vutti{စရာဒီဟိ ဓာတူဟိ တေ စရာဒယော ဟောန္တိ ယထာက္ကမံ။ စရဂတိဘက္ခနေသု စရန္တိ ဧတ္ထာတိ စစ္စရံ=ဝီထိစတုက္ကံ၊ အင်္ဂဏဉ္စ၊ ဒရ-ဝိဒါရဏေ၊ ဒရီယတီတိ ဒဒ္ဒရံ=ဝါဒိတ္တံ၊ ဘေရီ စ။ ဇရ-ဝယောဟာနိယံ၊ အဇရီတိ ဇဇ္ဇရော=ဇိဏ္ဏော။ ဂရ ဃရ-သေစနေ၊ ဂရတိ သိဉ္စတီတိ ဂဂ္ဂရော=ဘိန္နဿရော၊ ဟံသဿရော စ။ မရ-ပါဏစာဂေ၊ မရတီတိ မမ္မရော=သုက္ခပဏ္ဏံ၊ ပတ္ထပဏ္ဏာနံ သဒ္ဒေါ စ။}

\sutta{1003}{153}{ပီတော ကွရော။}
\vutti{ပီ-တပ္ပနေတီမသ္မာ ကွရော ဟောတိ။ အပ္ပိဏီတိ ပီဝရံ=ထူလံ။}

\sutta{1004}{154}{စီဝရာဒယော။}
\vutti{စီဝရအာဒယော ကွရန္တာ နိပစ္စန္တေ။ စိနာတိဿ ဒီဃတ္တံ။ စီယတီတိ စီဝရံ=ကာသာဝံ။ သမ-ဥပသမေ၊ နဒါဒိတ္တာ ဝီ၊ ပရိဠာဟံ သမေတီတိ သံဝရီ=ရတ္တိ။ ဓာဿ ဤ၊ ဇာလကုမိနာဒီနိ ဓာရေတီတိ ဓီဝရော=ကောဋ္ဋော၊ တာယတိဿ ဤ၊ ယေန ကေနစိ အတ္တာနံ တာယတီတိ တီဝရော=ဟီနဇာတိ။ နယတိဿီ၊ နယန္တိ ဧတ္ထ သတ္တာတိ နီဝရံ=ဃရံ၊ ဧဝမညေပိ။}

\sutta{1005}{155}{ကုတော ကြိရော။}
\vutti{ကု-သဒ္ဒေတီမသ္မာ ကြိရော ဟောတိ။ ကဝတိ နဒတီတိ ကုရရော=ပက္ခီ၊ ဣတ္ထိယံ ဝီမှိ ကုရရီ။}

\sutta{1006}{156}{ဝသာသာ ဆရော။}
\vutti{ဝသ-နိဝါသေ၊ အသ-ခေပနေ၊ ဧတေဟိ ဆရော ဟောတိ။ ဝသန္တိ ဧတ္ထာတိ ဝစ္ဆရော=ဝဿော။ သံပုဗ္ဗော၊ သံဝသန္တိ ဧတ္ထာတိ သံဝစ္ဆရော=သောဣ။ အသတိ ဝိဿဇ္ဇေတီတိ အစ္ဆရာ=ဒေဝကညာ၊ အင်္ဂုလိဖောဋနဉ္စ။}

\sutta{1007}{157}{မသာ ဆေရော စ။}
\vutti{မသ-အမသနေတီမသ္မာ ဆေရော ဟောတိ ဆရော စ။ တဏှာယ ပရာမသနံ မစ္ဆေရံ=သကသမ္ပတ္တိနိဂုဟနံ၊ မစ္ဆရံ=တဒေဝ။}

\sutta{1008}{158}{ဓူ ဝါတော သရော။}
\vutti{ဓုနာတိ ဝါတီဟိ သရော ဟောတိ။ ဓုနာတီတိ ဓူသရော=လူခေါ ဤသံပဏ္ဍု စ။ ဝါတိ ဂစ္ဆတီတိ ဝါသရော=ဒိဝသော။}

\sutta{1009}{159}{ဘမာဒီဟျရော။}
\vutti{ဘမ တသ မန္ဒ ကန္ဒာဒီဟိ အရော ဟောတိ။ ဘမတီတိ ဘမရော=မဓုကရော။ တသတိ တန္တံ ဂဏှာတီတိ တသရော=သုတ္တဝေဋ္ဌနော။ မန္ဒန္တိ မောဒန္တိ ဧတ္ထာတိ မန္ဒရော=ပဗ္ဗတော။ ကန္ဒတိ အဝှာယတီတိ ကန္ဒရော=ဒရီ။ ဒိဝဿ ဧတ္တံ၊ ဒေဝန္တိ ကီဠန္တိ ဧတေနာတိ ဒေဝရော=ပတိရော ဘာတာ။}

\sutta{1010}{160}{ဝဒိဿ ဗဒစ။}
\vutti{ဝဒတိသ္မာ အရော ယောတိ၊ ဝဒတိဿ ဗဒါဒေသော စ။ ဝဒန္တိ ဧတေနာတိ ဗဒရော=ကက္ကန္ဓူဖလံ။ ဝီမှိ ဗဒရီ=ကက္ကန္ဓူ။}

\sutta{1011}{161}{ဝဒ ဇနာနံ ဌငစ။}
\vutti{ဝဒ ဇနေဟိ အရော ဟောတိ၊ ဌင စာန္တာဒေသော။ ဝဒတီတိ ဝဌရော=မူဠော ဝဌရံ=ထူလံ။ ဇာယတီတိ ဇဌရံ=ဥဒရံ။}

\sutta{1012}{162}{ပစိဿိဌငစ။}
\vutti{ပစတိသ္မာ အရော ဟောတိ ဣဌငစာန္တာဒသော။ ပစန္တိ ဧတေနာတိ ပိဌရော=ထာလီ။}

\sutta{1013}{163}{ဝကာ အရဏ။}
\vutti{ဝက ကုက-အာဒါနေတိမသ္မာ အရဏဟောတိ။ ဝကေတိ အာဒဒါတိ ဧတာယာတိ ဝါကရာ=မိဂဗန္ဓနီ။}

\sutta{1014}{164}{သိင်္ဂျင်္ဂါဂ မဇ္ဇကလာလာ အာရော။}
\vutti{သိင်္ဂိဣတိ နာမဓာတု၊ အင်္ဂ-ဂမနတ္ထော၊ အဂ-ကုဋိလဂမနေ၊ မဇ္ဇသံသုဒ္ဓိယံ၊ ကလ-သင်္ချာနေ၊ အလ-ဗန္ဓနေ၊ ဧတေဟိ အာရော ဟောတိ။ ဝိဇ္ဈနတ္ထေန သိင်္ဂံ ဝိယ သိင်္ဂံ=နာဂရိကဘာဝသင်္ခါတဿ ကိလေသသိင်္ဂဿေတံ နာမံ၊ တံကရောတိ သိင်္ဂံ ဝါ ပယုတ္တံ၊ တံ ကရောတိ ရာဂီသု ပဘဝတီတိ ဝါ၊ “ဓာတွတ္ထေ နာနာမသ္မီ ”တိ (၅.၁၂) ဣ၊ ပုဗ္ဗ သရလောပေါ၊ သိင်္ဂိ၊ တတော အာရော၊ “သရောလောပေါ သရေ ”တိ (၁.၂၆) ဣလောပေါ၊ ပုဗ္ဗေ “ဝိပ္ပဋိသေဓေ ”တိ (၁.၂၂) အနိဋ္ဌပ္ပဋိသေဓော၊ ဧတ္ထ ဟိ အာရတော အညတ္ထ သာဝကာသပုဗ္ဗသရလောပေါဝ၊ ဣပစ္စယတော အညတ္ထ စ၊ သိင်္ဂါရော=ကိလေသသိင်္ဂကရဏံ၊ ဝိလာသောတိ ဝုတ္တံ ဟောတိ။ အင်္ဂတိ ဝိနာသံ ဂစ္ဆတီတိ အင်္ဂါရော=ဒဍ္ဎကဋ္ဌံ။ အဂန္တိ ဂစ္ဆန္တိ ဧတ္ထာတိ အဂါရံ=ဃရံ။ လီဟနေန အတ္တနော သရီရံ မဇ္ဇတိ နိမ္မလတ္တံ ကရောတီတိ မဇ္ဇာရော=ဗိဠာရော။ ကလာတိ နိဒ္ဒေသာ လဿ ဠတ္တံ၊ ဧတေန ဂုဏံ ကလီယတိ ပရိမီယတီတိ ကဠာရော=ပိင်္ဂလော။ ဒီဃတ္တံ အလတိ ဗန္ဓတီတိ အဠာရော=ဝင်္ကော ဝိသာလော စ။}

\sutta{1015}{165}{ကမိဿဿု စ။}
\vutti{ကမ-ဣစ္ဆာယမိစ္စသ္မာ အာရော ဟောတိ၊ အဿ ဥ စ။ ကာမီယတီတိ ကုမာရော=ဗာလော။}

\sutta{1016}{166}{ဘိင်္ဂါရာဒယော။}
\vutti{ဘိင်္ဂါရပ္ပဘုတယော အာရန္တာ နိပစ္စန္တေ။ ဘရ=ဘရဏေ၊ ဘရဏံ ဓာရဏံ ပေါသနဉ္စ၊ ဓာရဏတ္ထဿ ဘရတိဿ ဘိင်္ဂါဒေသော၊ ဘရတိ ဒဓာတိ ဥဒကန္တိ ဘိင်္ဂါရော=ဟေမဘာဇနံ။ က္လေဒ က္လိဒ-အလ္လ ဘာဝေ၊ လ လောပေါ၊ က္လေဒယတီတိ ကေဒါရံ=ခေတ္တံ၊ (ကေ ဇလေ သတိ ဒါရော ဝိဒါရဏမဿာတိ ဝါ ကေဒါရံ=တဒေဝ၊ ဗဟုလာဓိကာရာ သတ္တမိယာ န လောပေါ။) ဝိဒ-လာဘေတီမသ္မာ ကုပုဗ္ဗာ အာရော ဒဿ ဠတ္တံ ဣဿ ဧတ္တာဘာဝေါ သမာသေ ကုဿ ဩ စ နိပစ္စန္တေ၊ ကုံ ပထဝိံ ဝိန္ဒတိ တတြုပ္ပန္တာယာတိ ကောဝိဠာရော=ဒိဂုဏပတ္တော။}

\sutta{1017}{167}{ကရာ မာရော။}
\vutti{ကရောတိသ္မာ မာရော ဟောတိ။ လောဟကိစ္စံ ကရောတီတိ ကမ္မာရော=လောဟကာရော။}

\sutta{1018}{168}{ပုသ သရေဟိ ခရော။}
\vutti{ပုသ သရေဟိ ခရော ဟောတိ။ ပေါသီယတိ ဇလေနာတိ ပေါက္ခရံ=ပဒုမံ။ သရတိ ဝိကာရံ ဂစ္ဆတီတိ သက္ခရာ=ဥစ္ဆုဝိကာရော။}

\sutta{1019}{169}{သရ ဝသ ကလာ ကီရော ဝဿုဋ စ။}
\vutti{ဧတေဟိ ကီရော ဟောတိ ဝဿ ဥဋ စ။ သရီယတီတိ သရီရံ=ဒေဟော။ ဝသန္တိ ဝါသံ ကရောန္တိ ဧတေနာတိ ဥသီရံ=ဗီရဏမူလံ။ အနေန ထူလာဒိ ကလီယတိ ပရိမီယတိတိ ကလိရော=အင်္ကုရော။}

\sutta{1020}{170}{ဂမ္ဘီရာဒယော။}
\vutti{ဂမ္ဘီရအာဒယော ကီရန္တာ နိပစ္စန္တေ။ ဂမိဿ ဘုက၊ မလောပေါ ဝါ၊ ပထဝိံ၊ ဘိန္ဒိတွာ ဂစ္ဆတိ ပဝတ္တတီတိ ဂမ္ဘီရော၊ ဂဘီရော= အဂါဓော ၊ ကုလိဿ လဿ ဠော၊ ပါဒေ ကုလတိ ပတ္ထရတီတိ ကုဠီရော=ကက္ကဋော၊ ဧဝမညေပိ။}

\sutta{1021}{171}{ခဇ္ဇ ဝလ္လ မသာ ဦရော။}
\vutti{ခဇ္ဇ-မဇ္ဇနေ၊ ဝလ ဝလ္လ-သံဝရဏေ၊ သမ-အာမသနေ၊ ဧတေဟိ ဦရော ဟောတိ၊ ခဇ္ဇီယတီတိ ခဇ္ဇူရော၊ ဝီမှိ ခဇ္ဇူရီ=ရုက္ခဝိသေသော။ ဝလ္လီယတိ သံဝရီယတီတိ ဝလ္လူရော=သုက္ခမံသော။ မသီယတီတိ မသူရော=ဝီဟိဝိသေသော။}

\sutta{1022}{172}{ကပ္ပူရာဒယော။}
\vutti{ကပ္ပူရအာဒယော ဦရန္တာ နိပစ္စန္တေ။ တုဋ္ဌိမုပ္ပာဒေတုံ ကပ္ပတိ သက္ကောတီတိ ကပ္ပူရံ=ဃနသာရော။ ကရောတိဿ အဿု၊ ကိဗ္ဗိသံ ကရောတီတိ ကုရူရော=ပါပကာရီ။ ပသ-ဗာဓနေ၊ ပသတိ ပီဠေတီတိ ပသူရော=ဒုဋ္ဌော၊ ဗျဉ္ဇနံ၊ ဧဝံနာမကော စ၊ ဧဝမညေပိ။}

\sutta{1023}{173}{ကဌ စကာ ဩရော။}
\vutti{ကဌ-ကိစ္ဆဇီဝနေ၊ စက-ပရိဝိတက္ကနေ၊ ဧတေဟိ ဩရော ဟောတိ။ ကဌတိ ကိစ္ဆေန ဇီဝတီတိ ကဌောရော=ထဒ္ဓေါ။ စကတိ ပရိဝိတက္ကေတီတိ စကောရော=ပက္ခိဝိသေသော။}

\sutta{1024}{174}{မောရာဒယော။}
\vutti{မောရအာဒယော ဩရန္တာ နိပစ္စန္တေ။ မီ-ဟိံသာယံ၊ ဤလောပေါ၊ မယတိ ဟိံသတီတိ မောရော=မယူရော။ ကသ-ဂမနေ၊ အဿိ ၊ ကသတိ ဂစ္ဆတီတိ ကိသောရော=ပဌမဝယော အဿော။ မဟီယတိ ပူဇီယတီတိ မဟောရော=ဝမ္မိကော၊ ဧဝမညေပိ။}

\sutta{1025}{175}{ကုတော ဧရက။}
\vutti{ကု-သဒ္ဒေတီမသ္မာ ဧရက ဟောတိ။ “ယုဝဏ္ဏာနမိယငုဝင သရေ ”တိ (၅.၁၃၆) ဥဝင၊ ကဝတိ နဒတီတိ ကုဝေရော=ဝေဿဝဏော။}

\sutta{1026}{176}{ဘူ သူဟိ ရိက။}
\vutti{ဘူသတ္တာယံ၊ သူ-ပသဝနေ၊ ဧတေဟိ ရိက ဟောတိ။ ဘဝတီတိ ဘူရိ=ပဟူတံ၊ ငီမှိ ဘူရီ=မေဓာ။ သဝတိ ဟိတံ ပသဝတီတိ သူရိ=ဝိစက္ခဏော။}

\sutta{1027}{177}{မီ ကသီ နီဟိ ရု။}
\vutti{မီ-ဟိံသာယမိစ္စသ္မာ၊ ကပုဗ္ဗာ သယတိသ္မာ၊ နယတိသ္မာ စ ရု ဟောတိ။ ရံသီဟိ အန္ဓကာရံ မီယတိ ဟိံသတီတိ မေရု=သိရေရု၊ ကေ ဇလေ သယတိ ပဝတ္တတီတိ ကသေရု=တိဏဝိသေသော။ အတ္တနိဿိတေ သုန္ဒရတ္တံ နေတိ ပါပေတီတိ နေရု=ပဗ္ဗတော။}

\sutta{1028}{178}{သိနာ ဧရု။}
\vutti{သိနာ-သောစေယျေတီမသ္မာ ဧရု ဟောတိ။ သိနာတိ သုစိံ ကရောတီတိ သိနေရု=ပဗ္ဗတရာဇာ။}

\sutta{1029}{179}{ဘီ ရုဟိ ရုက။}
\vutti{ဘီ-သယေ၊ ရု-သဒ္ဒေ၊ ဧတေဟိ ရုက ဟောတိ။ ဘာယန္တိ ဧတသ္မာတိ ဘီရု=ဘယာနကော။ ရဝတီတိ ရုရု=မိဂေါ။}

\sutta{1030}{180}{တမာ ဗူလော။}
\vutti{တမ-ဘူသနေတီမသ္မာ ဗူလော ဟောတိ။ မုခံ တမေတိ ဘူသေတီတိ တမ္ဗူလံ=မုခဘူသနံ။}

\sutta{1031}{181}{သိတော လကဝါလာ။}
\vutti{သိ-သေဝါယမိစ္စသ္မာ လကဝါလဣစ္စေတေ ပစ္စယာ ဟောန္တိ။ သတ္တေဟိ သေဝီယတီတိ သိလာ=ပါသာဏော၊ သေလော=ပဗ္ဗတော။ ဇလံ သေဝတီတိ သေဝါလော=ဇလတိဏံ။}

\sutta{1032}{182}{မင်္ဂ ကမ သမ္ဗ သဗ သက ဝသ ပိသ ကေဝ ကလ ပလ္လ ကဌ ပဋ ကုဏ္ဍ မဏ္ဍာ အလော။}
\vutti{မင်္ဂ-မင်္ဂလျေ၊ ကမ-ဣစ္ဆာယံ၊ သမ္ဗ-မဏ္ဍနေ၊ သဗဣတိ အဿေဝ ကတမလောပဿ နိဒ္ဒေသော၊ သက-သတ္တိယ၊ ဝသ-နိဝါသေ၊ ပိသဂမနေ၊ ကေဝ-သေဝနေ၊ ကလ-သင်္ချာနေ၊ ပလ္လ-ဂမနေ၊ ကဌ ကိစ္ဆဇီဝနေ၊ ပဋ-ဂမနတ္ထော၊ ကုဏ္ဍ-ဒါဟေ၊ မဏ္ဍ-ဘူသနေ၊ ဧတေဟိ အလော ဟောတိ။ မင်္ဂန္တိ သတ္တာ ဧတေန ဝုဒ္ဓိံ ဂစ္ဆန္တီတိ မင်္ဂလံ=ပသတ္ထံ။ ကာမီယတီတိ ကမလံ-ပင်္ကဇံ။ သမ္ပတိ မဏ္ဍေတီတိ သမ္ဗလံ=ပါထေယျံ။ သဗလံ=ဝိသဘာဂ ဝဏ္ဏဝန္တံ။ သက္ကောတိ ဝတ္တုန္တိ သကလံ=သဗ္ဗံ။ ဝသတီတိ ဝသလော=သုဒ္ဒေါ။ ပိယဘာဝံ ပိသတိ ဂစ္ဆတီတိ ပေသလော=ပိယသီလော။ ကေဝတိ ပဝတ္တတီတိ ကေဝလံ=သကလံ။ ကလီယတိ ပရိမီယတိ ဥဒကမေတေနာတိ ကလလံ=အပတ္ထိန္နံ၊ ပလ္လတိ အာဂစ္ဆတိ ဥဒကမေတသ္မာတိ ပလ္လလံ=အပ္ပောဒကော သရော။ ကဌန္တိ ဧတ္ထ ဒုက္ခေန ယန္တီတိ ကဌလံ=ကပါလခဏ္ဍံ၊ ပဋတိ ဝုဒ္ဓိံ ဂစ္ဆတီတိ ပဋလံ=သမူ- ဟော ။ ဃံသေန ကုဏ္ဍတိ ဒဟတီတိ ကုဏ္ဍလံ=ကဏ္ဏာဘရဏံ။ မဏ္ဍီယတိ ပရိစ္ဆေဒကရဏဝသေန ဘူသီယတီတိ မဏ္ဍလံ=သမန္တတော ပရိစ္ဆိန္နံ။}

\sutta{1033}{183}{မုသာ ကလော။}
\vutti{မုသတိသ္မာ ကလော ဟောတိ။ မုသတိ ဧတေနာတိ မုသလော=အယောဂ္ဂေါ။}

\sutta{1034}{184}{ထလာဒယော။}
\vutti{ထလအာဒယော ကလန္တာ နိပစ္စန္တေ။ ဌဿ ထော၊ ပုဗ္ဗသရလောပေါ၊ တိဋ္ဌန္တိ ဧတ္ထာတိ ထလံ=ဥန္နတပ္ပဒေသော။ ပါ-ပါနေ၊ ဥပုဗ္ဗော၊ ဒွိဘာဝသရလောပါ၊ ဥဒကံ ပိဝတီတိ ဥပ္ပလံ=ကုဝလယံ။ ပတိဿ ပါဋံ၊ ပတတိ ဂစ္ဆတိ ပရိပါကန္တိ ပါဋလံ=ဖလံ၊ တမ္ဗဝဏ္ဏံ ကုသုမဉ္စ။ ဗံဟိဿ နိဂ္ဂဟီတလောပေါ၊ ဗံဟတိ ဝုဒ္ဓိံ ဂစ္ဆတီတိ ဗဟလံ=ဃနံ။ စုပိဿ ဥဿ အတ္တံ၊ စုပတိ ဧကတ္ထ န တိဋ္ဌတီတိ စပလော=အနဝဋ္ဌိတော၊ ဧဝမညေပိ။}

\sutta{1035}{185}{ကုလာ ကာလော စ။}
\vutti{ကုလ-ပတ္ထာရေတီမသ္မာ ကာလော ဟောတိ ကလော စ။ ကုလတိ အတ္တနော သိပ္ပံ ပတ္ထရတီတိ ကုလာလော=ကုမ္ဘကာရော။ ကုလတိ ပက္ခေ ပသာရေတီတိ ကုလလော=ပက္ခိဇာတိ။}

\sutta{1036}{186}{မုဠာလာဒယော။}
\vutti{မုဠာလအာဒယော ကာလန္တာ နိပစ္စန္တေ။ မီလ-နိမီလနေ၊ ဥတ္တဠတာနိ၊ ဥဒ္ဓဋမတ္တေ နိမီလတီတိ မုဠာလံ=ဘိသံ။ ဗလ-ပါဏနေ၊ ဣတ္တ- ဠတ္တာနိ ၊ မူသိကာဒိခါဒနေန ဗလတိ ဇီဝတီတိ ဗိဠာလော=မဇ္ဇာရော။ ကပ္ပိဿ သံယောဂါဒိလောပေါ၊ ကပ္ပန္တိ ဇီဝိကံ ဧတေနာတိ ကပါလံ=ဃဋာဒိခဏ္ဍံ။ ပီ တပ္ပနေ။ “ယုဝဏ္ဏာနမိယငုဝင သရေ ”တိ (၅.၁၃၆) ဣယင၊ အတ္တနော ဖလေန သတ္တေ သန္တပ္ပေတီတိ ပိယာလော-ရုက္ခော။ ကုဏ-သဒ္ဒေ၊ ဝါတသမုဋ္ဌိတာ ဝီစိမာလာ ဧတ္ထ ကုဏန္တိ နဒန္တီတိ ကုဏာလော=ဧကော မဟာသရော။ ဝိသ-ပဝိသနေ၊ ပဝိသန္တိ ဧတ္ထာတိ ဝ္သာလော=ဝိတ္ထိဏ္ဏော။ ပလ-ဂမနေ၊ ဝါတေန ပလတိ ဂစ္ဆတီတိ ပလာလံ=သဿာနမုပနီတဓညာနံ နာဠပတ္တာနိ။ သရတိဿ သိဂေါ၊ သသာဒယော သရတိ ဟိံသတီတိ သိဂါလော=ကောတ္ထု၊ ဧဝမညေပိ။}

\sutta{1037}{187}{စဏ္ဍ ပတာ ဏာလော။}
\vutti{စဏ္ဍ-စဏ္ဍိကျေ၊ ပတ ပထ-ဂမနေ၊ ဧတေဟိ ဏာလော ဟောတိ။ စဏ္ဍေတိ ပီဠေတီတိ စဏ္ဍာလော=မာတင်္ဂေါ၊ ပတတိ အဓောဂစ္ဆတီတိ ပါတာလံ=ရသာတလံ။}

\sutta{1038}{188}{မာဒိတော လော။}
\vutti{မာ-မာနေ၊ ဣ-အဇ္ဈေနဂတီသု၊ ပီ-တပ္ပနေ၊ ဒူ-ပရိတာပေ၊ ဧဝမာဒီဟိ လော ဟောတိ။ မီယတိ ပရိမီယတီတိ မာလာ=ပန္တိ။ ဧတိ ဂစ္ဆတီတိ ဧလာ=သုခုမေလာ။ ပိဏေတိ တပ္ပေတိ ဧတ္ထာတိ ပေလာ=အာသိတ္တကူပဓာနံ။ ဒူယတိ ပရိတာပေတီတိ ဒေါလာ=ကီဠနယာနကံ။ ကလသင်္ချာနေ၊ ကလနံ ကလ္လံ=ယုတ္တံ။}

\sutta{1039}{189}{အန သလ ကလ ကုက သဌ မဟာ ဣလော။}
\vutti{အန-ပါဏနေ၊ သလ-ဂမနေ၊ ကလ-သင်္ချာနေ၊ ကုက ဝက-အာဒနေ၊ သဌ-ကိတဝေ၊ အရဟ မဟ-ပူဇာယံ၊ ဧတေဟိ ဣလော ဟောတိ။ အနတိ ပဝတ္တတီတိအနိလော=မာလုတော။ သလတိ ဂစ္ဆတီတိ သလိလံ=ဇလံ။ ကလတိ ပဝတ္တတီတိ ကလိလံ=ဂဟနံ။ ကုကတိ အတ္တနော နာဒေန သတ္တာနံ မနံ ဂဏှာတီတိ ကောကိလော=ပရပုဏ္ဍော။ သဌတိ ဝဉ္စေတီတိ သဌိလော=သဌော။ မဟီယတိ ပူဇီယတီတိ မယိလာ=ဣတ္ထီ။}

\sutta{1040}{190}{ကုဋာ ကိလော။}
\vutti{ကုဋ-ကောဋိလျေတီမသ္မာ ကိလော ဟောတိ။ အကုဋိ ကုဋိလတ္တမဂမီတိ ကုဋိလော=ဝင်္ကော။}

\sutta{1041}{191}{သိထိလာဒယော။}
\vutti{သိထိလအာဒယော ကိလန္တာ နိပစ္စန္တေ။ သဟ ခမာယံ၊ သဟိဿ သိထတ္တံ၊ သဟိတုမလန္တိ သိထိလံ=အဒဠှံ။ ကမ္ပိဿ သံယောဂါဒိလောပေါ၊ ပရဒုက္ခေ သတိ ကမ္ပတီတိ ကပိလော=ဣသိ။ ကဗ-ဝဏ္ဏေ၊ ဗဿ ပေါ၊ အကဗိ နီလာဒိဝဏ္ဏတ္တမဂမီတိ ကပိလော=ဝဏ္ဏဝိသေသော။ မထိဿ မိထော၊ မထီယတီတိ မိထိလာ=ပူရီ၊ ဧဝမညေပိ။}

\sutta{1042}{192}{စဋ ကဏ္ဍ ဝဋ္ဋ ပုထာ ကုလော။}
\vutti{စဋ-ဘေဒနေ၊ ကဏ္ဍ-ဆေဒနေ၊ ဝဋ္ဋ-ဝတ္တနေ၊ ပုထ ပထ-ဝိတ္ထာရေ၊ ဧတေဟိ ကုလော ဟောတိ။ စဋတိ မိတ္တေ ဘိန္ဒတီတိ စဋုလော=စာဋုကာရီ။ ကဏ္ဍီယတိ ဆိန္ဒီယတီတိ ကဏ္ဍုလော=ရုက္ခော။ ဝဋ္ဋတီတိ ဝဋ္ဋုလော=ပရိမဏ္ဍလော။ အပတ္ထရီတိ ပုထုလော=ဝိတ္ထာရော။}

\sutta{1043}{193}{တုမုလာဒယော။}
\vutti{တုမုလအာဒယော ကုလန္တာ နိပစ္စန္တေ။ တမ ခေဒနေ၊ အဿု၊ အတမိ ဝိတ္ထိဏ္ဏတ္တမဂမီတိ တုမုလော=ပတ္ထဋော။ တမ္ဘိဿ ဍုက၊ တမီယတိ ဝိကာရမာပါဒီယတီတိ တဏ္ဍုလော=ဝီဟိသာရော။ နိပုဗ္ဗဿ စိနာတိဿ ဣလောပေါ၊ အတ္ထိကေဟိ နိစီယတေတိ နိစုလော=ဟိဇ္ဇလော၊ ဧဝမညေပိ။}

\sutta{1044}{194}{ကလ္လ ကပ တက္က ပဋာ ဩလော။}
\vutti{ကလ္လ-သဒ္ဒေ၊ ကပ-အစ္ဆာဒနေ၊ တက္က-ဝိတက္ကေ၊ ပဋ-ဂမနေ၊ ဧတေဟိ ဩလော ဟောတိ။ ဝါတဝေဂေန သမုဒ္ဒတော ဥဋ္ဌဟိတွာ ကလ္လတိ နဒတီတိ ကလ္လောလော=မဟာဝီစိ။ ကပတိ ဒန္တေ အစ္ဆာဒတီတိ ကပေါလော=ဝဒနေကဒေသော။ တက္ကီယတီတိ တက္ကောလံ=ကောလကံ။ ပဋတိ ဗျာဓိမေတေန ဂစ္ဆတီတိ ပဋောလော=တိတ္တကော။}

\sutta{1045}{195}{အင်္ဂါ ဥလောလိ။}
\vutti{အင်္ဂ-ဂမနတ္ထော၊ ဧတသ္မာ ဥလဥလီ ဟောန္တိ။ အင်္ဂန္တိ ဧတေန ဇာနန္တီတိ အင်္ဂုလံ=ပမာဏံ။ အင်္ဂတိ ဥဂ္ဂစ္ဆတီတိ အင်္ဂလိ=ကရသာခါ။}

\sutta{1046}{196}{အဉ္ဇာလိ။}
\vutti{အဉ္ဇ-ဗျတ္တိ မက္ခန ဂတိ ကန္တီသု၊ ဧတသ္မာ အလိ ဟောတိ။ အဉ္ဇေတိ ဘတ္တိမနေန ပကာသေတီတိ အဉ္ဇလိ=ကရပုဋော။}

\sutta{1047}{197}{ဆဒါ လိ။}
\vutti{ဆဒ-သံဝရဏေ၊ ဧတသ္မာ လိ ဟောတိ။ ဆာဒေတီတိ ဆလ္လီ=သကလိကာ။}

\sutta{1048}{198}{အလ္လျာဒယော။}
\vutti{အလ္လိအာဒယော လိအန္တာ နိပစ္စန္တေ။ အရ-ဂမနေ၊ အရတိ ပဝတ္တတီတိ အလ္လိ=ရုက္ခော။ နယတိဿ ဧတ္တာဘာဝေါ၊ အတ္တိကေဟိ နီယတီတိ ၄ နီလိ၊ ငီမှိ နီလီ=ဂစ္ဆဇာတိ။ “သရမှာ ဒွေ ”တိ (၁.၃၄) လဿ ဒွိဘာဝေ ရဿတ္တေ စ နိလ္လီတိပိ ဟောတိ။ ပါလိဿ ပါ၊ ပါလေတိ ရက္ခတီတိ ပါလိ၊ ငီမှိ ပါလီ=ပန္တိ။ ပါလိဿ ပလော၊ ပါလေတိ ရက္ခတီတိ ပလ္လိ=ကုဋိ။ စုဒ-စောဒနေ၊ ဩတ္တာဘာဝေါ၊ စောဒီယတီတိ စုလ္လိ=ဥဒ္ဓနံ၊ ဧဝမညေပိ။}

\sutta{1049}{199}{ပိလာဒီဟျဝေါ။}
\vutti{ပိလ-ဝတ္တနေ၊ ပလ္လ-ဂမနေ၊ ပဏ-ဗျဝဟာရထုတီသု၊ ဧဝမာဒီဟိ အဝေါ ဟောတိ။ ပိလျတေတိ ပေလဝေါ=လဟု။ ပလ္လတီတိ ပလ္လဝေါ=ကိသလယံ။ ပဏီယတီတိ ပဏဝေါ=မုဒင်္ဂေါ။ ဧဝမညေပိ။}

\sutta{1050}{200}{သာဠဝါဒယော။}
\vutti{သာဠဝအာဒယော အဝန္တာ နိပစ္စန္တေ။ သလ-ဂမနတ္ထော၊ ဥပန္တဿ ဒီဃော ဠတ္တဉ္စ နိပါတနာ။ သလတိ ပဝတ္တတီတိ သာဠ ဝေါ=အဘိသင်္ခတံ ဗဒရာဒိဖလခါဒနီယံ။ ကိတ-နိဝါသေ၊ ဧတ္တာဘာဝေါ၊ ကေတတီတိ ကိတဝေါ=ဇူတကာရော၊ စောရော စ။ မူ-ဗန္ဓနေ၊ ဦဿ ရဿတ္တံ၊ တုဉ စာဝဿ၊ မုနာတိ ဗန္ဓတီတိ မုတဝေါ=စဏ္ဍာလော။ ဝလ ဝလ္လ-သံဝရဏေ၊ ဠတ္တံ၊ ဝလတိ၊ ဝလျတေတိ ဝါ ဝဠဝါ=တုရင်္ဂကန္တာ။ မုရ-သံဝေဠနေ၊ မုရီယတီတိ မုရဝေါ=မုဒင်္ဂေါ၊ ဧဝမညေပိ။}

\sutta{1051}{201}{သရာ အာဝေါ။}
\vutti{သရတိသ္မာ အာဝေါ ဟောတိ။ သရတိ ပဝတ္တတီတိ သရာဝေါ=ဘာဇနဝိသေသော။}

\sutta{1052}{202}{အလ မလ ဗိလာ ဏုဝေါ။}
\vutti{အလ-ဗန္ဓနေ ၊ မလ မလ္လ-ဓာရဏေ၊ ဗိလ-ဘေဒနေ၊ ဧတေဟိ ဏုဝေါ ဟောတိ။ လတာဟိ အလီယတီတိ အာလုဝေါ=ဂစ္ဆဇာတိ။ မလတိ ဓာရေတီတိ မာလုဝေါ=ပတ္တလတာ။ ဗိလတိ ဘိန္ဒတီတိ ဗေလုဝေါ=ရုက္ခော။}

\sutta{1053}{203}{ဂါတွီ ဝေါ။}
\vutti{ကာ ဂါ-သဒ္ဒေတီမသ္မာ ဤဝေါ ဟောတိ။ ဂါယန္တိ ဧတာယာတိ ဂီဝါ=ဂလော။}

\sutta{1054}{204}{သုတော ကွတွာ။}
\vutti{သု-သဝနေတီမသ္မာ ကွ ကွာ ဟောန္တိ။ သုဏာတီတိ သုဝေါ=ကီရော။ သုဝါ=သုဏော။}

\sutta{1055}{205}{ဝိဒွါ။}
\vutti{ဝိဒတိသ္မာ ကွာ ပရရူပဗာဓနတ္ထံ။ “ဗျဉ္ဇနေ ဒီဃရဿာ ”တိ (၁.၃၉) ရဿတ္တံ၊ ဝိဒတိ ဇာနာတီတိ ဝိဒွါ=ဝိဒူ။}

\sutta{1056}{206}{ထုတော ရေဝေါ။}
\vutti{ထု-အဘိတ္ထဝေ၊ ဧတသ္မာ ရေဝေါ ဟောတိ။ ထဝတိ သိဉ္စတီတိ ထေဝေါ=ဖုသိတံ။}

\sutta{1057}{207}{သမာ ရိဝေါ။}
\vutti{သမ-ဥပသမေ၊ ဧတသ္မာ ရိဝေါ ဟောတိ။ သမေတိ ဥပသမေတီတိ သိဝေါ=ဥမာပတိ၊ သိဝါ=သိဂါလော၊ သိဝံ=သန္တိ။}

\sutta{1058}{208}{ဆဒါ ရဝိ။}
\vutti{ဆဒ-သံဝရဏေ ၊ ဧတသ္မာ ရဝိ ဟောတိ။ ဆာဒေတီတိ ဆဝိ=ဇုတိ။}

\sutta{1059}{209}{ပူရ တိမာ ကိသော ရဿော စ။}
\vutti{ပူရ-ပူရဏေ၊ တိမ-တေမနေ၊ ဧတေဟိ ကိသော ဟောတိ ဦဿ ရဿော စ။ ပူရေတီတိ ပုရိသော=ပုမာ။ (ပုရေ ဥစ္စဋ္ဌာနေ သေတိ ပဝတ္တတီတိ ဝါ ပုရိသော=သောဝ။) တေမေတီတိ တိမိသံ=တမော။}

\sutta{1060}{210}{ကရာ ဤသော။}
\vutti{ကရောတိသ္မာ ဤသော ဟောတိ။ ကရီယတီတိ ကရီသံ=ဂူထံ။}

\sutta{1061}{211}{သိရီသာဒယော။}
\vutti{သိရီသအာဒယော ဤသန္တာ နိပစ္စန္တေ။ သရတိဿ အဿိ၊ သပ္ပဒဋ္ဌကာလာဒီသု သရီယတီတိ သိရီသော=ရုက္ခော။ ပူရိဿ ရဿတ္တံ၊ ပူရေတီတိ ပုရီသံ=ဂူထံ။ တလိဿ ဒီဃော၊ တလတိ သတ္တာနံ ပတိဋ္ဌာနံ ဘဝတီတိ တာလီသံ=ဩသဓိဝိသေသော၊ ဧဝမညေပိ။}

\sutta{1062}{212}{ကရာ ရိဗ္ဗိသော။}
\vutti{ကရောတိသ္မာ ရိဗ္ဗိသော ဟောတိ။ ကရီယတီတိ ကိဗ္ဗိသံ=ပါပံ။}

\sutta{1063}{213}{သသာသ ဝသ ဝိသ ဟန ဝန မနာန ကမာ သော။}
\vutti{သသ-ဂတိ ဟိံသာ ဝိဿသ ပါဏနေသု၊ အသ-ခေပနေ၊ ဝသ-နိဝါသေ၊ ဝိသ-ပဝိသနေ၊ ဟန-ဟိံ သာယံ၊ ဝန သန-သမ္ဘတ္တိယံ၊ မန-ဉာဏေ၊ အန-ပါဏနေ၊ ကမ-ဣစ္ဆာယံ၊ ဧတေဟိ သော ဟောတိ၊ သသန္တိ ဇီဝန္တိ သတ္တာ ဧတေနာတိ သဿံ=ကလမာဒိ၊ အသတိ ခိပတီတိ အဿော=ဟယော ။ ဝသန္တိ ဧတ္ထာတိ ဝဿံ=သံဝစ္ဆရော။ ဝိသတီတိ ဝေဿော=တတိယဝဏ္ဏော။ ဟညတေတိ ဟံသော=သိတစ္ဆဒေါ။ ဝနောတိ ပတ္ထရတီတိ ဝံသော=သန္တာနော၊ ဝေဠု စ။ မညတေတိ မံသံ=ပိသိတံ၊ အနတိ ဇီဝတိ ဧတေနာတိ အံသော=ဧကာဋ္ဌာသော၊ ဘုဇသိရော စ။ ကာမီယတီတိ ကံသော=ပရိမာဏံ။}

\sutta{1064}{214}{အာမိ ထု ကု သီတော သက။}
\vutti{အာပုဗ္ဗော မိ-ပက္ခေပေ၊ ထု-အဘိတ္ထဝေ ကု-သဒ္ဒေ၊ သီ-သယေ၊ ဧတေဟိ သက ဟောတိ။ အာမီယတိ အန္တော ပက္ခိပီယတီတိ အာမိသံ=ဘက္ခံ။ ထဝီယတီတိ ထုသော=ဝီဟိတစော။ ကဝတိ ဝါတေန နဒတီတိ ကုသော=တိဏဝိသေသော။ သယန္တိ ဧတ္ထ ဦကာတိ သီသံ=မုဒ္ဓါ၊ ကာလတိပု စ။}

\sutta{1065}{215}{ဖဿာဒယော။}
\vutti{ဖဿအာဒယော သကအန္တာ နိပစ္စန္တေ။ ဖုသ-သမ္ဖဿေ၊ ဥဿတ္ထံ၊ ဖုသတီတိ ဖဿော=ကာယဝိညာဏဝိသယော။ ဖုဿော=နက္ခတ္တံ။ ပုသ ပေါသနေ၊ ပေါသီယတီတိ ပုဿံ=ဖလဝိသေသော။ ဘူ-သတ္တာယံ၊ ဘူဿ ရဿော၊ အဘဝီတိ ဘုသံ=တုစ္ဆဓညံ၊ အံကိဿ ဥက၊ အင်္ကေတိ အနေန အညေတိ အင်္ကုသော=ဂဇပတောဒေါ။ ဖာယ-ဝုဒ္ဓိယံ၊ ပပုဗ္ဗော၊ ယလောပေါ၊ ဖာယတိ ဝုဒ္ဓိံ ဂစ္ဆတီတိ ပပ္ဖာသံ=ဒေဟကောဋ္ဌာသဝိသေသော။ ကလိသ္မာ သဿ မာဉ၊ ကုလိသ္မာ စ၊ ကလီယတိ ပရိမီယတီတိ ကမ္မာသော=သဗလော၊ ကမ္မာသံ=ပါပံ။ ကုလတိ ပတ္ထရတီတိ ကုမ္မာသော=ဘက္ခဝိသေသော။ မနိဿ ဇူက၊ မညတိ သဓနတ္တံ ဧတာယာတိ မဉ္ဇူသာ=ကဋ္ဌပေဠာ။ ပီဿ ယူက၊ ပိဏေတီတိ ပီယူသံ=အမတံ။ ကုလ-သံဝရဏေ၊ ဣက၊ ကုလီယတိ သံဝရီယတီတိ ကုလိသံ=ဝဇိရံ ။ ဗလ-သံဝရဏေ၊ ဣက၊ လဿ ဠတ္တဉ္စ၊ ဗလတိ ဧတေန မစ္ဆေ ဂဏှာတီတိ ဗဠိသော=မစ္ဆဝေဓနံ။ မဟိဿ ဧက၊ မဟီယတီတိ မဟေသီ=ကတာဘိသေကာ ပဓာနိတ္ထီ၊ ဧဝမညေပိ။}

\sutta{1066}{216}{သုတော ဏိသက။}
\vutti{သုဏာတိသ္မာ ဏိသက ဟောတိ။ သုဏာတီတိ သုဏိသာ=ပုတ္တဘရိယာ။}

\sutta{1067}{217}{ဝေတာတ ယု ပနာလ ကလ စမာ အသော။}
\vutti{ဝေတ-သုတ္တိယော ဓာတု၊ အတ-သာတစ္စဂမနေ၊ ယု-မိဿနေ၊ ပနထုတိယံ၊ အလ-ဗန္ဓနေ၊ ကလ-သင်္ချာနေ၊ စမ-အဒနေ၊ ဧတေဟိ အသော ဟောတိ။ ဝေတတိ ပဝတ္တတီတိ ဝေတသော=ဝါနီရော။ အတတိ ဝါတေရိတော နိစ္စံ ဝေဓတ္တံ ယာတီတိ အတသော=ဝနပ္ပတိဝိသေသော ဝီမှိ အဘသီ=ဂစ္ဆဝိသေသော။ ယဝီယတိ မိဿီယတီတိ ယဝသော=ပသုဃာသော။ ပညတေ ထဝီယတေတိ ပနသော=ကဏ္ဍဏီဖလော။ အလီယတိ ဗန္ဓိယတီတိ အလသော=မန္ဒကာရီ။ ကလီယတီတိ ကလသော=ကုမ္ဘော။ စမတိ အဒတိ အနေနာတိ စမသော=ဟောမဘာဇနံ။}

\sutta{1068}{218}{ဝယ ဒိဝ ကရ ကရေဟျသဏသကပါသကသာ။}
\vutti{ဝယတျာဒီဟိ အသဏအာဒယော ဟောန္တိ ယထာက္ကမံ။ ဝယတိ ဂစ္ဆတီတိ ဝါယသော=ကာကော။ ဒိဗ္ဗန္တိ ဧတ္ထာတိ ဒိဝသော=ဒိနံ။ ကရီယတီတိ ကပ္ပာသော=သုတ္တသမ္ဘဝေါ။ ကိဗ္ဗိသံ ကရောတီတိ ကက္ကသော=ဖရုသော။}

\sutta{1069}{219}{သသ မသ ဒံသာသာ သု။}
\vutti{သသာဒီဟိ သု ဟောတိ။ သသတိ ဇီဝတီတိ သဿု=ဇယမ္ပတီနံ မာတာ။ မသီယတီတိ မဿု=ပုရိသမုခေ ပဝဒ္ဓလောမာနိ။ “လောပေါ ”တိ (၁.၃၉) နိဂ္ဂဟီတလောပေါ၊ ဒံသီယတိ ဗန္ဓမနေနာတိ ဒဿု=စောရော။ အသီယတိ ခိပီယတီတိ အဿု=ဗပ္ပော။}

\sutta{1070}{220}{ဝိဒါ ဒသုက။}
\vutti{ဝိဒိသ္မာ ဒသုက ဟောတိ။ ဝိဒတိ ဇာနာတီတိ ဝိဒ္ဒသု=ဝိဒွါ။}

\sutta{1071}{221}{သသာ ရီဟော။}
\vutti{သသတိသ္မာ ရီဟော ဟောတိ။ သသတိ ဟိံသတီတိ သီဟော=ကေသရီ။}

\sutta{1072}{222}{ဇီဝါမာ ဟော ဝမာ စ။}
\vutti{ဇီဝ-ပါဏဓာရဏေ၊ အမ-ဂမနေ၊ ဧတေဟီ ဟော ဟောတိ၊ ဝမာ စာန္တာဒေသာ ယထာက္ကမံ၊ အာဒေသဝိဓာနံ ပန ပရရူပဗာဓနတ္ထံ။ “ဗျဉ္ဇနေ ဒီဃရဿာ ”တိ (၁.၃၉) ရဿတ္တံ၊ ဇီဝန္တိ ဧတာယာတိ ဇိဝှာ=ရသနာ။ အမတိ ပဝတ္တတီတိ အမှံ=အသ္မာ။ ပပုဗ္ဗေ အမတိ ပဝတ္တတီတိ ပမှံ=ပခုမံ။}

\sutta{1073}{223}{တဏှာဒယော။}
\vutti{တဏှအာဒယော ဟန္တာ နိပစ္စန္တေ။ တသ-ပိပါသာယံ၊ သဿ ဏတ္တံ၊ ဧဝမုပရိ စ၊ တသတိ ပါတုမိစ္ဆတိ ဧတာယာတိ တဏှာ=လော ဘော။ ကသ-ဝိလေခနေ၊ ကသတီတိ ကဏှော=ကာဠော။ ဇုတ-ဒိတ္တိယံ၊ တဿ ဏတ္တံ၊ ဩတ္တာဘာဝေါ စ၊ ဇောတေတီတိ ဇုဏှာ=စန္ဒ- ပဘာ ။ မီလိဿ ဠော၊ နိမီလန္တျနေန အက္ခီနီတိ မီဠှံ-ဂူထံ။ ဂါဟိဿ ဠော၊ ဂယှတီတိ ဂါဠှံ၊ ဒဟိဿ ဠော၊ ဒဟတီတိ ဒဠှံ၊ ဗဟိဿ ဠော၊ ဒီဃော စ၊ ဗဟတိ ဝုဒ္ဓိံ ဝစ္ဆတီတိ ဗာဠှံ၊ ဧတေ တယော ဒဠှတ္ထာ။ ဂမိဿ အဿိ၊ ဂစ္ဆတီတိ ဂိမှော-နိဒါဃော။ ပဋကလာနံ အက စ၊ ပဋတိ ယာတီတိ ပဋဟော=ဘေရိဝိသေသော။ ကလီယတိ ပရိမီယတိ အနေန သူရဘာဝေါတိ ကလဟော=ဝိဝါဒေါ။ ကဋဝရာနံ အာက၊ ကဋန္တိ ဧတ္ထ ဩသဓာဒိံ မဒ္ဒန္တီတိ ကဋာဟော=ဘာဇနဝိသေသော။ ဝရီယတီတိ ဝရာဟော=သူကရော။ လုနာတိဿ ဩ၊ လုနာတိ ဧတေနာတိ လောဟံ=အယာဒိ။ ဧဝမညေပိ။}

\sutta{1074}{224}{ပဏုဿဟာ ဟိဟီ ဏောဠင စ။}
\vutti{ပဏာ ဥပုဗ္ဗသဟာ စ ဟိဟီ ဟောန္တိ ယထာက္ကမံ၊ ဏဩဠင စာန္တာဒေသာ၊ အာဒေသဝိဓာနသာမတ္ထိယာ ပရရူပါဘာဝေါ၊ ပဏီယတိ ဝေါဟရီယတီတိ ပဏှိ=ပါဒဿ ပစ္ဆာဘာဂေါ။ ဥဿဟတီတိ ဥဿာဠှီ-ဝီရိယံ။}

\sutta{1075}{225}{ခီ မိ ပီ စု မာ ဝါကာဟိ ဠော ဥဿ ဝါ ဒီဃော စ။}
\vutti{ခီ-ခယေ၊ မိ-ပက္ခေပေ၊ ပီ-တပ္ပနေ၊ စု-စဝနေ၊ မာ-မာနေ၊ ဝီ ဝါ-ဂမနေ၊ ကာ ဂါ-သဒ္ဒေ၊ ဧတေဟိ ဠော ဟောတိ၊ ဥကာရဿ ဝါ ဒီဃော စ။ ခယီယတီတိ ခေလော=လာလာ။ မီယတိ ပက္ခိပီယတီတိ မေဠာ=မသိ။ ပိဏေတီတိ ပေဠာ=ဘာဇနဝိသေသော။ စဝတီတိ စူဠာ=သိခါ။ စောဠော=ပိလောတိကော။ မီယတိ ပရိမီယတီတိ မာဠော=ဧကကူဋသင်္ဂဟိတော အနေကကောဏဝန္တော ပဋိဿယဝိသေသော။ ဝါတိ ဂစ္ဆတီတိ ဝါဠော=စဏ္ဍမိဂေါ။ ကာယတိ ဖရုသံ ဝဒတီတိ ကာဠော=ကဏှော၊ ဝီမှိ ကာဠီ=ကဏှာ။}

\sutta{1076}{226}{ဂုတော ဠက စ။}
\vutti{ဂု-သဒ္ဒေတီမသ္မာ ဠက ဟောတိ ဠော စ။ ဂဝတိ ပဝတ္တတိ ဧတေနာတိ ဂုဠော=ဥစ္ဆုဝိကာရော။ ဂေါဠော=လကုဏ္ဍကော။}

\sutta{1077}{227}{ပင်္ဂုဠာဒယော။}
\vutti{ပင်္ဂုဠအာဒယော ဠက အန္တာ နိပစ္စန္တေ။ ခဉ္ဇ-ဂတိဝေကလ္လေ၊ ပင်္ဂုအာဒေသော၊ အခဉ္ဇိ ဂတိဝေကလ္လမာပဇ္ဇီတိ ပင်္ဂဠော=ပီဌသပ္ပီ။ ကရောတိသ္မာ ဠဿ ခဉ၊ ကိဗ္ဗိသံ ကရောတီတိ ကက္ခဠော=ကုရူရော။ ကုကတိဿ ကုက၊ ကုကျတိ ပါပကာရီဟိ အာဒီယတီတိ ကုက္ကုဠံ=သင်္ကု သံကိဏ္ဏော သောဗ္ဘော။ ကုက္ကုဠော=ထုသဂ္ဂီ။ မံကိဿ ဥက၊ ဗိန္ဒု လောပေါ စ၊ မံကေတိ ဝနံ မဏ္ဍေတီတိ မကုဠော=အဝိကသိတကုသုမံ။}

\sutta{1078}{228}{ပါတော ဠိ။}
\vutti{ပါတိသ္မာ ဠိ ဟောတိ။ အတ္ထံ ပါတိ ရက္ခတီတိ ပါဠိ=တန္တိ။}

\sutta{1079}{229}{ဝီတော ဠု။}
\vutti{ဝီတိသ္မာ ဠု ဟောတိ။ ဝေတိ ပဝတ္တတီတိ ဝေဠု=ဝေဏု။}

\begin{jieshu}
ဣတိ အဝဂ္ဂပစ္စယဝိဓာနံ။
\end{jieshu}


\begin{jieshu}
ဣတိ မောဂ္ဂလ္လာနေ ဗျာကရဏေ ဝုတ္တိယံ

ဏွာဒိကဏ္ဍော သတ္တမော။
\end{jieshu}

\chapter*{နိဂမကထာ}
\addcontentsline{toc}{chapter}{နိဂမကထာ}

\begin{song}
သုတ္တံ ဓာတု ဂဏော ဏွာဒိ၊ နာမလိင်္ဂါနုသာသနံ။\\
ယဿ တိဋ္ဌတိ ဇီဝှဂ္ဂေ၊ သ ဗျာကရဏကေသရီ။
\end{song}

\begin{jieshu}
သမတ္တာ စာယံ မောဂ္ဂလ္လာနဝုတ္တိ

သတ္တဟိ ဘာဏဝါရေဟိ။
\end{jieshu}


\begin{song}
၁.ယဿ ရညော ပဘာဝေန၊ ဘာဝိတတ္တယမာကုလံ။\\
အနာကုလံ ဒုလဒ္ဓီဟိ၊ ပါပဘိက္ခူဟိ သဗ္ဗသော။

၂.လင်္ကာယ မုနိရာဇဿ၊ သာသနံ သာဓု သဏ္ဌိကံ။\\
ပုဏ္ဏစန္ဒသမာယောဂါ၊ ဝါရိဓီဝ ဝိဝဒ္ဓတေ။

၃.ပရက္ကမဘုဇေ တသ္မိံ၊ သဒ္ဓါဗုဒ္ဓိဂုဏောဒိတေ။\\
မနုဝံသဒ္ဓဇာကာရေ၊ လင်္ကာဒီပံ ပသာသတိ။

၄.မောဂ္ဂလ္လာနေန ထေရေန၊ ဓီမတာ သုစိဝုတ္တိနာ။\\
ရစိတံ ယံ သုဝိညေယျ-မသန္ဒိဒ္ဓ’မနာကုလံ။

၅.အသေသဝိသယဗျာပိ၊ ဇိနဗျပ္ပထ နိဿယံ။\\
သဒ္ဓသတ္ထ’မနာယာသ-သာဓိယံ ဗုဒ္ဓိဝဒ္ဓနံ။

၆.တဿ ဝုတ္တိ သမာသေန၊ ဝိပုလတ္ထပကာသနီ။\\
ရစိတာ ပုန တေနေဝ၊ သာသနုဇ္ဇောတကာရိနာတိ။
\end{song}


\begin{jieshu}
မောဂ္ဂလ္လာနဗျာကရဏံ နိဋ္ဌိတံ။
\end{jieshu}


% 附录部分(注释状态,按需启用)
%\appendix % 附录开始
%\include{appendixA} % 语法术语表
%\include{appendixB} % 语法规则索引
%\include{appendixC} % 经典例句汇编

% 后记部分(无页码编号)
\backmatter

% 后记章节(注释状态,按需启用)
%\include{epilogue} % 后记内容

% 参考文献系统
%\nocite{*} % 引用所有文献(注释状态)
\printbibliography % 打印参考文献(使用biblatex)
\markboth{参考文献}{参考文献} % 页眉修复
\addcontentsline{toc}{chapter}{参考文献} % 添加到目录

% 索引系统
\printindex % 利用makeindex工具生成索引

%%%%%%%%%%%%%%%%%%%%%%%%%%%%%%%%%%%%%%%%%%%%%%%%%%%%%%%%%%%%%%%

\end{document}