\chapter*{Moggallānasuttapāṭho}
\addcontentsline{toc}{chapter}{Moggallānasuttapāṭho}
\kaishi

\section{saññādikaṇḍo paṭhamo}
\markboth{Moggallānasuttapāṭhe}{saññādikaṇḍo paṭhamo}
\begin{suttalist}
\suttaitem{1}{1}{aādayo titālīsa vaṇṇā.}
\suttaitem{2}{2}{dasā-do sarā.}
\suttaitem{3}{3}{dvedve savaṇṇā.}
\suttaitem{4}{4}{pubbo rasso.}
\suttaitem{5}{5}{paro dīgho.}
\suttaitem{6}{6}{kādayo byañjanā.}
\suttaitem{7}{7}{pañca pañcakā vaggā.}
\suttaitem{8}{8}{bindu niggahītaṃ.}
\suttaitem{9}{9}{iyuvaṇṇā jhalā nāmassante.}
\suttaitem{10}{10}{pitthiyaṃ.}
\suttaitem{11}{11}{ghā.}
\suttaitem{12}{12}{go syālapane.}
\begin{jieshu}   Iti saññā.
\end{jieshu}
\suttaitem{13}{13}{vidhibbisesanantassa.}
\suttaitem{14}{14}{sattamiyaṃ pubbassa.}
\suttaitem{15}{15}{pañcamiyaṃ parassa.}
\suttaitem{16}{16}{ādissa.}
\suttaitem{17}{17}{chaṭṭhiyantassa.}
\suttaitem{18}{18}{ṅānubandho.}
\suttaitem{19}{19}{ṭānubandhā-nekavaṇṇā sabbassa.}
\suttaitem{20}{20}{ña kānubandhādyantā.}
\suttaitem{21}{21}{mānubandho sarānamantā paro.}
\suttaitem{22}{22}{vippaṭisedhe.}
\suttaitem{23}{23}{saṅketo-navayavo-nubandho.}
\suttaitem{24}{24}{vaṇṇaparena savaṇṇo-pi.}
\suttaitem{25}{25}{ntu vantumantāvantutavantusambandhī.}
\begin{jieshu}   Iti paribhāsā.
\end{jieshu}
\suttaitem{26}{26}{saro lopo sare.}
\suttaitem{27}{27}{paro kvaci.}
\suttaitem{28}{28}{na dve vā.}
\suttaitem{29}{29}{yuvaṇṇānameo luttā.}
\suttaitem{30}{30}{yavā sare.}
\suttaitem{31}{31}{eonaṃ.}
\suttaitem{32}{32}{gossāvaṅa.}
\suttaitem{33}{33}{byañjane dīgharassā.}
\suttaitem{34}{34}{saramhā dve.}
\suttaitem{35}{35}{catutthadutiyesvesaṃ tatiyapaṭhamā.}
\suttaitem{36}{36}{vī-tisse-ve vā.}
\suttaitem{37}{37}{eonama vaṇṇe.}
\suttaitem{38}{38}{niggahītaṃ.}
\suttaitem{39}{39}{lopo.}
\suttaitem{40}{40}{parasarassa.}
\suttaitem{41}{41}{vagge vagganto.}
\suttaitem{42}{42}{yevahisu ño.}
\suttaitem{43}{43}{ye saṃssa.}
\suttaitem{44}{44}{mayadā sare.}
\suttaitem{45}{45}{va-na-ta-ra-gā cā-gamā.}
\suttaitem{46}{46}{chā ḷo.}
\suttaitem{47}{47}{tadaminādīni.}
\suttaitem{48}{48}{tavagga-va-ra-ṇānaṃ ye cavagga-ba-ya-ñā.}
\suttaitem{49}{49}{vaggalasehi te.}
\suttaitem{50}{50}{hassa vipallāso.}
\suttaitem{51}{51}{ve vā.}
\suttaitem{52}{52}{tathanarānaṃ ṭaṭhaṇalā.}
\suttaitem{53}{53}{saṃyogādi lopo.}
\suttaitem{54}{54}{vicchābhikkhaññesu dve.}
\suttaitem{55}{55}{syādilopo pubbasse-kassa.}
\suttaitem{56}{56}{sabbādīnaṃ vītihāre.}
\suttaitem{57}{57}{yāvabodhaṃ sambhame.}
\suttaitem{58}{58}{bahulaṃ.}
\end{suttalist}
\begin{jieshu}
iti moggallāne byākaraṇe saññādikaṇḍo paṭhamo.
\end{jieshu}

\section{syādikaṇḍo dutiyo}
\markboth{Moggallānasuttapāṭhe}{ syādikaṇḍo dutiyo}
\begin{suttalist}
\suttaitemmulti{59}{1}{dve dvekā-nekesu nāmasmā si yo, aṃ yo, nā hi, sa naṃ, smā hi, sa naṃ, smiṃ su.}
\suttaitem{60}{2}{kamme dutiyā.}
\suttaitem{61}{3}{kāladdhānamaccantasaṃyoge.}
\suttaitem{62}{4}{gatibodhāhārasaddatthākammakabhajjādīnaṃ payojje.}
\suttaitem{63}{5}{harādīnaṃ vā.}
\suttaitem{64}{6}{na khādādīnaṃ.}
\suttaganaitem{65}{1}{vahissā-niyantuke.}
\suttaganaitem{66}{2}{bhakkhissāhiṃsāyaṃ.}
\suttaitem{67}{7}{dhyādīhi yuttā.}
\suttaitem{68}{8}{lakkhaṇitthambhūtavicchāsvabhinā.}
\suttaitem{69}{9}{patiparīhi bhāge ca.}
\suttaitem{70}{10}{anunā.}
\suttaitem{71}{11}{sahatthe.}
\suttaitem{72}{12}{hīne.}
\suttaitem{73}{13}{upena.}
\suttaitem{74}{14}{sattamyādhikye.}
\suttaitem{75}{15}{sāmitte-dhinā.}
\suttaitem{76}{16}{kattukaraṇesu tatiyā.}
\suttaitem{77}{17}{sahatthena.}
\suttaitem{78}{18}{lakkhaṇe.}
\suttaitem{79}{19}{hetumhi.}
\suttaitem{80}{20}{pañcamīṇe vā.}
\suttaitem{81}{21}{guṇe.}
\suttaitem{82}{22}{chaṭṭhī hetvatthehi.}
\suttaitem{83}{23}{sabbādito sabbā.}
\suttaitem{84}{24}{catutthī sampadāne.}
\suttaitem{85}{25}{tādatthye.}
\suttaitem{86}{26}{pañcamyavadhismā.}
\suttaitem{87}{27}{apaparīhi vajjane.}
\suttaitem{88}{28}{paṭinidhipaṭidānesu patinā.}
\suttaitem{89}{29}{rite dutiyā ca.}
\suttaitem{90}{30}{vinā-ññatra tatiyā ca.}
\suttaitem{91}{31}{puthanānāhi.}
\suttaitem{92}{32}{sattamyādhāre.}
\suttaitem{93}{33}{nimitte.}
\suttaitem{94}{34}{yambhāvo bhāvalakkhaṇaṃ.}
\suttaitem{95}{35}{chaṭṭhī cā-nādare.}
\suttaitem{96}{36}{yato niddhāraṇaṃ.}
\suttaitem{97}{37}{paṭhamā-tthamatte.}
\suttaitem{98}{38}{āmantaṇe.}
\suttaitem{99}{39}{chaṭṭhī sambandhe.}
\suttaitem{100}{40}{tulyatthena vā tatiyā.}
\suttaitem{101}{41}{ato yonaṃ ṭāṭe.}
\suttaitem{102}{42}{ninaṃ vā.}
\suttaitem{103}{43}{smāsminnaṃ.}
\suttaitem{104}{44}{sassā-ya catutthiyā.}
\suttaitem{105}{45}{ghapate-kasmiṃ nādīnaṃ ya-yā.}
\suttaitem{106}{46}{ssā vā te-ti-māmūhi.}
\suttaitem{107}{47}{namhi nuka dvādīnaṃ sattarasannaṃ.}
\suttaitem{108}{48}{bahukatinnaṃ.}
\suttaitem{109}{49}{ṇṇaṃṇṇannaṃ tito jhā.}
\suttaitem{110}{50}{ubhinnaṃ.}
\suttaitem{111}{51}{suña sassa.}
\suttaitem{112}{52}{ssaṃ-ssā-ssāyesvi-tare-ka-ññe-timāna-mi.}
\suttaitem{113}{53}{tāya vā.}
\suttaitem{114}{54}{tetimāto sassa ssāya.}
\suttaitem{115}{55}{ratyādīhi ṭo smino.}
\suttaitem{116}{56}{suhisu-bhasso.}
\suttaitem{117}{57}{ltupitādīnamā simhi.}
\suttaitem{118}{58}{ge a ca.}
\suttaitem{119}{59}{ayūnaṃ vā dīgho.}
\suttaitem{120}{60}{ghabrahmādite.}
\suttaitem{121}{61}{nā-mmādīhi.}
\suttaitem{122}{62}{rasso vā.}
\suttaitem{123}{63}{gho ssaṃssāssāyaṃtiṃsu.}
\suttaitem{124}{64}{ekavacanayosvaghonaṃ.}
\suttaitem{125}{65}{ge vā.}
\suttaitem{126}{66}{sismiṃ nā-napuṃsakassa.}
\suttaitem{127}{67}{gossā-ga-si-hi-naṃsu gāva-gavā.}
\suttaitem{128}{68}{sumhi vā.}
\suttaitem{129}{69}{gavaṃ sena.}
\suttaitem{130}{70}{gunnaṃ ca naṃnā.}
\suttaitem{131}{71}{nāssā.}
\suttaitem{132}{72}{gāvumhi.}
\suttaitem{133}{73}{yaṃ pīto.}
\suttaitem{134}{74}{naṃ jhīto.}
\suttaitem{135}{75}{yonaṃ none pume.}
\suttaitem{136}{76}{no.}
\suttaitem{137}{77}{smiṃno ni.}
\suttaitem{138}{78}{ambādīhi.}
\suttaitem{139}{79}{kammādito.}
\suttaitem{140}{80}{nāsse-no.}
\suttaitem{141}{81}{jhalā sassa no.}
\suttaganaitem{142}{3}{ito kvaci sassa ṭānubandho.}
\suttaitem{143}{82}{nā smāssa.}
\suttaitem{144}{83}{lā yonaṃ vo pume.}
\suttaitem{145}{84}{jantādito no ca.}
\suttaitem{146}{85}{kūto.}
\suttaitem{147}{86}{lopo-musmā.}
\suttaitem{148}{87}{na no sassa.}
\suttaitem{149}{88}{yolopa-nisu dīgho.}
\suttaitem{150}{89}{sunaṃhisu.}
\suttaitem{151}{90}{pañcādīnaṃ cuddasannama.}
\suttaitem{152}{91}{yvādo ntussa.}
\suttaitem{153}{92}{ntassa ca ṭa vaṃse.}
\suttaitem{154}{93}{yosu jhissa pume.}
\suttaitem{155}{94}{vevosu lussa.}
\suttaitem{156}{95}{yomhi vā kvaci.}
\suttaitem{157}{96}{pumā-lapane vevo.}
\suttaitem{158}{97}{smāhisminnaṃ mhābhimhi.}
\suttaitem{159}{98}{suhisvasse.}
\suttaitem{160}{99}{sabbādīnaṃ naṃmhi ca.}
\suttaganaitemmulti{161}{4}{pubbaparāvaradakkhiṇuttarādharāni vavatthāyamasaññāyaṃ.}
\suttaitem{162}{100}{saṃsānaṃ.}
\suttaitem{163}{101}{ghapā sassa ssā vā.}
\suttaitem{164}{102}{smino ssaṃ.}
\suttaitem{165}{103}{yaṃ.}
\suttaitem{166}{104}{tiṃ sabhā-parisāya.}
\suttaitem{167}{105}{padādīhi si.}
\suttaitem{168}{106}{nāssa sā.}
\suttaitem{169}{107}{kodhādīhi.}
\suttaitem{170}{108}{atena.}
\suttaitem{171}{109}{sisso.}
\suttaitem{172}{110}{kvace vā.}
\suttaitem{173}{111}{aṃ napuṃsake.}
\suttaitem{174}{112}{yonaṃ ni.}
\suttaitem{175}{113}{jhalā vā.}
\suttaitem{176}{114}{lopo.}
\suttaitem{177}{115}{jantuhetvīghapehi vā.}
\suttaitem{178}{116}{ye passivaṇṇassa.}
\suttaitem{179}{117}{gasīnaṃ.}
\suttaitem{180}{118}{asaṅkhyehi sabbāsaṃ.}
\suttaitem{181}{119}{ekatthatāyaṃ.}
\suttaitem{182}{120}{pubbasmā-mādito.}
\suttaitem{183}{121}{nā-to-mapañcamiyā.}
\suttaitem{184}{122}{vā tatiyāsattamīnaṃ.}
\suttaitem{185}{123}{rājassi nāmhi.}
\suttaitem{186}{124}{sunaṃhisū.}
\suttaitem{187}{125}{imassānitthiyaṃ ṭe.}
\suttaitem{188}{126}{nāmha-ni-mi.}
\suttaitem{189}{127}{simha-napuṃsakassā-yaṃ.}
\suttaitem{190}{128}{tyatetānaṃ tassa so.}
\suttaitem{191}{129}{massā-mussa.}
\suttaitem{192}{130}{ke vā.}
\suttaitem{193}{131}{ta tassa no sabbāsu.}
\suttaitemmulti{194}{132}{ṭa sa-smā-smiṃ-ssāya-ssaṃ-ssā-saṃ-mhā-mhi-svi-massa ca.}
\suttaitem{195}{133}{ṭe sissisismā.}
\suttaitem{196}{134}{dutiyassa yossa.}
\suttaitem{197}{135}{ekaccādīha-to.}
\suttaitem{198}{136}{na nissa ṭā.}
\suttaitem{199}{137}{sabbādīhi.}
\suttaitem{200}{138}{yonameṭa.}
\suttaitem{201}{139}{nāññaṃ ca nāmappadhānā.}
\suttaitem{202}{140}{tatiyatthayoge.}
\suttaitem{203}{141}{catthasamāse.}
\suttaitem{204}{142}{veṭa.}
\suttaitem{205}{143}{pubbādīhi chahi.}
\suttaitem{206}{144}{manādīhi smiṃsaṃnāsmānaṃ sisoosāsā.}
\suttaganaitem{207}{5}{sumedhādīna-mavuddhica.}
\suttaganaitemmulti{208}{6}{sara-vayā-ya-vāsa-cetā jalā-saya-kkhaya-\\loha-paṭa-manesu.}
\suttaitem{209}{145}{sato saba bhe.}
\suttaitem{210}{146}{bhavato vā bhonto ga-yo-nāse.}
\suttaitem{211}{147}{sissaggito ni.}
\suttaitem{212}{148}{ntassaṃ.}
\suttaitem{213}{149}{bhūto.}
\suttaitem{214}{150}{mahantārahantānaṃ ṭā vā.}
\suttaitem{215}{151}{ntussa.}
\suttaitem{216}{152}{aṃṅaṃ napuṃsake.}
\suttaitem{217}{153}{himavato vā o.}
\suttaitem{218}{154}{rājādiyuvāditvā.}
\suttaganaitem{219}{7}{dhammo vā-ññatthe.}
\suttaganaitem{220}{8}{imo bhāve.}
\suttaitem{221}{155}{vā-mhā-naṅa.}
\suttaitem{222}{156}{yonamāno.}
\suttaitem{223}{157}{āyono ca sakhā.}
\suttaitem{224}{158}{ṭe smino.}
\suttaitem{225}{159}{nonāsesvi.}
\suttaitem{226}{160}{smānaṃsu vā.}
\suttaitem{227}{161}{yosvaṃhisu cāraṅa.}
\suttaitem{228}{162}{ltupitādīnamase.}
\suttaitem{229}{163}{naṃmhi vā.}
\suttaitem{230}{164}{ā.}
\suttaitem{231}{165}{salopo.}
\suttaitem{232}{166}{suhisvāraṅa.}
\suttaitem{233}{167}{najjā yosvāma.}
\suttaitem{234}{168}{ṭi katimhā.}
\suttaitem{235}{169}{ṭa pañcādīhi cuddasahi.}
\suttaitem{236}{170}{ubha-gohi ṭo.}
\suttaitem{237}{171}{āraṅasmā.}
\suttaitem{238}{172}{ṭoṭe vā.}
\suttaitem{239}{173}{ṭā nāsmānaṃ.}
\suttaitem{240}{174}{ṭi smino.}
\suttaitem{241}{175}{divādito.}
\suttaitem{242}{176}{rassāraṅa.}
\suttaitem{243}{177}{pitādīnamanatvādīnaṃ.}
\suttaitem{244}{178}{yuvādīnaṃ suhisvānaṅa.}
\suttaitem{245}{179}{nonānesvā.}
\suttaitem{246}{180}{smāsmiṃnaṃ nāne.}
\suttaitem{247}{181}{yonaṃ none vā.}
\suttaitem{248}{182}{ito-ññatthe pume.}
\suttaitem{249}{183}{ne smino kvaci.}
\suttaitem{250}{184}{pumā.}
\suttaitem{251}{185}{nāmhi.}
\suttaitem{252}{186}{sumhā ca.}
\suttaitem{253}{187}{gassaṃ.}
\suttaitem{254}{188}{sāssaṃ-se cānaṅa.}
\suttaitem{255}{189}{vattahā sanannaṃ nonānaṃ.}
\suttaitem{256}{190}{brahmassu vā.}
\suttaitem{257}{191}{nāmhi.}
\suttaitem{258}{192}{pumakammathāmaddhānaṃ vā sasmāsu ca.}
\suttaitem{259}{193}{yuvā sassino.}
\suttaitem{260}{194}{no-ttā-tumā.}
\suttaitem{261}{195}{suhisu naka.}
\suttaitem{262}{196}{smāssa nā brahmā ca.}
\suttaitem{263}{197}{ime-tāna-menā-nvādese dutiyāyaṃ.}
\suttaitem{264}{198}{kissa ko sabbāsu.}
\suttaitem{265}{199}{ki sasmiṃsu vā-nitthiyaṃ.}
\suttaitem{266}{200}{kimaṃsisu saha napuṃsake.}
\suttaitem{267}{201}{imassidaṃ vā.}
\suttaitem{268}{202}{amussāduṃ.}
\suttaitem{269}{203}{sumhā-mussā-smā.}
\suttaitem{270}{204}{naṃmhi ticatunnamitthiyaṃ tissa-catassā.}
\suttaitem{271}{205}{tisso catasso yomhi savibhattīnaṃ.}
\suttaitem{272}{206}{tīṇicattāri napuṃsake.}
\suttaitem{273}{207}{pume tayocattāro.}
\suttaitem{274}{208}{caturo vā catussa.}
\suttaitem{275}{209}{maya-masmā-mhāssa.}
\suttaitem{276}{210}{naṃ-sesva-smākaṃ-mamaṃ.}
\suttaitem{277}{211}{simha-haṃ.}
\suttaitem{278}{212}{tumhassa tuvaṃ tvamamhi ca.}
\suttaitem{279}{213}{tayātayīnaṃ tva vā tassa.}
\suttaitem{280}{214}{smāmhi tvamhā.}
\suttaitem{281}{215}{ntantūnaṃ nto yomhi paṭhame.}
\suttaitem{282}{216}{taṃ naṃmhi.}
\suttaitem{283}{217}{totātitā sasmāsmiṃnāsu.}
\suttaitem{284}{218}{ṭaṭāaṃ ge.}
\suttaitem{285}{219}{yomhi dvinnaṃ duve dve.}
\suttaitem{286}{220}{duvinnaṃ naṃmhi vā.}
\suttaitem{287}{221}{rājassa raññaṃ.}
\suttaitem{288}{222}{nāsmāsu raññā.}
\suttaitem{289}{223}{raññoraññassarājino se.}
\suttaitem{290}{224}{smiṃmhi raññerājini.}
\suttaitem{291}{225}{samāse vā.}
\suttaitem{292}{226}{smiṃmhi tumha-mhānaṃ tayi-mayi.}
\suttaitem{293}{227}{aṃmhi taṃ-maṃ-tavaṃ-mamaṃ.}
\suttaitem{294}{228}{nāsmāsu tayā-mayā.}
\suttaitem{295}{229}{tava-mama-tuyhaṃ-mayhaṃ se.}
\suttaitem{296}{230}{ṅaṃ-ṅākaṃ naṃmhi.}
\suttaitem{297}{231}{dutiye yomhi vā.}
\suttaitem{298}{232}{apādā-do padate-kavākye.}
\suttaitem{299}{233}{yonaṃhisva-pañcamyā vo-no.}
\suttaitem{300}{234}{teme nāse.}
\suttaitem{301}{235}{anvādese.}
\suttaitem{302}{236}{sapubbā paṭhamantā vā.}
\suttaitem{303}{237}{na ca-vā-ha-he-vayoge.}
\suttaitem{304}{238}{dassanatthe-nā-locane.}
\suttaitem{305}{239}{āmantaṇaṃ pubba-masantaṃ-va.}
\suttaitem{306}{240}{na sāmaññavacanamekatthe.}
\suttaitem{307}{241}{bahūsu vā.}
\end{suttalist}
\begin{jieshu}
iti moggallāne byākaraṇe syādikaṇḍo dutiyo.
\end{jieshu}

\section{samāsakaṇḍo tatiyo }
\markboth{Moggallānasuttapāṭhe}{samāsakaṇḍo tatiyo}
\begin{suttalist}
\suttaitem{308}{1}{syādi syādine-katthaṃ.}
\suttaitemmulti{309}{2}{asaṅkhyaṃ vibhatti-sampatti-samīpa-sākalyā-bhāva-yathā-pacchā-yugapadatthe.}
\suttaitem{310}{3}{yathā na tulye.}
\suttaitem{311}{4}{yāvā-vadhāraṇe.}
\suttaitem{312}{5}{payyapā-bahi-tiro-pure-pacchā vā pañcamyā.}
\suttaitem{313}{6}{samīpā-yāmesva-nu.}
\suttaitem{314}{7}{tiṭṭhagvādīni.}
\suttaitem{315}{8}{ore-pari-pati-pāre-majjhe-heṭṭhu-ddhā-dhonto vā chaṭṭhiyā.}
\suttaitem{316}{9}{taṃ napuṃsakaṃ.}
\suttaitem{317}{10}{amādi.}
\suttaitem{318}{11}{visesanamekatthena.}
\suttaitem{319}{12}{naña.}
\suttaitem{320}{13}{ku-pā-dayo nicca-masyā-dividhimhi.}
\suttaganaitem{321}{9}{pādayo gatādyatthe paṭhamāya.}
\suttaganaitem{322}{10}{accādayo kantādyatthe dutiyāya.}
\suttaganaitem{323}{11}{avādayo kuṭṭhādyatthe tatiyāya.}
\suttaganaitem{324}{12}{pariyādayo gilānādyatthe catutthiyā.}
\suttaganaitem{325}{13}{nyādayo kantādyatthe pañcamiyā.}
\suttaitem{326}{14}{cī kriyatthehi.}
\suttaitem{327}{15}{bhūsanā-darā-nādaresva-laṃ-sā-sā.}
\suttaitem{328}{16}{aññe ca.}
\suttaitem{329}{17}{vā-neka-ññatthe.}
\suttaitem{330}{18}{tattha gahetvā tena paharitvā yuddhe sarūpaṃ.}
\suttaitem{331}{19}{catthe.}
\suttaitem{332}{20}{samāhāre napuṃsakaṃ.}
\suttaitem{333}{21}{saṅkhyādi.}
\suttaitem{334}{22}{kvace-kattañca chaṭṭhiyā.}
\suttaitem{335}{23}{syādīsu rasso.}
\suttaitem{336}{24}{gha-passa-ntassā-ppadhānassa.}
\suttaitem{337}{25}{gossu.}
\suttaitem{338}{26}{itthiyamatvā.}
\suttaitem{339}{27}{nadādito ṅī.}
\suttaganaitem{340}{14}{goto vā.}
\suttaitem{341}{28}{yakkhāditvinī ca.}
\suttaitem{342}{29}{ārāmikādīhi.}
\suttaganaitem{343}{15}{saññāyaṃ mānuso.}
\suttaitem{344}{30}{yuvaṇṇehi nī.}
\suttaitem{345}{31}{ktimhā-ññatthe.}
\suttaitem{346}{32}{gharaṇyādayo.}
\suttaganaitem{347}{16}{ācariyā vā yalopo ca.}
\suttaitem{348}{33}{mātulāditvānī bhariyāyaṃ.}
\suttaganaitem{349}{17}{abhariyāyaṃ khattiyā vā.}
\suttaganaitem{350}{18}{punnāmasmā yogā apālakantā.}
\suttaitemmulti{351}{34}{upamā-saṃhita-sahita-saññata-saha-sapha-vāma-\\lakkhaṇāditū-rutū.}
\suttaitem{352}{35}{yuvā ti.}
\suttaitem{353}{36}{ntantūnaṃ ṅīmhi to vā.}
\suttaitem{354}{37}{bhavato bhoto.}
\suttaitem{355}{38}{gossā-vaṅa.}
\suttaitem{356}{39}{puthussa pathavaputhavā.}
\suttaitem{357}{40}{samāsantva.}
\suttaitem{358}{41}{pāpādīhi bhūmiyā.}
\suttaitem{359}{42}{saṅkhyāhi.}
\suttaitem{360}{43}{nadīgodāvarīnaṃ.}
\suttaitem{361}{44}{asaṅkhyehi cā-ṅgulyā-naññā-saṅkhyatthesu.}
\suttaitem{362}{45}{dīghā-ho-vasse-kadesehi ca rattyā.}
\suttaitem{363}{46}{gotvacatthe cālope.}
\suttaitem{364}{47}{rattindiva-dāragava-caturassā.}
\suttaitem{365}{48}{āyāme-nugavaṃ.}
\suttaitem{366}{49}{akkhismā-ññatthe.}
\suttaitem{367}{50}{dārumyaṅgulyā.}
\suttaitem{368}{51}{ci vītihāre.}
\suttaitem{369}{52}{lti-tthi-yūhi ko.}
\suttaitem{370}{53}{vā-ññato.}
\suttaitem{371}{54}{uttarapade.}
\suttaitem{372}{55}{imassidaṃ.}
\suttaitem{373}{56}{puṃ pumassa vā.}
\suttaitem{374}{57}{ṭa ntantūnaṃ.}
\suttaitem{375}{58}{a.}
\suttaitem{376}{59}{manā-dyā-pādīna-mo maye ca.}
\suttaitem{377}{60}{parassa saṅkhyāsu.}
\suttaitem{378}{61}{jane puthassu.}
\suttaitem{379}{62}{so chassā-hā-yatane vā.}
\suttaitem{380}{63}{ltupitādīna-māraṅa-raṅa.}
\suttaitem{381}{64}{vijjāyonisambandhānamā tatra catthe.}
\suttaitem{382}{65}{putte.}
\suttaitem{383}{66}{cismiṃ.}
\suttaitem{384}{67}{itthiyaṃ bhāsitapumitthī pumeve-katthe.}
\suttaitem{385}{68}{kvaci paccaye.}
\suttaitem{386}{69}{sabbādayo vuttimatthe.}
\suttaitem{387}{70}{jāyāya jayaṃ patimhi.}
\suttaitem{388}{71}{saññāya-mudo-dakassa.}
\suttaitem{389}{72}{kumbhādīsu vā.}
\suttaitem{390}{73}{sotādīsūlopo.}
\suttaitem{391}{74}{ṭa nañassa.}
\suttaitem{392}{75}{ana sare.}
\suttaitem{393}{76}{nakhādayo.}
\suttaitem{394}{77}{nago vā-ppāṇini.}
\suttaitem{395}{78}{sahassa soññatthe.}
\suttaitem{396}{79}{saññāyaṃ.}
\suttaitem{397}{80}{appaccakkhe.}
\suttaitem{398}{81}{akāle sakatthe.}
\suttaitem{399}{82}{ganthantā-dhikye.}
\suttaitem{400}{83}{samānassa pakkhādīsu vā.}
\suttaitem{401}{84}{udare iye.}
\suttaitem{402}{85}{rīrikkhakesu.}
\suttaitem{403}{86}{sabbādīnamā.}
\suttaitem{404}{87}{ntakimimānaṃ ṭākīṭī.}
\suttaitem{405}{88}{tumhā-mhānaṃ tā-me-kasmiṃ.}
\suttaitem{406}{89}{taṃ-ma-maññatra.}
\suttaitem{407}{90}{ve-tasse-ṭa.}
\suttaitem{408}{91}{vidhādīsu dvissa du.}
\suttaitem{409}{92}{di guṇādīsu.}
\suttaitem{410}{93}{tīsva.}
\suttaitem{411}{94}{ā saṅkhyāyā-satādo-naññatthe.}
\suttaitem{412}{95}{tisse.}
\suttaitem{413}{96}{cattālīsā-do vā.}
\suttaitem{414}{97}{dvissā ca.}
\suttaitem{415}{98}{bācattālīsā-do.}
\suttaitem{416}{99}{vīsatidasesu pañcassa paṇṇapannā.}
\suttaitem{417}{100}{catussa cuco dase.}
\suttaitem{418}{101}{chassa so.}
\suttaitem{419}{102}{ekaṭṭhānamā.}
\suttaitem{420}{103}{ra saṅkhyāto vā.}
\suttaitem{421}{104}{chatīhi ḷo ca.}
\suttaitem{422}{105}{catuttha-tatiyāna-maḍḍhu-ḍḍhatiyā.}
\suttaitem{423}{106}{dutiyassa saha diyaḍḍhadivaḍḍhā.}
\suttaitem{424}{107}{sare kada kussu-ttaratthe.}
\suttaitem{425}{108}{kā-ppatthe.}
\suttaitem{426}{109}{purise vā.}
\suttaitem{427}{110}{pubbā-para-jja-sāya-majjhehā-hassa ṇho.}
\end{suttalist}
\begin{jieshu}
iti moggallāne byākaraṇe samāsakaṇḍo tatiyo.
\end{jieshu}
%校对至此
\section{ṇādikaṇḍo catuttho}
\markboth{Moggallānasuttapāṭhe}{ṇādikaṇḍo catuttho}
\begin{suttalist}
\suttaitem{428}{1}{ṇo vā pacce.}
\suttaitem{429}{2}{vacchādito ṇānaṇāyanā.}
\suttaganaitem{430}{19}{katā ṇiyova.}
\suttaganaitem{431}{20}{kaṇho brāhmaṇe.}
\suttaitem{432}{3}{kattikāvidhavādīhi ṇeyyaṇerā.}
\suttaitem{433}{4}{ṇya diccādīhi.}
\suttaitem{434}{5}{ā ṇi.}
\suttaitem{435}{6}{rājato ñño jātiyaṃ.}
\suttaitem{436}{7}{khattā yiyā.}
\suttaitem{437}{8}{manuto ssasaṇa.}
\suttaitem{438}{9}{janapadanāmasmā khattiyā raññe ca ṇo.}
\suttaitem{439}{10}{ṇya kurusivīhi.}
\suttaitem{440}{11}{ṇa rāgā tena rattaṃ.}
\suttaitem{441}{12}{nakkhatte-ninduyuttena kāle.}
\suttaitem{442}{13}{sā-ssa devatā puṇṇamāsī.}
\suttaitem{443}{14}{tamadhīte taṃ jānāti kaṇikā ca.}
\suttaitem{444}{15}{tassa visaye dese.}
\suttaitem{445}{16}{nivāse tannāme.}
\suttaitem{446}{17}{adūrabhave.}
\suttaitem{447}{18}{tena nibbatte.}
\suttaitem{448}{19}{tamīdhatthi.}
\suttaitem{449}{20}{tatra bhave.}
\suttaitem{450}{21}{ajjādīhi tano.}
\suttaitem{451}{22}{purāto ṇo ca.}
\suttaitem{452}{23}{amātvacco.}
\suttaitem{453}{24}{majjhāditvimo.}
\suttaitem{454}{25}{kaṇa ṇeyya ṇeyyaka yiyā.}
\suttaitem{455}{26}{ṇiko.}
\suttaitemmulti{456}{27}{tamassa sippaṃ sīlaṃ paṇyaṃ paharaṇaṃ payojanaṃ.}
\suttaitem{457}{28}{taṃ hanta rahati gacchatuñchati carati.}
\suttaitemmulti{458}{29}{tena kataṃ kītaṃ baddhamabhisaṅkhataṃ saṃsaṭṭhaṃ hataṃ hanti jitaṃ jayati dibbati khaṇati tarati carati vahati jīvati.}
\suttaitem{459}{30}{tassa saṃvattati.}
\suttaitem{460}{31}{tato sambhūtamāgataṃ.}
\suttaitem{461}{32}{tattha vasati vidito bhatto niyutto.}
\suttaitem{462}{33}{tassidaṃ.}
\suttaitem{463}{34}{ṇo.}
\suttaitem{464}{35}{gavādīhi yo.}
\suttaitem{465}{36}{pitito bhātari reyyaṇa.}
\suttaitem{466}{37}{mātito ca bhaginiyaṃ cho.}
\suttaitem{467}{38}{mātāpitūsvā-maho.}
\suttaitem{468}{39}{hite reyyaṇa.}
\suttaitem{469}{40}{nindā-ññāta-ppapaṭibhāgarassa dayāsaññāsu ko.}
\suttaganaitem{470}{21}{vatthito ivatthe eyyo.}
\suttaganaitem{471}{22}{silāya ṇeyyo ca.}
\suttaganaitem{472}{23}{sākhādīhi iyo.}
\suttaganaitem{473}{24}{mukhādīhi yo.}
\suttaganaitem{474}{25}{ākasmike bhidheye īyo.}
\suttaganaitem{475}{26}{sakkarādīhi ṇo.}
\suttaganaitem{476}{27}{aṅgulyādīhi ṇiko.}
\suttaitem{477}{41}{tamassa parimāṇaṃ ṇiko ca.}
\suttaitem{478}{42}{yate-tehi ttako.}
\suttaitem{479}{43}{sabbā cā-vanthu.}
\suttaitem{480}{44}{kimhā rati rīva rīvataka rittakā.}
\suttaitem{481}{45}{sañjātaṃ tārakāditvito.}
\suttaitem{482}{46}{māne matto.}
\suttaitem{483}{47}{taggho cuddhaṃ.}
\suttaitem{484}{48}{ṇo ca purisā.}
\suttaitem{485}{49}{ayubhadvitīhaṃse.}
\suttaitemmulti{486}{50}{saṅkhyāya saccutīsā-sa, dasantā-dhikā-smiṃ satasahasse ḍo.}
\suttaitem{487}{51}{tassa pūraṇe-kādasādito vā.}
\suttaitem{488}{52}{ma pañcādikatīhi.}
\suttaitem{489}{53}{satādīnami ca.}
\suttaitem{490}{54}{chā ṭṭhaṭṭhamā.}
\suttaitem{491}{55}{ekā kākya-sahāye.}
\suttaitem{492}{56}{vacchādīhi tanutte taro.}
\suttaitem{493}{57}{kimhā niddhāraṇe ratara ratamā.}
\suttaitem{494}{58}{tena datte liyā.}
\suttaitem{495}{59}{tassa bhāvakammesu tta tā ttana ṇya ṇeyyaṇiya ṇiyā.}
\suttaitem{496}{60}{bya vaddhadāsā vā.}
\suttaitem{497}{61}{naṇa yuvā bo ca vassa.}
\suttaitem{498}{62}{aṇvāditvimo.}
\suttaitem{499}{63}{bhāvā tena nibbatte.}
\suttaitem{500}{64}{tara tami-ssikiyiṭṭhātisaye.}
\suttaitem{501}{65}{tannissite llo.}
\suttaitem{502}{66}{tassa vikārāvayavesu ṇa ṇika ṇeyya mayā.}
\suttaitem{503}{67}{jatuto ssaṇa vā.}
\suttaitem{504}{68}{samūhe kaṇa ṇa ṇikā.}
\suttaitem{505}{69}{janādīhi tā.}
\suttaitem{506}{70}{iyo hite.}
\suttaitem{507}{71}{cakkhvādito sso.}
\suttaitem{508}{72}{ṇyo tattha sādhu.}
\suttaitem{509}{73}{kammā niyaññā.}
\suttaitem{510}{74}{kathāditviko.}
\suttaitem{511}{75}{pathādīhi ṇeyyo.}
\suttaitem{512}{76}{dakkhiṇāyā-rahe.}
\suttaitem{513}{77}{rāyo tumantā.}
\suttaitem{514}{78}{tamettha-ssa-tthīti mantu.}
\suttaitem{515}{79}{vantvavaṇṇā.}
\suttaitem{516}{80}{daṇḍāditvika ī vā.}
\suttaganaitem{517}{28}{uttamīṇe va dhanā iko.}
\suttaganaitem{518}{29}{asannihite atthā.}
\suttaganaitem{519}{30}{tadantā ca.}
\suttaganaitem{520}{31}{vaṇṇantā īyeva.}
\suttaganaitem{521}{32}{hattha dantehi jātiyaṃ.}
\suttaganaitem{522}{33}{vaṇṇato brahmacārimhi.}
\suttaganaitem{523}{34}{pokkharādito dese.}
\suttaganaitem{524}{35}{nāvāyi-ko.}
\suttaganaitem{525}{36}{sukhadukkhā ī.}
\suttaganaitem{526}{37}{sikhādīhi vā.}
\suttaganaitem{527}{38}{balā bāhūrupubbā ca.}
\suttaitem{528}{81}{tapādīhi ssī.}
\suttaitem{529}{82}{mukhādito ro.}
\suttaganaitem{530}{39}{dantassu ca unnatadante.}
\suttaitem{531}{83}{tundyādīhi bho.}
\suttaitem{532}{84}{saddhāditva.}
\suttaitem{533}{85}{ṇo tapā.}
\suttaitem{534}{86}{ālvabhijjhādīhi.}
\suttaitem{535}{87}{picchāditvilo.}
\suttaitem{536}{88}{sīlādito vo.}
\suttaganaitem{537}{40}{aṇṇā niccaṃ.}
\suttaganaitem{538}{41}{gāṇḍirājīhi saññāyaṃ.}
\suttaitem{539}{89}{māyāmedhāhi vī.}
\suttaitem{540}{90}{sissare āmyuvāmī.}
\suttaitem{541}{91}{lakkhyā ṇo a ca.}
\suttaitem{542}{92}{aṅgā no kalyāṇe.}
\suttaitem{543}{93}{so lomā.}
\suttaitem{544}{94}{imiyā.}
\suttaitem{545}{95}{to pañcamyā.}
\suttaitem{546}{96}{ito tetto kuto.}
\suttaitem{547}{97}{abhyādīhi.}
\suttaitem{548}{98}{ādyādīhi.}
\suttaitem{549}{99}{sabbādito sattamyā tratthā.}
\suttaitem{550}{100}{katthe-tthakutrā-trakve-hidha.}
\suttaitem{551}{101}{dhi sabbā vā.}
\suttaitem{552}{102}{yā hiṃ.}
\suttaitem{553}{103}{tā haṃ ca.}
\suttaitem{554}{104}{kuhiṃ kahaṃ.}
\suttaitem{555}{105}{sabbe-kañña ya tehi kāle dā.}
\suttaitem{556}{106}{kadā kudā sadā-dhune-dāni.}
\suttaitem{557}{107}{ajjasajjvaparajjve-tarahikarahā.}
\suttaitem{558}{108}{sabbādīhi pakāre thā.}
\suttaitem{559}{109}{kathamitthaṃ.}
\suttaitem{560}{110}{dhā saṅkhyāhi.}
\suttaitem{561}{111}{vekā jjhaṃ.}
\suttaitem{562}{112}{dvitīhedhā.}
\suttaitem{563}{113}{tabbati jātiyo.}
\suttaitem{564}{114}{vārasaṅkhyāya kkhattuṃ.}
\suttaitem{565}{115}{katimhā.}
\suttaitem{566}{116}{bahumhā dhā ca paccāsattiyaṃ.}
\suttaitem{567}{117}{sakiṃ vā.}
\suttaitem{568}{118}{so vīcchā pakāresu.}
\suttaitem{569}{119}{abhūtatabbhāve karāsabhūyoge vikārā cī.}
\suttaitem{570}{120}{dissantaññepi paccayā.}
\suttaitem{571}{121}{aññasmiṃ.}
\suttaitem{572}{122}{sakatthe.}
\suttaitem{573}{123}{lopo.}
\suttaitem{574}{124}{sarānamādissā-yuvaṇṇassā e o ṇānubandhe.}
\suttaitem{575}{125}{saṃyoge kvaci.}
\suttaitem{576}{126}{majjhe.}
\suttaitemmulti{577}{127}{kosajjājjava pārisajja sohajja maddavārissāsabhājañña theyya bāhusaccā.}
\suttaitem{578}{128}{manādīnaṃ saka.}
\suttaitem{579}{129}{uvaṇṇassā-vaṅa sare.}
\suttaitem{580}{130}{yamhi gossa ca.}
\suttaitem{581}{131}{lopo-vaṇṇivaṇṇānaṃ.}
\suttaitem{582}{132}{rānubandhe-ntasarādissa.}
\suttaitem{583}{133}{kisamahatamime kasamahā.}
\suttaitem{584}{134}{āyussā-yasa mantumhi.}
\suttaitem{585}{135}{jo vuddhassiyiṭṭhesu.}
\suttaitem{586}{136}{bāḷhantikapasatthānaṃ sādha neda sā.}
\suttaitem{587}{137}{kaṇakanā-ppayuvānaṃ.}
\suttaitem{588}{138}{lopo vīmantuvantūnaṃ.}
\suttaitem{589}{139}{ḍe satissa tissa.}
\suttaitem{590}{140}{etasseṭa ttake.}
\suttaitem{591}{141}{ṇikassi yo vā.}
\suttaitem{592}{142}{adhātussa ke-syādito ghe-ssi.}
\end{suttalist}
\begin{jieshu}
iti moggallāne byākaraṇe ṇādikaṇḍo catuttho.
\end{jieshu}

\section{khādikaṇḍo pañcamo}
\markboth{Moggallānasuttapāṭhe}{khādikaṇḍo pañcamo}
\begin{suttalist}
\suttaitem{593}{1}{tijamānehi khasā khamāvīmaṃsāsu.}
\suttaitem{594}{2}{kitā tikicchāsaṃsayesu cho.}
\suttaitem{595}{3}{nindāyaṃ gupabadhā bassa bhoca.}
\suttaitem{596}{4}{tuṃsmā lopo cicchāyaṃ te.}
\suttaitem{597}{5}{īyo kammā.}
\suttaitem{598}{6}{upamā-nācāre.}
\suttaitem{599}{7}{ādhārā.}
\suttaitem{600}{8}{kattutā-yo.}
\suttaitem{601}{9}{cyatthe.}
\suttaitem{602}{10}{saddādīni karoti.}
\suttaitem{603}{11}{namotvasso.}
\suttaitem{604}{12}{dhātvatthe nāmasmi.}
\suttaitem{605}{13}{saccādīhāpi.}
\suttaitem{606}{14}{kriyatthā.}
\suttaitem{607}{15}{curādito ṇi.}
\suttaitem{608}{16}{payojakabyāpāre kāpi ca.}
\suttaitem{609}{17}{kyo bhāvakammesvaparokkhesu mānantatyādīsu.}
\suttaitem{610}{18}{kattari lo.}
\suttaitem{611}{19}{maṃ ca rudhādīnaṃ.}
\suttaitem{612}{20}{ṇiṇāpyāpīhi vā.}
\suttaitem{613}{21}{divādīhi yaka.}
\suttaitem{614}{22}{tudādīhi ko.}
\suttaitem{615}{23}{jyādīhiknā.}
\suttaitem{616}{24}{kyādīhi kṇā.}
\suttaitem{617}{25}{svādīhi kṇo.}
\suttaitem{618}{26}{tanāditvo.}
\suttaitem{619}{27}{bhāvakammesu tabbānīyā.}
\suttaitem{620}{28}{ghyaṇa.}
\suttaitem{621}{29}{āsse ca.}
\suttaitem{622}{30}{vadādīhi yo.}
\suttaganaitem{623}{42}{bhujānne.}
\suttaitem{624}{31}{kicca ghacca bhacca bhabba leyyā.}
\suttaganaitem{625}{43}{saññāyaṃ bharā.}
\suttaitem{626}{32}{guhādīhi yaka.}
\suttaitem{627}{33}{kattari ltuṇakā.}
\suttaitem{628}{34}{āvī.}
\suttaitem{629}{35}{āsiṃsāya-mako.}
\suttaitem{630}{36}{karā ṇano.}
\suttaitem{631}{37}{hāto vīhikālesu.}
\suttaitem{632}{38}{vidā kū.}
\suttaitem{633}{39}{vito ñāto.}
\suttaitem{634}{40}{kammā.}
\suttaitem{635}{41}{kva caṇa.}
\suttaitem{636}{42}{gamā rū.}
\suttaitemmulti{637}{43}{samānaññabhavantayāditūpamānā disā kammerīrikkhākā.}
\suttaitem{638}{44}{bhāvakārake svaghaṇaghakā.}
\suttaitem{639}{45}{dādhātvi.}
\suttaitem{640}{46}{vamādīhyathu.}
\suttaitem{641}{47}{kvi.}
\suttaitem{642}{48}{ano.}
\suttaitem{643}{49}{itthiyamaṇa tti ka yakayā ca.}
\suttaitem{644}{50}{jāhāhi ni.}
\suttaitem{645}{51}{karā ririyo.}
\suttaitem{646}{52}{i ki tī sarūpe.}
\suttaitem{647}{53}{sīlā-bhikkhaññā-vassakesu ṇī.}
\suttaitemmulti{648}{54}{thāvari-ttara, bhaṅgura, bhidura, bhāsura, bhassarā.}
\suttaitem{649}{55}{kattari bhūte ktvantuttāvī.}
\suttaitem{650}{56}{kto bhāvakammesu.}
\suttaitem{651}{57}{kattari cārambhe.}
\suttaitemmulti{652}{58}{ṭhā-sa, vasa, silisa, sī, ruha, jara, janīhi.}
\suttaitem{653}{59}{gamanatthā kammakādhāre ca.}
\suttaitem{654}{60}{āhāratthā.}
\suttaitemmulti{655}{61}{tuṃ tāye tave bhāve bhavissati kriyāyaṃ tadatthāyaṃ.}
\suttaitem{656}{62}{paṭisedhe-laṃkhalūnaṃ, tunaktvāna, ktvā vā.}
\suttaitem{657}{63}{pubbe-kakattukānaṃ.}
\suttaitem{658}{64}{nto kattari vattamāne.}
\suttaitem{659}{65}{māno.}
\suttaitem{660}{66}{bhāvakammesu.}
\suttaitem{661}{67}{te ssapubbā-nāgate.}
\suttaitem{662}{68}{ṇvādayo.}
\suttaitem{663}{69}{khachasānamekassarodi dve.}
\suttaitem{664}{70}{parokkhāyañca.}
\suttaitem{665}{71}{ādismā sarā.}
\suttaitem{666}{72}{na puna.}
\suttaitem{667}{73}{yathiṭṭhaṃ syādino.}
\suttaitem{668}{74}{rasso pubbassa.}
\suttaitem{669}{75}{lopo-nādibyañjanassa.}
\suttaitem{670}{76}{khachasesvassi.}
\suttaitem{671}{77}{gupissussa.}
\suttaitem{672}{78}{catuttha dutiyānaṃ tatiyapaṭhamā.}
\suttaitem{673}{79}{kavaggahānaṃ cavaggajā.}
\suttaitem{674}{80}{mānassa vī parassa ca maṃ.}
\suttaitem{675}{81}{kitassā-saṃsaye ti vā.}
\suttaitem{676}{82}{yuvaṇṇāname o paccaye.}
\suttaitem{677}{83}{lahussupāntassa.}
\suttaitem{678}{84}{assā ṇānubandhe.}
\suttaitem{679}{85}{na te kānubandhanāgamesu.}
\suttaitem{680}{86}{vā kvaci.}
\suttaitem{681}{87}{aññatrā pi.}
\suttaitem{682}{88}{pye sissā.}
\suttaitem{683}{89}{eonamayavā sare.}
\suttaitem{684}{90}{āyāvā ṇānubandhe.}
\suttaitem{685}{91}{āssā ṇāpimhi yuka.}
\suttaitem{686}{92}{padādīnaṃ kvaci.}
\suttaitem{687}{93}{maṃ vā rudhādīnaṃ.}
\suttaitem{688}{94}{kvimhi lopo-nta byañjanassa.}
\suttaitem{689}{95}{pararūpamayakāre byañjane.}
\suttaitem{690}{96}{manānaṃ niggahītaṃ.}
\suttaitem{691}{97}{na brūsso.}
\suttaitem{692}{98}{kagā cajānaṃ ghānubandhe.}
\suttaitem{693}{99}{hanassa ghāto ṇānubandhe.}
\suttaitem{694}{100}{kvimhi gho paripaccāsamohi.}
\suttaitem{695}{101}{parassa ghaṃse.}
\suttaitem{696}{102}{jiharānaṃ gī.}
\suttaitem{697}{103}{dhāssa ho.}
\suttaitem{698}{104}{ṇimhi dīgho dusassa.}
\suttaitem{699}{105}{guhissa sare.}
\suttaitem{700}{106}{muhabahānañca te kānubandhe tve.}
\suttaitem{701}{107}{vahassussa.}
\suttaitem{702}{108}{dhāssa hi.}
\suttaitem{703}{109}{gamādirānaṃ lopo-ntassa.}
\suttaitem{704}{110}{vacādīnaṃ vassuṭa vā.}
\suttaitem{705}{111}{assu.}
\suttaitem{706}{112}{vaddhassa vā.}
\suttaitem{707}{113}{yajassa yassa ṭiyī.}
\suttaitem{708}{114}{ṭhāssi.}
\suttaitem{709}{115}{gāpānamī.}
\suttaitem{710}{116}{janissā.}
\suttaitem{711}{117}{sāsassa sisa vā.}
\suttaitem{712}{118}{karassā tave.}
\suttaitem{713}{119}{tuṃtunatabbesu vā.}
\suttaitem{714}{120}{ñāssa ne jā.}
\suttaitem{715}{121}{sakāpānaṃ kuṇakū ṇe.}
\suttaitem{716}{122}{nito cissa cho.}
\suttaitem{717}{123}{jarasadānamīma vā.}
\suttaitemmulti{718}{124}{disassa passa dassa dasa da dakkhā.}
\suttaitem{719}{125}{samānā ro rīrikkhakesu.}
\suttaitem{720}{126}{dahassa dassa ḍo.}
\suttaitem{721}{127}{anaghaṇasvāparīhi ḷo.}
\suttaitem{722}{128}{atyādintesvatthissa bhū.}
\suttaitem{723}{129}{aāssaādīsu.}
\suttaitem{724}{130}{ntamānāntiyiyuṃ svādilopo.}
\suttaitem{725}{131}{pādito ṭhāssa vā ṭhaho kvaci.}
\suttaitem{726}{132}{dāssi yaṅa.}
\suttaitem{727}{133}{karotissa kho.}
\suttaitem{728}{134}{purā smā.}
\suttaitem{729}{135}{nito kamassa.}
\suttaitem{730}{136}{yuvaṇṇānamiyaṅuvaṅa sare.}
\suttaitem{731}{137}{aññādissāssī kye.}
\suttaitem{732}{138}{tanassā vā.}
\suttaitem{733}{139}{dīgho sarassa.}
\suttaitem{734}{140}{sā-nantarassa tassa ṭho.}
\suttaitem{735}{141}{kasassima ca vā.}
\suttaitem{736}{142}{dhastotrastā.}
\suttaitem{737}{143}{pucchādito.}
\suttaitemmulti{738}{144}{sāsa, vasa, saṃsa, sasā tho.}
\suttaitem{739}{145}{dho dhahabhehi.}
\suttaitem{740}{146}{dahā ḍho.}
\suttaitem{741}{147}{bahassuma ca.}
\suttaitem{742}{148}{ruhādīhi ho ḷa ca.}
\suttaitem{743}{149}{muhā vā.}
\suttaitem{744}{150}{bhidādito no ktaktavantūnaṃ.}
\suttaitem{745}{151}{dātvinno.}
\suttaitem{746}{152}{kirādīhi ṇo.}
\suttaitem{747}{153}{tarādīhi riṇṇo.}
\suttaitem{748}{154}{go bhanjādīhi.}
\suttaitem{749}{155}{susā kho.}
\suttaitem{750}{156}{pacā ko.}
\suttaitem{751}{157}{mucā vā.}
\suttaitem{752}{158}{lopo vaḍḍhā ktissa.}
\suttaitem{753}{159}{kvissa.}
\suttaitem{754}{160}{ṇiṇāpīnaṃ tesu.}
\suttaitem{755}{161}{kvaci vikaraṇānaṃ.}
\suttaitem{756}{162}{mānassa massa.}
\suttaitem{757}{163}{ñi lasse.}
\suttaitem{758}{164}{pyo vā tvāssa samāse.}
\suttaitem{759}{165}{tuṃyānā.}
\suttaitem{760}{166}{hanā racco.}
\suttaitem{761}{167}{sāsādhikarā ca ca riccā.}
\suttaitem{762}{168}{ito cco.}
\suttaitem{763}{169}{disā vānavāsa ca.}
\suttaitem{764}{170}{ñi byañjanassa.}
\suttaitem{765}{171}{rā nassa ṇo.}
\suttaitem{766}{172}{na ntamānatyādīnaṃ.}
\suttaitem{767}{173}{gamayamisāsadisānaṃ vā cchaṅa.}
\suttaitem{768}{174}{jaramarānamīyaṅa.}
\suttaitem{769}{175}{ṭhāpānaṃ tiṭṭha pivā.}
\suttaitemmulti{770}{176}{gamavadadānaṃ ghamma vajja dajjā.}
\suttaitemmulti{771}{177}{karassa sossa kubba kuru kayirā.}
\suttaitem{772}{178}{gahassa gheppo.}
\suttaitem{773}{179}{ṇo niggahītassa.}
\end{suttalist}
\begin{jieshu}
iti moggallāne byākaraṇe khādikaṇḍo pañcamo.
\end{jieshu}

\section{tyādikaṇḍo chaṭṭho}
\markboth{Moggallānasuttapāṭhe}{tyādikaṇḍo chaṭṭho}
\begin{suttalist}
\suttaitemmulti{774}{1}{vattamāne ti anti, si tha, mi ma, te ante, se vhe, e mhe.}
\suttaitemmulti{775}{2}{bhavissati ssati ssanti, ssasi ssatha, ssāmi ssāma, ssate ssante, ssase ssavhe, ssaṃ ssāmhe.}
\suttaitem{776}{3}{nāme garahāvimhayesu.}
\suttaitemmulti{777}{4}{bhūte ī uṃ, o ttha, iṃ mhā, ā ū, se vhaṃ, a mhe.}
\suttaitemmulti{778}{5}{anajjatane ā ū, o ttha, a mhā, ttha tthuṃ, se vhaṃ, iṃ muse.}
\suttaitemmulti{779}{6}{parokkhe a u, e ttha, a mha, ttha re, ttho vho, i mhe.}
\suttaitemmulti{780}{7}{eyyādo vā tipattiyaṃ ssā ssaṃsu, sse ssatha, ssaṃ ssāmhā, ssatha ssiṃsu, ssase ssavhe, ssiṃ ssāmhase.}
\suttaitemmulti{781}{8}{hetuphalesveyya eyyuṃ, eyyāsi eyyātha, eyyāmi eyyāma, etha eraṃ, etho eyyāvho, eyyaṃ eyyāmhe.}
\suttaitem{782}{9}{pañcapatthanāvidhīsu.}
\suttaitemmulti{783}{10}{tu antu, hi tha, mima, taṃ antaṃ, ssu vho, e āmadhasa.}
\suttaitem{784}{11}{satyarahesveyyādī.}
\suttaitem{785}{12}{sambhāvane vā.}
\suttaitem{786}{13}{māyoge īāādī.}
\suttaitemmulti{787}{14}{pubbāparacchakkāna mekānekesu tumhāmhasesesu dvedve majjhimuttamapaṭhamā.}
\suttaitem{788}{15}{āīssādīsvau vā.}
\suttaitem{789}{16}{aādīsvāho brūssa.}
\suttaitem{790}{17}{bhūssa vuka.}
\suttaitem{791}{18}{pubbassa a.}
\suttaitem{792}{19}{ussaṃ svāhā vā.}
\suttaitem{793}{20}{tyantīnaṃ ṭaṭū.}
\suttaitem{794}{21}{īādo vacassoma.}
\suttaitem{795}{22}{dāssa daṃ vā mimesvadvitte.}
\suttaitem{796}{23}{karassa sossa kuṃ.}
\suttaitem{797}{24}{kā īādīsu.}
\suttaitem{798}{25}{hāssa cāhaṅa ssena.}
\suttaitem{799}{26}{labhavasacchidabhidarudānaṃ cchaṅa.}
\suttaitem{800}{27}{bhuja mūca vaca visānaṃ kkhaṅa.}
\suttaitem{801}{28}{āīādīsu harassā.}
\suttaitem{802}{29}{gamissa.}
\suttaitem{803}{30}{ḍaṃsassa ca chaṅa.}
\suttaitem{804}{31}{hūssa he hehi hohī ssatyādo.}
\suttaitem{805}{32}{ṇānāsu rasso.}
\suttaitem{806}{33}{āīūmhāssāssamhānaṃ vā.}
\suttaitem{807}{34}{kusaruhehī-ssa chi.}
\suttaitem{808}{35}{aīssaādīnaṃ byañjanassiu.}
\suttaitem{809}{36}{brūto tissīu.}
\suttaitem{810}{37}{kyassa.}
\suttaitemmulti{811}{38}{eyyāthasseaāīthānaṃ oaaṃtthatthovhoka.}
\suttaitem{812}{39}{uṃssiṃ svaṃsu.}
\suttaitem{813}{40}{eottā suṃ.}
\suttaitem{814}{41}{hūto resuṃ.}
\suttaitem{815}{42}{ossa aitthattho.}
\suttaitem{816}{43}{si.}
\suttaitem{817}{44}{dīghā īssa.}
\suttaitem{818}{45}{mhātthāna mha.}
\suttaitem{819}{46}{iṃssa ca siu.}
\suttaitem{820}{47}{eyyuṃssuṃ.}
\suttaitem{821}{48}{hissa-to lopo.}
\suttaitem{822}{49}{kyassa sse.}
\suttaitem{823}{50}{atthiteyyādicchannaṃ sa su sa satha saṃ sāma.}
\suttaitem{824}{51}{ādidvinnamiyāiyuṃ.}
\suttaitem{825}{52}{tassa tho.}
\suttaitem{826}{53}{sihisvaṭa.}
\suttaitem{827}{54}{mimānaṃ vā mimhā ca.}
\suttaitem{828}{55}{esuṅa.}
\suttaitem{829}{56}{īādo dīgho.}
\suttaitem{830}{57}{himimesvassa.}
\suttaitem{831}{58}{sakā ṇāssa kha īādo.}
\suttaitem{832}{59}{sse vā.}
\suttaitem{833}{60}{tesu suto kṇokṇānaṃ roṭa.}
\suttaitem{834}{61}{ñāssa sanāssa nāyo timhi.}
\suttaitem{835}{62}{ñāmhi jaṃ.}
\suttaitem{836}{63}{eyyāssiyāñā vā.}
\suttaitem{837}{64}{īssatyādīsu knālopo.}
\suttaitem{838}{65}{ssassa hi kamme.}
\suttaitem{839}{66}{etismā.}
\suttaitem{840}{67}{hanā cha khā.}
\suttaitem{841}{68}{hato ha.}
\suttaitem{842}{69}{dakkhakhahehi hohīhi lopo.}
\suttaitem{843}{70}{kayireyyasseyyumādīnaṃ.}
\suttaitem{844}{71}{ṭā.}
\suttaitem{845}{72}{ethassā.}
\suttaitem{846}{73}{labhā iṃīnaṃ thaṃthā vā.}
\suttaitem{847}{74}{gurupubbā rassā re-nte-ntinaṃ.}
\suttaitem{848}{75}{eyyeyyāseyyannaṃ ṭe.}
\suttaitem{849}{76}{ovikaraṇassu paracchakke.}
\suttaitem{850}{77}{pubbacchakke vā kvaci.}
\suttaitem{851}{78}{eyyāmasse muca.}
\end{suttalist}
\begin{jieshu}
iti moggallāne byākaraṇe tyādikaṇḍo chaṭṭho.
\end{jieshu}

\section{ṇvādikaṇḍo sattamo}
\markboth{Moggallānasuttapāṭhe}{ṇvādikaṇḍo sattamo}
\begin{suttalist}
\suttaitemmulti{852}{1}{cara dara kara raha jana sana tala sāda sādha kasāsa caṭā ya vāhi ṇu.}
\suttaitemmulti{853}{2}{bhara mara cara tara ara gara hana tana mana bhama kita dhana baṃha kambamba cakkha bhikkha saṃkindanda yaja paṭāṇāsa vasa pasa paṃsa bandhā u.}
\suttaitem{854}{3}{bandhā ū vadho ca.}
\suttaitem{855}{4}{jambādayo.}
\suttaitem{856}{5}{tapusa vidha kura putha mudā ku.}
\suttaitem{857}{6}{sindhādayo.}
\suttaitem{858}{7}{i.}
\suttaitem{859}{8}{dadhyādayo.}
\suttaitem{860}{9}{yuvaṇṇupantā ki.}
\suttaitemmulti{861}{10}{vapa vara vasa rasa nabha hara hana paṇā īṇa.}
\suttaitem{862}{11}{bhū gamā īṇa.}
\suttaitem{863}{12}{tanda lakkhā ī.}
\suttaitem{864}{13}{gamā ro.}
\suttaitem{865}{14}{i bhī kā karāra vaka saka vāhi ko.}
\suttaitem{866}{15}{ūkādako.}
\suttaitem{867}{16}{bhītvā nako.}
\suttaitem{868}{17}{siṅghā āṇi kāṭakā.}
\suttaitem{869}{18}{karāditvako.}
\suttaitem{870}{19}{bala pate hyāko.}
\suttaitem{871}{20}{sāmākādayo.}
\suttaitemmulti{872}{21}{vicchā la gama musā kiko.}
\suttaitem{873}{22}{kiṃ kaṇikādayo.}
\suttaitem{874}{23}{i sā kīko.}
\suttaitem{875}{24}{kama padā ṇuko.}
\suttaitem{876}{25}{maṇḍa salā ṇūko.}
\suttaitem{877}{26}{lūkādayo.}
\suttaitem{878}{27}{kasā sako.}
\suttaitem{879}{28}{karā tiko.}
\suttaitem{880}{29}{isā ṭhakana.}
\suttaitem{881}{30}{samā kho.}
\suttaitem{882}{31}{mukhādayo.}
\suttaitemmulti{883}{32}{aja vaja muda gada gamā gaka.}
\suttaitem{884}{33}{siṅgādayo.}
\suttaitem{885}{34}{agā gi.}
\suttaitem{886}{35}{yāvalā gu.}
\suttaitem{887}{36}{phegvādayo.}
\suttaitem{888}{37}{janā gho.}
\suttaitem{889}{38}{meghādayo.}
\suttaitem{890}{39}{cusara varā co.}
\suttaitem{891}{40}{marā cuīcīca.}
\suttaitem{892}{41}{kusa pasā chika.}
\suttaitem{893}{42}{kasausā chuka.}
\suttaitemmulti{894}{43}{asa masa vasa kuca kacā cho.}
\suttaitem{895}{44}{gucchādayo.}
\suttaitem{896}{45}{arā ju uṭa ca.}
\suttaitem{897}{46}{rajjādayo.}
\suttaitem{898}{47}{gidhā jhaka.}
\suttaitem{899}{48}{vañcyādayo.}
\suttaitem{900}{49}{kama yajā ño.}
\suttaitem{901}{50}{puññaṃ.}
\suttaitemmulti{902}{51}{arahāñño hāssa hira ca.}
\suttaitem{903}{52}{kira tarā kīṭo.}
\suttaitem{904}{53}{sakādīhyaṭo.}
\suttaitemmulti{905}{54}{makuṭāvāṭa kavāṭa kukkuṭā.}
\suttaitem{906}{55}{kamusa kusa kasā ṭho.}
\suttaitem{907}{56}{kuṭṭhādayo.}
\suttaitem{908}{57}{vara karā aṇḍo.}
\suttaitem{909}{58}{manantā ḍo.}
\suttaitem{910}{59}{kuṇḍādayo.}
\suttaitemmulti{911}{60}{tija kasa tasa dakkhā kiṇo jassa kho ca.}
\suttaitem{912}{61}{vīādito ṇi.}
\suttaitem{913}{62}{gahādīhya ṇi.}
\suttaitem{914}{63}{rīvībhāhi ṇu.}
\suttaitem{915}{64}{khāṇvādayo.}
\suttaitem{916}{65}{kvāditoṇo.}
\suttaitem{917}{66}{suvīhi ṇaka.}
\suttaitem{918}{67}{tiṇādayo.}
\suttaitem{919}{68}{ravaṇa varaṇa pūraṇādayo.}
\suttaitem{920}{69}{pāvasā ati.}
\suttaitemmulti{921}{70}{dhāhisi tana jana jara gama sacā tu.}
\suttaitem{922}{71}{arissuṭa ca.}
\suttaitem{923}{72}{pitvādayo.}
\suttaitem{924}{73}{jana karā ratu.}
\suttaitem{925}{74}{sakā unto.}
\suttaitem{926}{75}{kapā oto.}
\suttaitem{927}{76}{vasādīhyanto.}
\suttaitem{928}{77}{hisīnaṃ muka ca.}
\suttaitem{929}{78}{hara ruha kulā ito.}
\suttaitem{930}{79}{bharādīhyato.}
\suttaitem{931}{80}{kirādīhyā-taka.}
\suttaitem{932}{81}{amādīhya-tto.}
\suttaitem{933}{82}{vādīhi to.}
\suttaitem{934}{83}{gharādīhi taka.}
\suttaitem{935}{84}{nettādayo.}
\suttaitem{936}{85}{samādīhyatho.}
\suttaitem{937}{86}{upavasā vassoṭa ca.}
\suttaitem{938}{87}{ramā thaka.}
\suttaitem{939}{88}{titthādayo.}
\suttaitem{940}{89}{vasa masa kusā thu.}
\suttaitem{941}{90}{saka vasā thi.}
\suttaitem{942}{91}{vīto thika.}
\suttaitem{943}{92}{sārismā rathi.}
\suttaitem{944}{93}{tā tā ithi.}
\suttaitem{945}{94}{isā thī.}
\suttaitemmulti{946}{95}{ruda khuda muda mada chida sūda sapa kamā daka.}
\suttaitem{947}{96}{kundādayo.}
\suttaitem{948}{97}{dadā du.}
\suttaitem{949}{98}{khanāna dama ramā dho.}
\suttaitem{950}{99}{muddhādayo.}
\suttaitem{951}{100}{sīto dhuka.}
\suttaitemmulti{952}{101}{varāra kara tara dara yamaajja mithasakā kuno.}
\suttaitem{953}{102}{ajā ino.}
\suttaitem{954}{103}{vipinādayo.}
\suttaitem{955}{104}{kirā kano.}
\suttaitem{956}{105}{dī ji i mīhi naka.}
\suttaitem{957}{106}{si dhā vī vāhi no.}
\suttaitem{958}{107}{ūnādayo.}
\suttaitem{959}{108}{vīpatā tano.}
\suttaitem{960}{109}{ramā tanaka.}
\suttaitem{961}{110}{sū bhāhi nuka.}
\suttaitem{962}{111}{dhāsse ca.}
\suttaitem{963}{112}{vattā ṭāva dhamāsehyani.}
\suttaitem{964}{113}{yuto ni.}
\suttaitem{965}{114}{camāpa pā vapā po.}
\suttaitem{966}{115}{yu thu kunaṃ dīgho ca.}
\suttaitem{967}{116}{khipa supa nī sū pūhi paka.}
\suttaitem{968}{117}{sippādayo.}
\suttaitem{969}{118}{sāsā apo.}
\suttaitem{970}{119}{viṭapādayo.}
\suttaitem{971}{120}{gupā pho.}
\suttaitem{972}{121}{gara sarādīhi bo.}
\suttaitem{973}{122}{nimbādayo.}
\suttaitem{974}{123}{darā bi.}
\suttaitemmulti{975}{124}{kara sara sala kala valla vasā abho.}
\suttaitem{976}{125}{gadā rabho.}
\suttaitem{977}{126}{usarā sā kato.}
\suttaitem{978}{127}{ito bhaka.}
\suttaitem{979}{128}{garāvā bho.}
\suttaitem{980}{129}{sobbhādayo.}
\suttaitemmulti{981}{130}{usa kusa pada sukhā kumo.}
\suttaitem{982}{131}{vaṭumādayo.}
\suttaitem{983}{132}{gudhā umo.}
\suttaitem{984}{133}{paṭha carā amimā.}
\suttaitem{985}{134}{hi dhūhi maka.}
\suttaitem{986}{135}{tīto rīsano ca.}
\suttaitemmulti{987}{136}{khī su vī yā gā hi sā lū khu hu mara dhara kara ghara jamā ma sāmā mo.}
\suttaitem{988}{137}{asmādayo.}
\suttaitem{989}{138}{nīto mi.}
\suttaitem{990}{139}{ūmi bhūmi nimi rasmi.}
\suttaitem{991}{140}{mā chāhi yo.}
\suttaitem{992}{141}{janissa jā ca.}
\suttaitem{993}{142}{hadayādayo.}
\suttaitemmulti{994}{143}{khī si sinī sī su vī ku sū hi raka.}
\suttaitem{995}{144}{hici du minaṃ dīgho ca.}
\suttaitem{996}{145}{dhātā namī ca.}
\suttaitem{997}{146}{bhadrādayo.}
\suttaitemmulti{998}{147}{mandaṅka sasā sa ma dha catā uro.}
\suttaitem{999}{148}{vidhurādayo.}
\suttaitemmulti{1000}{149}{timaruharudhabadhamadamandavajā jarucakasā kiro.}
\suttaitem{1001}{150}{thirādayo.}
\suttaitem{1002}{151}{dadagarehi dura bharā.}
\suttaitem{1003}{152}{cara dara jara gara marehite.}
\suttaitem{1004}{153}{pīto kvaro.}
\suttaitem{1005}{154}{cīvarādayo.}
\suttaitem{1006}{155}{kuto kraro.}
\suttaitem{1007}{156}{vasāsā charo.}
\suttaitem{1008}{157}{masā chero ca.}
\suttaitem{1009}{158}{dhūvāto saro.}
\suttaitem{1010}{159}{bhamādīhyaro.}
\suttaitem{1011}{160}{vadissa bada ca.}
\suttaitem{1012}{161}{vadajanānaṃ ṭhaṅa ca.}
\suttaitem{1013}{162}{pacissiṭhaṅa ca.}
\suttaitem{1014}{163}{vakā araṇa.}
\suttaitemmulti{1015}{164}{sigyaṅgāga majjakalā lā āro.}
\suttaitem{1016}{165}{kamissa ssu ca.}
\suttaitem{1017}{166}{bhiṅgā (ṅkā) rādayo.}
\suttaitem{1018}{167}{karā māro.}
\suttaitem{1019}{168}{pusa sarehi kharo.}
\suttaitem{1020}{169}{sara vasa kalā kīro vassuṭa ca.}
\suttaitem{1021}{170}{gabbhīrādayo.}
\suttaitem{1022}{171}{khajja valla masā ūro.}
\suttaitem{1023}{172}{kappūrādayo.}
\suttaitem{1024}{173}{kaṭha cakā oro.}
\suttaitem{1025}{174}{morādayo.}
\suttaitem{1026}{175}{kuto eraka.}
\suttaitem{1027}{176}{bhūsūhi rika.}
\suttaitem{1028}{177}{mīkasīnīhi ru.}
\suttaitem{1029}{178}{sinā eru.}
\suttaitem{1030}{179}{bhīruhi ruka.}
\suttaitem{1031}{180}{tamā būlo.}
\suttaitem{1032}{181}{sito lakavālā.}
\suttaitemmulti{1033}{182}{maṅga kama samba saba saka vasa visa keva kala palla kaṭha paṭa kuṇḍa maṇḍā alo.}
\suttaitem{1034}{183}{musā kalo.}
\suttaitem{1035}{184}{thalādayo.}
\suttaitem{1036}{185}{kulā kālo ca.}
\suttaitem{1037}{186}{muḷālādayo.}
\suttaitem{1038}{187}{caṇḍa patā ṇālo.}
\suttaitem{1039}{188}{mādito lo.}
\suttaitemmulti{1040}{189}{ana sana kala kuka saṭha mahā ilo.}
\suttaitem{1041}{190}{kuṭā kilo.}
\suttaitem{1042}{191}{sithilādayo.}
\suttaitem{1043}{192}{caṭa kaṇḍa vaṭṭa puthā kulo.}
\suttaitem{1044}{193}{tumulādayo.}
\suttaitemmulti{1045}{194}{kalla kapa takka paṭā olo.}
\suttaitem{1046}{195}{aṅgā ulo li.}
\suttaitem{1047}{196}{añjā li.}
\suttaitem{1048}{197}{chadā li.}
\suttaitem{1049}{198}{alyādayo.}
\suttaitem{1050}{199}{pilādī hya vo.}
\suttaitem{1051}{200}{sāḷavādayo.}
\suttaitem{1052}{201}{sarā āvo.}
\suttaitem{1053}{202}{ala mala bilā ṇuvo.}
\suttaitem{1054}{203}{gātvīvo.}
\suttaitem{1055}{204}{suto kva kvā.}
\suttaitem{1056}{205}{vidvā.}
\suttaitem{1057}{206}{thuto re vo.}
\suttaitem{1058}{207}{samā rivo.}
\suttaitem{1059}{208}{chadā ravi.}
\suttaitem{1060}{209}{pūra timā kiso rasso ca.}
\suttaitem{1061}{210}{karā īso.}
\suttaitem{1062}{211}{sirīsādayo.}
\suttaitem{1063}{212}{karā ribbi so.}
\suttaitemmulti{1064}{213}{sasāsa vasa visa hana vana manāna kamā so.}
\suttaitem{1065}{214}{āmi thu kusīto saka.}
\suttaitem{1066}{215}{phassādayo.}
\suttaitem{1067}{216}{suto ṇisaka.}
\suttaitemmulti{1068}{217}{ve tā ta yu panā la kala camā aso.}
\suttaitemmulti{1069}{218}{vaya diva kara kare hya saṇasakapāsa kasā.}
\suttaitem{1070}{219}{sasa masa daṃsā sā su.}
\suttaitem{1071}{220}{vidā dasuka.}
\suttaitem{1072}{221}{sasā rīho.}
\suttaitem{1073}{222}{jīvāmā ho vamā ca.}
\suttaitem{1074}{223}{taṇhādayo.}
\suttaitem{1075}{224}{paṇussahā hīhī ṇolaṅa ca.}
\suttaitemmulti{1076}{225}{khī mi pī cu mā vā kāhi ḷo ussa vā dīgho ca.}
\suttaitem{1077}{226}{guto ḷaka ca.}
\suttaitem{1078}{227}{paṅguḷādayo.}
\suttaitem{1079}{228}{pāto ḷi.}
\suttaitem{1080}{229}{vīto ḷu.}
\end{suttalist}
\begin{jieshu}
iti moggallāne byākaraṇe ṇvādikaṇḍo sattamo.
\end{jieshu}
\begin{jieshu}
Moggallānasuttapā​ṭho niṭṭhi​to.
\end{jieshu}
